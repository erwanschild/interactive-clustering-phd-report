\section{Espoirs portés sur la méthode proposée.}
\label{section:3.3-ESPOIRS-METHODE}

	% Rappel de la proposition.
	Nous avons proposé une méthodologie d'annotation basée sur une interaction entre l'Homme et la machine dans le but de soulager l'expert métier dans son intervention.
	
	% Annonce des espoirs.
	Grâce à une telle approche, nous pouvons espérer que :
	\begin{todolist}
		\item Les experts métiers n'auront désormais plus besoin de bagages analytiques ou techniques pour intervenir dans un projet d'annotation ;
		\item Les experts métiers pourront désormais participer à la modélisation d'une base d'apprentissage en ayant des discussions pragmatiques sur les cas d'usages des données manipulées ;
		\item Une telle méthodologie d'annotation permettra d'obtenir efficacement une base d'apprentissage stable et pertinente pour entraîner une modèle de classification d'intention ;
		\item Une telle méthodologie d'annotation sera réaliste en terme de délais et d'investissement financier.
	\end{todolist}
	
	% Transition au chapitre 4.
	Nous allons explorer diverses pistes pour confirmer ou infirmer ces espoirs dans le \textsc{Chapitre~\ref{chapter:4-ETUDES}}, et nous détaillerons nos conclusions dans un guide d'utilisation qui sera présenté dans le \textsc{Chapitre~\ref{chapter:5-GUIDE}}.