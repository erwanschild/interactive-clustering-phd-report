\chapter*{Introduction}
    \label{chapter:INTRODUCTION}
	\todo[inline]{CHAPITRE À REFORMULER FAÇON SWALES}

    %%%%%--------------------------------------------------------------------
    %%%%% Section I.1:
    %%%%%--------------------------------------------------------------------
    \section{"\textit{Asset centrality}"}
	\todo[inline]{SECTION À RÉDIGER}

    \begin{todolist}
    
        \item Des enjeux ou problèmes actuels

            •	Accessibilité à l'information :
                o	Grosses bases documentaires, pas toujours ordonnées ;

            •	Relations client à distance
                o   Besoin d'un accessibilité h24 ;

        \item Utilisation de plus en plus fréquente des chatbots
        
            •   Description succinte ;

            •   Cas d'usage usuels ;

            •   Tous les canaux d'utilisation ;

            •   Avantages et Dérives potentiels de l'utilisation (emploi, biais, pertinence, ergonomie, ...) ;

        \item Révolution techniques fréquentes (règles, classification, modèles)

            •	Moteurs de règles :
                o	Basé sur la détecté de mots clés,
                o	(+) facile à mettre en œuvre,
                o	(-) peu robuste au langage naturel,
                o	Paramétrage des réponses ;

            •	Paramétrage intentions-entités :
                o	Classification d’intention et/ou détection d’entités,
                o	(+) plus robuste au langage naturel, facile à paramétrer, réponses contrôlées,
                o	(-) demande de l’entrainement, des données, …,
                o	Paramétrage des réponses ;

            •	Génération de réponse :
                o	Réseau de neurones avec attention,
                o	Transformers,
                o	(+) plus robuste,
                o	(-) plus complexe à mettre en œuvre, réponses non contrôlées,
                o	Réponses non paramétrées ;

            •	Approche hybride :
                o   Cumul des trois approches pour cumuler certains avantages suivant les besoins ;
	\end{todolist}

    %%%%%--------------------------------------------------------------------
    %%%%% Section I.2:
    %%%%%--------------------------------------------------------------------
	
	\section{"\textit{Estabilishing a Niche}"}
	\todo[inline]{SECTION À RÉDIGER}

    \begin{todolist}

        \item Cadre industriel

            •	Algorithme fixe

            •	Données spécifiques

        \item GAP: Besoins de données

            •	Collecte de données spécifiqus au domaine traité :
                o   extraction de base de données (solution simple),
                o   collecte manuelle (organisation complexe, biais de collecte),
                o   scraping (pas toujours fiable) ;

            •	Nombreux biais :
                o   Biais,,
                o   Réglementation,
                o   Compétences (NOTRE COEUR DU SUJET),
                o   ...
    \end{todolist}

    %%%%%--------------------------------------------------------------------
    %%%%% Section I.3:
    %%%%%--------------------------------------------------------------------
	
	\section{"\textit{Occupying the Niche}"}
	\todo[inline]{SECTION À RÉDIGER}

    \begin{todolist}
        \item Etude de l'organisation d'une entreprise pour concevoir ses jeux de données
        \item Etude de l'état de l'art pour concevoir des jeux de données
        \item proposition/contribution : une méthode adaptée pour un cadre industriel
    \end{todolist}