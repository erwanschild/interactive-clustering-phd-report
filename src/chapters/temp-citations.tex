# Citation ergonomiques

	## Usability Assessment: How to Measure the Usability of Products, Services, and Systems
	@book{kortum2016usability,
		title={Usability Assessment: How to Measure the Usability of Products, Services, and Systems},
		author={Kortum, P.},
		isbn={9780945289494},
		lccn={2016050386},
		series={Users' Guides to Human Factors},
		url={https://books.google.fr/books?id=hp4nMQAACAAJ},
		year={2016},
		publisher={Human Factors and Ergonomics Society}
	}
	> Lise de LISE
		>> Définitions de Efficacité, Efficience, Charge mental, ...


	## Usability inspection methods > Heuristic evaluation
	@article{nielsen1994heuristic,
		title={Heuristic evaluation},
		author={Nielsen, Jakob},
		journal={Usability Inspection Mehods},
		year={1994},
		publisher={John Wiley\&Sons}
	}
	> Abstract : https://tecfa.unige.ch/tecfa/teaching/LMRI41/PrincipesNielsen.html


	## Success rate: the simplest usability metric	
	@article{nielsen2001success,
		title={Success rate: the simplest usability metric},
		author={Nielsen, Jakob},
		journal={Jakob Nielsen’s Alertbox},
		volume={18},
		pages={3--5},
		year={2001}
	}
	> Abstract : Le plus important, c'est d'arriver à faire la tâche demandée.


	## Usability engineering
	@book{nielsen1994usability,
		added-at = {2012-08-31T17:36:49.000+0200},
		address = {San Francisco, Calif.},
		author = {Nielsen, Jakob},
		biburl = {https://www.bibsonomy.org/bibtex/2d419237200065ce477ca64d62dd8bdc9/schmidt2},
		description = {Usability Engineering: Amazon.de: Jakob Nielsen: Englische Bücher},
		interhash = {c136d3ba53e7a92672e5236a289fa327},
		intrahash = {d419237200065ce477ca64d62dd8bdc9},
		isbn = {0125184069 9780125184069},
		keywords = {bib classic hci toread ui usability ux},
		publisher = {Morgan Kaufmann Publishers},
		refid = {41525304},
		timestamp = {2012-08-31T17:36:49.000+0200},
		title = {Usability engineering},
		url = {http://www.amazon.de/gp/product/0125184069/ref=oh_details_o02_s00_i00},
		year = 1994
	}
	> Learnability : Mesure de base, doit être pris en main facilement


	## Development of NASA-TLX (Task Load Index): Results of Empirical and Theoretical Research
	@incollection{HART1988139,
		title = {Development of NASA-TLX (Task Load Index): Results of Empirical and Theoretical Research},
		editor = {Peter A. Hancock and Najmedin Meshkati},
		series = {Advances in Psychology},
		publisher = {North-Holland},
		volume = {52},
		pages = {139-183},
		year = {1988},
		booktitle = {Human Mental Workload},
		issn = {0166-4115},
		doi = {https://doi.org/10.1016/S0166-4115(08)62386-9},
		url = {https://www.sciencedirect.com/science/article/pii/S0166411508623869},
		author = {Sandra G. Hart and Lowell E. Staveland},
		abstract = {The results of a multi-year research program to identify the factors associated with variations in subjective workload within and between different types of tasks are reviewed. Subjective evaluations of 10 workload-related factors were obtained from 16 different experiments. The experimental tasks included simple cognitive and manual control tasks, complex laboratory and supervisory control tasks, and aircraft simulation. Task-, behavior-, and subject-related correlates of subjective workload experiences varied as a function of difficulty manipulations within experiments, different sources of workload between experiments, and individual differences in workload definition. A multi-dimensional rating scale is proposed in which information about the magnitude and sources of six workload-related factors are combined to derive a sensitive and reliable estimate of workload.}
	}
	> Mental Workload : nombre de facteurs, difficulté de la tâche, nombre de tâche concurrentes, l'habitude de réaliser cette tâche, facteurs de l'environnement
	> Abstract:
		>> Mental - Combien d'activité mentale et perceptuelle était nécessaire ? La tâche était-elle facile ou exigeante, simple ou complexe, exigeante ou indulgente ?
		>> Physique - Combien d'activité physique était nécessaire ? La tâche était-elle facile ou exigeante ? Lent ou rapide ? Lâche ou épuisant ? Reposant ou laborieux ?
		>> Temporel - Quelle pression de temps avez-vous ressentie en raison de la vitesse ou du rythme auquel les tâches ou les éléments de tâche se sont produits ? Le rythme était-il lent et tranquille ou rapide et frénétique ?
		>> Frustration - Dans quelle mesure vous êtes-vous senti insécurisé, découragé, irrité, stressé et ennuyé par rapport à sûr, satisfait, satisfait, détendu et complaisant pendant la tâche ?
		>> Effort- À quel point avez-vous dû travailler (mentalement et physiquement) pour atteindre votre niveau de performance ?
		>> Performance- Dans quelle mesure pensez-vous avoir réussi à atteindre les objectifs de la tâche définie par l'expérimentateur (ou vous-même) ? Quel est votre niveau de satisfaction ?
	
	## Jones, G., M. Hocine, J. Salomon, W. Dab et L. Temime, 2015. Demographic and occupational predictors of stress and fatigue in french intensive-care registered nurses and nurses’ aides : A cross-sectional study. International journal of nursing studies, 52(1) :250–259. 23
	> La fatigue est décrite comme un épuisement et/ou un inconfort corporel associés à une activité prolongée (Jones et al. (2015)).

	## Elkosantini, S. et D. Gien, 2009. Integration of human behavioural aspects in a dynamic model for a manufacturing system. International Journal of Production Research, 47(10) :2601–2623. 23, 45, 70, 77, 88, 98
	> Dans le travail de Elkosantini et Gien (2009), les auteurs ont défini la fatigue comme un état de capacité réduite à travailler après une période d’activité

----------------------------------------------------------------------------
# Citation temps annotations
				
	## Active Learning with Real Annotation Costs
	@article{article,
		author = {Settles, Burr and Craven, Mark and Friedland, Lewis},
		year = {2008},
		month = {01},
		pages = {},
		title = {Active learning with real annotation costs}
	}
	> Note: Abstract : annotation time is not constant / ...
	> URL https://d1wqtxts1xzle7.cloudfront.net/57893949/settles.nips08-libre.pdf?1543610656=&response-content-disposition=inline%3B+filename%3DActive_Learning_with_Real_Annotation_Cos.pdf&Expires=1686043449&Signature=HoHdaYL3hbW9ToOvKL02b8LyWvZHPFlVZ46s6DYalzjnh7Q1BpFFcQVzCsNIbH3DBG3ub1YiXqWw22xjFL-YTh9S6uDrcU77R9Eb1vZZFCMd8zdwxkdr~0llTjZRpHZfrHSirY4xW9r9enoBBzUMkwZYRWu2IHRf0LvfOLc0oWF4bQXI7hWu8TlJg9zs-cxo49d8jGYUo2bZ6Tlkh7prPIkx8zSxg-~jXKXwPiD1MHQizYMjfK1D3JWOlZku0EJers1OVnseFV~7n-lXNQ9iXmzSfqa5e8UQHxrRpNSutR~aEAOU2BKd1J0jrIFMTYlw-USRXvoVqDIahnVhM6FK-w__&Key-Pair-Id=APKAJLOHF5GGSLRBV4ZA
	
	
	## PASCAL Recognizing Textual Entailment task (Dagan et al., 2006)
	@inproceedings{inproceedings,
		author = {Dagan, Ido and Glickman, Oren and Magnini, Bernardo},
		year = {2005},
		month = {01},
		pages = {177-190},
		title = {The PASCAL recognising textual entailment challenge},
		isbn = {978-3-540-33427-9},
		doi = {10.1007/11736790_9}
	}
	> Note: The RTE task is defined as recognizing, given two text fragments, whether the meaning of one text can be inferred (en- tailed) from the other
	
	
	## Contextual correlates of synonymy
	@article{10.1145/365628.365657,
		author = {Rubenstein, Herbert and Goodenough, John B.},
		title = {Contextual Correlates of Synonymy},
		year = {1965},
		issue_date = {Oct. 1965},
		publisher = {Association for Computing Machinery},
		address = {New York, NY, USA},
		volume = {8},
		number = {10},
		issn = {0001-0782},
		url = {https://doi.org/10.1145/365628.365657},
		doi = {10.1145/365628.365657},
		journal = {Commun. ACM},
		month = {oct},
		pages = {627–633},
		numpages = {7}
	}
	> URL : https://dl.acm.org/doi/10.1145/365628.365657
	> Note : Ordonner la similarité de paires de mots => score inter-annotateur très fort !
	
	
	## Contextual correlates of semantic similarity
	@article{doi:10.1080/01690969108406936,
		author = { George A.   Miller  and  Walter G.   Charles },
		title = {Contextual correlates of semantic similarity},
		journal = {Language and Cognitive Processes},
		volume = {6},
		number = {1},
		pages = {1-28},
		year  = {1991},
		publisher = {Routledge},
		doi = {10.1080/01690969108406936},
		URL = {https://doi.org/10.1080/01690969108406936},
		eprint = {https://doi.org/10.1080/01690969108406936}
	}
	> URL : https://doi.org/10.1080/01690969108406936
	> Note : Ordonner la similarité de paires de mots => score inter-annotateur très fort !
	
	## Fitts, P.M. (1964). Perceptual-motor skills learning. In: A.W. Melton (Ed.), Categories of human learning. New York: Academic Press, 243-285.
	## Anderson (1983, 1995)
	
	
	## 
		Purves, D., Cabeza, R., Huettel, S. A., LaBar, K. S., Platt, M. L., Woldorff, M. G., & Brannon, E. M. (2008). Cognitive neuroscience (Vol. 6, No. 4). Sunderland: Sinauer Associates, Inc. 
		
		P600 = 600ms après un stimulus pour une phrase
		+ 600ms pour avoir le traitement de la concordance
		=> ~ 4-5 sec si très automatique
		=> sous-entend qu'il y a un traitement cognitif supplémentaire que je ne peux pas qualifier (par exemple : création d'un modèle mentale ?)
		=> mais moins long que le reste