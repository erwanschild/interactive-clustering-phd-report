\chapter{Proposition d'un Clustering Interactif}
    \label{chapter:2_CLUSTERING_INTERACTIF}

    \minitoc

    Transition / Récap : On a vu que :
    
    •	Le travail est principalement manuel : comment peut-on l’assister ?
    
    •	La définition de la structure de classes est un mélange de connaissances métiers et de regroupement manuel sur la base de patterns linguistiques commun : faisons du clustering !
    
    •	Mais le clustering est souvent peu pertinent pour un usage métier : intégrons-y des interactions avec la machine !
    
    •	Les intercations non guidées sont fastidieuses : ajoutons de l’active learning !

    Nous proposons donc notre version d'intercative clustering.

    NB : la démonstration de cette proposition sera démontrée dans la partie 3.

    %%%%%--------------------------------------------------------------------
    %%%%% Section 2.1:
    %%%%%--------------------------------------------------------------------
    \section{Description théorique de la méthode}

        Nous proposons de combiner les techniques vues précédemment :

        •	Clustering sous contraintes
            o	Kmeans : Classique, Incontournable, Rapide, Efficace
            o	Hiérarchique : Lent mais facile à implémenter
            o	Spectral : Permet des topologies complexes
            o	DBScan : Classique, Incontournable, Rapide, Efficace, Peu d’hyperparamètres
            o	Affinity propagation : 
            o	Metric learning : Lent mais plus adapté au corpus
            o	…
    
        •	Echantillonnage de contraintes à annoter 
            o	Random ou Pseudo-random
            o	Farhtest : Scinder les gros clusters
            o	Closest : Redéfinir la position des frontières de clusters
            o	…
    
        •	Annotation de contraintes
            o	MUST-LINK / CANNOT-LINK / SKIP
            o	« Répondriez-vous de la même manière à ces deux demandes ? »

        •	Boucle itérative entre clustering, échantillonnage et annotation
            o	Améliorer le résultat précédent
            o	Autant de boucle que « nécessaire »
            o	Avoir le clustering le plus efficace pour avoir de bon résultats
            o	Avoir l’échantillonnage le plus efficace pour améliorer le plus efficacement
            o	Avoir une annotation sans ambiguïté pour ne pas biaiser la construction itérative

        •	Analyses
            o	Analyse de l’évolution de l’accord clustering->annotation
            o	Analyse des patterns linguistiques pertinents
            o	Analyse de la formation de clusters (taille, répartition, …)
        •	NB : Réutilisation de schéma etc
    
    %%%%%--------------------------------------------------------------------
    %%%%% Section 2.2:
    %%%%%--------------------------------------------------------------------
    \section{Espoirs de la méthode proposée (??ETAT DE L'ART??)}

        •	Moins de formations, d’ateliers, …

        •	Se concentrer sur son domaine de compétence (i.e. pas de datascience pour les experts métiers)

        •	Permettre de trouver la base d’apprentissage

        •	Méthode réaliste / pas trop coûteuse•	…
    
    %%%%%--------------------------------------------------------------------
    %%%%% Section 2.3:
    %%%%%--------------------------------------------------------------------
    \section{Description technique et implémentation}

        •	cognitivefactory.interactive-clustering : Gestion des données
    
        •	cognitivefactory.interactive-clustering : Gestion des contraintes + conflits

        •	cognitivefactory.interactive-clustering : Algorithmes de clustering

        •	cognitivefactory.interactive-clustering : Algorithmes de sampling

        •	cognitivefactory.interactive-clustering-gui : Interface d’annotation

        •	cognitivefactory.interactive-clustering-gui : Interface d’analyse

        •	NB : captures d’écrans pour donner un aperçu, puis redirection vers les annexes
    
    %%%%%--------------------------------------------------------------------
    %%%%% Section 2.1:
    %%%%%--------------------------------------------------------------------
    \section{Protocole d’utilisation : Mode d'emploi associé (??CONCLUSION??)}

        •	Collecte des données

        •	Itération de clustering > échantillonnage > annotation

        •	A chaque conflit : correction nécessaire

        •	A la fin d’un clustering : caractériser la pertinence métier avec FMC

        •	A chaque itération : voir l’évolution par rapport à la précédente

        NB : la démonstration de cette proposition de protocole sera démontrée dans la partie 3.