\section{Les nombreux défis de l'annotation}
\label{section:2.3-DEFIS-ANNOTATION}

	%%%
	%%% Introduction: annoncer la complexité due (1) aux données (2) à la tâche et (3) aux humains.
	%%%
	
	\todo[inline]{SECTION: À RÉDIGER: \\
		- Les données doivent être représentatives (Qualité, Biais, Equilibrage, ...) ; \\
		- Donc la tâche est complexe (...) ; \\
		- Donc les opérateurs régulent leur charge de travail () !
	}
	
	
	%%%
	%%% Subsection 2.3.1: Défis concernant le besoin de qualité des données.
	%%%
	\subsection{Défis concernant le problème de qualité des données}
	\label{section:2.3.1-DEFIS-ANNOTATION-ASPECT-DONNEES}
		\todo[inline]{SECTION: À RÉDIGER}
	
	
	%%%
	%%% Subsection 2.3.2: Défis concernant la complexité inhérente à la tâche d'annotation.
	%%%
	\subsection{Défis concernant la complexité inhérente à la tâche d'annotation}
	\label{section:2.3.2-DEFIS-ANNOTATION-ASPECT-COMPLEXITE}
		\todo[inline]{SECTION: À RÉDIGER}
	
	
	%%%
	%%% Subsection 2.3.3: Défis concernant les différences de comportements intra- et inter-annotateurs.
	%%%
	\subsection{Défis concernant les différences de comportements intra- et inter-annotateurs}
	\label{section:2.3.3-DEFIS-ANNOTATION-ASPECT-HUMAIN}
		\todo[inline]{SECTION: À RÉDIGER}
	
	
	%%%
	%%% Conclusion.
	%%%
	\begin{leftBarSummary}
		\begin{todolist}
			\item[\itemok] L'enjeu d'un projet d'annotation consiste à avoir des données de qualité qui soient représentatives du problème à traiter ;
			\item[\itemok] Or la tâche d'annotation et son exigence de qualité engendre de la complexité, et donc une charge de travail élevée ;
			\item[\itemok] Pour réguler cette charge de travail élevée, chaque opérateur va adapter sa tâche pour la rendre supportable, créant alors des différences de comportement.
			% [bayerl-paul:2011:what-determines-intercoder] : plus on est nombreux, moins on est d'accord
		\end{todolist}
	\end{leftBarSummary}