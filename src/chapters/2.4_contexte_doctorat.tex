\section{Contexte du doctorat: comment assister la conception d'une base d'apprentissage pour un agent conversationnel bancaire en français ?}
\label{section:2.4-CONTEXTE-DOCTORAT}


	% Introduction: Domaine d'application au NLP / chatbot.
	Durant ce doctorat, nous nous sommes intéressés à la conception d'assistants conversationnels (\textit{chatbot}) dans le domaine bancaires.
	Ces assistants sont notamment utiles pour accéder à des bases documentaires et pour entretenir la relation client à distance.
	\begin{leftBarInformation}
		L'\textsc{Annexe~\ref{annex:B-ANNEXE-CHATBOT}}
	\end{leftBarInformation}
	
	Dans ce domaine, le contrôle Toutefois, ces robots conversationnels 
	% besoin de
	
	\todo[inline]{SECTION: À CONFIRMER: \\
		- Chatbot: Task-oriented, entraîné avec intention et entitées et règles de dialogues \\
		- Données: pas de données linguistiques open source en français pour le domaine bancaire ==> besoin de créer le jeu de données \\
		- Modélisation: Phase longue en approche essai-erreur \\
		- Experts: Experts avec compétences métiers mais peu de connaissance en IA \\
		- Objectif du doctorat: trouver une alternative à ce processus manuel.
	}
	
	\subsection{Contexte industriel}
		Besoin : 
	
	\subsection{Idées}
		manuel = "trop long" et non-supervisé = "pas assez pertinant", donc intéressons nous au semi-supervisé.
		
		semi-supervisé = 


%	- On aide `ANNOTATE`, mais c'est `MODELIZE` qui est complexe
%		- **supervisé**: ça marche pas
%		- **non-supervisé**: (clustering) c'est pas terrible !
%		- **semi-supervisé**: c'est pertinent mais c'est long / beaucoup sur l'annotation
%	- **Regroupement automatique**:
%		> - [xu-tian:2015:comprehensive-survey-clustering] limite du clustering
%	- **Apprentissage actif**: utiliser un modèle pour sélectionner les données intéressante
%		> - [settles:2010:active-learning-literature] base active learning
%		> - [lampert-etal:2019:constrained-distance-based] A RELIRE !