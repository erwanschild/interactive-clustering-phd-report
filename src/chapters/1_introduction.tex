\chapter{Introduction}
\label{chapter:1-INTRODUCTION}
	\todo[inline]{CHAPITRE: TITRE À TROUVER: "\textit{Introduction}"}

	% RÉSUMÉ DES ÉPISODES PRÉCÉDENTS: 
	\todo[inline]{CHAPITRE: INTRODUCTION À RÉDIGER}

	% ANNONCE DU BUT DU CHAPITRE: 
	\todo[inline]{CHAPITRE: INTRODUCTION À RÉDIGER}

	% TABLE DES MATIÈRES DU CHAPITRE
    \minitoc

    %%%%%--------------------------------------------------------------------
    %%%%% Section 1.1:
    %%%%%--------------------------------------------------------------------
    \section{(\textit{Asset centrality})}
	\label{section:1.1-INTRODUCTION-ASSET-CENTRALITY}
		\todo[inline]{SECTION: TITRE À TROUVER: "\textit{Asset centrality}"}
		\todo[inline]{SECTION: À RÉDIGER: \\
			- Utilisation intensive de l'IA / des chatbots, et description des nombreuses opportunités liées ; \\
			- Mais aussi, mystification de l'IA : le grand publique ne se sait pas comment ça marche, comme ça s'entraîne, ont des craintes sur les capacités des modèles, ...
		}

    %%%%%--------------------------------------------------------------------
    %%%%% Section 1.2:
    %%%%%--------------------------------------------------------------------
    \section{(\textit{Establishing a Niche})}
	\label{section:1.2-INTRODUCTION-ESTABLISHING-A-NICHE}
		\todo[inline]{SECTION: TITRE À TROUVER: "\textit{Establishing a Niche}"}
		\todo[inline]{SECTION: À RÉDIGER: \\
			- Pour démystifier : \\
			- 1. Chat-oriented ou Task-oriented ; \\
			- 2. Conception par approche symbolique ou par approche générative ; \\
			- 3. Besoin d'annotation pour avoir des données de qualité.
		}

    %%%%%--------------------------------------------------------------------
    %%%%% Section 1.3:
    %%%%%--------------------------------------------------------------------
    \section{(\textit{Gap})}
	\label{section:1.3-INTRODUCTION-GAP}
		\todo[inline]{SECTION: TITRE À TROUVER: "\textit{Gap}"}
		\todo[inline]{SECTION: À RÉDIGER: \\
			- Peu de travaux sur la conception d'un jeu de données : en recherche les données sont publiques, en entreprises les données sont privées ; \\
			- Nombreuses pistes d'amélioration, mais peu sont exploitées en pratique ; \\
			- Défis d'organisation, de gestion de coûts, de complexité, de qualité, ... \\
			- Conception trop manuelle, experts pas à leur place, ...
		}

    %%%%%--------------------------------------------------------------------
    %%%%% Section 1.4:
    %%%%%--------------------------------------------------------------------
    \section{(\textit{Occupying the Niche})}
	\label{section:1.4-INTRODUCTION-OCCUPYING-THE-NICHE}
		\todo[inline]{SECTION: TITRE À TROUVER: "\textit{Occupying the Niche}"}
		\todo[inline]{SECTION: À RÉDIGER: \\
			- Besoin de recentrer l'activité des experts métiers ; \\
			- Besoin d'assister la conception d'un jeu de données ; \\
			- Nous proposons donc une méthode itérative et semi-supervisée.
		}
	
	%%%%%--------------------------------------------------------------------
    %%%%% TRANSITION:
    %%%%%--------------------------------------------------------------------
	\todo[inline]{TRANSITION: Annonce du plan ?}