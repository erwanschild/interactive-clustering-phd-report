\Annex{Jeux de données utilisés dans nos études}
\label{annex:A-ANNEXE-DATASET}

	% INTRODUCTION DE L'ANNEXE.
	Pour les différentes études réalisées au cours de ce doctorant (cf. \textsc{Chapitre~\ref{chapter:4-ETUDES}}), nous avons utilisé les deux jeux de données suivants.

	% TABLE DES MATIÈRES DE L'ANNEXE.
	\minitoc


	%%%%%--------------------------------------------------------------------
	%%%%% Annexe A.1: \texttt{Bank Cards}: Jeu d'entraînement en français d'assistants conversationnels traitant des demandes courantes sur les cartes bancaires
	%%%%%--------------------------------------------------------------------
	\section{\texttt{Bank Cards}: Jeu d'entraînement en français d'assistants conversationnels traitant des demandes courantes sur les cartes bancaires}
	\label{annex:A.1-DATASET-BANK-CARDS}
		
		
		% Description.
		\paragraph{Description :}
		Cet ensemble de données représente des exemples de demandes usuelles des clients concernant la gestion des cartes bancaires.
		Il peut être utilisé comme jeu d'entraînement pour un petit assistant conversationnel destiné à traiter ces demandes courantes.
		
		% Contenu.
		\paragraph{Contenu :}
		Les questions sont formulées en français.
		L'ensemble de données est divisé en $10$ intentions (classes) dont un aperçu est disponible dans la \textsc{Table~\ref{table:A.1-DATASET-BANK-CARDS}}.
		Ces intentions sont construites de telle manière que toutes les questions issues d'une même intention ont la même réponse ou action.
		La version \texttt{1.0.0} du jeu de données contient de $50$ questions par intention, soit un total de $500$ questions ;
		La version \texttt{2.0.0} du jeu de données contient de $100$ questions par intention, soit un total de $1~000$ questions.
		
		\begin{table}[!htb]
			\begin{center}
			\begin{scriptsize}
			\begin{tabular}{|c|c|c|c|}
			
				\hline
				% ENTETE DU TABLEAU
				\rowcolor{colorTableHeader!15}
				\textbf{Intention}
					& \textbf{Définition}
					& \textbf{Exemple}
					\tabularnewline
					\hline
				% alerte\_perte\_vol\_carte
				\multirow{2}{*}{\texttt{alerte\_perte\_vol\_carte}}
					& Affichage de la procédure de blocage
					& \textit{Comment signaler une perte}
					\tabularnewline
					& d'une carte perdue ou volée
					& \textit{de carte de paiement ?}
					\tabularnewline
					\hline
				% carte\_avalee
				\multirow{2}{*}{\texttt{carte\_avalee}}
					& Affichage de la procédure de
					& \textit{Comment récupérer}
					\tabularnewline
					& récupération d'une carte avalée
					& \textit{une carte avalée ?}
					\tabularnewline
					\hline
				% commande\_carte
				\multirow{2}{*}{\texttt{commande\_carte}}
					& Affichage des cartes disponibles,
					& \textit{Je souhaite changer}
					\tabularnewline
					& de la procédure de commande,
					& \textit{de carte bancaire.}
					\tabularnewline
					\hline
				% consultation\_solde
				\multirow{2}{*}{\texttt{consultation\_solde}}
					& Affichage d'une synthèse des
					& \textit{Où retrouver le solde}
					\tabularnewline
					& soldes bancaires du client.
					& \textit{ de mon compte ?}
					\tabularnewline
					\hline
				% couverture\_assurrance
				\multirow{2}{*}{\texttt{couverture\_assurrance}}
					& Affichage d'une synthèse des garanties
					& \textit{Que couvre ma carte bancaire}
					\tabularnewline
					& d'assurances de la carte bancaire du client
					& \textit{en cas d'hospitalisation ?}
					\tabularnewline
					\hline
				% deblocage\_carte
				\multirow{2}{*}{\texttt{deblocage\_carte}}
					& Affichage de gestion du statut
					& \textit{ma carte de paiement est}
					\tabularnewline
					& des cartes du client.
					& \textit{bloquée, que faire ?}
					\tabularnewline
					\hline
				% gestion\_carte\_virtuelle
				\multirow{2}{*}{\texttt{gestion\_carte\_virtuelle}}
					& Affichage de gestion des cartes
					& \textit{Comment faire pour créer une}
					\tabularnewline
					& virtuelles du client.
					& \textit{carte de paiements virtuelle ?}
					\tabularnewline
					\hline
				% gestion\_decouvert
				\multirow{2}{*}{\texttt{gestion\_decouvert}}
					& Affichage d'une synthèse des autorisations
					& \textit{Est-ce que j'ai un}
					\tabularnewline
					& de découverts de leur procédure de gestion
					& \textit{découvert autorisé ?}
					\tabularnewline
					\hline
				% gestion\_plafond
				\multirow{2}{*}{\texttt{gestion\_plafond}}
					& Affichage de gestion des plafonds
					& \textit{Le plafond de ma carte est}
					\tabularnewline
					& des cartes du client.
					& \textit{trop bas, que faire ?}
					\tabularnewline
					\hline
				% gestion\_sans\_contact
				\multirow{2}{*}{\texttt{gestion\_sans\_contact}}
					& Affichage de gestion des
					& \textit{Je veux désactiver le}
					\tabularnewline
					& fonctionnalités des cartes du client.
					& \textit{sans contact sur ma carte.}
					\tabularnewline
					\hline
			\end{tabular}
			\end{scriptsize}
			\end{center}
			\caption{
				Présentation du jeu de données \texttt{Bank Cards} avec quelques exemples.
				La version \texttt{2.0.0} contient $100$ questions par intention.
			}
			\label{table:A.1-DATASET-BANK-CARDS}
		\end{table}
		
		% Origine.
		\paragraph{Origine :}
		Le périmètre des intentions est inspiré d'un chatbot actuellement en production.
		Les données ont été sélectionnées aléatoirement et reformulées manuellement pour garantir la confidentialité des utilisateurs : aucune données personnelles ne subsistent dans ce jeu de données.
		Enfin, deux réviseurs extérieurs à l'équipe de recherche, ayant un profil d'analystes métiers du domaine bancaire, ont validé le périmètre et le contenu de ces intentions.
		
		% Disponibilité.
		\paragraph{Disponibilité :}
		Le jeu de données est archivé sur la plateforme \texttt{Zenodo} et est accessible ici: \cite{schild:2022:french-trainset-chatbots}


	%%%%%--------------------------------------------------------------------
	%%%%% Annexe A.2: \texttt{MLSUM} (The Multilingual Summarization Corpus): Échantillon de titre d'article de journaux en français associés à leur classification thématique
	%%%%%--------------------------------------------------------------------
	\section{\texttt{MLSUM} (The Multilingual Summarization Corpus): Échantillon de titre d'article de journaux en français associés à leur classification thématique}
	\label{annex:A.2-DATASET-MLSUM-SUBSET-SCHILD}
		
		
		% Description.
		\paragraph{Description :}
		Cet ensemble de données est constitué d'articles de journaux avec leur titre, leur résumé et leur classification thématique.
		Nous l'utilisons (1) pour estimer le temps nécessaire pour annoter la similarité des titres avec des contraintes (\texttt{MUST-LINK}, \texttt{CANNOT-LINK}) et (2) pour tester la méthodologie de \textit{clustering} interactif (annotation de contraintes et \textit{clustering} sous contraintes).
		
		% Contenu.
		\paragraph{Contenu :}
		Les titres de journaux sont formulés en français.
		L'ensemble de données est divisé en $14$ thèmes (classes) dont un aperçu est disponible dans la \textsc{Table~\ref{table:A.2-DATASET-MLSUM-SUBSET-SCHILD}}.
		La version \texttt{1.0.0 [subset: fr+train+filtered]} contient $744$ articles.
		
		\begin{table}[!htb]
			\begin{center}
			\begin{scriptsize}
			\begin{tabular}{|c|c|c|c|}
			
				\hline
				% ENTETE DU TABLEAU
				\rowcolor{colorTableHeader!15}
				\textbf{Thème}
					& \textbf{Définition}
					& \textbf{Exemple}
					& \textbf{Taille}
					\tabularnewline
					\hline
				
				% arts
				\multirow{2}{*}{\texttt{arts}}
					& Actualités artistiques (spectacles,
					& \textit{La rencontre de l'art et de la }
					& \multirow{2}{*}{$50$}
					\tabularnewline
					& oeuvres, événements, expositions)
					& \textit{gastronomie au château du Feÿ}
					&
					\tabularnewline
					\hline
				% disparitions
				\multirow{2}{*}{\texttt{disparitions}}
					& Actualités nécrologiques 
					& \textit{Le traducteur Jean-Pierre}
					& \multirow{2}{*}{$48$}
					\tabularnewline
					& (décès ou disparition)
					& \textit{Carasso est mort à 73 ans}
					&
					\tabularnewline
					\hline
				% ecologie
				\multirow{2}{*}{\texttt{ecologie}}
					& Actualités sur la pollution
					& \textit{Comment Lyon a banni les}
					& \multirow{2}{*}{$34$}
					\tabularnewline
					& et la transition écologique
					& \textit{pesticides de ses parcs et jardins}
					&
					\tabularnewline
					\hline
				% economie
				\multirow{2}{*}{\texttt{economie}}
					& Actualités économiques,
					& \textit{La guerre des prix s'intensifie}
					& \multirow{2}{*}{$41$}
					\tabularnewline
					& financières et boursières
					& \textit{sur le marché du mobile en Israël}
					&
					\tabularnewline
					\hline
				% education
				\multirow{2}{*}{\texttt{education}}
					& Actualités liées à l'éducation
					& \textit{Plainte de parents d'élève sur des notes}
					& \multirow{2}{*}{$62$}
					\tabularnewline
					& et à la filière enseignante
					& \textit{jugées trop basses au bac}
					&
					\tabularnewline
					\hline
				% emploi
				\multirow{2}{*}{\texttt{emploi}}
					& Actualités liées au marché du
					& \textit{Plus d'un tiers des CDI prennent}
					& \multirow{2}{*}{$54$}
					\tabularnewline
					& travail et aux actions syndicales
					& \textit{ fin avant la première année}
					&
					\tabularnewline
					\hline
				% immobilier
				\multirow{2}{*}{\texttt{immobilier}}
					& Actualités liées au marché de
					& \textit{Depuis la fin des années 2000, l'accession}
					& \multirow{2}{*}{$65$}
					\tabularnewline
					& l'immobilier et logements locatifs
					& \textit{à la propriété se complique en France}
					&
					\tabularnewline
					\hline
				% meteo
				\multirow{2}{*}{\texttt{meteo}}
					& Actualités météorologiques
					& \textit{L'Eure et l'est de la France}
					& \multirow{2}{*}{$35$}
					\tabularnewline
					& (bulletins, catastrophes, canicule)
					& \textit{balayés par les intempéries}
					&
					\tabularnewline
					\hline
				% musiques
				\multirow{2}{*}{\texttt{musiques}}
					& Actualités liées aux chanteurs,
					& \textit{Opéra : Elsa Dreisig,}
					& \multirow{2}{*}{$55$}
					\tabularnewline
					& concerts et sorties d'albums
					& \textit{une soprano à voix nue}
					&
					\tabularnewline
					\hline
				% police-justice
				\multirow{2}{*}{\texttt{police-justice}}
					&Actualités liées aux affaires
					& \textit{Bygmalion : Nicolas Sarkozy}
					& \multirow{2}{*}{$67$}
					\tabularnewline
					& policières et aux tribunaux
					& \textit{directement visé}
					&
					\tabularnewline
					\hline
				% politique
				\multirow{2}{*}{\texttt{politique}}
					& Actualités de la scène
					& \textit{Le Sénat donne son aval à la}
					& \multirow{2}{*}{$52$}
					\tabularnewline
					& politique et législative
					& \textit{prolongation de l'état d'urgence}
					&
					\tabularnewline
					\hline
				% sante
				\multirow{2}{*}{\texttt{sante}}
					& \multirow{2}{*}{Actualités sanitaires}
					& \textit{Chine : un nouveau}
					& \multirow{2}{*}{$70$}
					\tabularnewline
					&
					& \textit{cas de grippe aviaire H7N9}
					&
					\tabularnewline
					\hline
				% sciences
				\multirow{2}{*}{\texttt{sciences}}
					& Actualités scientifiques
					& \textit{L'ordinateur quantique}
					& \multirow{2}{*}{$47$}
					\tabularnewline
					& et vulgarisation
					& \textit{au banc d'essai}
					&
					\tabularnewline
					\hline
				% sport
				\multirow{2}{*}{\texttt{sport}}
					& \multirow{2}{*}{Actualités sportives}
					& \textit{F1 : Webber partira}
					& \multirow{2}{*}{$64$}
					\tabularnewline
					&
					& \textit{en tête à Monaco}
					&
					\tabularnewline
					\hline
			\end{tabular}
			\end{scriptsize}
			\end{center}
			\caption{
				Présentation du jeu de données échantillonné à partir de \texttt{MLSUM} avec quelques exemples.
			}
			\label{table:A.2-DATASET-MLSUM-SUBSET-SCHILD}
		\end{table}
		
		% Origine.
		\paragraph{Origine :}
		L'ensemble de données \texttt{MLSUM} a été proposé par \cite{scialom-etal:2020:mlsum-multilingual-summarization}.
		Notre ensemble de données en est un échantillon (\textit{sélectionner au hasard de $75$ articles dans les $14$ sujets les plus utilisés}) filtré (\textit{conserver les articles qui ont un sujet évident par rapport à leur titre, sans leur corps}).
		Deux relecteurs ont travaillé sur cette tâche afin de limiter la subjectivité du filtrage.
		L'échantillon final contient $744$ articles après relecture.
		
		% Disponibilité.
		\paragraph{Disponibilité :}
		Le jeu de donnée original est archivé sur \texttt{arXiv} et est accessible ici : \cite{scialom-etal:2020:mlsum-multilingual-summarization}.
		L'échantillon réalisé par nos soin est archivé sur la plateforme \texttt{Zenodo} et est accessible ici: \cite{schild-adler:2023:subset-mlsum-multilingual}.