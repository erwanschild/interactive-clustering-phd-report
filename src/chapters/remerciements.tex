\begin{ThesisAcknowledgments}

	% Introduction.
	J'ai pu travailler pendant près de quatre ans sur un sujet passionnant, et, grâce au soutien et à la contribution de nombreuses personnes, je peux désormais vous présenter le fruit de mes intenses réflexions.
	
	Par conséquent, je souhaite exprimer toute ma gratitude envers :
	\begin{todolist}
		\item[\itemok] Toi, lecteur, qui me fait l'immense honneur de lire ce manuscrit ! \faThumbsUp
		% PARTENAIRES.
		\item[\itemok] \textsc{Euro Information} (partenaire industriel), le \textsc{LORIA} (partenaire académique) et l'\textsc{ANRT} :
		pour avoir contribué au financement et à l’encadrement de cette thèse CIFRE, me permettant ainsi de travailler dans de bonnes conditions ;
		% ENCADRANTS.
		\item[\itemok] \textsc{Jean-Charles}, mon encadrant académique :
		pour tout ce que tu as pu m'apprendre au cours de ces quatre années, pour ta disponibilité, et pour la confiance que tu m'as accordée tout au long de ce doctorat ;
		\item[\itemok] \textsc{Florian}, mon encadrant industriel :
		pour avoir subi à ma place une bonne partie de la paperasse administrative (\faGrinTongueWink) et pour m'avoir fait l'honneur d'être le premier doctorant chez \textsc{Euro Information} ;
		% COLLEGUES
		\item[\itemok] \textsc{Gautier} :
		pour ton aide inestimable tout au long de mon doctorat, pour m'avoir aidé à structurer mon sujet, pour m'avoir appris comment rédiger un article scientifique, et pour tes remarques toujours pertinentes sur mon travail ;
		\item[\itemok] \textsc{Amélie} : pour toutes les discussions que nous avons eues, pour l’intérêt que tu as porté à mes travaux, et pour tes encouragements ;
		\item[\itemok] \textsc{Mathieu} et toute l'\textsc{équipe H330} : pour la bonne ambiance au bureau ainsi que le sérieux de nos réunions d'équipe (\faGrinTongueWink) ;
		% OUTILS
		\item[\itemok] Ma \textit{machine à café}, ma \textit{boîte de bretzels} et le \textit{moteur de recherche Google}, sans qui cette thèse aurait été beaucoup plus compliquée que prévu ! \faGrinBeamSweat
		% ANNOTATEURS
		\item[\itemok] \textsc{Marie}, \textsc{Jeremy} et \textsc{William} : pour m'avoir aidé dans la révision de mes jeux de données ;
		\item[\itemok] \textsc{Adrien}, \textsc{Amélie}, \textsc{Arthur}, \textsc{Baptiste}, \textsc{Bourhan}, \textsc{Iris}, \textsc{Julien}, \textsc{Lise}, \textsc{Marceau}, \textsc{Marie}, \textsc{Mathieu}, \textsc{Quentin}, \textsc{Robin} et \textsc{Thibaud} : pour avoir participé en tant que \st{cobayes} opérateurs lors de mes expériences d'annotation de contraintes ; 
		% ETUDIANTS
		\item[\itemok] \textsc{Clémentine}, \textsc{Thomas} et \textsc{Thomas}, puis \textsc{David}, \textsc{Esther} et \textsc{Marc} (les élèves de l'École d'Ingénieurs Télécom Physique Strasbourg) : pour avoir contribué aux développements logiciels de mon \texttt{Clustering Interactif} lors de vos projets étudiants ;
		% RELECTRICE
		\item[\itemok] \textsc{M'a su paire be relek triss} : sang qui m'ont manu skry ceux raie il ysible ! \footnote{
			Ma superbe relectrice, sans qui mon manuscrit serait illisible !
		} !
		% AMIS
		\item[\itemok] \textsc{Tous mes amis et ma famille} : vous qui m'avez \st{supporté} soutenu jusqu'au bout (\textit{Spéciale dédicace à ma très chère consoeur et à mes parents que j'aime tout beaucoup \faGrinWink}) ;
		\item[\itemok] et \textit{bien entendu}, \textsc{merci à ma merveilleuse épouse} : pour avoir été une oreille attentive, pour tes petites attentions, et pour ta capacité naturelle à me faire sourire au quotidien : merci d'exister ! \textcolor{colorDarkPastelRed}{\faHeart}
	\end{todolist}

\end{ThesisAcknowledgments}