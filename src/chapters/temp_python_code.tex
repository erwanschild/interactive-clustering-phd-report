\usepackage{listings}


	% from https://gist.github.com/a1ip/de2c0d4d3d6339cda809
	\usepackage{listings}
	\usepackage{color}

	\definecolor{mygreen}{rgb}{0,0.6,0}
	\definecolor{mygray}{rgb}{0.5,0.5,0.5}
	\definecolor{mymauve}{rgb}{0.58,0,0.82}

	\lstset{ %
	  backgroundcolor=\color{white},   % choose the background color
	  basicstyle=\footnotesize,        % size of fonts used for the code
	  breaklines=true,                 % automatic line breaking only at whitespace
	  captionpos=b,                    % sets the caption-position to bottom
	  commentstyle=\color{mygreen},    % comment style
	  escapeinside={\%*}{*)},          % if you want to add LaTeX within your code
	  keywordstyle=\color{blue},       % keyword style
	  stringstyle=\color{mymauve},     % string literal style
	}

\begin{document}

	\lstlistoflisting


	\section{test 1}
	% https://www.overleaf.com/learn/latex/Code_listing#Using_listings_to_highlight_code

		\begin{lstlisting}[language=Python, caption=Python example]
			import numpy as np
				
			def incmatrix(genl1,genl2):
				m = len(genl1)
				n = len(genl2)
				M = None #to become the incidence matrix
				VT = np.zeros((n*m,1), int)  #dummy variable
				
				#compute the bitwise xor matrix
				M1 = bitxormatrix(genl1)
				M2 = np.triu(bitxormatrix(genl2),1) 

				for i in range(m-1):
					for j in range(i+1, m):
						[r,c] = np.where(M2 == M1[i,j])
						for k in range(len(r)):
							VT[(i)*n + r[k]] = 1;
							VT[(i)*n + c[k]] = 1;
							VT[(j)*n + r[k]] = 1;
							VT[(j)*n + c[k]] = 1;
							
							if M is None:
								M = np.copy(VT)
							else:
								M = np.concatenate((M, VT), 1)
							
							VT = np.zeros((n*m,1), int)
				
				return M
		\end{lstlisting}

	\section{test 2}
	% from https://gist.github.com/a1ip/de2c0d4d3d6339cda809

		\begin{lstlisting}[language=python]
			>>> from numpy import *
			>>> from numpy.fft import *
			>>> signal = array([-2., 8., -6., 4., 1., 0., 3., 5.])
			>>> fourier = fft(signal)
			>>> N = len(signal)
			>>> timestep = 0.1 # if unit=day -> freq unit=cycles/day
			>>> freq = fftfreq(N, d=timestep) # freqs corresponding to 'fourier'
			>>> freq
			array([ 0. , 1.25, 2.5 , 3.75, -5. , -3.75, -2.5 , -1.25])
			>>> fftshift(freq) # freqs in ascending order
			array([-5. , -3.75, -2.5 , -1.25, 0. , 1.25, 2.5 , 3.75])
		\end{lstlisting}

\end{document}