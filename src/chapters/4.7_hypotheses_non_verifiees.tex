\section{Autres hypothèses non vérifiées}
\label{section:4.7-HYPOTHESES-NON-VERIFIEES}

	%%%
	%%% Introduction / Transition.
	%%%
	Lors des études précédentes, nous avons vérifié un certain nombre d'hypothèses et avons exploré plusieurs détails pratiques pour mettre en oeuvre une méthodologie d'annotation basée sur le \textit{clustering} interactif.
	Toutefois, certains points n'ont pas pu être étudiés en profondeurs lors de ce doctorat, par manque de temps ou de moyens.
	Nous exposons ici un ensemble de pistes intéressantes pouvant nourrir de futurs travaux afin d'améliorer la notre méthode.
	
	
	%%%
	%%% Subsection 4.7.1: Etude du nombre de clusters optimal.
	%%%
	\subsection{Etude du nombre de clusters optimal.}
	\label{section:4.7.1-HYPOTHESES-NON-VERIFIEES-NOMBRE-CLUSTERS}
	
		% Problème ouvert de la recherche: Estimer le nombre optimal de clusters.
		Un problème ouvert de la recherche lors de l'utilisation d'algorithmes de \textit{clustering} concerne le choix du nombre de \textit{clusters} à trouver.
		En effet, à part une connaissance à priori du nombre de thématiques présentes dans le jeu de données, il est difficile d'estimer le nombre optimal de \textit{clusters}, d'autant plus que celui-ci peut changer en fonction de la granularité de modélisation requise pour répondre au cas d'usage.
		
		% Pistes déjà explorées.
		Nous avons déjà exploré partiellement deux pistes :
		\begin{itemize}
			\item l'\textbf{exploration du graphe de contraintes} : en effet, il est possible d'estimer le nombre maximal de \textit{clusters} grâce aux composants connexes de contraintes \texttt{MUST-LINK}, et d'estimer le nombre minimal de \textit{clusters} grâce à la coloration du graphe de contraintes \textit{CANNOT-LINK} ;
			\item les \textbf{études de pertinence} avec l'analyse des patterns linguistiques et le résumé thématique des \textit{clusters} (cf. \textsc{Section~\ref{section:4.4-HYPOTHESE-PERTINENCE}}) : ces deux approches permettent de rapidement constater si les thématiques obtenues sont trop générales (\textit{i.e. s'il n'y a pas assez de clusters}) ou si elles semblent trop spécifiques (\textit{i.e. s'il y en a trop}).
		\end{itemize}
		
		% Piste potentielles à explorer.
		Toutefois, pour aller plus loin, deux pistes potentielles pourraient être explorées :
		\begin{itemize}
			\item l'exploration brute du nombre de \textit{clusters} par la \textbf{méthode du coude} : bien que ces approches sont plus coûteuses en temps de calcul, elles permettent d'estimer le nombre de \textit{clusters} pour lequel la stabilité du \textit{clustering} est la plus élevée ;
			\item l'utilisation d'algorithme n'ayant pas de nombre de clusters en paramètres comme des versions contraintes de \textbf{DBscan} (par exemple dans sa version \textit{C-DBScan}, \cite{ruiz-etal:2010:densitybased-semisupervised-clustering}) ou de la \textbf{propagation par affinité} (\cite{givoni-frey:2009:semisupervised-affinity-propagation}) : ces alternatives semblent prometteuses car elles retirent la complexité due à ce paramétrage abstrait.
		\end{itemize}
		
		\begin{leftBarInformation}
			L'étude de \texttt{C-DBScan} a été en partie réalisée dans le cadre d'un projet étudiant avec l'école d'ingénieur Télécom Physique Strasbourg.
			Les résultats montraient que le temps de calcul était similaire à celui du KMeans (dans sa version \texttt{COP}).
			La difficulté d'utilisation résidait plutôt sur la définition du rayon de voisinage \texttt{eps} à parcourir pour établir des liens entre données.
			Celui-ci peut être estimé en analysant la densité vectorielle du jeu de données.
			Le code informatique est disponible dans \cite{schild:2022:cognitivefactory-interactiveclustering} (\textit{\textit{Pull Request} en attente pour une version \texttt{0.6.0}}).
		\end{leftBarInformation}
	
	
	%%%
	%%% Subsection 4.7.2: Etude d'autres méthodes de vectorisation.
	%%%
	\subsection{Etude d'autres méthodes de vectorisation}
	\label{section:4.7.2-HYPOTHESES-NON-VERIFIEES-VECTORISATION}
	
		% Introduction.
		Au début de ce doctorat, nous avons conclu que les algorithmes de vectorisation n'avaient pas d'impact réel sur l'efficience de notre méthodologie d'annotation.
		Toutefois, les modèles de langues se sont largement développés, et il est fort probable que l'utilisation d'un \textbf{modèle pré-entraîné} permettent désormais d'avoir un gain de performance.
		
		% Piste potentielles à explorer.
		Nous pourrions par exemple tester les \textbf{architectures à base de \textit{Transformers}} (\cite{uszkoreit:2017:transformer-novel-neural}) comme \texttt{BERT} (\cite{devlin-etal:2019:bert-pretraining-deep}) et essayer différents modèles pré-entraînés sur des données françaises pour compléter nos études réalisées dans \cite{schild:2021:cognitivefactory-interactiveclusteringcomparativestudy}
	
	
	%%%
	%%% Subsection 4.7.3: Etude d'autres méthodes d'échantillonnage.
	%%%
	\subsection{Etude d'autres méthodes d'échantillonnage}
	\label{section:4.7.3-HYPOTHESES-NON-VERIFIEES-ECHANTILLONNAGE}
	
		% Introduction.
		Comme nous avons pu le voir dans \textsc{Section~\ref{section:4.6-HYPOTHESE-ROBUSTESSE}}, il peut-être intéressant d'introduire un mécanisme de création de redondance dans le graphe de contraintes annotées pour identifier les erreurs d'annotation.
		Un tel mécanisme n'a pas encore été implémenté mais pourrait facilement être intégré aux implémentations \texttt{Python} déjà existantes (\cite{schild:2022:cognitivefactory-interactiveclustering}).
		
		% Piste potentielles à explorer.
		Pour ce faire, le parcours de graphe et la création de cycle permettraient de vérifier la présence de conflits et ainsi de \textbf{provoquer des phases de revues de contraintes} si cela est nécessaire.
		Une telle page de revue pourrait aussi être complétée par des analyses complémentaires, comme l'estimation du taux de contraintes n'ayant pas de redondance et représentant ainsi des erreurs cachées potentielles.
	
	
	%%%
	%%% Subsection 4.7.4: Etude ergonomique de l'interface d'annotation.
	%%%
	\subsection{Etude ergonomique de l'interface d'annotation}
	\label{section:4.7.4-HYPOTHESES-NON-VERIFIEES-ERGONOMIQUE}
	
	% Introduction.
	L'application web développée au cours de ce doctorat (\cite{schild-etal:2022:cognitivefactory-interactiveclusteringgui}) permet d'essayer rapidement notre méthodologie d'annotation.
	Cependant, cette dernière n'a pu faire l'objet d'études poussées pour estimer la meilleure disposition des composants ou l'intérêt de certaines fonctionnalités d'annotation.
	
	% Piste potentielles à explorer.
	Parmi les pistes potentielles à explorer, nous avons évoqué la possibilité d'\textbf{annoter plusieurs contraintes} dans une même interface (\textit{par exemple : annoter visuellement un mini-graphes de $4$ données plutôt que d'annoter simplement un couple de données}) et le besoin de \textbf{réaliser des analyses rapides} sur les \textit{clusters} ou sur le graphe de contraintes (voir \textsc{Section~\ref{section:4.4-HYPOTHESE-PERTINENCE}} et \textsc{Section~\ref{section:4.5-HYPOTHESE-RENTABILITE}}).
	Toutes ces idées pourraient être l'objet d'études dédiées avec des groupes d'annotateurs différentes pour voir l'impact sur les performances et les biais de conception de modèles.