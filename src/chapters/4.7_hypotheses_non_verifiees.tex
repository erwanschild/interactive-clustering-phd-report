\section{Autres hypothèses non vérifiées}
\label{section:4.7-HYPOTHESES-NON-VERIFIEES}

	%%%
	%%% Introduction / Transition.
	%%%
	Lors des études précédentes, nous avons vérifié un certain nombre d'hypothèses et avons exploré plusieurs détails pratiques pour mettre en oeuvre une méthodologie d'annotation basée sur le \texttt{Clustering Interactif}.
	Toutefois, certains points n'ont pas pu être étudiés en profondeur lors de ce doctorat, par manque de temps ou de moyens.
	Nous exposons ici un ensemble de pistes intéressantes pouvant nourrir de futurs travaux afin d'améliorer notre méthode.
	
	
	%%%
	%%% Subsection 4.7.1: Étude du nombre de \textit{clusters} optimal.
	%%%
	\subsection{Étude du nombre de \textit{clusters} optimal}
	\label{section:4.7.1-HYPOTHESES-NON-VERIFIEES-NOMBRE-CLUSTERS}
	
		% Problème ouvert de la recherche: Estimer le nombre optimal de \textit{clusters}.
		Un problème ouvert de la recherche lors de l'utilisation d'algorithmes de \textit{clustering} concerne le choix du nombre de \textit{clusters} à trouver.
		En effet, à part une connaissance \textit{a priori} du nombre de thématiques présentes dans le jeu de données, il est difficile d'estimer le nombre optimal de \textit{clusters}, d'autant plus que celui-ci peut changer en fonction de la granularité de modélisation requise pour répondre au cas d'usage.
		
		% Pistes déjà explorées.
		Nous avons déjà exploré partiellement deux pistes :
		\begin{itemize}
			\item l'\textbf{exploration du graphe de contraintes} : en effet, il est possible d'estimer le nombre maximal de \textit{clusters} grâce aux composants connexes de contraintes \texttt{MUST-LINK}, et d'estimer le nombre minimal de \textit{clusters} grâce à la coloration du graphe de contraintes \texttt{CANNOT-LINK} ;
			\item les \textbf{études de pertinence} avec l'analyse des patterns linguistiques et le résumé thématique des \textit{clusters} (cf. \textsc{Section~\ref{section:4.4-HYPOTHESE-PERTINENCE}}) : ces deux approches permettent de rapidement évaluer si les thématiques obtenues sont trop générales (\textit{i.e. s'il n'y a pas assez de clusters}) ou si elles semblent trop spécifiques (\textit{i.e. s'il y en a trop}).
		\end{itemize}
		
		% Piste potentielles à explorer.
		Toutefois, pour aller plus loin, deux pistes potentielles pourraient être explorées :
		\begin{itemize}
			\item l'exploration brute du nombre de \textit{clusters} par la \textbf{méthode du coude} : bien que ces approches sont plus coûteuses en temps de calcul, elles permettent d'estimer le nombre de \textit{clusters} pour lequel la stabilité du \textit{clustering} est la plus élevée ;
			\item l'utilisation d'algorithmes n'ayant pas de nombre de \textit{clusters} en paramètres, comme des versions contraintes de \texttt{DBScan} (par exemple dans sa version \texttt{C-DBScan}, \cite{ruiz-etal:2010:densitybased-semisupervised-clustering}) ou de la \textbf{propagation par affinité} (\cite{givoni-frey:2009:semisupervised-affinity-propagation}) : ces alternatives semblent prometteuses car elles retirent la complexité due à ce paramétrage abstrait.
		\end{itemize}
		
		\setcounter{localCounterOfFootnoteValue}{\value{footnote}}
		\begin{leftBarInformation}
			L'étude de \texttt{C-DBScan} a été en partie réalisée dans le cadre d'un projet étudiants avec l'École d'Ingénieurs Télécom Physique Strasbourg (au cours de l'année 2022).
			Les résultats montraient que le temps de calcul était similaire à celui du \texttt{KMeans} (dans sa version \texttt{COP}).
			La difficulté d'utilisation résidait plutôt dans la définition du rayon de voisinage \texttt{eps} à parcourir pour établir des liens entre données.
			Celui-ci peut être estimé en analysant la densité vectorielle du jeu de données.
			Le code informatique est disponible dans \cite{schild:2022:cognitivefactory-interactiveclustering} \footnotemark.
		\end{leftBarInformation}
		% Rattraper les footnote.
			\stepcounter{localCounterOfFootnoteValue}
			\footnotetext[\value{localCounterOfFootnoteValue}]{
				Implémentation de \texttt{C-DBScan} : \textit{Pull Request} en attente pour une version \texttt{0.6.0} après ajout de documentation et de tests unitaires.
			}
	
	
	%%%
	%%% Subsection 4.7.2: Étude d'autres méthodes de vectorisation.
	%%%
	\subsection{Étude d'autres méthodes de vectorisation}
	\label{section:4.7.2-HYPOTHESES-NON-VERIFIEES-VECTORISATION}
	
		% Introduction.
		Au début de ce doctorat, nous avons conclu que les algorithmes de vectorisation n'avaient pas d'impact réel sur l'efficience de notre méthodologie d'annotation.
		Toutefois, les modèles de langues se sont largement développés, et il est fort probable que l'utilisation d'un \textbf{modèle pré-entraîné} permette désormais d'avoir un gain de performance.
		
		% Piste potentielles à explorer.
		Nous pourrions par exemple tester les \textbf{architectures à base de \textit{Transformers}} (\cite{uszkoreit:2017:transformer-novel-neural}) comme \texttt{BERT} (\cite{devlin-etal:2019:bert-pretraining-deep}) et essayer différents modèles pré-entraînés sur des données françaises pour compléter nos études réalisées dans \cite{schild:2021:cognitivefactory-interactiveclusteringcomparativestudy}
	
	
	%%%
	%%% Subsection 4.7.3: Étude d'autres méthodes d'échantillonnage.
	%%%
	\subsection{Étude d'autres méthodes d'échantillonnage}
	\label{section:4.7.3-HYPOTHESES-NON-VERIFIEES-ECHANTILLONNAGE}
	
		% Introduction.
		Comme nous avons pu le voir dans \textsc{Section~\ref{section:4.6-HYPOTHESE-ROBUSTESSE}}, il peut-être intéressant d'introduire un mécanisme de création de redondance dans le graphe de contraintes annotées pour identifier les erreurs d'annotation.
		Un tel mécanisme n'a pas encore été implémenté mais pourrait facilement être intégré aux implémentations \texttt{Python} déjà existantes (\cite{schild:2022:cognitivefactory-interactiveclustering}).
		
		% Piste potentielles à explorer.
		Pour ce faire, le parcours de graphe et la création de cycle permettraient de vérifier la présence de conflits et ainsi de \textbf{provoquer des phases de revues de contraintes} si cela est nécessaire.
		Une telle page de revue pourrait aussi contenir des analyses complémentaires, comme l'estimation du taux de contraintes n'ayant pas de redondance et représentant ainsi des erreurs cachées potentielles.
	
	
	%%%
	%%% Subsection 4.7.4: Étude de techniques de transfert d'apprentissage.
	%%%
	\subsection{Étude de techniques de transfert d'apprentissage}
	\label{section:4.7.4-HYPOTHESES-NON-VERIFIEES-TRANSFERT-APPRENTISSAGE}
	
		% Introduction.
		Dans la \textsc{Section~\ref{section:2.3-DEFIS-ANNOTATION}}, nous avions déjà évoqué le fait que la modélisation d'un phénomène peut être assisté par des techniques telles que la pré-annotation (\cite{dandapat-etal:2009:complex-linguistic-annotation}) ou le transfert d'apprentissage (\cite{zhuang-etal:2021:comprehensive-survey-transfer}).
		Nous pourrions nous inspirer davantage de ces approches pour démarrer plus efficacement les premières itérations d'un \texttt{Clustering Interactif}.
	
		% Piste potentielles à explorer.
		Voici quelques idées inspirées de ces méthodes :
		\begin{itemize}
			\item \textbf{pré-annoter} certaines contraintes simples à l'aide de règles (\textit{basées par exemple sur la présence de mots de vocabulaire en commun}) ou grâce à l'utilisation d'un modèle déjà disponible ; 
			\item \textbf{introduire des données synthétiques ou empruntées} à d'autres bases d'apprentissage pour initialiser le \textit{clustering}, et permettre ainsi ajouter d'emblée des connaissances générales dans la modélisation.
		\end{itemize}
	
	
	%%%
	%%% Subsection 4.7.5: Étude ergonomique de l'interface d'annotation.
	%%%
	\subsection{Étude ergonomique de l'interface d'annotation}
	\label{section:4.7.5-HYPOTHESES-NON-VERIFIEES-ERGONOMIQUE}
	
		% Introduction.
		L'application web développée au cours de ce doctorat (\cite{schild-etal:2022:cognitivefactory-interactiveclusteringgui}) permet d'essayer rapidement notre méthodologie d'annotation.
		Cependant, cette dernière n'a pu faire l'objet d'études poussées pour estimer la meilleure disposition des composants ou l'intérêt de certaines fonctionnalités d'annotation.
		
		% Piste potentielles à explorer.
		Parmi les pistes potentielles à explorer, nous avons évoqué la possibilité d'\textbf{annoter plusieurs contraintes} dans une même interface (\textit{par exemple : annoter visuellement un mini-graphe de $4$ données plutôt que d'annoter simplement une paire de données}) et le besoin de \textbf{réaliser des analyses rapides} sur les \textit{clusters} ou sur le graphe de contraintes (voir \textsc{Section~\ref{section:4.4-HYPOTHESE-PERTINENCE}} et \textsc{Section~\ref{section:4.5-HYPOTHESE-RENTABILITE}}).
		Pour aller plus loin, \cite{bae-etal:2021:interactive-clustering-comprehensive} proposent d'autres listes d'interactions qui sont possibles d'avoir avec un algorithme de \textit{clustering}, notamment sur la manipulation de son résultat (\textit{fusion, suppression, verrouillage, ...}) et de ses hyperparamètres (\textit{nombre de \textit{clusters}, adaptation du vocabulaire autorisé, ...})
		
		% A essayer !
		Toutes ces idées pourraient être l'objet de développements et d'études dédiées avec des groupes d'annotateurs différents pour voir l'impact sur les performances et les biais de conception de modèles.