\section{Intuitions à l'origine d'un \textit{clustering} interactif}
\label{section:3.1-INTUITIONS-ORIGINES}


	
	% Avantages/Limites: xu-tian:2015:comprehensive-survey-clustering
	% Limite clustering: agarwal-etal:2011:issues-challenges-tools (hyperparamètres) ; steinbach-etal:2004:challenges-clustering-high (high data)
	% Plusieurs types d'interactions possibles : bae-etal:2021:interactive-clustering-comprehensive


	%%%
	%%% Subsection 3.1.1. Utiliser une approche non-supervisées pour créer une modélisation des données.
	%%%
	\subsection{Utiliser une approche non-supervisées pour créer une modélisation des données}
	\label{section:3.1.1-INTUITIONS-ORIGINES-NON-SUPERVISEES}
	
		\todo[inline]{A REDIGER}
	
	%%%
	%%% Subsection 3.1.2. Corriger l'approche non-supervisée à l'aide d'une annotation de contraintes.
	%%%
	\subsection{Corriger l'approche non-supervisée à l'aide d'une annotation de contraintes}
	\label{section:3.1.2-INTUITIONS-ORIGINES-SEMI-SUPERVISEES}
	
		\todo[inline]{A REDIGER}
	
	%%%
	%%% Subsection 3.1.3. Tirer parti des avantages de l'apprentissage actif pour optimiser les interactions Homme/machine.
	%%%
	\subsection{Tirer parti des avantages de l'apprentissage actif pour optimiser les interactions Homme/machine}
	\label{section:3.1.3-INTUITIONS-ORIGINES-APPRENTISSAGE-ACTIF}
	
		\todo[inline]{A REDIGER}


	\todo[inline]{ANCIENNE PARTIE (début)}
	% 1. L'annotation de contraintes est plus intuitif pour un expert métier.
	La pierre angulaire de notre méthode repose sur le fait qu'il est difficile pour un expert métier de classer une question suivant une modélisation abstraite prédéfinie :
	cela l'éloigne de ses compétences métiers initiales, nécessite en contre-partie de nombreuses formations, et introduit de nombreuses erreurs d'annotations.
	De fait, il semble plus adéquat de demander à l'expert métier de discriminer deux questions sur la base de leurs réponses :
	une telle approche demande a priori une charge de travail plus faible et semble plus intuitive car elle se rapproche des compétences réelles de l'annotateur (\textguillemets{\textit{est-ce les deux données traite du même cas d'usage ?}}).
	Ainsi, nous basons notre méthode sur l'annotation de contraintes sur les données.
	
	% 2. Un expert métier seul ne peut trouver une modélisation adéquate, il faut se reposer sur l'interaction homme-machine.
	Toutefois, l'annotation de contraintes semble elle aussi fastidieuse.
	En effet, pour faire émerger une base d'apprentissage, il faut annoter un grand nombre de contraintes et être attentifs aux éventuelles incohérences pour ne pas introduire de contraintes contradictoires.
	Pour assister l'expert dans cette tâche, nous avons donc décidé de l'intégrer dans une stratégie d'apprentissage actif en essayant de tirer parti des interactions possibles avec la machine.
	Ce choix est motivé entre autres par l'intuition qu'il est possible de coopérer avec la machine pour obtenir plus efficacement un résultat pertinent.

	% TR:
	C'est sur la combinaison de ces deux éléments que repose notre méthode d'annotation pour concevoir le jeu d'entraînement de notre assistant conversationnel.
	\todo[inline]{ANCIENNE PARTIE (fin)}
	
	%%%
	%%% Conclusion.
	%%%
	\begin{leftBarSummary}
		... :
		\begin{todolist}
			% Pression sur la qualité.
			\item[\itemok] ...
			\item[\itemok] ...
			\item[\itemok] ...
		\end{todolist}
	\end{leftBarSummary}