\chapter{Introduction}
\label{chapter:1-INTRODUCTION}


	%%%%%--------------------------------------------------------------------
	%%%%% ASSET CENTRALITY: Idéalisation de l'\texttt{IA} aux yeux du grand public.
	%%%%%--------------------------------------------------------------------
	\section*{Idéalisation de l'\texttt{IA} aux yeux du grand public}
	\addcontentsline{toc}{section}{
		\protect\numberline{}
		Idéalisation de l'\texttt{IA} aux yeux du grand public
	}
		
		%%% Démocratisation de l'utilisation d'IA.
		L'Intelligence Artificielle (\texttt{IA}) a connu une démocratisation massive ces dernières années.
		Elle est considérée comme une révolution majeure de notre société, à tel point qu'il devient presque impossible de s'en passer.
		\begin{itemize}
			\item Vous avez besoin de trouver votre chemin ? utilisez votre \texttt{GPS}.
			\item Vous avez un problème avec une commande ou besoin d'un service après-vente ? un bot informatique est disponible jour et nuit pour traiter votre demande.
			\item Vous ne savez plus quelle série regarder ? \texttt{Netflix} peut faire des suggestions personnalisées.
			\item Vous avez les mains pleines de farine et vous voulez lancer un minuteur ou écouter de la musique ? Demandez-le \texttt{OK Google} ou \texttt{Alexa}.
			\item Il vous manque une belle image pour votre présentation ? \texttt{DALL-E} peut la générer.
			\item Vous devez rédiger une dissertation en histoire-géographie ? \texttt{ChatGPT} s'en occupera.
			\item La classe \LaTeX{} proposée par votre école doctorale ne compile pas ? \texttt{ChatGPT} peut aussi identifier l'erreur et même la corriger...
		\end{itemize}
		
		%%% Mais des mythes se créent.
		Les modèles d'\texttt{IA} s'immiscent ainsi dans la plupart des activités de notre quotidien.
		Cependant, cette omniprésence est aussi source de confusion et d'incompréhension pour le grand public.
		En effet :
		\begin{itemize}
			% Craintes.
			\item L'\texttt{IA} peut être perçue comme une menace, soit parce qu'elle vole des emplois, soit parce qu'elle risquerait de devenir incontrôlable.
			Ces craintes sont notamment véhiculées par la culture populaire, à l'image de \texttt{Terminator} ou d'\texttt{Ultron} qui se sont retournés contre leur créateur.
			% Attentes trop hautes.
			\item Les attentes des utilisateurs sont parfois trop élevées par rapport aux capacités réelles du modèle.
			Il en résulte alors un sentiment de frustration, en particulier lorsque l'utilisateur exprime un besoin urgent, mais que le modèle se contente de répondre qu'il n'a pas compris la question et que vous devriez reformuler votre demande.
			% Confiance aveugle.
			\item La crédit accordé aux modèles d'\texttt{IA} est parfois excessif, au point que l'esprit critique des utilisateurs s'efface petit à petit.
			Cette confiance aveugle s'accentue avec les capacités spectaculaires des derniers modèles génératifs, pour lesquels il devient difficile d'identifier une fausse information tant leurs résultats sont crédibles (\textit{par exemple, \texttt{ChatGPT} peut vous mentir avec conviction en inventant certains détails dans ses réponses}). 
		\end{itemize}
		
		%%% Idée reçu : pas complexe à faire.
		L'idée reçue selon laquelle la conception d'un modèle d'\texttt{IA} est simple et peu onéreuse explique en partie ces confusions.
		Encore une fois, la culture populaire véhicule l'image d'une conception accessible à tous et à moindres coûts.
		En reprenant l'exemple d'\textit{Ultron}, il suffit au \texttt{Dr. Hank Pym} de \textguillemets{calquer ses schémas mentaux} pour créer le robot, comme si cela était un acte banal ; plus récemment, dans la série \texttt{Black Mirror} (S2 Ep1), le personnage de \textit{Martha} se procure un avatar de son conjoint décédé sans trop de difficultés en communiquant simplement les messages et les photos de ce dernier, mais aucune mention n'est faite sur le processus de conception, excepté qu'il est \textguillemets{expérimental}.
		Toutes ces illusions masquent à quel point la création d'un modèle d'\textit{IA} peut être longue est pénible.
		
		
	%%%%%--------------------------------------------------------------------
	%%%%% NICHE: Désillusion quant à la simplicité de conception d'un modèle d'\texttt{IA}.
	%%%%%--------------------------------------------------------------------
	\section*{Désillusion quant à la simplicité de conception d'un modèle d'\texttt{IA}}
	\addcontentsline{toc}{section}{
		\protect\numberline{}
		Désillusion quant à la simplicité de conception d'un modèle d'\texttt{IA}
	}
		
		%%% Introduction.
		En pratique, nous nous rendons vite compte qu'il faut traiter un certain nombre de défis pour mettre un modèle d'\texttt{IA} en production, et que plusieurs de ces défis sont encore aujourd'hui des problèmes ouverts du monde de la recherche.
		
		%%% Deep Learning.
		Tout d'abord, nous pourrions être tenté d'utiliser les dernières avancées publiés dans la littérature : nous pensons donc nécessairement aux modèles d'apprentissage profond (\textit{Deep Learning}), réputés pour leurs performances impressionnantes (\texttt{ChatGPT}, \texttt{LLaMA2}, \texttt{BLOOM}, ...).
		% Besoin de données
		De tels modèles tirent parti des immenses quantités de données mises à leur disposition : sans compter les difficultés liées à la collecte et aux droits d'utilisation de ces données, nous sommes rapidement confronté aux critères de qualité et aux biais de représentation.
		Un travail titanesque est alors nécessaire pour assainir, corriger, rééquilibrer puis documenter les jeux de données nécessaires à l'entraînement, représentant un coût non négligeable dans le processus de conception.
		
		% Besoin d'infrastructure technique.
		De plus, l'entraînement de ces modèles requiert une infrastructure technique colossale dont seules les grandes entreprises sont capables de se doter.
		Enfin, leur utilisation dans un environnement de production est pénalisée par des temps d'exécution lents ainsi que des erreurs de comportement non contrôlable et peu prévisibles.
		% Plutôt du Machine Learning.
		Par conséquent, les entreprises se tournent plus facilement vers des modèles d'\texttt{IA} plus traditionnels (\textit{Machine Learning}), mais moins onéreuses et plus facilement contrôlables.
		
		%%% Machine Learning.
		En tête de liste, nous retrouvons fréquemment les apprentissages supervisés.
		Ces techniques utilisent une modélisation abstraite du problème qui sera reproduite par un modèle sur la base des exemples de sa base d'apprentissage.
		% Exemple.
		Avec des telles approches, il est possible à un radar automatique d'identifier la marque d'une voiture flashée sur la base d'exemples de photos de voitures de sa base d'apprentissage ; en outre, il est aussi possible d'entraîner un modèles d'extraction pour lire automatiquement le numéro de la plaque d'immatriculation et ainsi savoir qui verbaliser en cas d'infraction.
		
		% Besoin d'une modélisation.
		L'inconvénient de cette méthode réside néanmoins dans le besoin de créer une modélisation abstraite du problème dans le but de le reproduire.
		À part dans de rares occasions où cette modélisation est triviale (\textit{dans notre exemple : les différentes marques de voitures existantes}), il est nécessaire de consulter des experts du domaine pour comprendre comment ces derniers manipulent les données au quotidien.
		% Exemple chatbot.
		Par exemple, pour traiter des commandes vocales ou des demandes clients avec un assistant conversationnel (\textit{chatbot}), il est possible de modéliser le texte en intention de dialogue, chaque intention conceptualisant l'action ou la demande à traiter (\textit{des requêtes comme \textguillemets{\textit{joue moi du jazz s'il te plaît !}} ou \textguillemets{\textit{peux-tu lancer une playlist de Noël sur l'enceinte du salon !}} peut être alors être modélisées par l'intention \texttt{jouer\_musique}}).
		Cependant, cette modélisation est généralement laborieuse à cause de la complexité intrinsèque des phénomènes à reproduire (\textit{dans notre exemple : la difficulté d'interprétation du langage naturel}), ce qui entretient le caractère subjectif de la tâche (\textit{avec les mots \textguillemets{\textit{peux-tu lancer ...}}, est-ce que l'assistant doit effectuer une action, ou doit-il exprimer s'il en est capable ?}).
		
		
	%%%%%--------------------------------------------------------------------
	%%%%% GAP: Complexité d'annotation et manque de valorisation de l'expertise métier.
	%%%%%--------------------------------------------------------------------
	\section*{Complexité d'annotation et manque de valorisation de l'expertise métier}
	\addcontentsline{toc}{section}{
		\protect\numberline{}
		Complexité d'annotation et manque de valorisation de l'expertise métier
	}
		
		\todo[inline]{A REDIGER}
		
		%- Peu de travaux sur la conception d'un jeu de données : en recherche les données sont publiques, en entreprises les données sont privées ;
		%- Nombreuses pistes d'amélioration, mais peu sont exploitées en pratique ; 
		%- Défis d'organisation, de gestion de coûts, de complexité, de qualité, ... 
		%- Conception trop manuelle, experts pas à leur place, ...
		
		
		
	%%%%%--------------------------------------------------------------------
	%%%%% OCCUPYING THE NICHE: Recherche d'une méthode alternative pour modéliser le texte en intentions
	%%%%%--------------------------------------------------------------------
	\section*{Recherche d'une méthode alternative pour modéliser le texte en intentions}
	\addcontentsline{toc}{section}{
		\protect\numberline{}
		Recherche d'une méthode alternative pour modéliser le texte en intentions
	}
		
		\todo[inline]{A REDIGER}
		
			%- Besoin de recentrer l'activité des experts métiers ;
			%- Besoin d'assister la conception d'un jeu de données ;
			%- Nous proposons donc une méthode itérative et semi-supervisée.
		
	%%%%%--------------------------------------------------------------------
	%%%%% PLAN: Annonce du plan de ce manuscrit.
	%%%%%--------------------------------------------------------------------
	\section*{Annonce du plan de ce manuscrit}
	\addcontentsline{toc}{section}{
		\protect\numberline{}
		Annonce du plan de ce manuscrit
	}
		
		% Introduction.
		Afin de traiter la problématique que nous venons d'exposer, nous organisons la discussion de ce manuscrit de la manière suivante :
		
		% Plan.
		\begin{itemize}
			% Chapitre 2: Revue de littérature.
			\item Au cours du \textsc{Chapitre~\ref{chapter:2-REVUE-DE-LITTERATURE}}, nous présentons en détails la tâche d'annotation, son organisation traditionnelle ainsi que les nombreux défis qu'elle comporte.
			Pour mieux illustrer nos propos, nous utilisons des exemples inspirés de l'univers de la bande dessinée.
			% Section 2.4: Contexte du doctorat.
			\item Nous complétons la revue de littérature en expliquant le contexte de ce doctorat en \textsc{Section~\ref{section:2.4-CONTEXTE-DOCTORAT}} : ce complément nous permet de mettre en évidence la difficulté d'intervention des experts métiers dans un projet traditionnel d'annotation.
			% Chapitre 3 : Présentation de la méthode.
			\item Le \textsc{Chapitre~\ref{chapter:3-CLUSTERING-INTERACTIF}} est dédié à la présentation de notre méthodologie d'annotation alternative basée sur un \texttt{Clustering Interactif}.
			La description de l'implémentation technique est consultable dans l'\textsc{Annexe~\ref{annex:C-ANNEXE-IMPLEMENTATIONS}}.
			% Chapitre 4 : Etude de la méthode
			\item Dans le \textsc{Chapitre~\ref{chapter:4-ETUDES}}, nous décrivons les six hypothèses que nous voulions vérifier sur notre méthodologie d'annotation : efficacité, efficience, coûts, pertinence, rentabilité et robustesse.
			\item Le \textsc{Chapitre~\ref{chapter:5-GUIDE}} fait le point sur l'ensemble des discussions et découvertes contenues des précédents chapitres, et comporte différents avis et conseils pratiques.
			Le chapitre entier est prévu pour être un guide d'utilisation synthétique de notre méthodologie d'annotation.
		\end{itemize}
		
		% Conclusion.
		Le \textsc{Chapitre~\ref{chapter:6-CONCLUSION}} dresse la conclusion et clôt la discussion en abordant des thématiques et perspectives plus générales.
