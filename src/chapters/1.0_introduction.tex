\chapter{Introduction}
\label{chapter:1-INTRODUCTION}


	%%%%%--------------------------------------------------------------------
	%%%%% ASSET CENTRALITY: Idéalisation de l'\texttt{IA} aux yeux du grand public.
	%%%%%--------------------------------------------------------------------
	\section*{Idéalisation de l'\texttt{IA} aux yeux du grand public}
	\addcontentsline{toc}{section}{
		\protect\numberline{}
		Idéalisation de l'\texttt{IA} aux yeux du grand public
	}
		
		%%% Démocratisation de l'utilisation d'IA.
		L'Intelligence Artificielle (\texttt{IA}) a connu une démocratisation massive ces dernières années.
		Elle est considérée comme une révolution majeure de notre société, à tel point qu'il devient presque impossible de s'en passer :
		\begin{itemize}
			\item Vous avez besoin de trouver votre chemin ? utilisez votre \texttt{GPS}.
			\item Vous avez un problème avec une commande ou besoin d'un service après-vente ? un bot informatique est disponible jour et nuit pour traiter votre demande.
			\item Vous ne savez plus quelle série regarder ? \texttt{Netflix} peut faire des suggestions personnalisées.
			\item Vous avez les mains pleines de farine et vous voulez lancer un minuteur ou écouter de la musique ? Demandez-le à \texttt{OK Google} ou \texttt{Alexa}.
			\item Il vous manque une belle image pour votre présentation ? \texttt{DALL-E} peut la générer.
			\item Vous devez rédiger une dissertation en histoire-géographie ? \texttt{ChatGPT} s'en occupera.
			\item La classe \LaTeX{} proposée par votre école doctorale ne compile pas \footnote{
				Toute ressemblance avec une situation réelle ou ayant existé est purement fortuite !
			} ? \texttt{ChatGPT} peut aussi identifier l'erreur et même la corriger...
		\end{itemize}
		
		%%% Mais des mythes se créent.
		Les modèles d'\texttt{IA} s'immiscent ainsi dans la plupart des activités de notre quotidien.
		%% NOTE (Thierry P.): Comme toute technologie, cette opmniprésence instaure aussi des dépendances, une diminution de certaines compétences, ... >> "Cependant, cette omniprésence, en plus de créer une dépendance à la technologie, est aussi source de confusion [...]" ?
		Cependant, cette omniprésence est aussi source de confusion et d'incompréhension pour le grand public.
		En effet :
		\begin{itemize}
			% Craintes.
			\item L'\texttt{IA} peut être perçue comme une menace, soit parce qu'elle vole des emplois, soit parce qu'elle risquerait de devenir incontrôlable.
			Ces craintes sont notamment véhiculées par la culture populaire, à l'image de \texttt{Terminator} ou d'\texttt{Ultron} qui se sont retournés contre leurs créateurs.
			% Attentes trop hautes.
			\item Les attentes des utilisateurs sont parfois trop élevées par rapport aux capacités réelles du modèle.
			Il en résulte alors un sentiment de frustration, en particulier lorsque l'utilisateur exprime un besoin urgent, mais que le modèle se contente de répondre qu'il n'a pas compris la question et que vous devriez reformuler votre demande.
			% Confiance aveugle.
			\item Le crédit accordé aux modèles d'\texttt{IA} est parfois excessif, au point que l'esprit critique des utilisateurs s'efface petit à petit.
			Les capacités spectaculaires des derniers modèles génératifs accentuent davantage cette confiance aveugle, et il devient même difficile d'identifier les fausses informations générées tant elles semblent crédibles (\textit{par exemple, \texttt{ChatGPT} peut vous mentir avec conviction en inventant certains détails dans ses réponses}). 
		\end{itemize}
		
		%%% Idée reçu : pas complexe à faire.
		L'idée reçue selon laquelle il est facile de concevoir un modèle d'\texttt{IA} explique en partie ces confusions.
		Encore une fois, la culture populaire véhicule cette image d'une conception accessible à tous et à moindres coûts.
		En reprenant l'exemple d'\textit{Ultron}, il suffit au \texttt{Dr. Hank Pym} de \textguillemets{calquer ses schémas mentaux} pour créer le robot, comme si cela était un acte banal ; plus récemment, dans la série \texttt{Black Mirror} (S2 Ep1), le personnage de \textit{Martha} se procure un avatar de son conjoint décédé sans trop de difficultés en communiquant simplement les messages et les photos de ce dernier, mais aucune mention n'est faite sur le processus de conception, excepté qu'il est \textguillemets{expérimental}.
		\newline
		
		%%% TR: annotation de données.
		\textbf{Toutes ces illusions masquent ainsi un point pourtant fondamental à la conception des modèles d'\texttt{IA} : la nécessité d'avoir des données de qualité pour l'entraînement.}
		
		
	%%%%%--------------------------------------------------------------------
	%%%%% NICHE: Désillusion quant à la simplicité de l'annotation de données.
	%%%%%--------------------------------------------------------------------
	\section*{Désillusion quant à la simplicité de l'annotation de données}
	\addcontentsline{toc}{section}{
		\protect\numberline{}
		Désillusion quant à la simplicité de l'annotation de données
	}
		
		% Modèle d'IA => besoins de données.
		Bien qu'il ne soit pas limité à celà, nous pouvons résumer l'apprentissage automatique comme étant majoritairement un ensemble de méthodes permettant de reproduire un phénomène par l'exemple.
		En d'autres termes, il faut des données en qualité et en quantité suffisante pour concevoir la base d'apprentissage d'un modèle d'\texttt{IA}.
		C'est là qu'intervient la \textbf{tâche d'annotation} : celle-ci consiste à demander à un expert du métier, c'est-à-dire à un spécialiste du phénomène, d'enrichir les données pour leur attribuer une signification, de leur fournir de la valeur ajoutée, et ainsi permettre à la machine de comprendre et reproduire le phénomène.
		\newline
		
		% Exemple assistant conversationnel.
		Pour illustrer nos propos, prenons l'exemple des assistants conversationnels (\textit{chatbot}).
		Ces assistants ont pour objectif de traiter automatiquement des requêtes exprimées en langage naturel.
		Pour ce faire, il est possible de réaliser une modélisation de textes en intentions de dialogue.
		À ce titre, des requêtes comme \textguillemets{\textit{joue moi du jazz s'il te plaît !}} ou \textguillemets{\textit{peux-tu lancer une playlist de Noël sur l'enceinte du salon !}} peuvent être modélisées par l'intention \texttt{jouer\_musique}.
		La base d'apprentissage d'un assistant conversationnel est alors constituée d'un ensemble de textes qui ont été annotés en intention.
		
		% Complexité cachée.
		L'aspect élémentaire de notre exemple cache cependant toute la difficulté de cette tâche :
		\begin{itemize}
			\item Il est possible d'avoir une vaste diversité d'intentions (\textit{jouer de la musique, allumer la lumière, consulter la météo, démarrer un minuteur, appeler sa maman, ...}) : la complexité grandit d'autant qu'il y a de cas d'usage modélisables ;
			\item L'interprétation du langage est complexe par essence : le vocabulaire employé peut être spécifique, une requête peut être ambiguë ou à double sens, une erreur grammaticale peut gêner la compréhension de la phrase, ...
			\item La modélisation en intentions est un exercice subjectif durant laquelle deux annotateurs peuvent avoir des avis différents : dans notre exemple, et avec les mots \textguillemets{\textit{peux-tu lancer ...}}, est-ce que l'assistant devrait effectuer une action, ou devrait-il simplement exprimer s'il en est capable ?
		\end{itemize}
		
		% Bilan sur cette complexité.
		Ainsi, l'annotation de textes en intentions met en évidence la complexité de cette tâche.
		Ben Hamner, cofondateur et directeur technique de l'entreprise \texttt{Kaggle}, résume ainsi cette problématique : \textbf{l'\texttt{IA} c'est \textguillemets{1\% d'écriture de code informatique, 9\% d'analyse de ce qui ne va pas dans le code informatique, et 90\% d'analyse de ce qui ne va pas dans les données d'entraînement}}.
		
		
	%%%%%--------------------------------------------------------------------
	%%%%% GAP: Intervention difficile des experts métiers dans les projets de conception de base d'apprentissage.
	%%%%%--------------------------------------------------------------------
	\section*{Intervention difficile des experts métiers dans les projets de conception de base d'apprentissage}
	\addcontentsline{toc}{section}{
		\protect\numberline{}
		Intervention difficile des experts métiers dans les projets de conception de base d'apprentissage
	}
		
		% Organisation MATTER.
		Pour absorber la complexité liée à la tâche d'annotation, un projet de conception d'une base d'apprentissage s'organise généralement de manière cyclique.
		\begin{itemize}
			\item En premier lieu, les experts métiers sont consultés lors d'ateliers d'idéation pour définir une première version de la modélisation (\textit{dans le cadre de notre exemple sur les assistants conversationnels, ils définiraient la liste des intentions possibles}).
			\item Dans un second temps, les experts métiers parcourent l'ensemble des données pour les annoter en fonction de la modélisation définie ;
			\item Si certaines frictions apparaissent (\textit{modélisation inadaptée, différences d'avis en annotateurs, ...}), le projet retourne à l'étape de modélisation pour proposer une nouvelle version révisée, et le cycle recommence...
			%% NOTE (Thierry P.): Donner un exemple en ba se page d'une modélisation inadaptée sur la base d'un chatbot ? par exemple, un processus client mal détaillé ?
		\end{itemize}
		
		% Complexité d'organisation.
		Une telle approche a l'avantage de régler la complexité de la tâche par de petits ajustements, s'apparentant ainsi aux méthodes agiles.
		Toutefois, elle possède les inconvénients d'être chronophage et onéreuse.
		En effet, à chaque remise en cause de la modélisation, toutes les données annotées sont potentiellement à revoir pour s'assurer de leur compatibilité avec la nouvelle modélisation proposée.
		D'autre part, la définition d'une modélisation stable et cohérente, ainsi que l'encadrement de ses remises en cause potentielles, relève du domaine analytique, qui n'est pas le domaine principal d'expertise des annotateurs intervenant dans le projet.
		Il est donc nécessaire de former les experts métiers à la tâche d'analyse de données pour que leurs remarques soient le plus pertinentes possibles lors des revues de modélisations.
		\newline
		
		% BAM : ça va pas !
		Nous touchons alors du doigt un paradoxe manifeste : \textbf{comment sommes-nous arrivés à la conclusion que la meilleure manière de faire intervenir un expert métier sur une tâche d'annotation, c'est en lui demandant une tâche pour laquelle il n'est pas expert ?}
		(\textit{En effet, si nous voulions par exemple faire un assistant conversationnel sur un domaine gastronomique, nous engagerions a priori un chef cuisinier ou un restaurateur ;
		toutefois, il semblerait incongru d'engager ces profils dans le but de réaliser des analyses statistiques ou d'être compétents sur des questions linguistiques.})
		
		% Constat : on a reposer la complexité sur l'annotateur.
		En conclusion, l'organisation traditionnelle des projets d'annotation ne semble pas résoudre la complexité de cette tâche : au contraire, elle semble plutôt l'ignorer en espérant qu'un opérateur humain suffisamment formé la résolve tout seul.
		%% NOTE (Thierry P.): "l'organisation traditionnelle des projets d'annotation ne semble pas optimale, et elle introduit de fait une complexité qui est généralement ignorée, mais subie par les opérateur humain intervenant dans le processus" ?
		
		
	%%%%%--------------------------------------------------------------------
	%%%%% OCCUPYING THE NICHE: Recherche d'une méthode alternative pour modéliser le texte en intentions tout en valorisant l'intervention de l'expert
	%%%%%--------------------------------------------------------------------
	\newpage
	\section*{Recherche d'une méthode alternative pour modéliser le texte en intentions tout en valorisant l'intervention de l'expert}
	\addcontentsline{toc}{section}{
		\protect\numberline{}
		Recherche d'une méthode alternative pour modéliser le texte en intentions tout en valorisant l'intervention de l'expert
	}
		
		% Objectifs.
		Au cours de ce doctorat, nous nous sommes demandé s'il était possible de changer la philosophie traditionnelle régissant les projets d'annotation, avec pour objectif de remettre l'expert métier au centre du processus.
		Pour cela, nous nous sommes restreints au cas d'usage de la modélisation du texte en intentions, en prenant comme contrainte le fait de toujours impliquer un expert métier pour ses vraies connaissances, et en lui demandant un minimum de connaissances analytiques ou techniques.
		
		% Proposition.
		Nous avons donc proposé une méthodologie d'annotation assistée par la machine et basée sur la caractérisation de contraintes de similarité.
		Dans ce manuscrit, nous organisons la présentation et l'étude de cette méthodologie de la manière suivante :
		
		% Plan.
		\begin{itemize}
			% Chapitre 2: Revue de littérature.
			\item Au cours du \textsc{Chapitre~\ref{chapter:2-REVUE-DE-LITTERATURE}}, nous présentons une revue de littérature sur la tâche d'annotation, son organisation traditionnelle ainsi que les nombreux défis qu'elle comporte.
			Pour mieux illustrer nos propos, nous utilisons des exemples inspirés de l'univers de la bande dessinée.
			% Section 2.4: Contexte du doctorat.
			\item Nous complétons la revue de littérature en expliquant le contexte de ce doctorat en \textsc{Section~\ref{section:2.4-CONTEXTE-DOCTORAT}} : ce complément nous permet de mettre en évidence la difficulté d'intervention des experts métiers dans un projet traditionnel d'annotation pour la conception d'assistants conversationnels.
			% Chapitre 3 : Présentation de la méthode.
			\item Le \textsc{Chapitre~\ref{chapter:3-CLUSTERING-INTERACTIF}} est dédié à la présentation de notre méthodologie d'annotation alternative basée sur un \texttt{Clustering Interactif}.
			La description de l'implémentation technique est consultable dans l'\textsc{Annexe~\ref{annex:C-ANNEXE-IMPLEMENTATIONS}}.
			% Chapitre 4 : Etude de la méthode
			\item Dans le \textsc{Chapitre~\ref{chapter:4-ETUDES}}, nous décrivons les six hypothèses que nous voulions vérifier sur notre méthodologie d'annotation : efficacité, efficience, coûts, pertinence, rentabilité et robustesse.
			\item Le \textsc{Chapitre~\ref{chapter:5-GUIDE}} fait le point sur l'ensemble des discussions et découvertes contenues des précédents chapitres, et comporte différents avis et conseils pratiques.
			Le chapitre entier est prévu pour être un guide d'utilisation synthétique de notre méthodologie d'annotation.
		\end{itemize}
		
		% Conclusion.
		Le \textsc{Chapitre~\ref{chapter:6-CONCLUSION}} dresse la conclusion et clôt la discussion en abordant des thématiques et perspectives plus générales.
