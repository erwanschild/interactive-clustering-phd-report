\chapter{Introduction}
\label{chapter:1-INTRODUCTION}


	%%%%%--------------------------------------------------------------------
	%%%%% ASSET CENTRALITY: Mythe des \texttt{IA} conçues rapidement et à moindres frais.
	%%%%%--------------------------------------------------------------------
	\section*{Mythe des \texttt{IA} conçues rapidement et à moindres frais}
	\addcontentsline{toc}{section}{
		\protect\numberline{}
		Mythe des \texttt{IA} conçues rapidement et à moindres frais
	}
		
		%%% Démocratisation de l'utilisation d'IA.
		L'Intelligence Artificielle (\texttt{IA}) a connu une démocratisation massive ces dernières années.
		Elle est considérée comme une révolution majeure de notre société, à tel point qu'il devient presque impossible de s'en passer.
		\begin{itemize}
			\item Vous avez besoin de trouver votre chemin ? utilisez votre \texttt{GPS}.
			\item Vous avez un problème avec une commande ou besoin d'un service après-vente ? un bot informatique est disponible nuit et jour pour traiter votre demande.
			\item Vous ne savez plus quelle série regarder ? \texttt{Netflix} peut faire des suggestions personnalisées.
			\item Vous avez les mains pleines de farine et vous voulez lancer un minuteur ou écouter de la musique ? Demandez-le \texttt{OK Google} ou \texttt{Alexa}.
			\item Il vous manque une belle image pour votre présentation ? \texttt{DALL-E} peut la générer.
			\item Vous devez rédiger une dissertation en histoire-géographie ? \texttt{ChatGPT} s'en occupera.
			\item La classe \LaTeX{} proposée par votre école doctorale ne compile pas ? \texttt{ChatGPT} peut aussi identifier l'erreur et même la corriger...
		\end{itemize}
		
		%%% Mais des mythes se créent.
		Les modèles d'\texttt{IA} s'immiscent ainsi dans la plupart des activités de notre quotidien.
		Cependant, cette omniprésence est aussi source de confusion et d'incompréhension.
		En effet :
		\begin{itemize}
			% Craintes.
			\item Aux yeux du grand public, l'\texttt{IA} peut être perçue comme une menace.
			Ces craintes sont notamment véhiculées par la culture populaire, à l'image de \texttt{Terminator} ou d'\texttt{Ultron} qui ont réussi à échapper au contrôle de leur créateur.
			% Attentes trop hautes.
			\item Les attentes des utilisateurs sont parfois trop élevées par rapport aux capacités réelles du modèle.
			Il en résulte alors un sentiment de frustration, en particulier lorsque l'utilisateur exprime un besoin urgent, mais que le modèle se contente de répondre qu'il n'a pas compris la question.
			% Confiance aveugle.
			\item La confiance accordée aux modèles d'\texttt{IA} est parfois excessive, au point que l'esprit critique des utilisateurs s'efface petit à petit.
			Cette confiance aveugle peut venir des capacités spectaculaires des derniers modèles génératifs, pour lesquels il est difficile d'identifier une fausse information tant le résultat est crédible (\textit{par exemple, considérons les hallucinations de \texttt{ChatGPT} sur certains faits ou personnages historiques}). 
		\end{itemize}
		
		%%% Idée reçu : pas complexe à faire.
		L'idée reçu selon laquelle la conception d'un modèle d'\texttt{IA} est simple et peu onéreuse explique en partie ces confusions.
		Encore une fois, la culture populaire véhicule l'image d'une conception accessible à tous et à moindres coûts.
		En reprenant l'exemple d'\textit{Ultron}, il suffit au \texttt{Dr. Hank Pym} de \textguillemets{calquer ses schémas mentaux} pour créer le robot, comme si cela était un acte banal ; plus récemment, dans la série \texttt{Black Mirror} (S2 Ep1), le personnage de \textit{Martha} se procure un avatar de son conjoint décédé sans trop de difficultés en communiquant simplement les messages et les photos de ce dernier, mais aucune mention n'est fait sur le processus de conception excepté qu'il est \textguillemets{expérimental}.
		
		
	%%%%%--------------------------------------------------------------------
	%%%%% NICHE: La réalité d'une conception complexe, subjective et onéreuse.
	%%%%%--------------------------------------------------------------------
	\section*{Réalité d'une conception complexe, subjective et onéreuse}
	\addcontentsline{toc}{section}{
		\protect\numberline{}
		Réalité d'une conception complexe, subjective et onéreuse
	}
		
		\todo[inline]{A REDIGER}
		
		%%% Mais c'est complexe à entrainer
			% Infrastructure technique
			% Besoin d'annotations/modélisation
		
		%%% Difficultés
			% organoisation complexe
			% difficulté à interpreter le langage
			% cas d'usage spécifiques à chaque domaine
			% subkjectivité de la tâche
		
		
	%%%%%--------------------------------------------------------------------
	%%%%% GAP: Complexité de conception encaissée par les annotateurs.
	%%%%%--------------------------------------------------------------------
	\section*{Complexité de conception encaissée par les annotateurs}
	\addcontentsline{toc}{section}{
		\protect\numberline{}
		Complexité de conception encaissée par les annotateurs
	}
		
		\todo[inline]{A REDIGER}
		
		%- Peu de travaux sur la conception d'un jeu de données : en recherche les données sont publiques, en entreprises les données sont privées ;
		%- Nombreuses pistes d'amélioration, mais peu sont exploitées en pratique ; 
		%- Défis d'organisation, de gestion de coûts, de complexité, de qualité, ... 
		%- Conception trop manuelle, experts pas à leur place, ...
		
		
		
	%%%%%--------------------------------------------------------------------
	%%%%% OCCUPYING THE NICHE: Recherche d'une méthode alternative pour la conception de \textit{chatbots}
	%%%%%--------------------------------------------------------------------
	\section*{Recherche d'une méthode alternative pour la conception de \textit{chatbots}}
	\addcontentsline{toc}{section}{
		\protect\numberline{}
		Recherche d'une méthode alternative pour la conception de \textit{chatbots}
	}
		
		\todo[inline]{A REDIGER}
		
			%- Besoin de recentrer l'activité des experts métiers ;
			%- Besoin d'assister la conception d'un jeu de données ;
			%- Nous proposons donc une méthode itérative et semi-supervisée.
		
	%%%%%--------------------------------------------------------------------
	%%%%% PLAN: Annonce du plan de ce manuscrit.
	%%%%%--------------------------------------------------------------------
	\section*{Annonce du plan de ce manuscrit}
	\addcontentsline{toc}{section}{
		\protect\numberline{}
		Annonce du plan de ce manuscrit
	}
		
		% Introduction.
		Afin de traiter la problématique que nous venons d'exposer, nous organisons la discussion de ce manuscrit de la manière suivante :
		
		% Plan.
		\begin{itemize}
			% Chapitre 2: Revue de littérature.
			\item Au cours du \textsc{Chapitre~\ref{chapter:2-REVUE-DE-LITTERATURE}}, nous présentons en détails la tâche d'annotation, son organisation traditionnelle ainsi que les nombreux défis qu'elle comporte.
			Pour mieux illustrer nos propos, nous utilisons des exemples inspirés de l'univers de la bande dessinée.
			% Section 2.4: Contexte du doctorat.
			\item Nous complétons la revue de littérature en expliquant le contexte de ce doctorat en \textsc{Section~\ref{section:2.4-CONTEXTE-DOCTORAT}} : ce complément nous permet de mettre en évidence la difficulté d'intervention des experts métiers dans un projet traditionnel d'annotation.
			% Chapitre 3 : Présentation de la méthode.
			\item Le \textsc{Chapitre~\ref{chapter:3-CLUSTERING-INTERACTIF}} est dédié à la présentation de notre méthodologie d'annotation alternative basée sur un \texttt{Clustering Interactif}.
			La description de l'implémentation technique est consultable dans l'\textsc{Annexe~\ref{annex:C-ANNEXE-IMPLEMENTATIONS}}.
			% Chapitre 4 : Etude de la méthode
			\item Dans le \textsc{Chapitre~\ref{chapter:4-ETUDES}}, nous décrivons les six hypothèses que nous voulions vérifier sur notre méthodologie d'annotation : efficacité, efficience, coûts, pertinence, rentabilité et robustesse.
			\item Le \textsc{Chapitre~\ref{chapter:5-GUIDE}} fait le point sur l'ensemble des discussions et découvertes contenues des précédents chapitres, et comporte différents avis et conseils pratiques.
			Le chapitre entier est prévu pour être un guide d'utilisation synthétique de notre méthodologie d'annotation.
		\end{itemize}
		
		% Conclusion.
		Le \textsc{Chapitre~\ref{chapter:6-CONCLUSION}} dresse la conclusion et clôt la discussion en abordant des thématiques et perspectives plus générales.
