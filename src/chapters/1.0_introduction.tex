\chapter{Introduction}
\label{chapter:1-INTRODUCTION}


	%%%%%--------------------------------------------------------------------
	%%%%% Section 1.1: 
	%%%%%--------------------------------------------------------------------
	\section*{?}
	\addcontentsline{toc}{section}{\protect\numberline{}S1.1}  % A RENOMMER
	\label{section:1.1-INTRODUCTION-}
		
		\todo[inline]{SECTION 1.1: À RÉDIGER}
		
		%%% Démocratisation de l'utilisation d'IA.
		Dans notre quotidien, il est devenu commun de croiser la route de modèles d'Intelligence Artificielle (\texttt{IA}).
		Pour s'en rendre compte, il suffit d'utiliser son \texttt{GPS}, de contacter le standard d'un service après-vente, de consulter les séries qu'on nous recommande sur une plateforme de \textit{streaming}, ou encore de donner l'ordre de fermer les volets à son enceinte connectée.
		Cette révolution est d'autant plus marquée avec la démocratisation des \texttt{IA} génératives ayant lieu depuis 2023.
		
		%%% Capacités folles de ces IA.
		\begin{leftBarImportantGray}
			Besoin d'une belle image pour illustrer pour une page de garde ? \texttt{DALL-E} (\cite{ramesh-etal:2021:zeroshot-texttoimage-generation}) peut la générer ! \\
			Besoin de rédiger une dissertation pour le prochain cours d'histoire-géo ? ou de corriger les erreurs de la classe \LaTeX{} proposée par son école doctorale ? \texttt{ChatGPT} (\cite{openai:2023:chatgpt}) en est capable ! \\
			Besoin de rédiger le compte rendu d'une réunion à distance ? \texttt{Microsoft Copilot} (\cite{microsoft-corporation:2023:microsoft-copilot}) s'en charge automatiquement !
		\end{leftBarImportantGray}
		
		
		%%% MAIS le grand public 
		Cependant, le regard du grand public sur les capacités et les perspectives d'utilisation de l'\texttt{IA} sont souvent 
			% C'est simple
			% Attente trop hautes
			% C'est dangeureux
		
		%%% Exemples de la pop culture.
			%\begin{leftBarExamples}
				% Terminator \cite{cameron:1984:terminator}
				% Ultron \cite{thomas-etal:1968:deliver-us-masters}
				% MODOK \cite{lee-etal:1967:if-this-be}
				% Black Mirror \cite{harris:2012:bientot-retour}
			%\end{leftBarExamples}
		
		
	%%%%%--------------------------------------------------------------------
	%%%%% Section 1.2:
	%%%%%--------------------------------------------------------------------
	\section*{?}
	\addcontentsline{toc}{section}{\protect\numberline{}S1.2}  % A RENOMMER
	\label{section:1.2-INTRODUCTION-}
		
		\todo[inline]{SECTION 1.2: À RÉDIGER}
		
		%%% Cas des assistants conversationnels.
			% C'est utilisable dans plein de domaine
			
		%%% Mais c'est complexe à entrainer
			% Infrastructure technique
			% Besoin d'annotations/modélisation
		
		%%% Difficultés
			% organoisation complexe
			% difficulté à interpreter le langage
			% cas d'usage spécifiques à chaque domaine
			% subkjectivité de la tâche
		
		
	%%%%%--------------------------------------------------------------------
	%%%%% Section 1.3:
	%%%%%--------------------------------------------------------------------
	\section*{?}
	\addcontentsline{toc}{section}{\protect\numberline{}S1.3}  % A RENOMMER
	\label{section:1.3-INTRODUCTION-}
		
		% 
		
		\todo[inline]{SECTION 1.3: À RÉDIGER}
		
			%- Peu de travaux sur la conception d'un jeu de données : en recherche les données sont publiques, en entreprises les données sont privées ;
			%- Nombreuses pistes d'amélioration, mais peu sont exploitées en pratique ; 
			%- Défis d'organisation, de gestion de coûts, de complexité, de qualité, ... 
			%- Conception trop manuelle, experts pas à leur place, ...
		
	%%%%%--------------------------------------------------------------------
	%%%%% Section 1.4:
	%%%%%--------------------------------------------------------------------
	\section*{?}
	\addcontentsline{toc}{section}{\protect\numberline{}S1.4}  % A RENOMMER
	\label{section:1.4-INTRODUCTION-}
		
		\todo[inline]{SECTION 1.4: À RÉDIGER}
		
			%- Besoin de recentrer l'activité des experts métiers ;
			%- Besoin d'assister la conception d'un jeu de données ;
			%- Nous proposons donc une méthode itérative et semi-supervisée.
		
	%%%%%--------------------------------------------------------------------
	%%%%% Section 1.5: Annonce du plan de ce manuscrit.
	%%%%%--------------------------------------------------------------------
	\section*{Annonce du plan de ce manuscrit}
	\addcontentsline{toc}{section}{\protect\numberline{}Annonce du plan de ce manuscrit}
	\label{section:1.5-INTRODUCTION-ANNONCE-PLAN}
		
		% Introduction.
		Afin de traiter la problématique que nous venons d'exposer, nous organisons la discussion de ce manuscrit de la manière suivante :
		
		% Plan.
		\begin{itemize}
			% Chapitre 2: Revue de littérature.
			\item Au cours du \textsc{Chapitre~\ref{chapter:2-REVUE-DE-LITTERATURE}}, nous présentons en détails la tâche d'annotation, son organisation traditionnelle ainsi que les nombreux défis qu'elle comporte.
			Pour mieux illustrer nos propos, nous utilisons des exemples inspirés de l'univers de la bande dessinée.
			% Section 2.4: Contexte du doctorat.
			\item Nous complétons la revue de littérature en expliquant le contexte de ce doctorat en \textsc{Section~\ref{section:2.4-CONTEXTE-DOCTORAT}} : ce complément nous permet de mettre en évidence la difficulté d'intervention des experts métier dans un projet traditionnel d'annotation.
			% Chapitre 3 : Présentation de la méthode.
			\item Le \textsc{Chapitre~\ref{chapter:3-CLUSTERING-INTERACTIF}} est dédié à la présentation de notre méthodologie d'annotation alternative basée sur un \texttt{Clustering Interactif}.
			La description de l'implémentation technique est consultable dans l'\textsc{Annexe~\ref{annex:C-ANNEXE-IMPLEMENTATIONS}}.
			% Chapitre 4 : Etude de la méthode
			\item Dans le \textsc{Chapitre~\ref{chapter:4-ETUDES}}, nous décrivons les six hypothèses que nous voulions vérifier sur notre méthodologie d'annotation : efficacité, efficience, coûts, pertinence, rentabilité et robustesse.
			\item Le \textsc{Chapitre~\ref{chapter:5-GUIDE}} fait le point sur l'ensemble des discussions et découvertes contenues des précédents chapitres, et comporte différents avis et conseils pratiques.
			Le chapitre entier est prévu pour être un guide d'utilisation synthétique de notre méthodologie d'annotation.
		\end{itemize}
		
		% Conclusion.
		Le \textsc{Chapitre~\ref{chapter:6-CONCLUSION}} dresse la conclusion et clôt la discussion en abordant des thématiques et perspectives plus générales.
