\begin{figure}[H]
\centering
\begin{tikzpicture}[
	node distance=1.5cm and 2.5cm,
	shorten >=1pt,
	>=stealth',
	auto,
	state/.style={
		circle,
		thick,
		minimum size=2.75cm,
		align=center,
	},
	edge/.style={
		align=center,
	},
]
    \node [
		state,
		accepting,
		draw=blue!75,
		fill=blue!20,
	] (Sf) {\scriptsize données \\ \scriptsize segmentées};
    \node [
		state,
		initial,
		draw=red!75,
		fill=red!20,
	] (S0) [above=of Sf] {\scriptsize données non \\ \scriptsize segmentées};
    \node [
		state,
		draw=orange!75,
		fill=orange!20,
	] (S2) [left=of Sf] {\scriptsize nouvelles \\ \scriptsize contraintes \\ \scriptsize à intégrer};
    \node [
		state,
		draw=orange!75,
		fill=orange!20,
	] (S1) [left=of S2] {\scriptsize échantillon \\ \scriptsize de segmentation \\ \scriptsize à vérifier};

    \path[->] (S0) edge [edge, left] node {\scriptsize clustering \\ \scriptsize non contraint} (Sf);
    \path[->] (Sf) edge [edge, bend left=30] node {\scriptsize nouvel \scriptsize échantillonnage} (S1);
    \path[->] (S1) edge [edge, left] node [pos=0.9] {\scriptsize annotation de \\ \scriptsize contraintes} (S2);
    \path[->] (S2) edge [edge, left] node [pos=1] {\scriptsize clustering \\ \scriptsize sous contraintes} (Sf);      
\end{tikzpicture}
	\caption{Diagramme d'état représentant les grandes étapes du clustering interactif.}
    \label{figure:CLUSTERING-INTERACTIF}
\end{figure}