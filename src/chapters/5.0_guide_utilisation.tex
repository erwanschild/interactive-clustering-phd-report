\chapter{Guide d'utilisation du \textit{Clustering Interactif}}
\label{chapter:5-GUIDE}

	%%%%%--------------------------------------------------------------------
	%%%%% Section 5.1:
	%%%%%--------------------------------------------------------------------
	\section{Organisation \texttt{ITTER}}
		\label{section:5.1-GUIDE-ITTER}
		\todo[inline]{SECTION À RÉDIGER}
	
	
	%%%%%--------------------------------------------------------------------
	%%%%% Section 5.2:
	%%%%%--------------------------------------------------------------------
	\section{Conseils pratiques}
		\label{section:5.2-GUIDE-CONSEILS}
		\todo[inline]{SECTION À RÉDIGER}
		
		
	% \cite{leech:1993:corpus-annotation-schemes}
		% 1. It should always be possible to come back to initial data (example BC). Note: can be hard after normalization (“l'arbre” ! “le arbre”, etc.)
		% 2. Annotations should be extractable from the text
		% 3. The annotation procedure should be documented (ex: Brown Corpus annotation guide, Penn Tree Bank annotation guide)
		% 4. Mention should be made of the annotator(s) and the way annotation was made (manual/automatic annotation, number of annotators, manually corrected/uncorrected...)
		% 5. Annotation is an act of interpretation (cannot be infallible)
		% 6. Annotation schemas should be as independent as possible on formalisms
		% 7. No annotation schema should consider itself a standard (it possibly becomes one)