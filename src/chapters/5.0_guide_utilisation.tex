\chapter{Bilan et Guide d'utilisation du \textit{Clustering Interactif}}
\label{chapter:5-GUIDE}
	
	%%%%%--------------------------------------------------------------------
	%%%%% Section 5.1:
	%%%%%--------------------------------------------------------------------
	\section{Présentation rapide du clustering interactif et de ses avantages}
		\label{section:5.1-GUIDE-PRESENTATION-RAPIDE}
		\todo[inline]{
			SECTION À RÉDIGER: \\
			- trouver une base d'apprentissage acceptable, puis corriger manuellement \\
			- intervention d'experts métiers sur la base de leur connaissance métiers \\
			- revues d'annotations basée sur leur connaissance métiers \\
			- aide à la modélisation par le clustering
		}
	
	
	%%%%%--------------------------------------------------------------------
	%%%%% Section 5.2:
	%%%%%--------------------------------------------------------------------
	\section{Conseils pour organiser l'annotation}
		\label{section:5.2-GUIDE-ANALYSE}
		\todo[inline]{
			SECTION À RÉDIGER: \\
			- bien définir l'objectif à modéliser (par action ? par objet ?) \\
			- laisser à la mach \\
			- faire référence aux maximes de \cite{leech:1993:corpus-annotation-schemes} ?
			1. VOIR LA DONNEE NON PRETRAITEE: It should always be possible to come back to initial data (example BC). Note: can be hard after normalization ("l'arbre !" vs "le arbre", etc.) \\
			2. ANNOTER DES DIFFERENCES VISIBLES: Annotations should be extractable from the text \\
			3. DOCUMENTER L'AJOUT DE CONTRAINTES: The annotation procedure should be documented (ex: Brown Corpus annotation guide, Penn Tree Bank annotation guide) \\
			4. DOCUMENTER LES COMPETENCES DES ANNOTATEURS: Mention should be made of the annotator(s) and the way annotation was made (manual/automatic annotation, number of annotators, manually corrected/uncorrected...) \\
			5. SUBJECTIVITE => 3 ANNOTATEURS: Annotation is an act of interpretation (cannot be infallible) \\
			6. MIEUX VAUT NE PAS LIER QUE LIER DE MANIERE AMBIGUE: Annotation schemas should be as independent as possible on formalisms \\
			7. PLUSIEURS VISIONS POSSIBLES, IL FAUT EN CHOISIR UNE ET S'Y TENIR: No annotation schema should consider itself a standard (it possibly becomes one)
		}
	
	
	%%%%%--------------------------------------------------------------------
	%%%%% Section 5.3:
	%%%%%--------------------------------------------------------------------
	\section{Conseils pour analyser les résultats}
		\label{section:5.3-GUIDE-ANALYSE}
		\todo[inline]{
			SECTION À RÉDIGER: \\
			- Cas d'arrêt quand le clustering stagne à 5\% : si change pas, alors l'annotation n'a plus d'effet... \\
			- utiliser le graphe de contraintes pour voir les données liées entres elles et les données isolées \\
			- Analyse avec résumer par LLM pour identifier facilement les thématiques qui se dégagent \\
			- Analyse avec FMC pour identifier le vocabulaire qui caractérise chaque thématique \\
			- utiliser ces analyses pour régler le nombres d clusters ?
		}
	
	
	%%%%%--------------------------------------------------------------------
	%%%%% Section 5.4:
	%%%%%--------------------------------------------------------------------
	\section{Conseils pour paramétrer la méthode}
		\label{section:5.4-GUIDE-PARAMETRAGE}
		\todo[inline]{
			SECTION À RÉDIGER: \\
			- Architecture parallele : pour gagner du temps\\
			- paramétrage optimal: simple + tfidf + kmeans + closest \\
			- taille de batch d'annotation dépendant de la taille du jeu de données (dépend du temps de clustering)
		}
	
	
	%%%%%--------------------------------------------------------------------
	%%%%% Section 5.5:
	%%%%%--------------------------------------------------------------------
	\section{Estimation des coûts du projet d'annotation}
		\label{section:5.5-GUIDE-COUTS}
		\todo[inline]{
			SECTION À RÉDIGER: \\
			- rappeler les équations de temps: environ 24 x taille de dataset, sans les revues d'annotations \\
			- besoin de 3 annotateurs pour confronter les vision et modéliser dans de bonnes conditions. organiser les revues sur la base des différences entre cas d'usage métier \\
			- ajouter de la redondance si le cas d'usage est complexe, mais ça ralenti \\
			- avantage par rapport aux méthode usuelles : moins abstrait/complexe, moins d'essai-erreur
		}