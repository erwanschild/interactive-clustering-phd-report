\DontFrameThisInToc
\Annex{Annexe des architectures d'assistants conversationnels (\textit{chatbot})}
\label{annex:B-ANNEXE-CHATBOTS}
	
	% INTRODUCTION DE L'ANNEXE.
	
	% Engoument pour les chatbots en 2019.
	Au début de ce doctorat (\texttt{octobre 2019}), nous pouvions noter que :
	\begin{itemize}
		\item selon \cite{costello-lodolce:2019:gartner-top-technologies}, \textguillemets{\textit{seuls $4$\% des clients de \texttt{Gartner} [déclaraient] utiliser des \textit{chatbot} sur leur lieu de travail, mais $40$\%} [avaient] \textit{l'intention de les mettre en oeuvre à court terme}} ;
		\item et selon \cite{goasduff:2019:chatbots-will-appeal}, \textguillemets{\textit{d'ici 2022, 70\% des employés} [interagiraient] \textit{quotidiennement avec les plateformes conversationnelles}}.
	\end{itemize}
	
	% Engoument pour les chatbots en 2023.
	Aujourd'hui (\texttt{octobre 2023}), le mot \textit{chatbot} est sur toutes les lèvres, surtout depuis la révolution des \texttt{IA} génératives lancées par \texttt{ChatGPT} (\cite{openai:2023:chatgpt}) :
	\begin{itemize}
		\item selon \cite{costello-lodolce:2022:gartner-predicts-chatbots}, une entreprise sur deux aurait actuellement recours à une forme de \textit{chatbot} pour gérer sa relation client, et \textguillemets{\textit{d'ici 2027, les \textit{chatbot} deviendront le principal canal de service client pour environ un quart des organisations}}.
	\end{itemize}

	% Annonce du plan.
	Dans cette annexe, nous allons détailler :
	\begin{itemize}
		\item une présentation succincte des approches principales de conception d'un assistant conversationnel, avec leurs avantages et leurs inconvénients (voir \textsc{Section~\ref{annex:B.1-ANNEXE-CHATBOTS-APPROCHES}}) ;
		\item une discussion sur le dilemme du conception entre flexibilité du dialogue et contrôle du comportement (voir \textsc{Section~\ref{annex:B.2-ANNEXE-CHATBOTS-DILEMME}}).
	\end{itemize}
	
	% TABLE DES MATIÈRES DE L'ANNEXE.
	\minitoc

	%%%%%--------------------------------------------------------------------
	%%%%% Annexe B.1: Approches principales pour concevoir un \textit{chatbot}
	%%%%%--------------------------------------------------------------------
	\newpage
	\section{Approches principales pour concevoir un \textit{chatbot}}
	\label{annex:B.1-ANNEXE-CHATBOTS-APPROCHES}
		%\cite{chen-etal:2017:survey-dialogue-systems}
		%\cite{adamopoulou-moussiades:2020:overview-chatbot-technology}.
		
		% Classification \textit{task-oriented} vs \textit{chat-oriented}.
		Selon \cite{chen-etal:2017:survey-dialogue-systems}, nous pouvons distinguer \textbf{deux approches principales} en fonction de l'usage de l'assistant :
		\begin{itemize}
			\item soit l'assistant est spécialisé pour une tâche bien déterminée (\textit{task-oriented}, \textsc{Section~\ref{annex:B.1.1-CHATBOTS-APPROCHES-TASK-ORIENTED}}) ;
			\item soit l'assistant est axé sur la fluidité de la conversation avec l'utilisateur (\textit{chat-oriented}, voir \texttt{Section~\ref{annex:B.1.2-CHATBOTS-APPROCHES-CHAT-ORIENTED}}).
		\end{itemize}
		
		% Figure illustrant ces deux approches.
		Ces deux approches sont traditionnellement représentées par deux architectures bien connues : celles-ci sont illustrés dans la \textsc{Figure~\ref{figure:B.2-ANNEXE-CHATBOT-APPROCHES}} et seront détaillées dans les sections suivantes.
		%
		\begin{figure}[H]
			\centering
			\includegraphics[width=0.95\textwidth]{figures/annexe-chatbots-architectures}
			\caption{
				Schéma illustrant les deux approches principales de conception d'un assistant conversationnel :
				\textbf{(1)} représente les \textbf{approches \textit{task-oriented}} à l'aide d'une architecture manipulant des états de dialogue ;
				\textbf{(2)} représente les \textbf{approches \textit{chat-oriented}} à l'aide d'une architecture à base de \textit{transformers}, encodant et décodant numériquement le dialogue et son contexte.
			}
			\label{figure:B.2-ANNEXE-CHATBOT-APPROCHES}
		\end{figure}
		
		%%%
		%%% Annexe B.1.1: Approches \textit{task-oriented}.
		%%%
		\newpage
		\subsection{Approches \textit{task-oriented}}
		\label{annex:B.1.1-CHATBOTS-APPROCHES-TASK-ORIENTED}
		
			\todo[inline]{
				A REDIGER: \textit{task-oriented}\\
				Spécialisé tache \\
				Quelques exemples \\
				Architecture gestion d'état: NLP+DST+PL+NLG \\
				Contrôle du dialogue pour une tâche déterminée \\
				Peu de flexibilité
			}
		
		
		%%%
		%%% Annexe B.1.2: Approches \textit{chat-oriented}.
		%%%
		\subsection{Approches \textit{chat-oriented}}
		\label{annex:B.1.2-CHATBOTS-APPROCHES-CHAT-ORIENTED}
		
			\todo[inline]{
				A REDIGER: \textit{chat-oriented}\\
				Axée dialogue \\
				Quelques exemples \\
				Architecture transformers: encodeur/décodeur, tout est numérique \\
				Pas de contrôle (hallucination) \\
				Grande flexibilité
			}
		
		
		%%%
		%%% Annexe B.1.3: Approches hybrides.
		%%%
		\subsection{Approches hybrides}
		\label{annex:B.1.3-CHATBOTS-APPROCHES-HYBRIDES}
		
			\todo[inline]{
				A REDIGER: \textit{hybride}\\
				LLM instruct \\
				DST entraînée \\
				NLG décodeur
			}
	
	
	%%%%%--------------------------------------------------------------------
	%%%%% Annexe B.2: Dilemme entre flexibilité du dialogue et contrôle du comportement.
	%%%%%--------------------------------------------------------------------
	\section{Dilemme entre flexibilité du dialogue et contrôle du comportement}
	\label{annex:B.2-ANNEXE-CHATBOTS-DILEMME}
		
		\todo[inline]{A REDIGER: niveau d'automatisation}
		% \cite{sheridan-verplank:1978:human-computer-control} repris par \cite{parasuraman-etal:2000:model-types-levels} \\ % 10 niveaux de contrôles
		
		
		
	
		% Exemple de projet task-oriented: \cite{yan-etal:2017:building-taskoriented-dialogue}
		
		% \cite{brabra-etal:2022:dialogue-management-conversational}: plusieurs système de gestion du dialogue : soit implémenté manuellement, soit entraîné, soit hybride.
		
		% Note auteur: une confusion fréquente est "task-based" => "not deep-learning"
		
		%%%%%--------------------------------------------------------------------
		%%%%% Annexe B.1: Classification des \textit{chatbots} suivant leurs objectifs.
		%%%%%--------------------------------------------------------------------

			% \subsection{Les \textit{chatbots} \textguillemets{\textit{task-oriented}}}
			% \label{annex:B.1.1-CHATBOT-CLASSIFICATION-TASK-ORIENTED}
			
				% % Définition.
					% Un \textit{chatbot} \textguillemets{\textit{task-oriented}} est conçu pour \textbf{accomplir une tâche spécifique}.
					% Le parcours de dialogue proposé est généralement prédéfini et comporte peu de digressions : l'assistant détermine l'action à réaliser grâce à l'énoncé de l'utilisateur, demande éventuellement des compléments d'informations si la requête n'est pas assez précise, puis effectue sa mission.
					% Le périmètre de fonctionnalités est généralement restreint pour contrôler les actions de l'assistant et s'assurer de son comportement.
				
				% % Exemples de fonctionnalités.
					% Gérer la relation client (\textit{suivi de commande, formulaire de satisfaction, ...}) ;
					% Accéder à des informations documentaires ou personnelles (\textit{accès à une documentation technique, accès au solde d'un compte bancaire, ...}) ;
					% Pré-remplir d'un formulaire (\textit{réserver un billet de voyage, payer ses contraventions, faire opposition à sa carte de crédit, ...}) ;
					 % Gérer la domotique (\textit{allume la lumière, joue de la musique, active l'alarme, ...}).
				
				% % Exemples d'assistants conversationnels connus.
					% \texttt{Assistant Virtuel SNCF} (\cite{sncf:2018:agent-virtuel-sncf}) pour la réservation de billets de train ;
					% \texttt{Google Assistant} (\cite{google:2016:google-assistant-your}) et \texttt{Alexa} (\cite{alexa-internet:2018:keyword-research-competitor}) pour la gestion de la domotique.
			
			
			% \subsection{Les \textit{chatbots} \textguillemets{\textit{chat-oriented}}}
			% \label{annex:B.1.2-CHATBOT-CLASSIFICATION-CHAT-ORIENTED}
			
				% % Définition.
					% Un \textit{chatbot} \textguillemets{\textit{chat-oriented}} se concentre davantage sur les interactions avec l'utilisateur dans le but d'\textbf{engager la conversation}.
					% Son objectif principale est de rendre le dialogue agréable.
					% Son périmètre de connaissance n'est en général pas restreint pour pouvoir facilement engager la conversation sur n'importe quel sujet.
			
				% % Exemples de fonctionnalités.
					% Offrir du divertissement (\textit{raconter une histoire ou une blague, organiser un jeu narratif, ...}) ;
					% Proposer une assistance générale (\textit{donner une définition, demander une explication, ...}) ;
					% Simplement discuter (\textit{mimer les interactions sociales}).
				
				% % Exemples d'assistants conversationnels connus.
					% \texttt{ELIZA} (\cite{weizenbaum:1966:eliza-computer-program}) pour simuler un entretien clinique en psychothérapie ;
					% \texttt{AI Dungeon} (\cite{latitude-inc.-oasis-tech-inc.:2019:ai-dungeon}) pour participer à la narration d'histoires interactives ;
					% \texttt{ChatGPT} (\cite{openai:2023:chatgpt}) et \texttt{BARD} (\cite{google:2023:bard-chat-based}) pour discuter avec des larges modèles de langues (\texttt{LLM}).
		
		
		%%%%--------------------------------------------------------------------
		%%%% Annexe B.2: Architectures usuelles des chatbots.
		%%%%--------------------------------------------------------------------
		% \section{Architectures usuelles des chatbots}
		% \label{annex:B.2-CHATBOT-ARCHITECTURES}

			% % Introduction: classification des \textit{chatbots}.
			% Toujours en s'inspirant de \cite{chen-etal:2017:survey-dialogue-systems} et de \cite{adamopoulou-moussiades:2020:overview-chatbot-technology}, nous pouvons distinguer deux types d'approches principales pour implémenter un \textit{chatbot} :
			% \begin{itemize}
				% \item les \textit{approches symboliques}, traditionnellement utilisées pour traiter le langage en essayant d'en réaliser un modélisation abstraite ;
				% \item les \textit{approches génératives}, reproduisant les capacités du langage en employant les capacités actuelles des réseaux de neurones sur des immenses quantités de données.
			% \end{itemize}
			
			% Ces deux types d'approches sont représentées dans la \textsc{Figure~\ref{figure:B.2-CHATBOT-ARCHITECTURES}}.
			
			% \begin{figure}[!htb]
				% \centering
				% \includegraphics[width=0.95\textwidth]{figures/annexe-chatbots-architectures}
				% \caption{
					% Schéma illustrant les deux architecture usuelles pour implémenter un assistant conversationnel :
					% \textbf{(1)} représente les \textbf{approches symboliques} avec un schéma d'architecture de gestion d'états de dialogue,
					% et \textbf{(2)} représente les \textbf{approches génératives} avec un schéma d'architecture à base de \textit{transformers}, composée d'un encodeur et d'un décodeur.
				% }
				% \label{figure:B.2-CHATBOT-ARCHITECTURES}
			% \end{figure}
			
			
			%%
			%% Subsection B.2.1. Les approches symboliques.
			%%
			% \subsection{Les approches symboliques}
			% \label{annex:B.2.1-CHATBOT-ARCHITECTURES-SYMBOLIQUE}
			% \todo[inline]{A REDIGER}
			
				% % Définition.
				% \paragraph{Définition :}
				
					% \cite{schuurmans-frasincar:2020:intent-classification-dialogue}  \\ % definition intention
				
				% % Exemples d'assistants conversationnels connus.
				% \paragraph{Exemples de moteurs connus ayant une approche symbolique :}
				
					% \texttt{RASA} \cite{bocklisch-etal:2017:rasa-open-source}
					% ou \texttt{WATSON} (\cite{hoyt-etal:2016:ibm-watson-analytics}).
			
			
			%%
			%% Subsection B.2.2. Les approches génératives.
			%%
			% \subsection{Les approches génératives}
			% \label{annex:B.2.2-CHATBOT-ARCHITECTURES-GENERATIVE}
			% \todo[inline]{A REDIGER}
			
				% % Définition.
				% \paragraph{Définition :}
				
					% \cite{uszkoreit:2017:transformer-novel-neural}  \\ % architecture transformers
					% \cite{ni-etal:2022:recent-advances-deep}  \\ % avancer en deep learning
					% \cite{openai:2023:chatgpt}  \\ % exemple
					% \cite{touvron-etal:2023:llama-open-foundation}  \\ % exemple
					% \cite{kaddour-etal:2023:challenges-applications-large} % plusieurs challenges
				
				% % Exemples d'assistants conversationnels connus.
				% \paragraph{Exemples de moteurs connus ayant une approche générative :}
					% \texttt{GPT} (\cite{openai:2023:chatgpt})
					% ou \texttt{LLAMA2} (\cite{touvron-etal:2023:llama-open-foundation}).