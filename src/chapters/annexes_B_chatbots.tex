\Annex{Concevoir un assistant conversationnel (\textit{chatbot})}
\label{annex:B-ANNEXE-CHATBOT}

% INTRODUCTION DE L'ANNEXE.

% Engoument pour les chatbots en 2019.
Au début de ce doctorat (\texttt{octobre 2019}), on pouvait noter que :
\begin{itemize}
	\item selon \cite{costello-lodolce:2019:gartner-top-technologies}, \textguillemets{seuls $4$\% des clients de \texttt{Gartner} [déclaraient à l'époque] utiliser des \textit{chatbot} sur leur lieu de travail, mais $40$\% [avaient] l'intention de les mettre en oeuvre à court terme} ;
	\item et selon \cite{goasduff:2019:chatbots-will-appeal}, \textguillemets{d'ici 2022, 70\% des employés [interagiraient] quotidiennement avec les plateformes conversationnelles}.
\end{itemize}

% Engoument pour les chatbots en 2023.
Aujourd'hui (\texttt{octobre 2023}), le mot \textit{chatbot} est sur toutes les lèvres, surtout la révolution des \texttt{IA} génératives lancées par \texttt{ChatGPT} (\cite{openai:2023:chatgpt}) :
\begin{itemize}
	\item selon \cite{costello-lodolce:2022:gartner-predicts-chatbots}, une entreprise sur deux aurait actuellement recours à une forme de \textit{chatbot}  pour gérer sa relation client, et \textguillemets{d'ici 2027, les \textit{chatbot} deviendront le principal canal de service client pour environ un quart des organisations}.
\end{itemize}

% Annonce du plan.
Dans cette annexe, nous allons détailler brièvement les fonctionnalités et architectures de ces assistants conversationnels.

% TABLE DES MATIÈRES DE L'ANNEXE.
\minitoc

%%%%%--------------------------------------------------------------------
%%%%% Annexe B.1: Classification des \textit{chatbots} suivant leurs objectifs.
%%%%%--------------------------------------------------------------------
\section{Classification des \textit{chatbots} suivant leurs objectifs}
\label{annex:B.1-CHATBOT-CLASSIFICATION}
	
	% Introduction: classification des \textit{chatbots}.
	En s'inspirant de \cite{chen-etal:2017:survey-dialogue-systems} et de \cite{adamopoulou-moussiades:2020:overview-chatbot-technology}, nous pouvons distinguer grossièrement les \textit{chatbots} selon deux catégories :
	\begin{itemize}
		\item les \textit{chatbots} axés sur une tâche précise (\textguillemets{\textit{task-oriented}}) ;
		\item les \textit{chatbots} axés sur la conversation (\textguillemets{\textit{chat-oriented}})
	\end{itemize}
		\cite{adamopoulou-moussiades:2020:overview-chatbot-technology}  \\ % plusieurs types de chatbots
	
	% Information : d'autres types existents.
	\begin{leftBarInformation}
		\cite{adamopoulou-moussiades:2020:overview-chatbot-technology} propose plusieurs types de catégorisation (\textit{périmètre de connaissances restreint ou non, personnalité propre ou non, typologies d'entrées/sorties, interaction avec un humain requise ou non, technologie propriétaire ou non}), mais nous ne nous attardons pas sur ces différentes visions car elles ne sont pas utiles à notre discussion.
	\end{leftBarInformation}
	
	
	%%%
	%%% Subsection B.1.1. Les \textit{chatbots} \textguillemets{\textit{task-oriented}}.
	%%%
	\subsection{Les \textit{chatbots} \textguillemets{\textit{task-oriented}}}
	\label{annex:B.1.1-CHATBOT-CLASSIFICATION-TASK-ORIENTED}
	
		% Définition
		\paragraph{Définition :}
		
			Un \textit{chatbot} \textguillemets{\textit{task-oriented}} est conçu pour \textbf{accomplir une tâche spécifique}.
			Le parcours de dialogue proposé est généralement prédéfini et comporte peu de digressions : l'assistant détermine l'action à réaliser grâce à l'énoncé de l'utilisateur, demande éventuellement des compléments d'informations si la requête n'est pas assez précise, puis effectue sa mission.
			Le périmètre de fonctionnalités est généralement restreint pour contrôler les actions de l'assistant et s'assurer de son comportement.
	
		% Exemples de fonctionnalités.
		\paragraph{Exemples de fonctionnalités :}
		
			\begin{itemize}
				\item Gérer la relation client (\textit{suivi de commande, formulaire de satisfaction, ...}) ;
				\item Accéder à des informations documentaires ou personnelles (\textit{accès à une documentation technique, accès au solde d'un compte bancaire, ...}) ;
				\item Pré-remplir d'un formulaire (\textit{réserver un billet de voyage, payer ses contraventions, faire opposition à sa carte de crédit, ...}) ;
				\item Gérer la domotique (\textit{allume la lumière, joue de la musique, active l'alarme, ...}).
			\end{itemize}
		
		% Exemples d'assistants conversationnels connus.
		\paragraph{Exemples d'assistants conversationnels connus :}
		
			\begin{itemize}
				\item \texttt{Assistant Virtuel SNCF} (\cite{sncf:2018:agent-virtuel-sncf}), pour la réservation de billets de train ;
				\item \texttt{Google Assistant} (\cite{google:2016:google-assistant-your}) ou \texttt{Alexa} (\cite{alexa-internet:2018:keyword-research-competitor}), pour la gestion de la domotique.
			\end{itemize}
	
	
	%%%
	%%% Subsection B.1.2. Les \textit{chatbots} \textguillemets{\textit{chat-oriented}}.
	%%%
	\subsection{Les \textit{chatbots} \textguillemets{\textit{chat-oriented}}}
	\label{annex:B.1.2-CHATBOT-CLASSIFICATION-CHAT-ORIENTED}
	
		% Définition
		\paragraph{Définition :}
		
			Un \textit{chatbot} \textguillemets{\textit{chat-oriented}} se concentre davantage sur les interactions avec l'utilisateur dans le but d'engager la conversation.
			Son objectif principale est de rendre le dialogue agréable.
			Son périmètre de connaissance n'est en général pas restreint pour pouvoir facilement engager la conversation sur n'importe quel sujet.
	
		% Exemples de fonctionnalités.
		\paragraph{Exemples de fonctionnalités :}
		
			\begin{itemize}
				\item Offrir du divertissement (\textit{raconter une histoire ou une blague, organiser un jeu narratif, ...}) ;
				\item Proposer une assistance générale (\textit{donner une définition, demander une explication, ...}) ;
				\item Simplement discuter (\textit{mimer les interactions sociales}).
			\end{itemize}
		
		% Exemples d'assistants conversationnels connus.
		\paragraph{Exemples d'assistants conversationnels connus :}
		
			\begin{itemize}
				\item \texttt{ELIZA} (\cite{weizenbaum:1966:eliza-computer-program}), simulateur d'un entretien clinique en psychothérapie ;
				\item \texttt{ChatGPT} (\cite{openai:2023:chatgpt}) ou \texttt{BARD} (\cite{google:2023:bard-chat-based}), discussion avec des larges modèles de langues (\texttt{LLM}) ;
				\item \texttt{AI Dungeon} (\cite{latitude-inc.-oasis-tech-inc.:2019:ai-dungeon}), un narrateur d'histoires interactives.
			\end{itemize}


%%%%%--------------------------------------------------------------------
%%%%% Annexe B.2: Architectures usuelles des chatbots.
%%%%%--------------------------------------------------------------------
\section{Architectures usuelles des chatbots}
\label{annex:B.2-CHATBOT-ARCHITECTURES}

	On distingue principalement deux approches.
	\cite{chen-etal:2017:survey-dialogue-systems} % génératif vs symbolic

	\paragraph{Approches symboliques}
	\todo[inline]{A REDIGER}
		\cite{schuurmans-frasincar:2020:intent-classification-dialogue}  \\ % definition intention
		\cite{bocklisch-etal:2017:rasa-open-source}  \\ % exemple
		\cite{hoyt-etal:2016:ibm-watson-analytics} % exemple

	\paragraph{Approche génératives}
	\todo[inline]{A REDIGER}
		\cite{uszkoreit:2017:transformer-novel-neural}  \\ % architecture transformers
		\cite{ni-etal:2022:recent-advances-deep}  \\ % avancer en deep learning
		\cite{openai:2023:chatgpt}  \\ % exemple
		\cite{touvron-etal:2023:llama-open-foundation}  \\ % exemple
		\cite{kaddour-etal:2023:challenges-applications-large} % plusieurs challenges

%%%%%--------------------------------------------------------------------
%%%%% Annexe B.3: Discussions sur le niveau d'automatisation.
%%%%%--------------------------------------------------------------------
\section{Discussions sur le niveau d'automatisation}
\label{annex:B.2-CHATBOT-DISCUSSION-AUTOMATISATION}
	
	\todo[inline]{A REDIGER}
		
	\begin{leftBarAuthorOpinion}
		On amalgame approche générative et chatbot \textit{chat-oriented}.
	\end{leftBarAuthorOpinion}
	\cite{sheridan-verplank:1978:human-computer-control} repris par \cite{parasuraman-etal:2000:model-types-levels} \\ % 10 niveaux de contrôles

\todo[inline]{A REDIGER}