\chapter{Étude de la méthode}
    \label{chapter:4-ETUDES}
	
	% RÉSUMÉ DES ÉPISODES PRÉCÉDENTES:
	Dans le chapitre précédent, nous avons présenté une méthode de création d'un jeu de données d'entrainement pour un assistant conversationnel, que nous appelons "\textit{clustering interactif}" :
	%
	\begin{leftBarImportantGreen}
		\begin{todolist}
			% 1. Structure de la méthode.
			\item[\itemok] La méthode proposée repose sur la combinaison entre un regroupement automatique des données par la machine et l'annotation de contraintes binaires par un expert métier pour corriger le regroupement proposé ;
			% 2. Enjeu 1 de la méthode : moins de technique.
			\item[\itemok] Une telle approche devrait limiter les pré-requis techniques actuellement exigés à un expert métier en les déléguant à la machine.
			% 2. Enjeu 2 de la méthode : plus de connaissance métier.
			\item[\itemok] En échange, l'expert se concentre d'avantage sur la transmission de ses connaissances avec une annotation caractérisant la similitude métier entre deux données.
			% 3. Divers.
			\item[\itemok] ...
		\end{todolist}
	\end{leftBarImportantGreen}
	\todo[inline]{divers à compléter (technique ? méthode ? ...).}
	
	% ANNONCE DU BUT DU CHAPITRE: TEST DE LA MÉTHODE.
	Comme nous l'avons détaillé dans le chapitre~\ref{chapter:2-ETAT-DE-L-ART}, des procédés d'annotation similaires existent pour des données facilement visualisables, comme dans le cadre du traitement d'images. Cependant, l'application d'une telle approche dans le cadre de la classification de données textuelles est peu détaillée dans la littérature. Ainsi, dans cette partie, nous étudierons la faisabilité d'un \textit{clustering interactif} pour des données textuelles en explorant les questions suivantes :
	%
	\begin{leftBarImportantRed}
		\begin{todolist}
			% 1. Efficacité.
			\item Peut-on obtenir une base d'apprentissage à l'aide de notre proposition d'implémentation de la méthodologie d'\textit{clustering} interactif ? (cf. hypothèse d'\textbf{efficacité} en section~\ref{section:4.1-HYPOTHESE-EFFICACITE})
			% 2. Efficience.
			\item Peut-on déterminer un paramétrage optimal de cette implémentation pour obtenir plus rapidement une base d'apprentissage ? (cf. hypothèse d'\textbf{efficience} en section~\ref{section:4.2-HYPOTHESE-EFFICIENCE})
			% 3. Coûts.
			\item D'après les données initiales, peut-on approximer l'investissement nécessaire pour obtenir une base d'apprentissage exploitable ? (cf. hypothèse sur les \textbf{coûts} en section~\ref{section:4.3-HYPOTHESE-COUTS})
			% 4. Pertinence.
			\item A un instant donné, peut-on estimer la pertinence métier d'une base d'apprentissage en cours de construction ? (cf. hypothèse de \textbf{pertinence} en section~\ref{section:4.4-HYPOTHESE-PERTINENCE})
			% 5. Impact.
			\item Au cours du processus de construction de la base d'apprentissage, peut-on aisément estimer les potentiels d'une étape de raffinage supplémentaire ? (cf. hypothèse de \textbf{rentabilité} en section~\ref{section:4.5-HYPOTHESE-RENTABILITE})
			% 6. Robustesse.
			\item Peut-on estimer l'influence d'une erreur ou d'une différence d'annotation dans la construction de la base d'apprentissage ? (cf. hypothèse de \textbf{robustesse} en section~\ref{section:4.6-HYPOTHESE-ROBUSTESSE})
		\end{todolist}
	\end{leftBarImportantRed}
	
	% ILLUSTRATION: SCHEMA DES HYPOTHESES
	Afin d'illustrer ces interrogations, nous vous proposons de considérer de la figure~\ref{figure:HYPOTHESE-00-DEFAULT}. Dans les sections suivantes, cette figure évoluera pour résumer les études réalisées.
	%
	\begin{figure}[!htb]
		\centering
		\includegraphics[width=0.8\textwidth]{figures/hypotheses-00-default}
		\caption{Illustration des études réalisées sur le \textit{clustering} interactif (\textit{étape 0/6}) en schématisant l'évolution de la performance (\textit{accord avec la vérité terrain calculé en v-measure}) d'une base d'apprentissage en cours de construction en fonction du nombre d'itérations de la méthode (\textit{nombre d'annotations par un expert métier}).}
		\label{figure:HYPOTHESE-00-DEFAULT}
	\end{figure}
	
	% PRÉAMBULE TECHNIQUE : CPU + scrips + datasets.
	\begin{leftBarInformation}
		Pour ces études, l'exécution des différentes expériences a été réalisée sur des CPU \textit{Intel(R) Xeon(R) CPU E5-2660 v4 \@ 2.00GHz} et parallélisé avec la librairie Python \textit{multiprocessing} (un worker par CPU).
		Les scripts d'exécution et d'analyse de ces expériences, rédigés au sein de notebooks Python et/ou R, sont disponibles dans~\cite{schild:cognitivefactory-interactive-clustering-comparative-study:2021}.
		Enfin, les jeux de données utilisés pour ces études sont détaillés en Annexe~\ref{annex:C-ANNEXE-DATASET}.
	\end{leftBarInformation}
	
	% TABLE DES MATIÈRES DU CHAPITRE
    \minitoc

    %%%%%--------------------------------------------------------------------
    %%%%% Section 4.1: Hypothèse d'efficacité.
    %%%%%--------------------------------------------------------------------
	\newpage
	\section{Évaluation de l'hypothèse d'efficacité}
\label{section:4.1-HYPOTHESE-EFFICACITE}
%  : « \textit{est-ce que la méthode permet d'annoter un jeu de données ?} »
	
	%%% Formulation des hypothèses:
	Nous aimerions vérifier l'hypothèse suivante :

	\begin{tcolorbox}[
		title=\faVial~\textbf{Hypothèse d'efficacité}~\faVial,
		colback=colorTcolorboxHypothesis!15,
		colframe=colorTcolorboxHypothesis!75,
		width=\linewidth
	]
		% Hypothèse.
		« \textbf{
			Une méthodologie d'annotation basée sur le \textit{clustering} interactif permet d'obtenir une base d'apprentissage pour un assistant conversationnel qui respecte la vision donnée par l'expert métier au cours de l'annotation.
		} » \\
		
		% Résumé de l'étude.
		Afin de vérifier cette hypothèse, nous mettrons en place une expérience de ré-annotation d'une base d'apprentissage (qui servira ici de vérité terrain) à l'aide de notre méthode, en simulant l'annotation d'un expert, et nous critiquerons l'évolution de la nouvelle base d'apprentissage obtenue et sa similitude avec la base d'apprentissage initiale.
		
		% Figure.
		La figure~\ref{figure:4.1-HYPOTHESE-EFFICACITE} illustre cette hypothèse et l'espoir de convergence d'une base d'apprentissage en cours de construction vers sa vérité terrain.
		%
		\begin{figure}[H]  % keep [H] to be in the tcolorbox.
			\centering
			\includegraphics[width=0.8\textwidth]{figures/hypotheses-01-efficacite}
			\caption{Illustration des études réalisées sur le \textit{clustering} interactif (\textit{étape 1/6}) en schématisant l'évolution de la performance (\textit{accord avec la vérité terrain calculé en v-measure}) d'une base d'apprentissage en cours de construction en fonction du nombre d'itérations de la méthode (\textit{nombre d'annotations par un expert métier}).}
			\label{figure:4.1-HYPOTHESE-EFFICACITE}
		\end{figure}

	\end{tcolorbox}
	
	%%%
	%%% Subsection 4.1.1: Étude de convergence vers une vérité terrain pré-établie en simulant l'annotation d'une base d'apprentissage et mesurant la vitesse de sa création
	%%%
	\subsection{Étude de convergence vers une vérité terrain pré-établie en simulant l'annotation d'une base d'apprentissage et mesurant la vitesse de sa création}
	\label{section:4.1.1-ETUDE-CONVERGENCE}
			
		% Référence articles.
		\begin{leftBarInformation}
			Cette étude a été l'objet d'une présentation à la conférence \texttt{EGC (Extraction et Gestion des Connaissances)}~\citep{schild:conception-interactive-clustering:2021}, et d'une extension dans le journal \texttt{IJDWM (International Journal of Data Warehousing and Mining)}~\citep{schild:extension-interactive-clustering:2022}.
			\footnote{Les résultats et la discussion de ces articles ont été mis à jour et réécrits pour mieux s'intégrer au discours ce manuscrit.}
		\end{leftBarInformation}

		%%% Protocole expérimental.
		\subsubsection{Protocole expérimental}
		
			% Objectif de l'expérience.
			Nous voulons vérifier qu'une méthodologie d'annotation basée sur notre implémentation du \textit{clustering} interactif permet de créer une base d'apprentissage pour un assistant conversationnel.
			Pour cela, nous prenons une base d'apprentissage employée pour entraîner un modèle de classification de textes\todo{référence, lien vers ANNEXE, + description conditions de création du JDD}, et nous utilisons ce jeu de données comme vérité terrain.
			L'objectif de cette expérience est de simuler la création de cette base d'apprentissage et de nous assurer que le résultat obtenu correspond à la vérité terrain.
			
			% Axiome.
			\begin{leftBarWarning}
				Dans le cadre de cette étude, nous supposons que l'expert métier connaît parfaitement le domaine traité dans ce jeu de données, et qu'il est capable de caractériser sans ambiguïté la similitude entre deux données issues de cet ensemble.
			\end{leftBarWarning}
			
			% Pseudo-code.
			Pour résumer le protocole expérimental que nous décrivons ci-dessous, vous pouvez vous référer au pseudo-code décrit dans Alg.~\ref{algorithm:4.1.1-ETUDE-CONVERGENCE-PROTOCOLE}.
			%
			\begin{algorithm}[!htb]
				\begin{algorithmic}[1]
					\Require jeu de données annoté (vérité terrain)
					\ForAll{arrangement d'algorithmes et de paramètres à tester}
						\State \textbf{initialisation (données)}: récupérer les données et la vérité terrain
						\State \textbf{initialisation (contraintes)}: créer une liste vide de contraintes
						\State \textbf{prétraitement}: supprimer le bruit dans les données
						\State \textbf{vectorisation}: transformer les données en vecteurs
						\State \textbf{clustering initial}: regrouper les données par similarité des vecteurs
						\State \textbf{évaluation}: estimer l'équivalence entre le clustering obtenu et la vérité terrain
						\Repeat
							\State \textbf{échantillonnage}: sélectionner de nouvelles contraintes à annoter
							\State \textbf{simulation d'annotation}: ajouter des contraintes en utilisant la vérité terrain
							\State \textbf{clustering}: regrouper les données par similarité avec les contraintes
							\State \textbf{évaluation}: estimer l'équivalence entre le clustering obtenu et la vérité terrain
						\Until{annotation de toutes les contraintes possibles}
						\State \textbf{évaluation finale}: espérer avoir un score d'équivalence de $100$\% entre le clustering obtenu et la vérité terrain
					\EndFor
					\Ensure arrangements d'algorithmes et de paramètres ayant un score d'équivalence de $100$\%
				\end{algorithmic}
				\caption{Description en pseudo-code du protocole expérimental de l'étude de convergence du \textit{clustering} interactif vers une vérité terrain pré-établie.}
				\label{algorithm:4.1.1-ETUDE-CONVERGENCE-PROTOCOLE}
			\end{algorithm}
			
			% Détails de l'expérience.
			\todo[inline]{description JDD}
			Lors de cette expérience, chaque tentative de la méthode commencera sur la version non labellisée de la vérité terrain à disposition, sans aucune contrainte connue à l'avance.
			Au fur et à mesure des itérations de la méthode, nous simulerons l'annotation de l'expert métier en comparant les labels de la vérité terrain : ainsi, deux données ont une contrainte \texttt{MUST-LINK} si elles ont le même label, et une contrainte \texttt{CANNOT-LINK} sinon.
			Cela traduit le prérequis d'avoir un annotateur qui soit capable, dans son domaine d'expertise, de différencier deux données selon leur ressemblance.
			Une tentative de l'application de notre méthode s'arrête lorsque toutes les contraintes possibles entre les données ont été annotés par l'expert.

			% Description implémentation de l'interactive clustering.
			Pour cette étude, nous essayons une tentative pour chaque combinaison de paramètre de notre implémentation du clustering interactif (cf. section~\ref{section:3.3-DESCRIPTION-IMPLEMENTATION}). Cela comprend les tâches et leurs paramètres respectifs suivants :
			%
			\begin{enumerate}
				\item le \textbf{prétraitement} des données, avec les niveaux suivants : \textbf{absent} (noté \texttt{prep.no}), \textbf{simple} (noté \texttt{prep.simple}), \textbf{avec lemmatisation} (noté \texttt{prep.lemma}) et \textbf{avec filtres} (noté \texttt{prep.filter}) ;
				\item la \textbf{vectorisation} des données, avec les niveaux suivants : \textbf{TF-IDF} (noté \texttt{vect.tfidf}) et \textbf{SpaCy} (noté \texttt{vect.frcorenewsmd}) ;
				\item le \textbf{clustering sous contraintes} des données, avec les niveaux suivants : \textbf{KMeans} (modèle \textit{COP} noté \texttt{clust.kmeans.cop}), \textbf{Hiérarchique} (lien \textit{single} noté \texttt{clust.hier.sing} ; lien \textit{complete} noté \texttt{clust.hier.comp} ; lien \textit{average} noté \texttt{clust.hier.avg} ; lien \textit{ward} noté \texttt{clust.hier.ward}) et \textbf{Spectral} (modèle \textit{SPEC} noté \texttt{clust.spec}). Le choix du nombre de clusters n'est pas étudié ici, et ce nombre est fixé au nombre de classes présentes dans la vérité terrain ;
				\item l'\textbf{échantillonnage} des contraintes à annoter, avec les niveaux suivants : \textbf{purement aléatoire} (noté \texttt{samp.random.full}), \textbf{pseudo-aléatoire} (noté \texttt{samp.random.same}), \textbf{même cluster et étant les plus éloignées} (noté \texttt{samp.farhtest.same}) et \textbf{clusters différents et étant les plus proches} (noté \texttt{samp.closest.diff}). Le choix de la taille d'échantillon n'est pas étudié ici, et cette taille est arbitrairement fixé à $50$.
			\end{enumerate}
			
			Il y a donc $192$ combinaisons testées, et chaque tentative est répétée $5$ fois pour contrer les aléas statistiques de certains algorithmes.
			Pour plus de détails sur ces algorithmes, référez-vous à la section~\ref{section:3.3-DESCRIPTION-IMPLEMENTATION} pour avoir accès à leur description, à leurs paramètres et aux choix d'implémentation.
			
			% Description de l'évaluation.
			Pour évaluer l'équivalence entre la vérité terrain et notre segmentation des données obtenue au cours de la méthode, nous nous intéresserons à l'évolution de la \texttt{v-measure} entre ces deux jeu de données.
			Si le score du calcul de la \texttt{v-measure} est de $100$\%, cela signifierait que le clustering final et la vérité terrain propose une segmentation identique des données, donc que la vérité terrain a pu être retrouvée, et donc qu'il est possible d'obtenir une base d'apprentissage pour un assistant conversationnel à l'aide d'une méthodologie d'annotation basée sur le \textit{clustering} interactif.
			
			% Référence scripts.
			\begin{leftBarInformation}
				Les scripts de l'expérience (\textit{notebooks} Python) sont disponibles dans un dossier dédié de~\cite{schild:cognitivefactory-interactive-clustering-comparative-study:2021}.
			\end{leftBarInformation}

		%%% Résultats.
		\subsubsection{Résultats obtenus}
			
			% Graphe d'évolution de la v-measure moyenne, min et max.
			La figure~\ref{figure:4.1.1-ETUDE-CONVERGENCE-EVOLUTION} et le tableau~\ref{table:4.1.1-ETUDE-CONVERGENCE-EVOLUTION} représentent l'évolution moyenne de la \texttt{v-measure} du clustering en fonction du nombre d'itération de la méthode. Les tentatives les plus rapides et les plus lentes sont représentées sur la figure.
							
			% Tendance: Forte dispersion, Croissance générale.
			Malgré une forte dispersion des résultats (écart-type de \texttt{v-measure} pouvant être supérieur à $20$\%, forte différence entre les tentatives la plus rapide et la plus lente) et quelques sauts de performances (cf. à-coups de la tentative la plus lente sur la figure), une convergence générale vers la vérité terrain peut être constatée.
			
			% Tendance à courts termes: Croissance linéaire
			A l'itération $0$, une tentative commence avec une moyenne de $19.05$\% de \texttt{v-measure}  entre son \textit{clustering} initial (sans contraintes) et la vérité terrain.
			Cette \texttt{v-measure} moyenne croît presque linéairement (pente de $0.97$) jusqu'à l'itération $75$ où elle atteint la performance de $92.08$\% (cf. tableau~\ref{table:4.1.1-ETUDE-CONVERGENCE-EVOLUTION}).

			% Tendance à longs termes: Asymptote.
			Au delà de l'itération $75$, la courbe de la \texttt{v-measure} moyenne tend vers une asymptote de $100$\% (cf. figure~\ref{figure:4.1.1-ETUDE-CONVERGENCE-EVOLUTION}).
			Cette asymptote est atteinte par toute les $960$ tentatives ($192$ combinaisons de paramètres, $5$ tentatives pour chaque combinaison), la tentative l'ayant atteinte le plus tôt à l'itération $19$ et celle le plus tard à l'itération $326$.
			La courbe se prolonge jusqu'à l'itération $394$ pour que toutes les tentatives puisse annoter toutes les contraintes possibles sur le jeu de données.
			
			%
			\begin{figure}[!htb]
				\centering
				\includegraphics[width=0.8\textwidth]{figures/etude-efficacite-evolution-moyenne-0par-iteration}
				\caption{Évolution de la moyenne de la \texttt{v-measure} entre un résultat obtenu et la vérité terrain en fonction du nombre d'itération de la méthode de \textit{clustering} interactif, moyenne réalisée itération par itération sur l'ensemble des tentatives.
				Représentation des tentatives ayant été les plus rapides (\textit{un prétraitement \texttt{prep.simple}, une vectorisation \texttt{vect.tfidf}, un clustering \texttt{clust.hier.comp} ou \texttt{clust.hier.ward}, et un échantillonnage \texttt{samp.closest.diff}}) et les plus lentes (\textit{un prétraitement \texttt{prep.no}, une vectorisation \texttt{vect.tfidf}, un clustering \texttt{clust.spec}, et un échantillonnage de contraintes \texttt{samp.farthest.same}}) pour atteindre $100$\% de \texttt{v-measure}.}
				\label{figure:4.1.1-ETUDE-CONVERGENCE-EVOLUTION}
			\end{figure}
			%
			\begin{table}[!htb]
				\begin{center}
				\begin{tabular}{|c|r|r|r|r|r|}
					\hline
					% ENTETE DU TABLEAU
					\multicolumn{2}{|c|}{ \shortstack{ Annotations } }
						& \multicolumn{4}{c|}{ \shortstack{ Performances (\texttt{v-measure}) } }
						\tabularnewline
						\hline
					\multicolumn{1}{|c|}{ \shortstack{ Itérations } }
						& \multicolumn{1}{c|}{ \shortstack{ Contraintes } }
						& \multicolumn{1}{c|}{ \shortstack{ Moyenne } }
						& \multicolumn{1}{c|}{ \shortstack{ Écart-type } }
						& \multicolumn{1}{c|}{ \shortstack{ Minimum } }
						& \multicolumn{1}{c|}{ \shortstack{ Maximum } }
						\tabularnewline
						\hline
					%
					$0$		& $0$		& $19.05$\% \footnotesize $(\pm0.43)$ \par	& $13.38$\% & $03.42$\% & $47.75$\%
					\tabularnewline
					\hline
					%
					$25$	& $1~250$	& $49.09$\% \footnotesize $(\pm0.82)$ \par	& $25.43$\% & $09.09$\% & $100.00$\%
					\tabularnewline
					\hline
					%
					$50$	& $2~500$	& $73.66$\% \footnotesize $(\pm0.77)$ \par	& $23.98$\% & $16.78$\% & $100.00$\%
					\tabularnewline
					\hline
					%
					$75$	& $3~750$	& $92.08$\% \footnotesize $(\pm0.54)$ \par	& $16.70$\% & $21.74$\% & $100.00$\%
					\tabularnewline
					\hline
					%
					$100$	& $5~000$	& $95.19$\% \footnotesize $(\pm0.41)$ \par	& $12.67$\% & $26.93$\% & $100.00$\%
					\tabularnewline
					\hline
					%
					$125$	& $6~250$	& $97.43$\% \footnotesize $(\pm0.29)$ \par	& $09.09$\% & $34.99$\% & $100.00$\%
					\tabularnewline
					\hline
					%
					$150$	& $7~500$	& $98.73$\% \footnotesize $(\pm0.23)$ \par	& $07.22$\% & $38.14$\% & $100.00$\%
					\tabularnewline
					\hline
					
				\end{tabular}
				\end{center}
				\caption{Détails de l'évolution de la moyenne de la \texttt{v-measure} entre un résultat obtenu et la vérité terrain en fonction du nombre d'itération de la méthode de \textit{clustering} interactif, moyenne réalisée itération par itération sur l'ensemble des tentatives.}
				\label{table:4.1.1-ETUDE-CONVERGENCE-EVOLUTION}
			\end{table}

		%%% Discussion
		\subsubsection{Discussion}
			
			%%% Principale conclusion : il y a convergence !
			La première et principale conclusion de cette étude concerne la preuve que la méthode est efficace.
			En effet, les différentes simulations ont bien convergé vers la vérité terrain (atteinte de l'asymptote à $100$\% de \texttt{v-measure}), montrant qu'il est possible pour un expert métier de créer une base d'apprentissage à l'aide d'une méthodologie d'annotation basée sur le \textit{clustering} interactif. \\
			
			
			%%% Avantages.
			Cette découverte permet de confirmer plusieurs espoirs portés sur la méthode. 
			
			% Avantage 1 : Émergence d'une modélisation sur la base des contraintes
			Tout d'abord, la vérité terrain a été retrouvée sans formaliser concrètement la structure de données.
			Là où une annotation par label aurait requis au préalable une définition des catégories possibles pour les données à étiqueter, la méthodologie employant le \textit{clustering} interactif a permis de faire émerger naturellement cette structure de données.
			Cette émergence provient directement des contraintes annotées par l'expert métier, traduisant ainsi ses connaissances à l'aide d'instructions simples : \textit{les données sont-elles ou non similaires ?}
			
			% Avantage 2 : annotations plus simples et plus concrètes
			De plus, ces contraintes ont été l'objet d'une annotation guidée par les besoins de la machine afin de s'améliorer d'itération en itération (voir la croissance globale de la \texttt{v-measure} sur la figure~\ref{figure:4.1.1-ETUDE-CONVERGENCE-EVOLUTION}).
			Ainsi, l'expert métier corrige la base d'apprentissage à chaque itération : soit en affinant les clusters en cours de construction, améliorant ainsi la cohérence des clusters (cf. pentes croissantes) ; soit en remaniant les clusters mal formés pour repartir sur de bonnes bases, détériorant la cohérence des clusters le temps de la réorganisation (cf. oscillations ou pentes décroissantes). \\

			
			%%% Limites.
			Néanmoins, différentes pistes sont encore à explorer pour rendre le \textit{clustering} interactif utilisable en situation réelle.
			
			% Limite 1 : Nombre d'annotations ==> besoin d'optimisation.
			D'une part, nous échangeons le besoin de définir une structure de données contre la nécessité d'annoter un grand nombre de contraintes : pour $500$ points de données, et en considérant que l'asymptote à $100$\% est atteinte en moyenne autour de l'itération $200$, il faudrait $10~000$ annotations de contraintes pour être exhaustif, ce qui correspond à près de $20$ fois plus de contraintes que de données.
			Bien que l'annotation binaire demande a priori une charge mentale plus faible à un annotateur, un tel volume représente tout de même une grande quantité de travail.
			\todo{
				Commentaire Gautier 22/05/2023 :
				(A DÉTAILLER AILLEURS ?)
				Oui, complètement d'accord ici, mais en fait ça va plus loin que ça non ?
				Déjà, on a une quantité de ressources allouées à la tâche en effet plus fabile (car choisir entre "similaire" et "non similaire" est clairement plus simple que d'assigner un label parmi N).
				Mais on a aussi une diminution des ressources allouées au maintien d'une stratégie d'annotation : en effet, pas besoin de définir à l'avance de type system ou autre, tout est construit à la volée. Ce deuxième point est particulierement intéressant à discuter je pense, car on sait normalement que le maintien d'objectifs en mémoire de travail peut aider à maintenir un niveau d'engagement sur une tâche cognitive. Du coup, ça pose d'autant plus la question de l'expérience utilisateur : annoter avec un CI sera-t-il moins engageant qu'annoter avec une méthode classique ?
			}
			Cela peut décourager les experts métiers en début de projet, surtout pour des projets ayant des jeux de données de plus grandes tailles.
			Toutefois, les résultats obtenus montrent une forte dispersion du nombre d'itérations nécessaire, et certaines tentatives ont été bien plus efficientes dans l'utilisation de leurs contraintes. La tentative la plus rapide a convergé à l'itération $19$, soit $950$ contraintes, ce qui est un volume d'annotation bien plus abordable !
			On peut donc espérer trouver un paramétrage optimal de la méthode permettant de diminuer significativement le nombre moyen de contraintes nécessaires afin d'obtenir une base d'apprentissage exploitable avec un volume d'annotations acceptable.
			Cet aspect fait l'objet de l'étude décrite dans la section~\ref{section:4.2-HYPOTHESE-EFFICIENCE} (hypothèse d'efficience).
			
			% Limite 2 : Exhaustivité des annotations ==> evaluation de la rentabilité.
			D'autre part, le choix d'annoter toutes les contraintes possibles sur les données (\textbf{annotation exhaustive}) n'est pas forcément judicieux.
			En effet, si nous nous référons à la figure~\ref{figure:4.1.1-ETUDE-CONVERGENCE-EVOLUTION}), une moyenne de $90$\% de \texttt{v-measure} est déjà atteinte autour de l'itération $75$, alors que l'asymptote à $100$\% n'est atteinte qu'au delà de l'itération $200$. Afin d'être plus efficient, il faudrait envisager une \textbf{annotation partielle} permettant d'obtenir rapidement $90$\% de \texttt{v-measure} (quitte à affiner le résultat manuellement pour combler la "perte" moyenne de $10$\% de \texttt{v-measure}).
			Cet aspect sera ajouté à l'objectif de l'étude décrite dans la section~\ref{section:4.2-HYPOTHESE-EFFICIENCE} (hypothèse d'efficience).
			
			% Limite 3 : Expert métier parfait ==> simuler les erreurs.
			Pour finir, nous avons supposé dans cette étude que l'annotateur est un expert métier connaissant parfaitement le domaine traité.
			Cette hypothèse forte n'est a priori pas valable en situation réelle : En effet, des erreurs d'annotations peuvent intervenir (ambiguïtés sur les données, méconnaissance du domaine, erreurs d'inattention, différence d'opinions entre annotateurs, ...), ce qui peut entraîner des divergences ou des incohérences dans la construction de la base d'apprentissage.
			Il semble donc nécessaire d'étudier les impacts de ces incohérences, ainsi que de proposer une méthode pour les prévenir ou les corriger.
			Cet aspect sera traité à la fin de ce chapitre dans la section~\ref{section:4.6-HYPOTHESE-ROBUSTESSE} (hypothèse de robustesse).


    %%%%%--------------------------------------------------------------------
    %%%%% Section 4.2: Hypothèse d'efficience.
    %%%%%--------------------------------------------------------------------
	\newpage
	\section{Évaluation de l'hypothèse d'efficience}
\label{section:4.2-HYPOTHESE-EFFICIENCE}
% « \textit{quels sont les paramètres optimaux pour minimiser la charge de travail de l'annotateur ?} »

	%%% Formulation des hypothèses:
	Suite à la validation de l'hypothèse d'efficacité (convergence de la méthode, cf. section~\ref{section:4.1-HYPOTHESE-EFFICACITE}), nous aimerions vérifier l'hypothèse suivante :

	\begin{tcolorbox}[
		title=\faVial~\textbf{Hypothèse d'efficience}~\faVial,
		colback=colorTcolorboxHypothesis!15,
		colframe=colorTcolorboxHypothesis!75,
		width=\linewidth
	]

		% Hypothèse.
		« \textbf{
			La vitesse de convergence du \textit{clustering} interactif peut être optimisée en ajustant différents paramètres afin de minimiser la charge de travail de l'opérateur. Nous étudierons en particulier l'influence du prétraitement des données, de la vectorisation des données, de l'échantillonnage des contraintes à annoter et du \textit{clustering} sous contraintes.
		} » \\
		
		% Résumé de l'étude.
		Afin de vérifier cette hypothèse, nous mettrons en place une expérience de ré-annotation d'une base d'apprentissage (qui servira ici de vérité terrain) à l'aide de notre méthode, en simulant l'annotation d'un expert, et nous réaliserons l'analyse statistique de la taille d'effet de différents paramètres sur la vitesse de convergence du \textit{clustering} itératif.
		
		% Figure.
		La figure~\ref{figure:4.2-HYPOTHESE-EFFICIENCE} illustre cette hypothèse et l'espoir d'une convergence "optimale" d'une base d'apprentissage en cours de construction vers sa vérité terrain.
		%
		\begin{figure}[H]  % keep [H] to be in the tcolorbox.
			\centering
			\includegraphics[width=0.8\textwidth]{figures/hypotheses-02-efficience}
			\caption{Illustration des études réalisées sur le \textit{clustering} interactif (\textit{étape 2/6}) en schématisant l'évolution de la performance (\textit{accord avec la vérité terrain calculé en v-measure}) d'une base d'apprentissage en cours de construction en fonction du nombre d'itérations de la méthode (\textit{nombre d'annotations par un expert métier}).}
			\label{figure:4.2-HYPOTHESE-EFFICIENCE}
		\end{figure}

	\end{tcolorbox}
	
	%%%
	%%% Subsection 4.2.1: Étude d'optimisation des paramètres de convergence.
	%%%
	\subsection{Étude d'optimisation des paramètres de convergence}
	\label{subsection:4.2.1-ETUDE-OPTIMISATION}
			
		% Référence articles.
		\begin{leftBarInformation}
			Cette étude a été l'objet d'une présentation à la conférence \texttt{EGC (Extraction et Gestion des Connaissances)}~\citep{schild:conception-interactive-clustering:2021}, et d'une extension dans le journal \texttt{IJDWM (International Journal of Data Warehousing and Mining)}~\citep{schild:extension-interactive-clustering:2022}.
			\footnote{Les résultats et la discussion ont été mis à jour et réécrits pour mieux s'intégrer au discours ce manuscrit.}
		\end{leftBarInformation}

		%%% Protocole expérimental.
		\subsubsection{Protocole expérimental : analyser la taille d'effet des paramètres d’implémentation sur la vitesse de création d'une base d'apprentissage}

			% Objectif de l'expérience.
			Nous voulons étudier l'influence des paramètres de notre implémentation du \textit{clustering} interactif sur la vitesse de création d'une base d'apprentissage pour un assistant conversationnel.
			Nous allons donc compléter le protocole expérimental de l'étude de convergence en section~\ref{subsection:4.1.1-ETUDE-CONVERGENCE} visant à simuler la création d'une base d'apprentissage\todo{référence, lien vers ANNEXE}.
			
			% Pseudo-code.
			Pour résumer le protocole expérimental que nous décrivons c-dessous, vous pouvez vous référer au pseudo-code décrit dans Alg.~\ref{algorithm:4.2.1-ETUDE-OPTIMISATION-PROTOCOLE}.
			%
			\begin{algorithm}[!htb]
				\begin{algorithmic}[1]
					\Require jeu de données annoté (vérité terrain)
					\ForAll{arrangement d'algorithmes et de paramètres à tester}
						\State \textbf{initialisation}: récupérer les données de la vérité terrain sans leur label, créer une liste vide de contraintes
						\State \textbf{prétraitement}: supprimer le bruit dans les données
						\State \textbf{vectorisation}: transformer les données en vecteurs
						\State \textbf{clustering initial}: regrouper les données par similarité
						\State \textbf{évaluation}: estimer l'équivalence entre le clustering obtenu et la vérité terrain
						\Repeat
							\State \textbf{échantillonnage}: sélectionner de nouvelles contraintes à annoter
							\State \textbf{simulation d'annotation}: ajouter des contraintes grâce à la comparaison des labels de la vérité terrain
							\State \textbf{clustering}: regrouper les données par similarité avec les contraintes
							\State \textbf{évaluation}: estimer l'équivalence entre le clustering obtenu et la vérité terrain
						\Until{annotation de toutes les contraintes possibles}
					\EndFor
					\State \textbf{analyse}: déterminer les tailles d'effets des algorithmes et paramètres
					\Ensure meilleurs arrangements d'algorithmes et de paramètres
				\end{algorithmic}
				\caption{Description en pseudo-code du protocole expérimental de l'étude d'optimisation de la convergence du \textit{clustering} interactif vers une vérité terrain pré-établie.}
				\label{algorithm:4.2.1-ETUDE-OPTIMISATION-PROTOCOLE}
			\end{algorithm}
			
			% Détails de l'expérience.
			En s'appuyant sur les résultats précédemment obtenus, nous allons analyser l'influence des différentes tâches employées (\textbf{prétraitement}, \textbf{vectorisation}, \textbf{clustering sous contraintes}, \textbf{échantillonnage}) et de leurs paramètres sur la vitesse de convergence vers la vérité terrain.
			% Description implémentation de l'interactive clustering.
			Nous avons toujours $192$ combinaisons testées, et chaque tentative est répétée $5$ fois pour contrer les aléas statistiques de certains algorithmes.
			Pour plus de détails sur ces algorithmes, référez-vous à la section~\ref{section:3.3-DESCRIPTION-IMPLEMENTATION}.
			
			% Description de l'évaluation et Seuils d'évaluation.
			Comme lors de l'étude sur la convergence de la méthode, nous nous intéresserons à l'évolution de la \texttt{v-measure} entre la vérité terrain et notre segmentation des données obtenue, et nous affinerons notre évaluation en portant attention aux trois seuils d'annotations suivants :
			\begin{enumerate}
				\item le cas d'une \textbf{annotation partielle}, correspondant au nombre d'itérations nécessaires à la méthode pour avoir $90$\% de \texttt{v-measure} entre le résultat obtenu et la vérité terrain, c'est-à-dire un état de semi-parcours vers une convergence totale\footnote{Le seuil de $90$\% a été choisi au cours de l'étude de convergence (cf. hypothèse d'efficacité, section~\ref{section:4.1-HYPOTHESE-EFFICACITE}).} ;
				\item le cas d'une \textbf{annotation suffisante}, correspondant au nombre d'itérations nécessaires à la méthode pour avoir $100$\% de \texttt{v-measure} entre le résultat obtenu et la vérité terrain, c'est-à-dire avoir suffisamment de contraintes annotées par l'expert métier pour retrouver la vérité terrain ;
				\item le cas d'une \textbf{annotation exhaustive}, correspondant au nombre d'itérations nécessaires à la méthode pour parcourir toutes les contraintes possibles sur les données, et ainsi retranscrire exhaustivement la vision de l'expert métier.
			\end{enumerate}
			
			% Description de l'analyse ANOVA.
			Enfin, nous utiliserons une \texttt{ANOVA} à mesures répétées afin de déterminer l’effet des paramètres de notre implémentation sur le nombre d’annotations requis pour converger vers la vérité terrain. Ces analyses sont réalisées à l'aide du logiciel R\todo{citation}, et le test de \texttt{Tukey (HSD)} est utilisé pour les comparaisons post-hoc.
			
			% Référence scripts.
			\begin{leftBarInformation}
				Les scripts de l'expérience (\textit{notebooks} Python) sont disponibles dans un dossier dédié de~\cite{schild:cognitivefactory-interactive-clustering-comparative-study:2021}.
			\end{leftBarInformation}

		%%% Résultats
		\subsubsection{Résultats obtenus}
		
			%%% Analyse d'une annotation partielle.
			Pour obtenir une \textbf{annotation partielle} (\textit{atteindre une \texttt{v-measure} de $90$\%}), la moyenne des itérations est de $59.04$ (min: $11$, max: $315$, écart-type: $42.14$), soit une moyenne de $2~951.81$ annotations (min: $550$, max: $15~750$, écart-type: $2~106.72$).
			La figure~\ref{figure:4.2.1-ETUDE-OPTIMISATION-HISTOGRAMME-ANNOTATION-PARTIELLE} représente la répartition de ces itérations au cours des différentes tentatives.
			On peut noter les deux cas intéressants suivants :
			%
			\begin{itemize}
				\item[$\bullet$] Les tentatives les plus rapides furent celles avec un prétraitement des données \texttt{prep.no} ou \texttt{prep.simple} ou \texttt{prep.lemma}, une vectorisation des données \texttt{vect.tfidf}, un clustering sous contraintes \texttt{clust.hier.sing}, et un échantillonnage de contraintes \texttt{samp.closest.diff}. Ces tentatives ont requis $11$ itérations, soit $550$ annotations, dont $299$ (respectivement $304$ et $281$) contraintes \texttt{MUST-LINK}.
				\item[$\bullet$] Les tentatives les plus lentes furent celles avec un prétraitement des données \texttt{prep.no}, une vectorisation des données \texttt{vect.tfidf}, un clustering sous contraintes \texttt{clust.spec}, et un échantillonnage de contraintes \texttt{samp.farthest.same}. Ces tentatives ont requis $315$ itérations, soit $15~750$ annotations, dont $1~032$ contraintes \texttt{MUST-LINK}.
			\end{itemize}
			%
			\begin{figure}[!htb]
				\centering
				\includegraphics[width=0.7\textwidth]{figures/etude-efficience-histogramme-annotation-partielle}
				\caption{Répartition des tentatives en fonction de l'itération de la méthode à laquelle elles atteignent le seuil d'une annotation partielle, c'est-à-dire l'itération à laquelle elles parviennent à $90$\% de \texttt{v-measure} entre un résultat obtenu et la vérité terrain. L'histogramme est réduit à $60$ pics pour simplifier l'affichage.}
				\label{figure:4.2.1-ETUDE-OPTIMISATION-HISTOGRAMME-ANNOTATION-PARTIELLE}
			\end{figure}
			%
			Le tableau~\ref{table:4.2.1-ETUDE-OPTIMISATION-ANOVA-ANNOTATION-PARTIELLE} retranscrit l'influence de chacun des paramètres sur le nombre d'itérations nécessaires pour atteindre une \textbf{annotation partielle} (\textit{atteindre une \texttt{v-measure} de $90$\%}).
			Les analyses de variance mettent en relief l'effet significatif sur cette convergence du prétraitement (\texttt{eta-carré}: $0.320$, \texttt{p-valeur}: $< 10^{-3}$), de la vectorisation (\texttt{eta-carré}: $0.388$, \texttt{p-valeur}: $< 10^{-3}$), du clustering (\texttt{eta-carré}: $0.866$, \texttt{p-valeur}: $< 10^{-3}$) et de l'échantillonnage (\texttt{eta-carré}: $0.968$, \texttt{p-valeur}: $< 10^{-3}$).
			L'analyse post-hoc de ces effets indique que le meilleur paramétrage moyen pour atteindre une \textbf{annotation partielle} repose sur la prétraitement \texttt{prep.simple}, le vectorisation \texttt{vect.tfidf}, le clustering \texttt{clust.hier.avg}, et l'échantillonnage \texttt{samp.closest.diff}. La moyenne du nombre d'itération requis pour ce paramétrage est de $19.00$ (écart-type: $0.79$), soit $950$ annotations (écart-type: $39.34$).
			%
			\begin{table}[!htb]
				\begin{center}
				\begin{tabular}{|c|c|c|c|c|c|c|}
					\hline
					% ENTETE DU TABLEAU
					\multicolumn{2}{|c|}{ \shortstack{Description des \\ facteurs analysés } }
						& \multicolumn{3}{c|}{ \shortstack{ Description \\ statistique } }
						& \multicolumn{2}{c|}{ \shortstack{ Description des \\ tailles d'effets } }
						\tabularnewline
						\hline

					Facteur
						& Niveau 
						& Moyenne
						& Rang
						& SE
						& \texttt{ $ \eta^{2} $ }
						& \texttt{p-valeur}
						\tabularnewline
						\hline
					
					% PRETRAITEMENT
					\multirow{4}{*}{prétraitement}
						& \texttt{prep.simple}
						& $61.90$
						& (1)
						& \multirow{4}{*}{ $0.32$ }
						& \multirow{4}{*}{ $0.320$ }
						& \multirow{4}{*}{ \shortstack{ $< 10^{-3}$ \\ ($***$) } }
						\tabularnewline
						\cline{2-4}
						
						& \texttt{prep.lemma}
						& $63.08$
						& (2)
						&
						&
						&
						\tabularnewline
						\cline{2-4}
						
						& \texttt{prep.no}
						& $63.70$
						& (2)
						&
						& 
						&
						\tabularnewline
						\cline{2-4}
						
						& \texttt{prep.filter}
						& $71.90$
						& (4)
						&
						&
						&
						\tabularnewline
						\hline
					
					% VECTORISATION
					\multirow{2}{*}{vectorisation}
						& \texttt{vect.tfidf}
						& $60.61$
						& (1)
						& \multirow{2}{*}{ $0.29$ }
						& \multirow{2}{*}{ $0.388$ }
						& \multirow{2}{*}{ \shortstack{$< 10^{-3}$ \\ ($***$) } }
						\tabularnewline
						\cline{2-4}
						
						& \texttt{vect.frcorenewsmd}
						& $63.08$
						& (2)
						&
						&
						&
						\tabularnewline
						\hline
					
					% CLUSTERING
					\multirow{6}{*}{clustering}
						& \texttt{clust.hier.avg}
						& $50.64$
						& (1)
						& \multirow{6}{*}{ $0.35$ }
						& \multirow{6}{*}{ $0.866$ }
						& \multirow{6}{*}{ \shortstack{ $< 10^{-3}$ \\ ($***$) } }
						\tabularnewline
						\cline{2-4}
						
						& \texttt{clust.kmeans.cop}
						& $52.43$
						& (2)
						&
						&
						&
						\tabularnewline
						\cline{2-4}
						
						& \texttt{clust.hier.sing}
						& $54.08$
						& (3)
						&
						& 
						&
						\tabularnewline
						\cline{2-4}
						
						& \texttt{clust.hier.ward}
						& $72.41$
						& (4)
						&
						& 
						&
						\tabularnewline
						\cline{2-4}
						
						& \texttt{clust.hier.comp}
						& $73.48$
						& (5)
						&
						&
						&
						\tabularnewline
						\cline{2-4}
						
						& \texttt{clust.spec}
						& $87.84$
						& (6)
						&
						& 
						&
						\tabularnewline
						\hline
					
					% ECHANTILLONNAGE
					\multirow{4}{*}{échantillonnage}
						& \texttt{samp.closest.diff}
						& $33.66$
						& (1)
						& \multirow{4}{*}{ $0.32$ }
						& \multirow{4}{*}{ $0.968$ }
						& \multirow{4}{*}{ \shortstack{ $< 10^{-3}$ \\ ($***$) } }
						\tabularnewline
						\cline{2-4}
						
						& \texttt{samp.random.same}
						& $48.24$
						& (2)
						&
						&
						&
						\tabularnewline
						\cline{2-4}
						
						& \texttt{samp.random.full}
						& $65.83$
						& (3)
						&
						& 
						&
						\tabularnewline
						\cline{2-4}
						
						& \texttt{samp.farhtest.same}
						& $112.86$
						& (4)
						&
						&
						&
						\tabularnewline
						\hline
				\end{tabular}
				\end{center}
				\caption{ANOVA du nombre d'itérations nécessaires pour l'obtention de $90$\% de v-mesure. Les (\textit{$*$}) dénotent le niveau de significativité ($\alpha=0.05$). Pour les effets significatifs, les chiffres précisés entre parenthèses dans la colonne \texttt{Moyenne} indiquent le classement des niveaux selon les analyses post-hoc.}
				\label{table:4.2.1-ETUDE-OPTIMISATION-ANOVA-ANNOTATION-PARTIELLE}
			\end{table}
			

			%%% Analyse d'une annotation suffisante.
			Pour obtenir une \textbf{annotation suffisante} (\textit{atteindre une \texttt{v-measure} de $100$\%}), la moyenne des itérations est de $76.29$ (min: $19$, max: $328$, écart-type: $46.44$), soit une moyenne de $3~801.19$ annotations (min: $950$, max: $16~400$, écart-type: $2~314.91$).
			La figure~\ref{figure:4.2.1-ETUDE-OPTIMISATION-HISTOGRAMME-ANNOTATION-SUFFISANTE} représente la répartition de ces itérations au cours des différentes tentatives.
			On peut noter les deux cas intéressants suivants :
			%
			\begin{itemize}
				\item[$\bullet$] Les tentatives les plus rapides furent celles avec un prétraitement des données \texttt{prep.simple}, une vectorisation des données \texttt{vect.tfidf}, un clustering sous contraintes \texttt{clust.hier.comp} ou \texttt{clust.hier.ward}, et un échantillonnage de contraintes \texttt{samp.closest.diff}. Ces tentatives ont requis $19$ itérations, soit $950$ annotations, dont $638$ (respectivement $641$) contraintes \texttt{MUST-LINK}.
				\item[$\bullet$] Les tentatives les plus lentes furent celles avec un prétraitement des données \texttt{prep.no}, une vectorisation des données \texttt{vect.tfidf}, un clustering sous contraintes \texttt{clust.spec}, et un échantillonnage de contraintes \texttt{samp.farthest.same}. Ces tentatives ont requis $394$ itérations, soit $16~400$ annotations, dont $1~309$ contraintes \texttt{MUST-LINK}.
			\end{itemize}
			%
			\begin{figure}[!htb]
				\centering
				\includegraphics[width=0.7\textwidth]{figures/etude-efficience-histogramme-annotation-suffisante}
				\caption{Répartition des tentatives en fonction de l'itération de la méthode à laquelle elles atteignent le seuil d'une annotation suffisante, c'est-à-dire l'itération à laquelle elles parviennent à $100$\% de \texttt{v-measure} entre un résultat obtenu et la vérité terrain. L'histogramme est réduit à $60$ pics pour simplifier l'affichage.}
				\label{figure:4.2.1-ETUDE-OPTIMISATION-HISTOGRAMME-ANNOTATION-SUFFISANTE}
			\end{figure}
			%
			Le tableau~\ref{table:4.2.1-ETUDE-OPTIMISATION-ANOVA-ANNOTATION-SUFFISANTE} retranscrit l'influence de chacun des paramètres sur le nombre d'itérations nécessaires pour atteindre une \textbf{annotation suffisante}.
			Les analyses de variance mettent en relief l'effet significatif sur cette convergence du prétraitement (\texttt{eta-carré}: $0.987$, \texttt{p-valeur}: $< 10^{-3}$), de la vectorisation (\texttt{eta-carré}: $0.991$, \texttt{p-valeur}: $< 10^{-3}$), du clustering (\texttt{eta-carré}: $0.997$, \texttt{p-valeur}: $< 10^{-3}$) et de l'échantillonnage (\texttt{eta-carré}: $0.998$, \texttt{p-valeur}: $< 10^{-3}$).
			L'analyse post-hoc de ces effets indique que le meilleur paramétrage moyen pour atteindre une \textbf{annotation suffisante} repose sur la prétraitement \texttt{prep.lemma}, le vectorisation \texttt{vect.tfidf}, le clustering \texttt{clust.kmeans.cop}, et l'échantillonnage \texttt{samp.closest.diff}. La moyenne du nombre d'itération requis pour ce paramétrage est de $34.60$ (écart-type: $7.44$), soit $1~730$ annotations (écart-type: $372.00$).
			%
			\begin{table}[!htb]
				\begin{center}
				\begin{tabular}{|c|c|c|c|c|c|c|}
					\hline
					% ENTETE DU TABLEAU
					\multicolumn{2}{|c|}{ \shortstack{Description des \\ facteurs analysés } }
						& \multicolumn{3}{c|}{ \shortstack{ Description \\ statistique } }
						& \multicolumn{2}{c|}{ \shortstack{ Description des \\ tailles d'effets } }
						\tabularnewline
						\hline

					Facteur
						& Niveau 
						& Moyenne
						& Rang
						& SE
						& \texttt{ $\eta^{2}$ }
						& \texttt{p-valeur}
						\tabularnewline
						\hline
					
					% PRETRAITEMENT
					\multirow{4}{*}{prétraitement}
						& \texttt{prep.lemma}
						& $72.86$
						& (1)
						& \multirow{4}{*}{ $0.32$ }
						& \multirow{4}{*}{ $0.276$ }
						& \multirow{4}{*}{ \shortstack{ $< 10^{-3}$ \\ ($***$) } }
						\tabularnewline
						\cline{2-4}
						
						& \texttt{prep.simple}
						& $73.30$
						& (2)
						&
						&
						&
						\tabularnewline
						\cline{2-4}
						
						& \texttt{prep.no}
						& $75.24$
						& (2)
						&
						& 
						&
						\tabularnewline
						\cline{2-4}
						
						& \texttt{prep.filter}
						& $83.77$
						& (4)
						&
						&
						&
						\tabularnewline
						\hline
					
					% VECTORISATION
					\multirow{2}{*}{vectorisation}
						& \texttt{vect.tfidf}
						& $71.16$
						& (1)
						& \multirow{2}{*}{ $0.36$ }
						& \multirow{2}{*}{ $0.366$ }
						& \multirow{2}{*}{ \shortstack{$< 10^{-3}$ \\ ($***$)} }
						\tabularnewline
						\cline{2-4}
						
						& \texttt{vect.frcorenewsmd}
						& $81.43$
						& (2)
						&
						&
						&
						\tabularnewline
						\hline
					
					% CLUSTERING
					\multirow{6}{*}{clustering}
						& \texttt{clust.kmeans.cop}
						& $62.23$
						& (1)
						& \multirow{6}{*}{ $0.42$ }
						& \multirow{6}{*}{ $0.700$ }
						& \multirow{6}{*}{ \shortstack{$< 10^{-3}$ \\ ($***$)} }
						\tabularnewline
						\cline{2-4}
						
						& \texttt{clust.hier.avg}
						& $65.13$
						& (2)
						&
						&
						&
						\tabularnewline
						\cline{2-4}
						
						& \texttt{clust.hier.sing}
						& $75.44$
						& (3)
						&
						& 
						&
						\tabularnewline
						\cline{2-4}
						
						& \texttt{clust.hier.ward}
						& $80.44$
						& (4)
						&
						& 
						&
						\tabularnewline
						\cline{2-4}
						
						& \texttt{clust.hier.comp}
						& $81.46$
						& (5)
						&
						&
						&
						\tabularnewline
						\cline{2-4}
						
						& \texttt{clust.spec}
						& $93.06$
						& (6)
						&
						& 
						&
						\tabularnewline
						\hline
					
					% ECHANTILLONNAGE
					\multirow{4}{*}{échantillonnage}
						& \texttt{samp.closest.diff}
						& $50.29$
						& (1)
						& \multirow{4}{*}{ $0.39$ }
						& \multirow{4}{*}{ $0.950$ }
						& \multirow{4}{*}{ \shortstack{$< 10^{-3}$ \\ ($***$)} }
						\tabularnewline
						\cline{2-4}
						
						& \texttt{samp.random.same}
						& $56.38$
						& (2)
						&
						&
						&
						\tabularnewline
						\cline{2-4}
						
						& \texttt{samp.random.full}
						& $71.95$
						& (3)
						&
						& 
						&
						\tabularnewline
						\cline{2-4}
						
						& \texttt{samp.farhtest.same}
						& $126.55$
						& (4)
						&
						&
						&
						\tabularnewline
						\hline
				\end{tabular}
				\end{center}
				\caption{ANOVA du nombre d'itérations nécessaires pour l'obtention de $100$\% de v-mesure. Les (\textit{$*$}) dénotent le niveau de significativité ($\alpha=0.05$). Pour les effets significatifs, les chiffres précisés entre parenthèses dans la colonne \texttt{Moyenne} indiquent le classement des niveaux selon les analyses post-hoc.}
				\label{table:4.2.1-ETUDE-OPTIMISATION-ANOVA-ANNOTATION-SUFFISANTE}
			\end{table}
			
			%%% Analyse d'une annotation exhaustive.
			Enfin, pour avoir une \textbf{annotation exhaustive} (\textit{annoter toutes les contraintes possibles}), la moyenne des itérations est de $88.98$ (min: $20$, max: $394$, écart-type: $68.21$), soit une moyenne de $4~431.34$ annotations (min: $1~000$, max: $19~656$, écart-type: $3~405.16$).
			La figure~\ref{figure:4.2.1-ETUDE-OPTIMISATION-HISTOGRAMME-ANNOTATION-EXHAUSTIVE} représente la répartition de ces itérations au cours des différentes tentatives.
			On peut noter les deux cas intéressants suivant :
			%
			\begin{itemize}
				\item[$\bullet$] Les tentatives les plus rapides furent celles avec un prétraitement des données \texttt{prep.no} ou \texttt{prep.lemma}, une vectorisation des données \texttt{vect.tfidf}, un algorithme de clustering sous contraintes \texttt{clust.hier.comp} ou \texttt{clust.hier.wars}, et un échantillonnage de contraintes \texttt{samp.closest.diff}. Ces tentatives ont requis $20$ itérations, soit $1~000$ annotations, dont $653$ (respectivement $668$) contraintes \texttt{MUST-LINK}.
				\item[$\bullet$] Les tentatives les plus lentes furent celles avec un prétraitement des données \texttt{prep.simple}, une vectorisation des données \texttt{vect.frcorenewsmd}, un clustering sous contraintes \texttt{clust.hier.sing}, et un échantillonnage de contraintes \texttt{samp.closest.diff}. Ces tentatives ont requis $394$ itérations, soit $19~656$ annotations, dont $682$ contraintes \texttt{MUST-LINK}.
			\end{itemize}
			%
			\begin{figure}[!htb]
				\centering
				\includegraphics[width=0.7\textwidth]{figures/etude-efficience-histogramme-annotation-exhaustive}
				\caption{Répartition des tentatives en fonction de l'itération de la méthode à laquelle elles atteignent le seuil d'une annotation exhaustive, c'est-à-dire l'itération à laquelle toutes les contraintes possibles entre les données ont été annotées. L'histogramme est réduit à $60$ pics pour simplifier l'affichage.}
				\label{figure:4.2.1-ETUDE-OPTIMISATION-HISTOGRAMME-ANNOTATION-EXHAUSTIVE}
			\end{figure}
			%
			Le tableau~\ref{table:4.2.1-ETUDE-OPTIMISATION-ANOVA-ANNOTATION-EXHAUSTIVE} retranscrit l'influence de chacun des paramètres sur le nombre d'itérations nécessaires pour atteindre une \textbf{annotation exhaustive}.
			Les analyses de variance mettent en relief l'effet significatif sur cette convergence du prétraitement (\texttt{eta-carré}: $0.909$, \texttt{p-valeur}: $< 10^{-3}$), de la vectorisation (\texttt{eta-carré}: $0.985$, \texttt{p-valeur}: $< 10^{-3}$), du clustering (\texttt{eta-carré}: $0.999$, \texttt{p-valeur}: $< 10^{-3}$) et de l'échantillonnage (\texttt{eta-carré}: $0.997$, \texttt{p-valeur}: $< 10^{-3}$).
			L'analyse post-hoc de ces effets indique que le meilleur paramétrage moyen pour atteindre une \textbf{annotation exhaustive} repose sur la prétraitement \texttt{prep.lemma}, le vectorisation \texttt{vect.tfidf}, le clustering \texttt{clust.kmeans.cop}, et l'échantillonnage \texttt{samp.random.same}. La moyenne du nombre d'itération requis pour ce paramétrage est de $32.60$ (écart-type: $1.14$), soit $1~630$ annotations (écart-type: $57.00$).
			%
			\begin{table}[!htb]
				\begin{center}
				\begin{tabular}{|c|c|c|c|c|c|c|}
					\hline
					% ENTETE DU TABLEAU
					\multicolumn{2}{|c|}{ \shortstack{Description des \\ facteurs analysés } }
						& \multicolumn{3}{c|}{ \shortstack{ Description \\ statistique } }
						& \multicolumn{2}{c|}{ \shortstack{ Description des \\ tailles d'effets } }
						\tabularnewline
						\hline

					Facteur
						& Niveau 
						& Moyenne
						& Rang
						& SE
						& \texttt{ $\eta^{2}$ }
						& \texttt{p-valeur}
						\tabularnewline
						\hline
					
					% PRETRAITEMENT
					\multirow{4}{*}{prétraitement}
						& \texttt{prep.lemma}
						& $85.89$
						& (1)
						& \multirow{4}{*}{ $0.42$ }
						& \multirow{4}{*}{ $0.052$ }
						& \multirow{4}{*}{ \shortstack{$< 10^{-3}$ \\ ($***$)} }
						\tabularnewline
						\cline{2-4}
						
						& \texttt{prep.filter}
						& $89.55$
						& (2)
						&
						&
						&
						\tabularnewline
						\cline{2-4}
						
						& \texttt{prep.simple}
						& $89.64$
						& (2)
						&
						& 
						&
						\tabularnewline
						\cline{2-4}
						
						& \texttt{prep.no}
						& $90.81$
						& (4)
						&
						&
						&
						\tabularnewline
						\hline
					
					% VECTORISATION
					\multirow{2}{*}{vectorisation}
						& \texttt{vect.tfidf}
						& $85.50$
						& (1)
						& \multirow{2}{*}{ $0.39$ }
						& \multirow{2}{*}{ $0.165$ }
						& \multirow{2}{*}{ \shortstack{$< 10^{-3}$ \\ ($***$)} }
						\tabularnewline
						\cline{2-4}
						
						& \texttt{vect.frcorenewsmd}
						& $92.46$
						& (2)
						&
						&
						&
						\tabularnewline
						\hline
					
					% CLUSTERING
					\multirow{6}{*}{clustering}
						& \texttt{clust.kmeans.cop}
						& $64.99$
						& (1)
						& \multirow{6}{*}{ $0.39$ }
						& \multirow{6}{*}{ $0.894$ }
						& \multirow{6}{*}{ \shortstack{$< 10^{-3}$ \\ ($***$)} }
						\tabularnewline
						\cline{2-4}
						
						& \texttt{clust.hier.avg}
						& $78.54$
						& (2)
						&
						&
						&
						\tabularnewline
						\cline{2-4}
						
						& \texttt{clust.hier.ward}
						& $81.31$
						& (3)
						&
						& 
						&
						\tabularnewline
						\cline{2-4}
						
						& \texttt{clust.hier.comp}
						& $82.49$
						& (3)
						&
						& 
						&
						\tabularnewline
						\cline{2-4}
						
						& \texttt{clust.spec}
						& $93.78$
						& (5)
						&
						&
						&
						\tabularnewline
						\cline{2-4}
						
						& \texttt{clust.hier.comp}
						& $132.75$
						& (6)
						&
						& 
						&
						\tabularnewline
						\hline
					
					% ECHANTILLONNAGE
					\multirow{4}{*}{échantillonnage}
						& \texttt{samp.random.same}
						& $57.23$
						& (1)
						& \multirow{4}{*}{ $0.42$ }
						& \multirow{4}{*}{ $0.930$ }
						& \multirow{4}{*}{ \shortstack{$< 10^{-3}$ \\ ($***$)} }
						\tabularnewline
						\cline{2-4}
						
						& \texttt{samp.random.full}
						& $72.80$
						& (2)
						&
						&
						&
						\tabularnewline
						\cline{2-4}
						
						& \texttt{samp.closest.diff}
						& $98.38$
						& (3)
						&
						& 
						&
						\tabularnewline
						\cline{2-4}
						
						& \texttt{samp.farhtest.same}
						& $132.75$
						& (4)
						&
						&
						&
						\tabularnewline
						\hline
				\end{tabular}
				\end{center}
				\caption{ANOVA du nombre d'itérations nécessaires pour annoter toutes les contraintes possibles. Les (\textit{$*$}) dénotent le niveau de significativité ($\alpha=0.05$). Pour les effets significatifs, les chiffres précisés entre parenthèses dans la colonne \texttt{Moyenne} indiquent le classement des niveaux selon les analyses post-hoc.}
				\label{table:4.2.1-ETUDE-OPTIMISATION-ANOVA-ANNOTATION-EXHAUSTIVE}
			\end{table}
		
		
			% Graphe d'évolution de la v-measure moyenne, min et max.
			La figure~\ref{figure:4.2.1-ETUDE-OPTIMISATION-EVOLUTION-PAR-FACTEURS} représente les évolutions moyennes de la \texttt{v-measure} du clustering en fonction du nombre d'itération de la méthode pour les différentes valeurs des facteurs analysés (prétraitement en haut à gauche, vectorisation en haut à droite, clustering en bas à gauche, échantillonnage en bas à droite).
			La figure~\ref{figure:4.2.1-ETUDE-OPTIMISATION-EVOLUTION-MEILLEUR-PARAMETRAGE} représente cette même évolution pour les meilleurs paramétrages moyens destinés à atteindre les trois seuils d'annotation définis (partiel, suffisant, exhaustif).
			%
			\begin{figure}[!htb]
				\centering
				\includegraphics[width=\textwidth]{figures/etude-efficience-evolution-moyenne-par-vmeasure-par-facteur}
				\caption{Évolution des moyennes du nombre d'itérations nécessaire de la méthode de \textit{clustering} interactif pour obtenir un seuil défini de \texttt{v-measure} entre un résultat obtenu et la vérité terrain, moyennes réalisées sur les différentes valeurs que peuvent prendre les facteurs analysés et affichées par facteur : \textbf{(1)} prétraitement, \textbf{(2)} vectorisation, \textbf{(3)} clustering et \textbf{(4)} échantillonnage. Le seuil d'annotation exhaustive (annoter toutes les contraintes possibles) n'étant pas exprimé en terme de \texttt{v-measure}, ce seuil n'est pas affiché ici.}
				\label{figure:4.2.1-ETUDE-OPTIMISATION-EVOLUTION-PAR-FACTEURS}
			\end{figure}
			%
			\begin{figure}[!htb]
				\centering
				\includegraphics[width=0.8\textwidth]{figures/etude-efficience-evolution-moyenne-5best-par-vmeasure}
				\caption{Évolution des moyennes du nombre d'itérations nécessaire de la méthode de \textit{clustering} interactif pour obtenu un seuil défini de \texttt{v-measure} entre un résultat obtenu et la vérité terrain, moyennes réalisées sur les différentes seuils d'annotations étudiés : l'annotation partielle (\textit{atteindre une \texttt{v-measure} de $90$\%}), l'annotation suffisante (\textit{atteindre une \texttt{v-measure} de $100$\%}) et l'annotation exhaustive (\textit{annoter toutes les contraintes possibles}).}
				\label{figure:4.2.1-ETUDE-OPTIMISATION-EVOLUTION-MEILLEUR-PARAMETRAGE}
			\end{figure}

		%%% Discussion
		\subsubsection{Discussion}

			% Rappel de l'objectif : être efficient.
			L'objectif de l'étude est de trouver une implémentation "efficiente" du \textit{clustering} interactif permettant d'obtenir une base d'apprentissage correctement annotée en un minimum d'annotation.
			Pour trouver si une telle implémentation existe et quels en sont les paramètres optimaux, nous avons analysé l'impact de différentes paramétrages sur les tâches principales de la méthode (\textbf{prétraitement}, \textbf{vectorisation}, \textbf{clustering sous contraintes}, \textbf{échantillonnage}) en nous basant sur des simulations d'annotation d'un jeu de données.
			
			% Première remarque : Choix d'un seuil à 90\% de v-measure.
			Dans l'optique d'être efficient, nous excluons le désir d'annoter \textbf{exhaustivement} le jeu de données car la charge de travail estimée est trop importante.
			(cf. discussion de la section~\ref{section:4.1-HYPOTHESE-EFFICACITE} (hypothèse d'efficacité))
			Nous préférons donc nous concentrer sur deux seuils d'annotation plus réalistes : celui d'une \textbf{annotation partielle} (atteindre $90$\% de \texttt{v-measure} avec la vérité terrain) et celui d'une \textbf{annotation suffisante} (atteindre $100$\% de \texttt{v-measure} avec la vérité terrain en un minimum de contraintes). 
			
			% Meilleur paramétrage.
			L'étude réalisée met en avant l'impact significatif des quatre tâches principales (\textbf{prétraitement}\todo{remarque sur la valeur de eta2}, \textbf{vectorisation}\todo{remarque sur la valeur de eta2}, \textbf{clustering sous contraintes}, \textbf{échantillonnage}) sur la vitesse de convergence de la méthode pour atteindre les seuils définis de $90$\% et $100$\% de \texttt{v-measure}. Il existe donc bien un paramétrage permettant d'optimiser l'implémentation proposée et de réduire le nombre de contraintes nécessaires à annoter :
			\begin{enumerate}
				\item pour une \textbf{annotation partielle} ($90$\% de \texttt{v-measure}), le meilleur paramétrage moyen est constitué du prétraitement simple (\texttt{prep.simple}), de la vectorisation TF-IDF (\texttt{vect.tfidf}), du clustering hiérarchique à lien moyen (\texttt{clust.hier.avg}) et de l'échantillonnage des données les plus proches dans des clusters différents (\texttt{sampl.closest.diff}). Avec ce paramétrage, il faut en moyenne $950$ annotations de contraintes pour obtenir une \texttt{v-measure} de $90$\% ;
				\item pour une \textbf{annotation suffisante} ($100$\% de \texttt{v-measure}), le meilleur paramétrage moyen est constitué du prétraitement avec lemmatisation (\texttt{prep.lemma}), de la vectorisation TF-IDF (\texttt{vect.tfidf}), du clustering KMeans (\texttt{clust.kmeans.cop}) et de l'échantillonnage des données les plus proches dans des clusters différents (\texttt{sampl.closest.diff}). Avec ce paramétrage, il faut en moyenne $1~750$ annotations de contraintes pour obtenir une \texttt{v-measure} de $100$\% ;
				\item le cas d'une \textbf{annotation exhaustive} (annoter toutes les contraintes possibles sur les données) n'est pas explicité ici mais peut se déduire des résultats décrits plus haut.
			\end{enumerate}


			%%% Avantages.
			Ainsi, cette étude permet de répondre à certaines limites discutées dans la section~\ref{section:4.1-HYPOTHESE-EFFICACITE} (hypothèse d'efficacité). 
			
			% Avantage 1: Optimisation du nombre de contraintes.
			En effet, l'optimisation des paramètres de l'implémentation du \textit{clustering} interactif permet de réduire considérablement le nombre de contraintes nécessaires pour obtenir une base d'apprentissage exploitable.
			En nous basant sur le tableau~\ref{table:4.1.1-ETUDE-CONVERGENCE-EVOLUTION} de l'étude de convergence, et dans le cadre de l'annotation d'un jeu de $500$ données, nous sommes passé d'un paramétrage moyen nécessitant $3~750$ (respectivement $10~000$) contraintes à un paramétrage optimisé ne nécessitant que $950$ (respectivement $1~750$) contraintes pour atteindre un seuil de $90$\% (respectivement $100$\%) \texttt{v-measure}.
			L'ordre de grandeur de la charge de travail demandée aux annotateurs est donc située entre $2$ et $4$ fois la taille du jeu de données.
			
			% Avantage 2: La méthode devient réaliste !
			En considérant que les annotations sont binaires et demandent a priori une charge mental plus faible que les annotations par attribution de label ("\textit{les données sont-elles similaires ?}" vs "\textit{quel est l'étiquette de cette donnée ?}"), nous pouvons conclure que la charge totale nécessaire à l'annotation avec une méthodologie basée sur le \textit{clustering} interactif est comparable à celles des méthodes traditionnelles.
			De plus, cette méthode ne demande pas de formalisation concrète de la structure de données à annoter pour faire émerger une base d'apprentissage au cours des itérations, donc le \textit{clustering} interactif devient une méthode d'annotation adaptée à l'activité des annotateurs.
			
			%%% Limites.
			Néanmoins, quelques pistes sont encore à explorer pour compléter cette analyse d'efficience.
			
			% Limite 1 : Coût temporel.
			D'une part, une étude de coût est à réaliser pour trancher le choix de paramètre optimaux réalistes. En effet, il est intéressant d'étudier le coût machine (temps CPU utilisé) et le coût humain (temps d'annotation) afin d'affiner les choix techniques et de compléter les arguments sur l'utilisation en situation réelle d'une méthodologie d'annotation basée sur le \textit{clustering} interactif.
			Cet aspect sera traité dans la section~\ref{section:4.3-HYPOTHESE-COUTS} (hypothèse des coûts).
			
			% Limite 2 : Valeur métier de ce 90\% (pas de vérité terrain en pratique).
			D'autre part, l'étude réalisée se base sur des seuils de performance par rapport à une vérité terrain.
			Or en situation réelle, cette comparaison avec la vérité terrain n'est pas possible car elle est précisément en cours de conception (la base d'apprentissage finale devant être la vérité terrain).
			De plus, un tel score n'est pas le plus explicite pour pour un expert métier pour qui un score de \texttt{v-measure} n'est pas révélateur de la pertinence métier de la segmentation proposée des données.
			Il manque donc une stratégie d'évaluation de pertinence de la base d'apprentissage en cours de construction et de la suffisance des annotations réalisées pour faire refléter la vision de l'annotateur dans le résultat.
			Cet aspect sera traité dans la section~\ref{section:4.4-HYPOTHESE-PERTINENCE} (hypothèse de pertinence).
			
			% Limite 3 : Expert métier parfait ==> simuler les erreurs.
			Pour finir, comme pour l'étude de convergence réalisé en section~\ref{section:4.1-HYPOTHESE-EFFICACITE}, nous avons supposé dans cette étude que l'annotateur est un expert métier connaissant parfaitement le domaine traité.
			Cette hypothèse forte n'est a priori pas valable en situation réelle : En effet, des erreurs d'annotations peuvent intervenir (ambiguïtés sur les données, méconnaissance du domaine, erreurs d'inattention, différence d'opinions entre annotateurs, ...), ce qui peut entraîner des divergences ou des incohérences dans la construction de la base d'apprentissage.
			Il semble donc nécessaire d'étudier les impacts de ces incohérences, ainsi que de proposer une méthode pour les prévenir ou les corriger.
			Cet aspect sera traité à la fin de ce chapitre dans la section~\ref{section:4.6-HYPOTHESE-ROBUSTESSE} (hypothèse de robustesse).


    %%%%%--------------------------------------------------------------------
    %%%%% Section 4.3: Hypothèse sur les coûts.
    %%%%%--------------------------------------------------------------------
	\newpage
	\section{Évaluation de l'hypothèse sur les coûts}
\label{section:4.3-HYPOTHESE-COUTS}
% : « \textit{combien dois-je investir ?} »

	%%% Introduction / Transition.
	Dans les deux sections précédentes, nous avons estimé le paramétrage du \textit{clustering} interactif le plus efficient pour atteindre $90$\% de \texttt{v-measure} avec la vérité terrain, correspondant à ce que nous appelons une annotation partielle.
	Toutefois, pour compléter l'étude de faisabilité technique de notre méthode, nous devons nous intéresser aux coûts (matériel et humain) à investir pour atteindre notre objectif.
	Nous aimerions donc vérifier l'hypothèse suivante :
	
	%%% Formulation des hypothèses:
	\begin{tcolorbox}[
		title=\faVial~\textbf{Hypothèse sur les coûts}~\faVial,
		colback=colorTcolorboxHypothesis!15,
		colframe=colorTcolorboxHypothesis!75,
		width=\linewidth
	]
		% Hypothèse.
		«\textbf{
			Il est possible d'\textbf{estimer les coûts nécessaires} d'une méthodologie d'annotation basée sur le \textit{clustering} interactif pour obtenir une base d'apprentissage exploitable. Nous étudierons en particulier les coûts relatifs au temps d'annotation, au temps de calculs des algorithmes, ainsi que la durée totale de la méthode en fonction de la taille du jeu de données.
		} » \\

		% Résumé des études.
		Afin de vérifier cette hypothèse, nous organiserons plusieurs expériences pour simuler ou déterminer ces durées : une étude du temps d'annotation par un expert métier (cf. section~\ref{section:4.3.1-ETUDE-COUTS-TEMPS-ANNOTATION}), une étude du temps de calcul des algorithmes (cf. section~\ref{section:4.3.2-ETUDE-COUTS-TEMPS-CALCUL}) et une étude du nombre de contraintes nécessaires (cf. section~\ref{section:4.3.3-ETUDE-COUT-NOMBRE-CONTRAINTES}). Nous conclurons l'estimation du temps total d'un projet d'annotation en section~\ref{section:4.3.4-ETUDE-COUTS-TOTAL}.
		
		% Figure.
		La figure~\ref{figure:4.3-HYPOTHESE-COUTS} illustre cette hypothèse et l'espoir de pouvoir caractériser la qualité de la base d'apprentissage en cours de construction en fonction d'un coût temporel au lieu d'un nombre abstrait d'itérations de la méthode. 
		%
		
		\begin{figure}[H]  % keep [H] to be in the tcolorbox.
			\centering
			\includegraphics[width=0.8\textwidth]{figures/hypotheses-03-couts}
			\caption{Illustration des études réalisées sur le \textit{clustering} interactif (\textit{étape 3/6}) en schématisant l'évolution de la performance (\textit{accord avec la vérité terrain calculé en v-measure}) d'une base d'apprentissage en cours de construction en fonction du nombre d'itérations de la méthode (\textit{nombre d'annotations par un expert métier}).}
			\label{figure:4.3-HYPOTHESE-COUTS}
		\end{figure}

	\end{tcolorbox}
	
	
	%%%
	%%% Subsection 4.3.1: Étude du temps d'annotation nécessaire pour traiter un lot de contraintes en chronométrant des opérateurs en situation réelle
	%%%
	\subsection{Étude du temps d'annotation nécessaire pour traiter un lot de contraintes en chronométrant des opérateurs en situation réelle}
	\label{section:4.3.1-ETUDE-COUTS-TEMPS-ANNOTATION}
	
		%%% Protocole expérimental.
		\subsubsection{Protocole expérimental}
		
			% Objectif de l'expérience.
			Nous voulons estimer le temps nécessaire à un opérateur pour annoter un lot de contraintes.
			Pour cela, nous allons chronométrer plusieurs experts métiers en train d'annoter un même échantillon et modéliser le nombre de contraintes par minute, ainsi que son évolution au cours de plusieurs sessions d'annotation.
			
			% Axiome.
			\begin{leftBarWarning}
				Dans cette étude, nous supposons que les annotateurs de l'expérience connaissent parfaitement le domaine traité dans le jeu de données, et qu'ils sont capables de caractériser sans ambiguïté la similitude entre deux données issues de cet ensemble.
				Afin de pourvoir faire cette hypothèse forte, et ainsi limiter les bruits dans l'analyse des résultats, le jeu de données devra traiter d'un sujet de culture générale (ne nécessitant donc pas de connaissance particulière) et des réviseurs supprimeront en amont et d'un commun accord les données trop spécifiques ou trop ambiguës.
			\end{leftBarWarning}
			
			% Objectif supplémentaire.
			Pour aller plus loin, nous aimerions aussi confirmer que le temps d'annotation est caractéristique d'une tâche "courte".
			En utilisant des approximations communes en neuroscience (cf. CITATION)\todo{CITATION: Purves et al. (2008). Principles of Cognitive Neuroscience}, nous pouvons estimer le temps d'annotation d'une contraintes à environ $5$ secondes.
			En effet, il faut $1~200$ms pour lire et traiter deux phrases (on utilise la \texttt{P600} pour estimer la réaction à un stimulus pour une phrase), auquel on ajoute à nouveau $600$ ms pour traiter la concordance entre les deux phrases.
			En considérant une réponse motrice au alentours de $1$ seconde (clic de bouton) et un délais application de $1$ seconde (rechargement de la page), nous sommes autour des $5$ secondes ($4.6$ estimé).
			Bien entendu, cette estimation reste approximative, et dépend fortement de la taille des phrases ainsi que de leur proximité sémantique.
			Toutefois, cela donne un ordre de grandeur pour notre étude du temps d'annotation.
			
			% Pseudo-code.
			Pour résumer le protocole expérimental que nous décrivons ci-dessous, vous pouvez vous référer au pseudo-code décrit dans Alg.~\ref{algorithm:4.3.1-ETUDE-COUTS-TEMPS-ANNOTATION-PROTOCOLE}.
			%
			\begin{algorithm}[!htb]
				\begin{algorithmic}[1]
					\Require jeu de données annoté (vérité terrain)
					\Require plusieurs réviseurs, plusieurs annotateurs
					\State \textbf{initialisation} définir et revoir le jeu de données entre réviseurs
					\State \textbf{échantillonnage} sélectionner une base de contraintes avec \texttt{samp.rand.full}
					\ForAll{annotateur}
						\While{la base de contraintes n'a pas été entièrement annotée}
							\State \textbf{chronomètre: START}
							\State \textbf{annotation}: annoter une partie des contraintes
							\State \textbf{revue}: revue des contraintes en conflits d'annotation
							\State \textbf{chronomètre: STOP}
							\State \textbf{mesure}: estimer la différence de chronomètre pour cette session
						\EndWhile
					\EndFor
					\State \textbf{modélisation}: entraîner un modèle linéaire généralisé du temps d'annotation
					\State \textbf{simulation}: écrire l'équation du temps d'annotation d'un lot de contraintes
					\Ensure modélisation du temps d'annotation d'un lot de contraintes
				\end{algorithmic}
				\caption{Description en pseudo-code du protocole expérimental de l'étude du temps d'annotation d'un lot de contraintes par un expert métier.}
				\label{algorithm:4.3.1-ETUDE-COUTS-TEMPS-ANNOTATION-PROTOCOLE}
			\end{algorithm}
			
			% Détails de l'expérience : préparation du jeu de données.
			Pour cette étude, nous procéderons en plusieurs étapes.
			D'abord, il faut choisir un jeu de données approprié : pour valider notre hypothèse forte sur les compétence de nos annotateurs, nous cherchons un jeu de données traitant d'un sujet de culture général.
			Pour cette expérience, nous avons donc choisi \texttt{MLSUM} : une collecte d'articles de journaux, classés par catégorie de publication et décrits par leur titre et leur résumé.
			Nous nous intéressons ici à la tâche de classification d'un titre d'article en fonction de sa catégorie de publication.
			Comme certains titres peuvent porter à confusion (un titre d'article n'étant pas toujours explicite sur son contenu), deux réviseurs sont chargés de choisir les données les plus explicites sur un échantillon d'un millier de données représentatives des catégories les plus communes.
			L'échantillon résultant, noté \texttt{MLSUM FR TRAIN SUBSET (v1.0.0-schild)}, est composé de $744$ titres d'articles rédigés en français et répartis en $14$ classes (\textit{économie}, \textit{sport}, ...).
			Pour plus de détails, consultez l'annexe~\ref{annex:C.2-DATASET-MLSUM-SUBSET-SCHILD}.
			
			% Détails de l'expérience : sélection des contraintes à annoter. 
			A partir de ces données, nous sélectionnons un lot de $1~000$ contraintes à annoter. Comme nous nous intéressons exclusivement au temps d'annotation pour cette expérience (et que nous ne regardons pas le nombre d'itérations de la méthode), nous utilisons l'échantillonnage purement aléatoire (\texttt{samp.rand.full}).			
			
			% Détails de l'expérience : annotations et consignes.
			Ensuite, un groupe de $14$ annotateurs vont annoter la sélection de $1~000$ contraintes en plusieurs sessions.
			Les directives données aux opérateurs sont les suivantes:
			\begin{itemize}
				\item \textbf{Contexte de l'opérateur} :
				« \textit{Vous êtes des \textbf{experts de la presse et de l’actualité} ; Vous voulez classer des articles dans des catégories en fonction de leur titre ; Vous ne savez pas précisément quelles catégories vous allez utiliser pour classer vos articles ; Mais vous savez \textbf{caractériser la similitude} de deux articles} » ;
				\item \textbf{Contexte sur le jeu de données} :
				« \textit{Le thème sont les catégories d’articles de presse ; La vérité terrain contient entre $10$ et $20$ catégories parmi les plus communes de la presse ; La vérité terrain contient entre $30$ et $100$ articles par catégorie ; Vous \textbf{pouvez regarder le jeu de données non annoté} autant que vous le voulez (disponible dans l'onglet \texttt{TEXTS} de l'application)} » ;
				\item \textbf{Objectif de l'expérience} :
				« \textit{Je veux savoir le temps nécessaire pour annoter un certain nombre de contraintes ; Autrement dit : \textbf{Pour annoter 1000 contraintes, combien de temps me faut-il ?}} » ;
				\item \textbf{Consignes d'annotations} :
				« \textit{Faites des séries de \textbf{15 minutes minimum} pour avoir de la régularité ; Si possible, \textbf{isolez-vous} pour ne pas être dérangé et ne pas fausser les résultats ; Pour chaque série, \textbf{notez le temps et le nombre de contraintes annotés} ; Si vous ne savez pas quoi annoter (trop ambigu, vocabulaire inconnu, ...), \textbf{passez au suivant sans annoter} (vous êtes sensés être des experts de la presse !)} ».
			\end{itemize}
			%
			Pour réaliser l'annotation, les opérateurs auront accès à l'application web développée au cours de ce doctorat.
			Des captures d'écran sont disponibles en figures~\ref{figure:4.3.1-ETUDE-COUTS-TEMPS-ANNOTATION-APPLICATION-ANNOTATION}et~\ref{figure:4.3.1-ETUDE-COUTS-TEMPS-ANNOTATION-APPLICATION-LISTE-CONTRAINTES}.
			Une description plus détaillée de l'application et de ses fonctionnalités est disponible en section~\ref{section:3.3-DESCRIPTION-IMPLEMENTATION}\todo{description à faire}
			%
			\begin{figure}[!htb]
				\centering
				\includegraphics[width=\textwidth]{figures/etude-temps-annotation-0application-annotation}
				\caption{Capture d'écran de l'application web permettant utilisant notre méthodologie de \textit{clustering} interactif pour annoter des contraintes (page d'annotation). Les deux textes à annoter sont disposés à gauche et droite de l'écran. Chacun dispose d'un cache à cocher si le texte n'est pas pertinent à analyser (\textit{ambigu, hors périmètre, incompréhensible, ...}).\\
				Les boutons à disposition permettent respectivement d'annoter un \texttt{MUST-LINK} si les données sont similaires (\textit{bouton en vert}), un \texttt{CANNOT-LINK} si les données ne sont pas similaire (\textit{bouton en rouge}), d'ignorer la contrainte pour laisser la main à l'algorithme de \textit{clustering} (\textit{bouton en bleu}), et d'ajouter un commentaire pour revoir la contrainte plus tard (\textit{case à choser et champ de texte libre}). Deux éléments déroulant permettent d'avoir des informations supplémentaires (\textit{metadata de sélection et de \textit{clustering}, représentation graphique des liens entre contraintes annotées}). Les boutons de navigation (\textit{boutons flèches et liste}) sont disponibles en bas de page.}
				\label{figure:4.3.1-ETUDE-COUTS-TEMPS-ANNOTATION-APPLICATION-ANNOTATION}
			\end{figure}
			\begin{figure}[!htb]
				\centering
				\includegraphics[width=\textwidth]{figures/etude-temps-annotation-0application-liste-contraintes}
				\caption{Capture d'écran de l'application web permettant utilisant notre méthodologie de \textit{clustering} interactif pour annoter des contraintes (page d'inventaire des contraintes à annoter).\\
				La partie supérieure permet d'identifier le nombre de textes et de contraintes sur le projet, ainsi que les boutons destinés à calculer les transitivités entre les contraintes et à approuver le travail réalisé si aucune transitivité n'entre en conflit avec un contrainte annotée. La partie inférieure liste l'ensemble des contraintes du projet, avec les annotations réalisées, l'itération à laquelle la contraintes a été sélectionnée et annotée, si elle est à revoir ou si une incohérence la concernant est détectée.}
				\label{figure:4.3.1-ETUDE-COUTS-TEMPS-ANNOTATION-APPLICATION-LISTE-CONTRAINTES}
			\end{figure}
			
			
			% Détails de l'expérience : modélisation.
			Une fois les sessions d'annotations terminées, nous entraînons un modèle linéaire généralisé (\textit{GLM}) pour estimer le temps d'annotation moyen pour un lot de contraintes (dont la taille est notée $\texttt{batch\_size}$).
			Ce modèle sera caractérisé par le coefficient de détermination généralisé \texttt{R²} de \textit{Cox et Snel}\todo{CITATION}, la log-vraisemblance \texttt{llf}\todo{CITATION} et la log-vraisemblance \texttt{llf\_null} du modèle \textit{null}.
			Nous discuterons aussi de l'évolution de la vitesse d'un opérateur au cours des différentes sessions d'annotation.

			% Référence scripts.
			\begin{leftBarInformation}
				Ces analyses sont réalisées en Python à l'aide des librairies \texttt{datetime} et \texttt{statsmodels} (\cite{seabold:2010}).
				Le projet à importer dans l'outil d'annotation ainsi que les scripts de l'expérience (\textit{notebooks} Python) sont disponibles dans un dossier dédié de~\cite{schild:cognitivefactory-interactive-clustering-comparative-study:2021}.
			\end{leftBarInformation}
			\todo{citation}

		%%% Résultats.
		\subsubsection{Résultats obtenus}
		
			% Taux de participation.
			Durant cette expérience, $14$ annotateurs ont participé à l'annotation de $1~000$ contraintes aléatoires sur un jeu de données.
			Par manque de disponibilités, $4$ annotateurs n'ont que partiellement réalisé leur tâche : nous avons toutefois intégré leurs participations car elles contenaient toutes au moins $150$ annotations.
			
			% Statistiques descriptives.
			D'après les observations, un annotateur réalisait en moyenne $170.7$ contraintes par session d'annotation (min: $43$, max: $547$, médiane: $138$, écart-type: $106.4$) ce qui lui demandait en moyenne $23.1$ minutes (min: $3.0$, max: $92.0$, écart-type: $14.4$).
			De plus, la vitesse d'annotation moyenne était de $7.7$ contraintes par minute (min: $3.5$, max: $14.3$, écart-type: $2.9$).
			
			% Modélisation du temps d'annotation.
			Le modèle linéaire généralisé entraîné sur les mesures de temps d'annotations (\texttt{R²}: $0.910$, \texttt{llf}: $-499.15$, \texttt{llf\_null}: $-539.95$) nous permet de déduire l'équation suivante :
			%
			\begin{equation}
				\texttt{annotation\_time}~[s]~
				\propto~7.8 \cdot \texttt{batch\_size}
			\end{equation}
		
			% Affichage du temps d'annotation.
			La figure~\ref{figure:4.3.1-ETUDE-COUTS-TEMPS-ANNOTATION-SIMULATION} représente cette modélisation du temps d'annotation en comparaison avec les mesures réalisées lors de l'expérience.
			\begin{figure}[!htb]
				\centering
				\includegraphics[width=\textwidth]{figures/etude-temps-annotation-1-modelisation-temps}
				\caption{Estimation du temps nécessaire (en minutes) pour annoter un lot de contraintes.}
				\label{figure:4.3.1-ETUDE-COUTS-TEMPS-ANNOTATION-SIMULATION}
			\end{figure}
		
			% Etude de cas.
			En ce qui concerne l'évolution de la vitesse d'annotation au cours des sessions, aucune tendance significative n'a été identifiée. 
			La figure~\ref{figure:4.3.1-ETUDE-COUTS-TEMPS-ANNOTATION-EXEMPLE} représente cette évolution de vitesse d'annotation pour quatre opérateurs (les deux plus rapides et les deux plus lents).
			Ces données sont l'objet d'une étude de cas dans la discussion ci-dessous.
			\begin{figure}[!htb]
				\centering
				\includegraphics[width=\textwidth]{figures/etude-temps-annotation-3-etude-de-cas}
				\caption{Etude de cas d'évolution de la vitesse d'annotation de contraintes (en contraintes par minutes) en fonction des différentes sessions d'annotations}
				\label{figure:4.3.1-ETUDE-COUTS-TEMPS-ANNOTATION-EXEMPLE}
			\end{figure}

		%%% Discussion.
		\subsubsection{Discussion}
		
		% Généralités sur la modélisation du temps d'annotation sur une session.
		L'étude réalisée avec $14$ annotateurs sur des lots de $1~000$ contraintes a permis d'estimer à $7.8 \cdot \texttt{batch\_size}$ le temps nécessaire (en secondes) pour annoter un lot de contraintes (cf. figure~\ref{figure:4.3.1-ETUDE-COUTS-TEMPS-ANNOTATION-SIMULATION})
		
		% Compliqué de comparer ...
		\begin{leftBarAuthorOpinion}
			Avant poursuivre la discussion, il est nécessaire de préciser qu'il est compliqué de comparer ces résultats.
			% Forte disparité des mesures.
			D'une part, il y a une forte disparité des mesures, et il est idyllique de penser qu'une étude sur $14$ annotateurs peut représenter la diversité du comportement humain sur une tâche aussi complexe que l'annotation de données textuelles.
			% Peu de repères concrets.
			D'autre part, il y a un manque de repères concrets dans la littérature scientifique, entre autre à cause des nombreux facteurs intervenant dans une tâche d'annotation (\textit{objectifs à réaliser, données à manipuler, nombre de choix proposés à l'opérateur, complexité sémantique des données, des compétences de l'opérateur, fréquence d'exécution de la tâche, ...}), mais aussi en raison du manque d'intérêt du temps nécessaire au profit de l'analyse de la cohérence et de la qualité intra-ou-inter-annotateur.
			% Diversité des données.
			De plus, les résultats peuvent différer en fonction des contraintes à caractériser : on peut supposer que des couples de données très similaires ou très différentes sont simples à annoter, mais que des données plus ambiguës peuvent nécessiter davantage de temps pour être intégrées et étiquetées.
			
			% Angle d'attaque.
			Pour pallier ce problème, nous proposons de comparer nos estimations de temps d'annotations grâce aux pistes ci-dessous. Certes, ces repères sont approximatifs, mais ils nous permettront de discuter des ordres de grandeurs à manipuler.
		\end{leftBarAuthorOpinion}
		
		% 8 secondes, ce n'est pas "court".
		\todo[inline]{A REDIGER: 8 secondes c'est pas courts}
			%%%% (OLD) tache courte : caractérisation rapide d'une similitude ou d'une différence entre deux données \todo{citation: Kahneman (2011), Hancock (1988)}.
			%%%% comparé à 5 secondes, il y a quelques chose en plus
			%%%% hypothèse 1 : il y a un traitement cognitif en plus
			%%%% hypothèse 2 : il y a un problème applicatif
			
		% ... mais 8 secondes, c'est mieux que 17 secondes.
		\todo[inline]{A REDIGER: 8 secondes c'est mieux que 17}
			% Article Cheap and Fast — But is it Good? (Word Sense Disambiguation)
			%%%% Selon~\todo{citation: Snow et al. (2008) / Yuret (2007)}, qui délègue à \texttt{Amazon Mechanical Turk} l'annotation de phrases pour catégoriser leur contexte parmi trois possibilités préformatés, il faut $8.59$ heures pour étiqueter $1~770$ données, soit $17.8$ secondes par annotation ;
			%%%% ajouter du contexte du AMT (pourquoi c'est utilisable, sont-ils payés, ...)
			
		
		% Autres analyses sans conclusions.
		Sur d'autres aspects, nous avons analysé l'évolution de la vitesse d'annotation au cours des sessions d'annotation, en espérant observer une accélération des annotations au fur et à mesure que l'annotateur s'habitue avec la tâche, ainsi qu'un effet de fatigue pour des sessions d'annotations trop longues.
		Cependant, aucune de nos analyses n'a montré de résultats significatifs (on peut constater la forte dispersion des résultats grâce à la figure~\ref{figure:4.3.1-ETUDE-COUTS-TEMPS-ANNOTATION-EXEMPLE}).
		Nous ne pouvons donc pas conclure sur de telles tendances.
		\begin{leftBarAuthorOpinion}
			Nos intuitions initiales concernaient deux points :
			\begin{itemize}
				\item la diminution du \textbf{temps d'adaptation} au cours de sessions d'annotations : au fur et à mesure qu'il annote, l'opérateur pourrait entrer plus facilement dans sa tâche, lui permettant d'atteindre plus rapidement sa vitesse de croisière et ainsi gagner en efficacité sur plusieurs sessions. D'après (CITATION: Anderson (2013)), ce temps d'adaptation pourrait se définir en trois étapes : une phase déclarative (\textit{besoin d'instructions détaillées, exécution lente et avec erreurs}), une phase associative (\textit{quelques rappels clés suffisent pour retrouver les instructions, donc gain de vitesse}) et une phase autonome (\textit{les consignes sont acquises, donc exécution rapide et sans erreur}) ;
				\item l'intervention d'un \textbf{effet de fatigue} : si une session d'annotation dure trop longtemps, l'opérateur pourrait perdre en efficacité par manque de concentration et augmenter ses chances de faire des erreurs. D'après (CITATION: Jones et al. (2015)), la fatigue est considérée comme un inconfort qui s'installe après une tâche excessive, et (CITATION: Elkosantini et Gien (2009)) décrit cet état de fatigue par des capacités de travail réduites.
			\end{itemize}
			Ces différentes intuitions ont aussi été remontées par les annotateurs de notre expériences, mais aucun effet significatif n'a pu être observé.
		\end{leftBarAuthorOpinion}
		\todo{citation temps d'adaptation: Anderson (2013)}
		\todo{citation effet de fatigue: Jones et al. (2015), Elkosantini et Gien (2009)}
		
		% remarque sur le nombre médian de contraintes à annoter
		\todo[inline]{A REDIGER: taille batch}
		Par extension, nous ne pouvons pas non plus conclure sur la taille optimale d’échantillon de contraintes à sélection pour une session d'annotation.
		Toutefois, ...
			%%%% nombre médian de contraintes annotées durant les sessions d'annotation de cette expérience est de $138$ (moyenne à $170.70$), nous pouvons par exemple fixer la taille par défaut des lots à $100$ contraintes ($14.9$ minutes) ou $150$ contraintes ($20.7$ minutes).
			%%%% Attention toutefois à ne pas faire des lots trop conséquents (au delà de $200$), d'une part pour garder l'aspect itératif et interactif de la méthode, et d'autre part pour ne pas atteindre un pallier de \textbf{fatigue de} l'annotateur.\todo{citation: Jones et al. (2015), Elkosantini et Gien (2009)}
		
		% Autres remontées applicatives.
		\todo[inline]{A REDIGER: ergo appli}
		Enfin, diverses remontées des opérateurs de l'expérience concerne l'ergonomie de l'application.
		En effet, il est logique de penser que la conception du logiciel peut grandement impacté 
	

	%%%
	%%% Subsection 4.3.2: Étude du temps de calcul nécessaire aux algorithmes implémentés en chronométrant des exécutions dans différentes situations
	%%%
	\subsection{Étude du temps de calcul nécessaire aux algorithmes implémentés en chronométrant des exécutions dans différentes situations}
	\label{section:4.3.2-ETUDE-COUTS-TEMPS-CALCUL}
	
		%%% Protocole expérimental.
		\subsubsection{Protocole expérimental}
		
			% Transition.
			Maintenant que nous avons pu modéliser le temps nécessaire à un expert pour annoter un lot de contraintes, nous nous intéressons au temps nécessaire à la machine pour interpréter ces annotations et proposer une nouvelle segmentation des données.
			
			% Objectif de l'expérience.
			Pour cela, nous allons chronométrer plusieurs exécutions des algorithmes intervenant dans notre implémentation du \textit{clustering} interactif, et nous évaluerons l'importance de leurs différents arguments d'entrée (la taille du jeu de données, le nombre de clusters et le nombre de contraintes annotées, ...).
			Nous profiterons aussi de ces modélisations du temps de calcul pour confirmer le choix de paramétrage réalisé lors de l'étude d'efficience en section~\ref{section:4.2-HYPOTHESE-EFFICIENCE}, et ainsi faire un compromis entre l'algorithme le plus efficient et l'algorithme le plus rapide.
			
			% Remarques.
			\begin{leftBarWarning}
				Pour utiliser des jeux de données de tailles différentes tout en maîtrisant leur contenu, nous avons dupliqués aléatoirement des données issues de nos jeux de référence en générant des fautes de frappes.
				Pour cette étude, nous faisons l'hypothèse que cela n'a pas d'impact majeur sur le temps d'exécution des différents algorithmes.
			\end{leftBarWarning}
			
			% Pseudo-code.
			Pour résumer le protocole expérimental que nous décrivons ci-dessous, vous pouvez vous référer aux pseudo-code décrit dans Alg.~\ref{algorithm:4.3.2-ETUDE-COUTS-TEMPS-CALCUL-PROTOCOLE}.
			%
			\begin{algorithm}[!htb]
				\begin{algorithmic}[1]
					\Require jeux de données annotés (vérité terrain) de tailles différentes
					\ForAll{arrangement d'algorithmes et de paramètres à tester}
						\State \textbf{initialisation}: récupérer ou générer le jeu de données
						\If{estimation de la tâche de \textbf{prétraitement}}
							\State \textbf{chronomètre: START}
							\State \textbf{prétraitement (à étudier)}: supprimer le bruit dans les données
							\State \textbf{chronomètre: STOP}
						\ElsIf {estimation de la tâche de \textbf{vectorisation}}
							\State \textbf{prétraitement}: supprimer le bruit dans les données avec \texttt{prep.simple}
							\State \textbf{chronomètre: START}
							\State \textbf{vectorisation (à étudier)}: transformer les données en vecteurs
							\State \textbf{chronomètre: STOP}
						\ElsIf {estimation de la tâche de \textbf{clustering}}
							\State \textbf{prétraitement}: supprimer le bruit dans les données avec \texttt{prep.simple}
							\State \textbf{vectorisation}: transformer les données en vecteurs avec \texttt{vect.tfidf}
							\State \textbf{échantillonnage initial}: sélectionner une base de contraintes avec \texttt{samp.rand.full}
							\State \textbf{simulation d'annotation}: ajouter des contraintes en utilisant la vérité terrain
							\State \textbf{chronomètre: START}
							\State \textbf{clustering (à étudier)}: regrouper les données par similarité
							\State \textbf{chronomètre: STOP}
						\ElsIf {estimation de la tâche d'\textbf{échantillonnage}}
							\State \textbf{prétraitement}: supprimer le bruit dans les données avec \texttt{prep.simple}
							\State \textbf{vectorisation}: transformer les données en vecteurs avec \texttt{vect.tfidf}
							\State \textbf{échantillonnage initial}: sélectionner une base de contraintes avec \texttt{samp.rand.full}
							\State \textbf{simulation d'annotation}: ajouter des contraintes en utilisant la vérité terrain
							\State \textbf{clustering initial}: regrouper les données par similarité avec \texttt{clust.kmeans.cop}
							\State \textbf{chronomètre: START}
							\State \textbf{échantillonnage (à étudier)}: sélectionner de nouvelles contraintes à annoter
							\State \textbf{chronomètre: STOP}
						\EndIf
						\State \textbf{mesure}: estimer la différence de chronomètre pour cet algorithme
					\EndFor
					\ForAll{algorithme à modéliser}
						\State \textbf{cadrage}: définir les facteurs et les interactions intervenant dans la modélisation
						\State \textbf{simplification}: restreindre la modélisation aux facteurs les plus corrélés
						\State \textbf{modélisation}: entraîner un modèle linéaire généralisé avec les facteurs retenus
						\State \textbf{simulation}: écrire l'équation du temps d'exécution avec des paramètres obtenus
					\EndFor
					\Ensure modélisation du temps d'exécution des différents algorithmes
				\end{algorithmic}
				\caption{Description en pseudo-code du protocole expérimental de l'étude du temps d'exécution des algorithmes du \textit{clustering} interactif}
				\label{algorithm:4.3.2-ETUDE-COUTS-TEMPS-CALCUL-PROTOCOLE}
			\end{algorithm}
			
			% Description de la vérité terrain.
			Nous utiliserons deux vérités terrains comme références pour cette expérience :
			\begin{itemize}
				\item le jeu de données \texttt{bank cards (v2.0.0)} : ce dernier traite des demandes les plus fréquentes des clients en ce qui concerne la gestion de leur carte bancaire. Il est composé de $1~000$ questions rédigées en français et réparties en $10$ classes (\texttt{perte ou vol de carte}, \texttt{carte avalée}, \texttt{commande de carte}, ...). Pour plus de détails, consultez l'annexe~\ref{annex:C.1-DATASET-BANK-CARDS} ;
				\item le jeu de données \texttt{MLSUM FR TRAIN SUBSET (v1.0.0-schild)} : ce dernier concerne les titres d'articles de journaux issus des catégories de publication les plus communes. Il est composé de $744$  titres d'articles rédigés et répartis en $14$ classes (\textit{économie}, \textit{sport}, ...). Pour plus de détails, consultez l'annexe~\ref{annex:C.2-DATASET-MLSUM-SUBSET-SCHILD} ;
			\end{itemize}
			
			% Description des tâches, des algorithmes et des contextes.
			Pour cette étude, nous lançons plusieurs exécutions de chaque algorithme de notre implémentation du \textit{clustering} interactif (cf. section~\ref{section:3.3-DESCRIPTION-IMPLEMENTATION}) avec différentes variations de contexte d'utilisation. Cela comprend les tâches, algorithmes et contextes d'utilisation suivants :
			%
			\begin{enumerate}
				% Prétraitement.
				\item le \textbf{prétraitement} des données...
					\begin{itemize}
						\item avec les algorithmes suivants : \textbf{simple} (noté \texttt{prep.simple}), \textbf{avec lemmatisation} (noté \texttt{prep.lemma}) et \textbf{avec filtres} (noté \texttt{prep.filter}) ;
						\item avec les contextes d'utilisation suivants : \textbf{nombre de données} (variant de $1~000$ à $5~000$ par pas de $1~000$, noté $\texttt{dataset\_size}$) ;
					\end{itemize}
				% Vectorisation.
				\item la \textbf{vectorisation} des données...
					\begin{itemize}
						\item avec les algorithmes suivants : \textbf{TF-IDF} (noté \texttt{vect.tfidf}) et \textbf{SpaCy} (noté \texttt{vect.frcorenewsmd}) ;
						\item avec les contextes d'utilisation suivants : \textbf{nombre de données} (variant de $1~000$ à $5~000$ par pas de $1~000$, noté $\texttt{dataset\_size}$) ;
						\item précédé par un prétraitement \textbf{simple} ;
					\end{itemize}
				% Clustering.
				\item le \textbf{\textit{clustering} sous contraintes} des données...
					\begin{itemize}
						\item avec les algorithmes suivants : \textbf{KMeans} (modèle \textit{COP} noté \texttt{clust.kmeans.cop}), \textbf{Hiérarchique} (lien \textit{single} noté \texttt{clust.hier.sing} ; lien \textit{complete} noté \texttt{clust.hier.comp} ; lien \textit{average} noté \texttt{clust.hier.avg} ; lien \textit{ward} noté \texttt{clust.hier.ward}) et \textbf{Spectral} (modèle \textit{SPEC} noté \texttt{clust.spec}) ;
						\item avec les contextes d'utilisation suivants : \textbf{nombre de données} (variant de $1~000$ à $5~000$ par pas de $1~000$, noté $\texttt{dataset\_size}$), le \textbf{nombre de contraintes annotés} (variant de $0$ à $5~000$ par pas de $500$, noté $\texttt{previous\_nb\_constraints}$) et le \textbf{nombre de \textit{clusters} à trouver} (variant de $5$ à $50$ par pas de $5$, noté $\texttt{algorithm\_nb\_clusters}$) ;
						\item précédé par un prétraitement \textbf{simple} et une vectorisation \textbf{TF-IDF} et un échantillonnage initial \textbf{purement aléatoire} ;
					\end{itemize}
				% Sampling.
				\item l'\textbf{échantillonnage} des contraintes à annoter...
					\begin{itemize}
						\item avec les algorithmes suivants : \textbf{purement aléatoire} (noté \texttt{samp.random.full}), \textbf{pseudo-aléatoire} (noté \texttt{samp.random.same}), \textbf{même cluster et étant les plus éloignées} (noté \texttt{samp.farhtest.same}) et \textbf{clusters différents et étant les plus proches} (noté \texttt{samp.closest.diff}) ;
						\item avec les contextes d'utilisation suivants : \textbf{nombre de données} (variant de $1~000$ à $5~000$ par pas de $1~000$, noté $\texttt{dataset\_size}$), le \textbf{nombre de contraintes annotés} (variant de $0$ à $5~000$ par pas de $500$, noté $\texttt{previous\_nb\_constraints}$), le \textbf{nombre de \textit{clusters} existant} (variant de $10$ à $50$ par pas de $10$, noté $\texttt{previous\_nb\_clusters}$) et le \textbf{nombre de contraintes à sélectionner} (variant de $50$ à $250$ par pas de $50$, noté $\texttt{algorithm\_nb\_constraints}$) ;
						\item précédé par un prétraitement \textbf{simple}, une vectorisation \textbf{TF-IDF}, un \textit{clustering} initial \textbf{KMeans} (modèle \textit{COP}) et un échantillonnage initial \textbf{purement aléatoire} ;
					\end{itemize}
			\end{enumerate}
			
			Il y a donc $8~825$ combinaisons d'algorithmes (\texttt{15} pour le prétraitement, $10$ pour la vectorisation, $3~330$ pour le \textit{clustering}, $5~550$ pour l'échantillonnage), et chaque combinaison est répétée $5$ fois pour contrer les aléas statistiques des exécutions.
			De plus, chaque jeu de données est généré $5$ fois pour contrer les aléas statistiques de création, donc il y a $220~625$ exécutions d'algorithmes ($375$ pour le prétraitement, $250$ pour la vectorisation, $82~500$ pour le \textit{clustering}, $137~500$ pour l'échantillonnage).
			
			% Description de la modélisation.
			Sur la base de ces mesures, nous cherchons à modéliser le temps d'exécution de chaque algorithme en fonction de son contexte d'utilisation (dépendant de ses arguments d'entrée), et les interactions doubles entre paramètres sont envisagées.
			Afin de réduire la complexité des modélisations, nous ordonnons les interactions de facteurs possibles en fonction de leur corrélation avec le temps mesuré (la corrélation \texttt{r} de \textit{Pearson}\todo{CITATION} est utilisée) et nous nous limitons aux variables responsables d'un maximum de la variance des mesures (la méthode d'\textit{Elbow}\todo{CITATION} est utilisée pour choisir les facteurs pertinents).
			Sur cette base, nous entraînons un modèle linéaire généralisé (\textit{GLM}) pour représenter le temps d'exécution moyen de l'algorithme : ce modèle sera caractérisé par le coefficient de détermination généralisé \texttt{R²} de \textit{Cox et Snel}\todo{CITATION}, la log-vraisemblance \texttt{llf}\todo{CITATION} et la log-vraisemblance \texttt{llf\_null} du modèle \textit{null}.
			Pour finir, nous discuterons des valeurs des coefficients obtenus sur l'impact du temps d'exécution.
			
			% Référence scripts.
			\begin{leftBarInformation}
				Ces analyses sont réalisées en Python à l'aide des librairies \texttt{datetime} et \texttt{statsmodels} (\cite{seabold:2010}).
				Les scripts de l'expérience (\textit{notebooks} Python) sont disponibles dans un dossier dédié de~\cite{schild:cognitivefactory-interactive-clustering-comparative-study:2021}.
			\end{leftBarInformation}

		%%% Résultats.
		\subsubsection{Résultats obtenus}
				
			%%% Prétraitements
			
			% Première analyse.
			En ce qui concerne la tâche de \textbf{prétraitement}, une première analyse montre que les modélisations des trois implémentations sont similaires (\texttt{p-valeur}: $> 0.980$). Nous ferons donc une seule modélisation.
			
			% Modélisation du temps de calcul (prep.simple + prep.lemma + prep.filter).
			Pour les algorithmes de prétraitements (\texttt{prep.simple}, \texttt{prep.lemma} et \texttt{prep.filter}), l'analyse de la corrélation des facteurs avec les mesures de temps d'exécution indique qu'une modélisation minimale et suffisante peut être réalisée à partir du facteur $\texttt{dataset\_size}$ (\texttt{r}: $0.997$).
			Le modèle linéaire généralisé retenu (\texttt{R²}: $> 0.999$, \texttt{llf}: $-432.43$, \texttt{llf\_null}: $-1~353.98$) nous permet de déduire l'équation suivante\todo{ref complexité théorique algo en annexe} :
			%
			\begin{equation}
				\texttt{computation\_time}(\texttt{prep})~[s]~
				\propto~6.55 \cdot 10^{-3} \cdot \texttt{dataset\_size}
			\end{equation}
			
			% Affichage du temps de calcul.
			La figure~\ref{figure:4.3.2-ETUDE-COUTS-TEMPS-CALCUL-MODELISATION-PREPROCESSING} représente cette modélisation du temps de calcul des algorithmes de prétraitements en comparaison avec les mesures réalisées lors de l'expérience.
			\newline
			%		
			\begin{figure}[!htb]
				\centering
				\includegraphics[width=0.8\textwidth]{figures/etude-temps-calcul-modelisation-1prep}
				\caption{Estimation du temps nécessaire (en minutes) pour effectuer une tâche de \textbf{prétraitement} en fonction du nombre de données à traiter. Les paramétrages \texttt{prep.simple}, \texttt{prep.lemma} et \texttt{prep.filter} ayant des temps de calculs similaires, leurs modélisations n'ont pas été séparées.}
				\label{figure:4.3.2-ETUDE-COUTS-TEMPS-CALCUL-MODELISATION-PREPROCESSING}
			\end{figure}
			
			%%% Vectorization
			
			% Première analyse.
			En ce qui concerne la tâche de \textbf{vectorisation}, une première analyse montre que les modélisations des deux implémentations sont différentiables  (\texttt{p-valeur}: $< 10^{-3}$). Nous ferons donc une modélisation par algorithme.
		
			% Modélisation du temps de calcul (vect.tfidf).
			Pour les algorithmes de vectorisation \texttt{vect.tfidf}, l'analyse de la corrélation des facteurs avec les mesures de temps d'exécution indique qu'une modélisation minimale et suffisante peut être réalisée à partir du facteur $\texttt{dataset\_size}$ (\texttt{r}: $0.977$).
			Le modèle linéaire généralisé retenu (\texttt{R²}: $> 0.999$, \texttt{llf}: $259.89$, \texttt{llf\_null}: $70.04$) nous permet de déduire l'équation suivante\todo{ref complexité théorique algo en annexe} :
			%
			\begin{equation}
				\texttt{computation\_time}(\texttt{vect.tfidf})~[s]~
				\propto~-9.16 \cdot 10^{-5} \cdot \texttt{dataset\_size}
			\end{equation}
			
			% Modélisation du temps de calcul (vect.frcorenewsmd).
			Pour les algorithmes de vectorisation \texttt{vect.frcorenewsmd}, l'analyse de la corrélation des facteurs avec les mesures de temps d'exécution indique qu'une modélisation minimale et suffisante peut être réalisée à partir du facteur $\texttt{dataset\_size}$ (\texttt{r}: $0.983$).
			Le modèle linéaire généralisé retenu (\texttt{R²}: $> 0.999$, \texttt{llf}: $-214.44$, \texttt{llf\_null}: $-399.39$) nous permet de déduire l'équation suivante\todo{ref complexité théorique algo en annexe} :
			%
			\begin{equation}
				\texttt{computation\_time}(\texttt{vect.frcorenewsmd})~[s]~
				\propto~4.62 \cdot 10^{-3} \cdot \texttt{dataset\_size}
			\end{equation}
			
			% Affichage du temps de calcul.
			La figure~\ref{figure:4.3.2-ETUDE-COUTS-TEMPS-CALCUL-MODELISATION-VECTORIZATION} représente ces modélisations de temps de calcul des algorithmes de vectorisation en comparaison avec les mesures réalisées lors de l'expérience.
			\newline
			%
			\begin{figure}[!htb]
				\centering
				\includegraphics[width=0.8\textwidth]{figures/etude-temps-calcul-modelisation-2vect}
				\caption{Estimation du temps nécessaire (en minutes) pour effectuer une tâche de \textbf{vectorisation} en fonction du nombre de données à traiter.}
				\label{figure:4.3.2-ETUDE-COUTS-TEMPS-CALCUL-MODELISATION-VECTORIZATION}
			\end{figure}
			
			%%% Clustering
			
			% Première analyse.
			En ce qui concerne la tâche de \textbf{\textit{clustering} sous contraintes}, une première analyse montre que les modélisations des six implémentations sont différentiables  (\texttt{p-valeur}: $<$ \texttt{$10^{-3}$}). Nous ferons donc une modélisation par algorithme.
			
			% Remarques: hiérarchique trop long.
			\begin{leftBarWarning}
				Plusieurs exécutions des algorithmes de type \textit{hiérarchique} ont été annulées pour les jeux données de tailles supérieures à $4~000$ car la durée excédé plusieurs heures.
				Nous limitons dons l'analyse de \texttt{clust.hier.sing}, \texttt{clust.hier.comp}, \texttt{clust.hier.avg} et \texttt{clust.hier.ward} aux tailles de $1~000$ à $3~000$.
			\end{leftBarWarning}
			
			% Modélisation du temps de calcul (clust.kmeans.cop).
			Pour les algorithmes du \textit{clustering} sous contraintes \texttt{clust.kmeans.cop}, l'analyse de la corrélation des facteurs avec les mesures de temps d'exécution indique qu'une modélisation minimale et suffisante peut être réalisée à partir du facteur $\texttt{dataset\_size}$ (\texttt{r}: $0.837$).
			Le second facteur le plus corrélé (mais non retenu) est l'interaction $\texttt{dataset\_size}^{2} \cdot algorithm\_nb\_clusters$ (\texttt{r}: $0.545$).
			Le modèle linéaire généralisé retenu (\texttt{R²}: $0.802$, \texttt{llf}: $-9.37 \cdot 10^{4}$, \texttt{llf\_null}: $-1.00 \cdot 10^{5}$) nous permet de déduire l'équation suivante\todo{ref complexité théorique algo en annexe} :
			%
			\begin{equation}
				\texttt{computation\_time}(\texttt{clust.kmeans.cop})~[s]~
				\propto~1.45 \cdot 10^{-1} \cdot \texttt{dataset\_size}
			\end{equation}
			
			% Modélisation du temps de calcul (clust.hier.sing).
			Pour les algorithmes du \textit{clustering} sous contraintes \texttt{clust.hier.sing}, l'analyse de la corrélation des facteurs avec les mesures de temps d'exécution indique qu'une modélisation minimale et suffisante peut être réalisée à partir du facteur $\texttt{dataset\_size}^{2}$ (\texttt{r}: $0.940$).
			Le second facteur le plus corrélé (mais non retenu) est l'interaction $\texttt{dataset\_size}^{2} \cdot algorithm\_nb\_clusters$ (\texttt{r}: $0.729$).
			Le modèle linéaire généralisé retenu (\texttt{R²}: $0.987$, \texttt{llf}: $-5.54 \cdot 10^{4}$, \texttt{llf\_null}: $-6.10 \cdot 10^{4}$) nous permet de déduire l'équation suivante\todo{ref complexité théorique algo en annexe} :
			%
			\begin{equation}
				\texttt{computation\_time}(\texttt{clust.hier.sing})~[s]~
				\propto~-5.00 \cdot 10^{-4} \cdot \texttt{dataset\_size}^{2}
			\end{equation}
			
			% Modélisation du temps de calcul (clust.hier.comp).
			Pour les algorithmes du \textit{clustering} sous contraintes \texttt{clust.hier.comp}, l'analyse de la corrélation des facteurs avec les mesures de temps d'exécution indique qu'une modélisation minimale et suffisante peut être réalisée à partir du facteur $\texttt{dataset\_size}^{2}$ (\texttt{r}: $0.938$).
			Le second facteur le plus corrélé (mais non retenu) est l'interaction $\texttt{dataset\_size}^{2} \cdot algorithm\_nb\_clusters$ (\texttt{r}: $0.736$).
			Le modèle linéaire généralisé retenu (\texttt{R²}: $0.984$, \texttt{llf}: $-5.56 \cdot 10^{4}$, \texttt{llf\_null}: $-6.11 \cdot 10^{4}$) nous permet de déduire l'équation suivante\todo{ref complexité théorique algo en annexe} :
			%
			\begin{equation}
				\texttt{computation\_time}(\texttt{clust.hier.comp})~[s]~
				\propto~-4.99 \cdot 10^{-4} \cdot \texttt{dataset\_size}^{2}
			\end{equation}

			% Modélisation du temps de calcul (clust.hier.avg).
			Pour les algorithmes du \textit{clustering} sous contraintes \texttt{clust.hier.avg}, l'analyse de la corrélation des facteurs avec les mesures de temps d'exécution indique qu'une modélisation minimale et suffisante peut être réalisée à partir du facteur $\texttt{dataset\_size}^{2}$ (\texttt{r}: $0.915$).
			Le second facteur le plus corrélé (mais non retenu) est l'interaction $\texttt{dataset\_size}^{2} \cdot algorithm\_nb\_clusters$ (\texttt{r}: $0.713$).
			Le modèle linéaire généralisé retenu (\texttt{R²}: $0.981$, \texttt{llf}: $-5.90 \cdot 10^{4}$, \texttt{llf\_null}: $-6.45 \cdot 10^{4}$) nous permet de déduire l'équation suivante\todo{ref complexité théorique algo en annexe} :
			%
			\begin{equation}
				\texttt{computation\_time}(\texttt{clust.hier.avg})~[s]~
				\propto~-8.51 \cdot 10^{-4} \cdot \texttt{dataset\_size}^{2}
			\end{equation}

			% Modélisation du temps de calcul (clust.hier.ward).
			Pour les algorithmes du \textit{clustering} sous contraintes \texttt{clust.hier.ward}, l'analyse de la corrélation des facteurs avec les mesures de temps d'exécution indique qu'une modélisation minimale et suffisante peut être réalisée à partir du facteur $\texttt{dataset\_size}^{2}$ (\texttt{r}: $0.945$).
			Le second facteur le plus corrélé (mais non retenu) est l'interaction $\texttt{dataset\_size}^{2} \cdot algorithm\_nb\_clusters$ (\texttt{r}: $0.734$).
			Le modèle linéaire généralisé retenu (\texttt{R²}: $0.989$, \texttt{llf}: $-5.57 \cdot 10^{4}$, \texttt{llf\_null}: $-6.14 \cdot 10^{4}$) nous permet de déduire l'équation suivante\todo{ref complexité théorique algo en annexe} :
			%
			\begin{equation}
				\texttt{computation\_time}(\texttt{clust.hier.ward})~[s]~
				\propto~-5.30 \cdot 10^{-4} \cdot \texttt{dataset\_size}^{2}
			\end{equation}
			
			% Modélisation du temps de calcul (clust.spec).
			Pour les algorithmes du \textit{clustering} sous contraintes \texttt{clust.spec}, l'analyse de la corrélation des facteurs avec les mesures de temps d'exécution indique qu'une modélisation minimale et suffisante peut être réalisée à partir du facteur $\texttt{dataset\_size}^{2}$ (\texttt{r}: $0.658$).
			Le second facteur le plus corrélé (mais non retenu) est l'interaction $\texttt{dataset\_size}^{2} \cdot algorithm\_nb\_clusters$ (\texttt{r}: $0.595$).
			Le modèle linéaire généralisé retenu (\texttt{R²}: $0.527$, \texttt{llf}: $-7.89 \cdot 10^{5}$, \texttt{llf\_null}: $-8.27 \cdot 10^{5}$) nous permet de déduire l'équation suivante\todo{ref complexité théorique algo en annexe} :
			%
			\begin{equation}
				\texttt{computation\_time}(\texttt{clust.spec})~[s]~
				\propto~8.18 \cdot 10^{-6} \cdot \texttt{dataset\_size}^{2}
			\end{equation}
			
			% Affichage du temps de calcul.
			La figure~\ref{figure:4.3.2-ETUDE-COUTS-TEMPS-CALCUL-MODELISATION-CLUSTERING} représente ces modélisations de temps de calcul des algorithmes de \textit{clustering} en comparaison avec les mesures réalisées lors de l'expérience.
			\newline
			%
			\begin{figure}[!htb]
				\centering
				\includegraphics[width=0.8\textwidth]{figures/etude-temps-calcul-modelisation-3clust}
				\caption{Estimation du temps nécessaire (en minutes) pour effectuer une tâche de \textbf{clustering} en fonction du nombre de données à traiter.}
				\label{figure:4.3.2-ETUDE-COUTS-TEMPS-CALCUL-MODELISATION-CLUSTERING}
			\end{figure}
			
			%%% Sampling
			
			% Première analyse.
			En ce qui concerne la tâche d'\textbf{échantillonnage de contraintes}, une première analyse montre que les modélisations des quatre implémentations sont différentiables  (\texttt{p-valeur}: $<$ \texttt{$10^{-3}$}). Nous ferons donc une modélisation par algorithme.
			
			% Modélisation du temps de calcul (samp.rand.full).
			Pour les algorithmes de l'échantillonnage de contraintes \texttt{samp.rand.full}, l'analyse de la corrélation des facteurs avec les mesures de temps d'exécution indique qu'une modélisation minimale et suffisante peut être réalisée à partir du facteur $\texttt{dataset\_size}^{2}$ (\texttt{r}: $0.993$).
			Le second facteur le plus corrélé (mais non retenu) est l'interaction $\texttt{dataset\_size}^{2} \cdot previous\_nb\_clusters$ (\texttt{r}: $0.791$).
			Le modèle linéaire généralisé retenu (\texttt{R²}: $> 0.999$, \texttt{llf}: $-4.52 \cdot 10^{4}$, \texttt{llf\_null}: $-1.17 \cdot 10^{5}$) nous permet de déduire l'équation suivante\todo{ref complexité théorique algo en annexe} :
			%
			\begin{equation}
				\texttt{computation\_time}(\texttt{samp.rand.full})~[s]~
				\propto~-8.20 \cdot 10^{-7} \cdot \texttt{dataset\_size}^{2}
			\end{equation}
			
			% Modélisation du temps de calcul (samp.rand.same).
			Pour les algorithmes de l'échantillonnage de contraintes \texttt{samp.rand.same}, l'analyse de la corrélation des facteurs avec les mesures de temps d'exécution indique qu'une modélisation minimale et suffisante peut être réalisée à partir du facteur $\texttt{dataset\_size}^{2}$ (\texttt{r}: $0.939$).
			Le second facteur le plus corrélé (mais non retenu) est l'interaction $\texttt{dataset\_size}^{2} \cdot algorithm\_nb\_constraints$ (\texttt{r}: $0.611$).
			Le modèle linéaire généralisé retenu (\texttt{R²}: $> 0.999$, \texttt{llf}: $-3.20 \cdot 10^{4}$, \texttt{llf\_null}: $-6.84 \cdot 10^{4}$) nous permet de déduire l'équation suivante\todo{ref complexité théorique algo en annexe} :
			%
			\begin{equation}
				\texttt{computation\_time}(\texttt{samp.rand.same})~[s]~
				\propto~1.85 \cdot 10^{-7} \cdot \texttt{dataset\_size}^{2}
			\end{equation}
			
			% Modélisation du temps de calcul (samp.farhtest.same).
			Pour les algorithmes de l'échantillonnage de contraintes \texttt{samp.farhtest.same}, l'analyse de la corrélation des facteurs avec les mesures de temps d'exécution indique qu'une modélisation minimale et suffisante peut être réalisée à partir du facteur $\texttt{dataset\_size}^{2}$ (\texttt{r}: $0.981$).
			Le second facteur le plus corrélé (mais non retenu) est l'interaction $\texttt{dataset\_size}^{2} \cdot previous\_nb\_clusters$ (\texttt{r}: $0.700$).
			Le modèle linéaire généralisé retenu (\texttt{R²}: $> 0.999$, \texttt{llf}: $-4.56 \cdot 10^{4}$, \texttt{llf\_null}: $-1.02 \cdot 10^{5}$) nous permet de déduire l'équation suivante\todo{ref complexité théorique algo en annexe} :
			%
			\begin{equation}
				\texttt{computation\_time}(\texttt{samp.farhtest.same})~[s]~
				\propto~5.19 \cdot 10^{-7} \cdot \texttt{dataset\_size}^{2}
			\end{equation}
			
			% Modélisation du temps de calcul (samp.closest.diff).
			Pour les algorithmes de l'échantillonnage de contraintes \texttt{samp.closest.diff}, l'analyse de la corrélation des facteurs avec les mesures de temps d'exécution indique qu'une modélisation minimale et suffisante peut être réalisée à partir du facteur $\texttt{dataset\_size}^{2}$ (\texttt{r}: $0.995$).
			Le second facteur le plus corrélé (mais non retenu) est l'interaction $\texttt{dataset\_size}^{2} \cdot previous\_nb\_clusters$ (\texttt{r}: $0.815$).
			Le modèle linéaire généralisé retenu (\texttt{R²}: $> 0.999$, \texttt{llf}: $-5.96 \cdot 10^{4}$, \texttt{llf\_null}: $-1.36 \cdot 10^{5}$) nous permet de déduire l'équation suivante\todo{ref complexité théorique algo en annexe} :
			%
			\begin{equation}
				\texttt{computation\_time}(\texttt{samp.closest.diff})~[s]~
				\propto~1.43 \cdot 10^{-6} \cdot \texttt{dataset\_size}^{2}
			\end{equation}
			
			% Affichage du temps de calcul.
			La figure~\ref{figure:4.3.2-ETUDE-COUTS-TEMPS-CALCUL-MODELISATION-SAMPLING} représente ces modélisations de temps de calcul des algorithmes d'échantillonnage en comparaison avec les mesures réalisées lors de l'expérience.
			\newline
			%
			\begin{figure}[!htb]
				\centering
				\includegraphics[width=0.8\textwidth]{figures/etude-temps-calcul-modelisation-4samp}
				\caption{Estimation du temps nécessaire (en minutes) pour effectuer une tâche d'\textbf{échantillonnage de contraintes} en fonction du nombre de données à traiter.}
				\label{figure:4.3.2-ETUDE-COUTS-TEMPS-CALCUL-MODELISATION-SAMPLING}
			\end{figure}

		%%% Discussion.
		\subsubsection{Discussion}
		
			% Rappel de l'objectif : estimer le temps d'exécution.
			Dans cette étude, nous avons estimé le temps de calcul des différents algorithmes implémentés afin de confirmer le choix de paramétrage pour une convergence optimal (cf. hypothèse d'efficience en section~\ref{section:4.2-HYPOTHESE-EFFICIENCE}).
			Ces estimations ont été réalisées sur la base de plusieurs exécutions et fonction de divers contextes d'utilisation : nombre de données, nombre de contraintes annotées, nombre de contraintes à sélectionner, nombre de \textit{clusters} existant, nombre de \textit{clusters} à trouver.
			
			% Remarque générale : Dépend principalement du nombre de données.
			En premier lieu, on peut constater que les différentes modélisations dépendent majoritairement de la taille du jeu de données manipulé ($\texttt{dataset\_size}$ ou $\texttt{dataset\_size}^{2}$) avec un score de corrélation \texttt{r} avec le temps mesuré généralement supérieur à $0.9$ et des modèles \textit{GLM} avec des coefficients de détermination généralisé \texttt{R²} généralement proches de $0.999$.
			Bien que d'autres facteurs peuvent intervenir dans ces estimations (notamment les interactions doubles entre la taille du jeu de données et le nombre de \textit{clusters} ou le nombre de contraintes), ces derniers semblent avoir un impact négligeable sur le temps d'exécution.
			
			% Note: remarque sur le nombre de contraintes.
			\begin{leftBarAuthorOpinion}
				Certains paramétrages de la méthode du \textit{clustering} interactif semblent cependant avoir un temps de calcul décroissant au cours des itérations, mais nous n'avons cependant pas pu montrer de tendances globales significatives.
				Il est probable que l'ajout de contraintes judicieusement placées permettent à certains algorithmes de \textit{clustering} de s'exécuter plus rapidement, notamment lorsque ceux-ci exploitent les composants connexes du graphe de contraintes (cf. section~\ref{section:3.3.2-GESTION-DES-CONTRAINTES}). En effet, :
				\begin{itemize}
					\item les \textit{clustering} hiérarchiques s'initialisent autant de \textit{clusters} que de groupes de données liées entre elles par des contraintes \texttt{MUST-LINK} : or s'il y a plus de contraintes, alors les composants connexes sont davantage développés, donc il y a moins de \textit{clusters} à initialiser et donc moins d'époques de l'algorithme ;
					\item le \textit{clustering} KMeans (modèle COP) attire auprès d'un barycentre l'ensemble des données liées par un \texttt{MUST-LINK} : or s'il y a plus de contraintes, alors il y a des données attirées, donc les noyaux de \textit{clusters} peuvent se stabiliser plus rapidement.  
				\end{itemize}
				Toutefois, ces suppositions n'ont pas pu être démontrées, et certains contre-exemples tendent à conclure que ces comportements sont très dépendants du jeu de données manipulé et de l'ordre d'ajout des contraintes. Par exemple :
				\begin{itemize}
					\item l'ajout d'un trop grand nombre de contraintes \texttt{CANNOT-LINK} peut engendrer un surplus de vérification pour estimer quelles formations de \textit{clusters} sont autorisées sans violer de contraintes ;
					\item l'algorithme KMeans (modèle COP) peut osciller autour de plusieurs noyaux de \textit{clusters} instables si les contraintes violent trop la similarité intrinsèque des données.
				\end{itemize}
			\end{leftBarAuthorOpinion}
			
			% Cas du clustering.
			En ce qui concerne la tâche de \textit{clustering}, on note des différences significatives dans les temps d'exécution des divers algorithmes implémentés.
			En effet, l'algorithme KMeans (modèle COP) est nettement plus rapide (complexité en $ \mathcal{O}(\texttt{dataset\_size}) $, nécessitant quelques dizaines de minutes pour $5~000$ données) que les implémentations du \textit{clustering} hiérarchique (complexité en $ \mathcal{O}(\texttt{dataset\_size}^{2}) $, nécessitant plusieurs heures dès $3~000$ données).
			Cette différence, visible en figure~\ref{figure:4.3.2-ETUDE-COUTS-TEMPS-CALCUL-MODELISATION-CLUSTERING}, a un réel impact sur l'expérience utilisateur de l'opérateur.
			En effet, bien qu'il soit théoriquement plus efficient pour atteindre une annotation suffisante (cf. hypothèse d'efficience en section~\ref{section:4.2-HYPOTHESE-EFFICIENCE}), l'usage d'un \textit{clustering} hiérarchique imposerait de longs temps d'attente à l'opérateur, interdisant des interactions rapides avec la machines.
			Or l'intérêt principal de notre méthodologie d'annotation à l'aide du \textit{clustering} interactif repose sur ces interactions homme-machine via l'ajout régulier de contraintes pertinentes (cf. hypothèse d'efficacité en section~\ref{section:4.1-HYPOTHESE-EFFICACITE}).
			Nous décidons donc d'exclure l'usage des algorithmes de \textit{clustering} hiérarchique au profit du \textit{clustering} KMeans (modèle COP).
			
			% Note: Cas du projet étudiant avec TPS.
			\begin{leftBarInformation}
				Dans le cadre du projet étudiant avec l'école Télécom Physique Strasbourg visant à implémenter d'autres algorithmes de \textit{clustering} sous contraintes, un résonnement similaire a été utilisé pour filtrer les algorithmes. Ainsi, l'implémentation de KMeans (modèle MPC) a été exclu (complexité en $ \mathcal{O}(\texttt{dataset\_size}^{3}) $) et l'implémentation de la propagation par affinité écarte la gestion des contraintes \texttt{CANNOT-LINK} pour avoir un temps d'exécution comparable au \textit{clustering} KMeans (modèle COP). L'algorithme DBScan (modèle C-DBScan) est quand à lui un rival possible avec une complexité théorique en $ \mathcal{O}(\texttt{dataset\_size}) $.
			\end{leftBarInformation}
			
			% Cas du prétraitement + vectorisation + échantillonnage.
			En ce qui concerne les tâches de prétraitements (figure~\ref{figure:4.3.2-ETUDE-COUTS-TEMPS-CALCUL-MODELISATION-PREPROCESSING}), de vectorisation (figure~\ref{figure:4.3.2-ETUDE-COUTS-TEMPS-CALCUL-MODELISATION-VECTORIZATION}), et d'échantillonnage de contraintes (cf. figure~\ref{figure:4.3.2-ETUDE-COUTS-TEMPS-CALCUL-MODELISATION-SAMPLING}) ont des complexités presque négligeables au regard des temps d'exécution du \textit{clustering} (pour $5~000$ données : moins de $2$ minute contre près de $12.1$ minutes pour \texttt{clust.kmeans.cop} et près de $3.5$ heures pour \texttt{clust.hier.sing}).
			Nous maintenons donc les paramétrages obtenus pour ces tâches en section~\ref{section:4.2-HYPOTHESE-EFFICIENCE} sans analyses complémentaires.
			
			% Conclusion.
			Pour conclure, dans l'optique d'atteindre de manière efficiente $90$\% de \texttt{v-measure}
			\footnote{$90$\% de \texttt{v-measure}: cas d'une annotation dite partielle, dont le paramétrage le plus efficient est constitué du prétraitement simple (\texttt{prep.simple}), de la vectorisation TF-IDF (\texttt{vect.tfidf}), du \textit{clustering} hiérarchique à lien moyen (\texttt{clust.hier.avg}) et de l'échantillonnage des données les plus proches dans des clusters différents (\texttt{sampl.closest.diff}}
			avec un coût global minimal, nous retenons l'usage du \textbf{paramétrage favori} constitué du prétraitement simple (\texttt{prep.simple}), de la vectorisation TF-IDF (\texttt{vect.tfidf}), du \textit{clustering} KMeans avec modèle COP (\texttt{clust.kmeans.cop}) et de l'échantillonnage des données les plus proches dans des clusters différents (\texttt{sampl.closest.diff}).
			On estime le temps d'exécution de ce paramétrage avec l'équation suivante\footnote{Temps du paramétrage favori : environ $2.6$ minutes pour $1~000$ données ; environ $13.3$ minutes pour $5~000$ données.} :
			%
			\begin{equation}
				\label{equation:4.3.2-ETUDE-COUTS-TEMPS-CALCUL-PARAMETRAGE-FAVORI}
				\texttt{computation\_time}(\texttt{settings.favorite})~[s]~
				\propto~1.52 \cdot 10^{-1} \cdot \texttt{dataset\_size} + 1.43 \cdot 10^{-6} \cdot \texttt{dataset\_size}^{2}
			\end{equation}
	
	%%%
	%%% Subsection 4.3.3: Étude du nombre de contraintes nécessaires à la convergence vers une vérité terrain pré-établie en fonction de la taille du jeu de données 
	%%%
	\subsection{Étude du nombre de contraintes nécessaires à la convergence vers une vérité terrain pré-établie en fonction de la taille du jeu de données}
	\label{section:4.3.3-ETUDE-COUT-NOMBRE-CONTRAINTES}
	
		%%% Protocole expérimental.
		\subsubsection{Protocole expérimental}
			
			% Transition.
			Avec les deux précédentes études, nous sommes capable d'estimer le temps nécessaire à un expert pour annoter des contraintes et le temps nécessaire à la machine pour proposer un nouveau \textit{clustering} adapté aux suggestions de l'expert.
			Pour poursuivre nos études et pouvoir estimer le coût total d'un projet d'annotation, il nous reste à estimer le nombre total de contraintes à devoir renseigner en fonction de la taille du jeu de données.
			
			% Objectif de l'expérience.
			Pour cela, nous allons simuler la création de cette base d'apprentissage en adaptant le protocole utilisé lors de notre étude d'efficacité (cf. section~\ref{section:4.1.1-ETUDE-CONVERGENCE}) :
			nous emploierons notre méthode de \textit{clustering} interactif avec notre \textbf{paramétrage favori}
			\footnote{Paramétrage favori (atteindre $90$\% de \texttt{v-measure} avec un coût minimal): prétraitement simple (\texttt{prep.simple}), vectorisation TF-IDF (\texttt{vect.tfidf}), \textit{clustering} KMeans avec modèle COP (\texttt{clust.kmeans.cop}) et échantillonnage des données les plus proches dans des clusters différents (\texttt{sampl.closest.diff})}
			sur des jeux de données de différentes tailles et mesurerons le nombre de contraintes nécessaires pour converger vers la vérité terrain.
			
			% Axiome et Remarque.
			\begin{leftBarWarning}
				Dans le cadre de cette étude, nous supposons que l'expert métier connaît parfaitement le domaine traité dans ce jeu de données, et qu'il est capable de caractériser sans ambiguïté la similitude entre deux données issues de cet ensemble.
				De plus, pour utiliser des jeux de données de tailles différentes tout en maîtrisant leur contenu, nous avons dupliqués aléatoirement des données issues de deux jeux de référence en générant des fautes de frappes.
				Pour cette étude, nous faisons l'hypothèse que cela n'a pas d'impact majeur sur le nombre de contraintes nécessaires pour converger vers la vérité terrain.
			\end{leftBarWarning}
			
			% Pseudo-code.
			Pour résumer le protocole expérimental que nous décrivons ci-dessous, vous pouvez vous référer au pseudo-code décrit dans Alg.~\ref{algorithm:4.3.3-ETUDE-COUT-NOMBRE-CONTRAINTES-PROTOCOLE}.
			%
			\begin{algorithm}[!htb]
				\begin{algorithmic}[1]
					\Require jeux de données annotés (vérité terrain) de tailles différentes
					\ForAll{jeux de données à tester}
						\State \textbf{initialisation (données)}: récupérer ou générer les données et la vérité terrain
						\State \textbf{initialisation (contraintes)}: créer une liste vide de contraintes
						\State \textbf{prétraitement}: supprimer le bruit dans les données avec \texttt{prep.simple}
						\State \textbf{vectorisation}: transformer les données en vecteurs avec \texttt{vect.tfidf}
						\State \textbf{clustering initial}: regrouper les données par similarité avec \texttt{clust.kmeans.cop}
						\State \textbf{évaluation}: estimer l'équivalence entre le \textit{clustering} obtenu et la vérité terrain
						\Repeat
							\State \textbf{échantillonnage}: sélectionner de nouvelles contraintes à annoter
							\State \textbf{simulation d'annotation}: ajouter des contraintes avec \texttt{samp.closest.diff}
							\State \textbf{clustering}: regrouper les données par similarité avec \texttt{clust.kmeans.cop}
							\State \textbf{évaluation}: estimer l'équivalence entre le \textit{clustering} obtenu et la vérité terrain
						\Until{annotation de toutes les contraintes possibles}
					\EndFor						
					\State \textbf{analyse}: entraîner un modèle linéaire généralisé du nombre de contraintes nécessaires
					\Ensure modélisation du nombre de contraintes nécessaires pour un jeu de données
				\end{algorithmic}
				\caption{Description en pseudo-code du protocole expérimental de l'étude du nombre de contraintes nécessaires pour converger vers une vérité terrain pré-établie avec notre paramétrage favori du \textit{clustering} interactif.}
				\label{algorithm:4.3.3-ETUDE-COUT-NOMBRE-CONTRAINTES-PROTOCOLE}
			\end{algorithm}
			
			% Description des jeux de données.
			Pour cette étude, nous utilisons comme références les jeux de données \textbf{TODO:JDD}\todo{TODO: reference dataset bank card} et \textbf{TODO:JDD}\todo{TODO: reference dataset mlsum}.
			La taille des jeux de données générée, noté $\texttt{dataset\_size}$, varie entre $1~000$ à $5~000$ par pas de $250$, et chaque taille de jeu est générée $3$ fois pour contrer les aléas statistiques de création.
			Il y a donc $51$ variations de chaque jeu de références, soit $102$ jeux utilisés de tailles différentes.
			
			% Description des tentatives de la méthode.
			Sur chacun de ces jeux générés, une tentative complète
			\footnote{Tentative complète : itérations d'échantillonnage, d'annotation et de \textit{clustering} jusqu'à annotation de toutes les contraintes possibles.}
			de la méthode du \textit{clustering} interactif utilisant notre paramétrage favori est exécuté, et chaque tentative est répétée $5$ fois pour contrer les aléas statistiques des exécutions.
			Il y a donc $510$ tentatives de \textit{clustering} interactif réalisées.
			
			% Description de l'évaluation.
			Pour chacune de ces tentatives, nous nous intéresserons au nombre de contraintes nécessaires pour atteindre le seuil d'annotation partielle (caractérisé par $90$\% de \texttt{v-measure} entre la vérité terrain et la segmentation des données obtenue), et nous entraînerons un modèle linéaire généralisé (\textit{GLM}) pour modéliser le nombre de contraintes requis en fonction de la taille du jeu de données (noté $\texttt{dataset\_size}$).
			Ce modèle sera caractérisé par le coefficient de détermination généralisé \texttt{R²} de \textit{Cox et Snel}\todo{CITATION}, la log-vraisemblance \texttt{llf}\todo{CITATION} et la log-vraisemblance \texttt{llf\_null} du modèle \textit{null}.
			Pour finir, nous discuterons des valeurs des coefficients obtenus sur l'impact du nombre d'itérations de la méthode à prévoir.

			% Référence scripts.
			\begin{leftBarInformation}
				Ces analyses sont réalisées en Python à l'aide de la librairie \texttt{statsmodels} (\cite{seabold:2010}).
				Les scripts de l'expérience (\textit{notebooks} Python) sont disponibles dans un dossier dédié de~\cite{schild:cognitivefactory-interactive-clustering-comparative-study:2021}.
			\end{leftBarInformation}

		%%% Résultats.
		\subsubsection{Résultats obtenus}
		
			% Modélisation du nombre de contraintes.
			Le modèle linéaire généralisé entraîné sur les mesures du nombre de contraintes requis pour atteindre $90$\% de \texttt{v-measure} (\texttt{R²}: $> 0.999$, \texttt{llf}: $-4~327.6$, \texttt{llf\_null}: $-4~942.9$) nous permet de déduire l'équation suivante :
			%
			\begin{equation}
				\label{equation:4.3.3-ETUDE-COUT-NOMBRE-CONTRAINTES}
				\texttt{constraints\_needed}(\texttt{settings.favorite})~[\#]~
				\propto~3.15 \cdot \texttt{dataset\_size}
			\end{equation}
			%
			La figure~\ref{figure:4.3.3-ETUDE-COUT-NOMBRE-CONTRAINTES} représente cette modélisation.
			\newline
			%
			\begin{figure}[!htb]
				\centering
				\includegraphics[width=0.8\textwidth]{figures/etude-nombre-contraintes-1-modelisation-nombre}
				\caption{Estimation du nombre moyen de contraintes nécessaire à notre \textbf{paramétrage favori} du \textit{clustering} interactif afin d'obtenir une annotation partielle (\textit{atteindre une \texttt{v-measure} de $90$\%}) en fonction de la taille du jeu de données à modéliser.}
				\label{figure:4.3.3-ETUDE-COUT-NOMBRE-CONTRAINTES}
			\end{figure}
		
			% Note de l'auteur.
			\begin{leftBarAuthorOpinion}
				% Estimation de points de références.
				On peut considérer les points de références suivants :
				%
				\begin{itemize}
					% Estimation grossière.
					\item le nombre de contraintes possibles (avec doublons) est de $\texttt{dataset\_size}^{2}$
					(\textit{caractériser chaque couple de données présent dans la matrice d'adjacence}) ;
					% Estimation sans doublon.
					\item le nombre de contraintes possibles (sans doublons) est de $\frac{1}{2} \cdot (\texttt{dataset\_size}^{2} - \texttt{dataset\_size})$
					(\textit{considérer la symétrie des contraintes, donc seul le triangle supérieur de la matrice d'adjacence a besoin d'être renseigné}) ;
					% Estimation minimale.
					\item le nombre minimal de contraintes à annoter sur une partition en $k$ \textit{clusters} $\{K_{1}, K_{2}, ..., K_{k}\} $ du jeu de données et en étant exhaustif est estimé à ${\displaystyle \sum\limits_{0 \leq i < k}{\|K_{i}\|-1} + \sum\limits_{0 \leq i < k}{k-1-i}} $ (
					\textit{considérer d'abord le chemin minimal pour créer des composants connexes avec des contraintes \texttt{MUST-LINK}, puis ajouter le nombre minimal des contraintes \texttt{CANNOT-LINK} pour distinguer les composants connexes en \textit{cluster}}).
				\end{itemize}
				%
				% Annonce de la figure.
				La figure~\ref{figure:4.3.3-ETUDE-COUT-NOMBRE-CONTRAINTES-EXEMPLES} illustre ces propos sur un jeu d'exemple comportant $10$ points de données réparties en $3$ classes, et met en avant l'explosion du nombre de contraintes possibles même sur un petit jeu de données (cf.~\ref{figure:4.3.3-ETUDE-COUT-NOMBRE-CONTRAINTES-EXEMPLES}~\textbf{(2)}).
				
				% Application de ces points de référence.
				Avec ces références, le nombre de contraintes est borné approximativement
				entre $1~035$ et $499~500$ pour un jeu de $1~000$ données équilibré en $10$ classes,
				et entre $6~175$ et $12~497~500$ pour un jeu de $5~000$ données équilibré en $50$ classes.
				%
				\begin{figure}[H]
					\centering
					\includegraphics[width=0.7\textwidth]{figures/etude-nombre-contraintes-2-bornes-limites}
					\caption{Exemple de caractérisation exhaustive d'un jeu de données ($10$ données, $3$ classes) en ajoutant un nombre minimal de contraintes (cf. \textbf{(1)}) ou en ajoutant toutes les contraintes possibles (cf. \textbf{(2)}).}
					\label{figure:4.3.3-ETUDE-COUT-NOMBRE-CONTRAINTES-EXEMPLES}
				\end{figure}
			\end{leftBarAuthorOpinion}

		%%% Discussion.
		\subsubsection{Discussion}
		
			% Rappel de l'objectif.
			L'objectif de cette étude était de déterminer le nombre moyen de contraintes à devoir annoter pour modéliser un jeu de données avec un accord $90$\% de \texttt{v-measure} avec la vérité terrain utilisée.
			Cette estimation, dépendant de la taille du jeu de données manipulé, est représentée par l'équation~\ref{equation:4.3.3-ETUDE-COUT-NOMBRE-CONTRAINTES}.
			
			% Discussion générale sur la pente.
			On peut constater que la relation entre la taille du jeu de données et le nombre de contraintes à annoter est linéaire (pente de $3.15$) : doubler la taille d'un jeu de données doublera donc la charge de travail incombant à l'expert métier.
			\todo[inline]{A REDIGER: Ca aurait pu être pire (cf. notes de l'auteur)}
			\todo[inline]{A REDIGER: Toutefois, ça parait beaucoup et ça pourrait gêner... cf. 4.3.4}
			
			% Influence du jeu de données.
			Bien évidemment, une telle estimation est sensible au jeu de données utilisé comme référence (cf. figure~\ref{figure:4.3.3-ETUDE-COUT-NOMBRE-CONTRAINTES}).
			Ici, la différence de pente mesurée est de $0.25$ (\texttt{p-valeur}: $> 0.999$), soit un écart moyen d'environ $8$\% par rapport à la modélisation moyenne.
			Toutefois, comme l'impact semble limité, nous maintenons la modélisation moyenne représentée par l'équation~\ref{equation:4.3.3-ETUDE-COUT-NOMBRE-CONTRAINTES} pour la suite de nos estimations de coûts.
		
			% Note de l'auteur.
			\begin{leftBarAuthorOpinion}
				Il n'y a pas davantage de matière à discussion pour cette étude, car le principal résultat (l'équation~\ref{equation:4.3.3-ETUDE-COUT-NOMBRE-CONTRAINTES}) est un résultat temporaire nécessaire à l'estimation du coût global d'un projet utilisant une méthodologie de \textit{clustering} interactif.
			\end{leftBarAuthorOpinion}
	
	%%%
	%%% Subsection 4.3.4: Estimation du temps total d'un projet d'annotation en combinant les précédentes études de coûts
	%%%
	\subsection{Estimation du temps total d'un projet d'annotation en combinant les précédentes études de coûts}
	\label{section:4.3.4-ETUDE-COUTS-TOTAL}
	
		% Equation finale.
		Résumons l'ensemble des modélisations réalisées lors des précédentes études (cf. sections \ref{section:4.3.1-ETUDE-COUTS-TEMPS-ANNOTATION}, \ref{section:4.3.2-ETUDE-COUTS-TEMPS-CALCUL} et \ref{section:4.3.3-ETUDE-COUT-NOMBRE-CONTRAINTES}) afin d'estimer le coût total d'un projet d'annotation employant une méthodologie basée sur le \textit{clustering} interactif et utilisant notre \textbf{paramétrage favori}
		\footnote{Paramétrage favori (atteindre $90$\% de \texttt{v-measure} avec un coût minimal).}.
		Dans les notations, $\texttt{dataset\_size}$ représente la taille du jeu de données à modéliser, et $\texttt{batch\_size}$ représente le nombre de contraintes que l'expert annote à chaque itération.

		%%% Résultats.
		\subsubsection{Synthèse des résultats}
			
			% Temps pour une itération.
			Tout d'abord, nous pouvons estimer le \textbf{temps moyen d'une itération de la méthode}, comprenant les temps de d'exécution des algorithmes (\textit{prétraitement}, \textit{vectorisation}, \textit{clustering}, \textit{échantillonnage}) et le temps d'annotation d'un lot de contraintes, grâce aux équations suivantes :
			\begin{equation}
				\label{equation:4.3.4-ETUDE-COUT-UNE-ITERATION}
				\begin{cases}
					% Computation time.
					\texttt{computation\_time}~[s]&
						~\propto~1.52 \cdot 10^{-1} \cdot \texttt{dataset\_size} + 1.43 \cdot 10^{-6} \cdot \texttt{dataset\_size}^{2} \\
					% Annotatim time.
					\texttt{annotation\_time}~[s]&
						~\propto~7.77 \cdot \texttt{batch\_size} \\
					% One iteration time.
					\texttt{one\_iteration\_time}~[s]&
						~\propto~\texttt{computation\_time} + \texttt{annotation\_time} \\
				\end{cases}
			\end{equation}
			
			% Nombre d'itérations.
			Ensuite, nous sommes en mesure d'anticiper le \textbf{nombre moyen de contraintes à annoter} pour modéliser le jeu de données avec un seuil de $90$\% de \texttt{v-measure}, et donc de déduire le nombre d'itérations nécessaire de la méthode, grâce aux équations suivantes :
			\begin{equation}
				\label{equation:4.3.4-ETUDE-COUT-NOMBRE-ITERATIONS}
				\begin{cases}
					% Constraints number.
					\texttt{constraints\_number\_needed}~[\#] &
						~\propto~3.15 \cdot \texttt{dataset\_size} \\
					% Iterations number.
					\texttt{iterations\_number\_needed}~[\#] &
						~\propto~\texttt{constraints\_number\_needed} / \texttt{batch\_size} \\
				\end{cases}
			\end{equation}
			
			% Temps total pour un projet.
			Enfin, il suffit de combiner les deux équations~\ref{equation:4.3.4-ETUDE-COUT-UNE-ITERATION} et~\ref{equation:4.3.4-ETUDE-COUT-NOMBRE-ITERATIONS} pour estimer le temps total nécessaire à un projet d'annotation utilisant le \textit{clustering} interactif pour converger vers $90$\% de \texttt{v-measure} :
			\begin{equation}
				\label{equation:4.3.4-ETUDE-COUT-TOTAL}
				\begin{cases}
					% Constraints number.
					\texttt{total\_time}~[s] &
						~\propto~\texttt{one\_iteration\_time} \cdot \texttt{iterations\_number\_needed}
				\end{cases}
			\end{equation}
			
			% Figure.
			La figure~\ref{figure:4.3.4-ETUDE-COUT-TOTAL} représente cette estimation globale en fonction de plusieurs taille de jeu de données et plusieurs tailles de lots d'annotation.
			%		
			\begin{figure}[!htb]
				\centering
				\includegraphics[width=0.8\textwidth]{figures/etude-temps-total-1-modelisation}
				\caption{Estimation du temps total nécessaire (en heures) pour modéliser un jeu de données avec notre \textbf{paramétrage favori} du \textit{clustering} interactif afin d'obtenir une annotation partielle (\textit{atteindre une \texttt{v-measure} de $90$\%}), en fonction de la taille du jeu de données et de la taille des lots d'annotations.}
				\label{figure:4.3.4-ETUDE-COUT-TOTAL}
			\end{figure}

		%%% Discussion finale.
		\subsubsection{Discussion finale}
		
		
			\todo[inline]{A REDIGER: (1) Rappel sur l'objectif des précédentes études}
			\todo[inline]{A REDIGER: (2) Annonce des résultats : c'est long !}
			\todo[inline]{A REDIGER: (3) Comparer avec une annotation classique}
			\todo[inline]{A REDIGER: (4) Augmenter la taille des lots d'annotation + Augmenter le nombre d'annotateurs + paralléliser annotation et clustering}
			\todo[inline]{A REDIGER: (5) Estimer la rentabilité d'une nouvelle itération pour savoir quand s'arrêter}
		
		
		\todo[inline]{A REDIGER : Calcul total pour le dataset de 500 points de données. Rejetter l'annotation suffisante car trop longue.}
		
		%%	A l'aide de la modélisation temporelle réalisée ci-dessus, nous pouvons estimer qu'un lot de $50$ contraintes (comme employé dans nos précédentes études) nécessite en moyenne $548$ secondes, soit $9.13$ minutes.
		%%	De plus, en considérant les résultats de l'étude d'efficience (cf. section~\ref{section:4.2-HYPOTHESE-EFFICIENCE}) sur un jeu de $500$ points de données, nous pouvons déduire que :
		%%	\begin{itemize}
		%%		\item l'obtention d'une annotation partielle (atteindre $90$\% de \texttt{v-measure}) peut se faire avec $950$ contraintes, soit $19$ itérations de $50$ contraintes, ce qui équivaut à une moyenne de $2.89$ heures d'annotations ;
		%%		\item l'obtention d'une annotation suffisante (atteindre $100$\% de \texttt{v-measure}) peut se faire en $1~730$ contraintes, soit $34.6$ itérations de $50$ contraintes, ce qui équivaut à une moyenne de $5.27$ heures d'annotations.
		%%	\end{itemize}
		
		\todo[inline]{A REDIGER : comparaison du total avec 5 points : le temps nécessaire à créer banking cards, le temps nécessaire à adapter MLSUM, et trois études de Snow et al. (2008)}
		
			% Commentaire sur la difficulté de comparer les durées totales.
		%%	En ce qui concerne les durées totales pour annoter un jeu de $500$ points de données, estimées respectivement à $2.89$ et à $5.27$ heures suivant les seuils d'annotation, il est difficile de discuter de leur compétitivité à cause du manque de repères concrets.
		%%	En effet, les références dépendent de la tâche d'annotation, des données manipulées, du nombre de choix proposés à l'opérateur, de la complexité sémantique des données, des compétences des annotateurs, ...
		%%	\footnote{De plus, la plupart des études des tâches d'annotation en \textit{machine learning} se concentrent sur la cohérence inter-annotateurs plutôt que sur le temps nécessaire.}
		%%	Pour pallier ce problème, nous proposons de comparer nos estimations de temps d'annotations grâce aux pistes ci-dessous.
		%%	Ces repères sont approximatifs et assez disparates, mais ils nous permettront de discuter des ordres de grandeurs à manipuler.
			
			% Cas de taches de complexité similaire.
		%%	D'une part, comparons nos estimations à trois exemples de tâches ayant une \textbf{complexité similaire} à la vérité terrain que nous manipulons lors de cette expérience. Ces points de repères font appel à la catégorisation du thème traité par des textes courts.
			%
		%%	\begin{enumerate}
		%%		% Conception du JDD bank cards.
		%%		\item Pour concevoir le jeu de données~\todo{citation: bank cards} ($10$ classes équilibrées, $500$ points de données) servant de vérité terrain aux études d'efficacité et d'efficience, nous avons requis approximativement $5$ heures, dont $2.5$ d'annotation
		%%		\footnote{La conception de $bank cards$ a demandé $\sim 2$ heure de définition du périmètre (1 personne), $\sim 2$ heures de collecte et d'annotation des étiquettes des données (1 personne), et $\sim 1$ heure de revue de la cohérence des annotations (2 personne)} ;
		%%		% Conception du JDD MLSUM subset.
		%%		\item Pour adapter le jeu de données~\todo{citation: mlsum} en~\todo{citation: mlsum subset SCHILD} ($14$ classes retenues, $744$ points de données sélectionnées sur un échantillon de $1~050$) servant de base à notre étude du coût de l'annotation, nous avons requis approximativement $5$ heures, dont $2$ d'annotation binaire
		%%		\footnote{L'adaptation de $TODO:mlsum$ a demandé $\sim 1$ heure de collecte des catégories les plus communes (1 personne), $\sim 2$ heures d'annotation binaire de la pertinence des données (2 personnes) et $\sim 2$ heures de revue de la cohérence des annotations (2 personne)} ;
		%%		% Article Cheap and Fast — But is it Good? (Word Sense Disambiguation)
		%%		\item Selon~\todo{citation: Snow et al. (2008) / Yuret (2007)}, qui délègue à \texttt{Amazon Mechanical Turk} l'annotation de phrases pour catégoriser leur contexte parmi trois possibilités préformatés, il faut $8.59$ heures pour étiqueter $1~770$ données. En pondérant cette approximation pour un jeu de $500$ points de données, on peut estimer le temps d'annotation à $2.43$ heures environ ;
		%%	\end{enumerate}
			%
		%%	À l'aide de ces trois exemples, on remarque qu'une annotation partielle ($950$ contraintes pour atteindre $90$\% de \texttt{v-measure}) nécessite une durée comparable à une annotation par label ($2.89$ heures vs. entre $2$ et $3$ heures), mais une annotation suffisante ($1~730$ contraintes pour atteindre $100$\% de \texttt{v-measure}) nécessite 2 à 3 fois plus de temps ($5.27$ heures).
		%%	Il est donc préférable de nous concentrer sur une \textbf{annotation partielle} afin de garder une méthode efficiente. \todo{lien vers la section 4.4 a faire dans la discussion globale de cette section}
				% Ce choix implique de pouvoir compléter la différence qualité en remaniant manuellement certains \textit{clusters} obtenus lors d'une phase de revue.
				% Cet aspect sera complété à partir de l'étude décrite dans la section~\ref{section:4.4-HYPOTHESE-PERTINENCE} (hypothèse de pertinence).
			
			% cas de taches de complexité différentes
		%%	D'autre part, comparons nos estimations à deux exemples de tâches ayant des \textbf{complexité différentes} (une premier presque triviale, une seconde plutôt complexe) :
			%
		%%	\begin{enumerate}
		%%		\setcounter{enumi}{3}
		%%		% Article Cheap and Fast — But is it Good? (Word Similarity)
		%%		\item Selon~\todo{citation: Snow et al. (2008) / Rubenstein et Goodenough (1965)}, qui délègue à \texttt{Amazon Mechanical Turk} l'annotation de couples de mots pour identifier par similarité des synonymes, il faut $0.17$ heures pour ordonner $300$ paires de données de la plus à la moins similair
		%%		\footnote{La tâche d'annotation de synonymes par ordonnancement peut s'apparenter à de l'annotation de contraintes en se demandant : « \textit{laquelle de ces deux paires est la plus adéquate ? »}. Cette tâche est relativement simple (similarité intrinsèque triviale, peu de vocabulaire).}. En pondérant cette approximation pour un lot de $950$ contraintes, on peut estimer le temps d'annotation à environ $0.53$ heures ;
		%%		% Article Cheap and Fast — But is it Good? (Recognizing Textual Entailment)
		%%		\item Encore selon~\todo{citation: Snow et al. (2008) / Dagan et Magnini (2005)}, qui délègue à \texttt{Amazon Mechanical Turk} l'annotation binaire de la véracité d'une implication, il faut $89.3$ heures pour étiqueter $8~000$ données
		%%		\footnote{La tâche d'annotation de la véracité d'une implication peut s'apparenter à de l'annotation de contraintes en se demandant : « \textit{est-ce que l'implication est vraie ? »}. Cette tâche est relativement complexe (implication logique non trivial). }. En pondérant cette approximation pour un lot de $950$ contraintes, on peut estimer le temps d'annotation à environ $10.60$ heures ;
		%%	\end{enumerate}
			%
		%%	Avec ces deux exemples, on constate que les variations peuvent être très grandes (allant du cinquième au quadruple !).
		%%	Il est donc important de nuancer nos précédentes conclusions en fonction de la complexité de la thématique traitée.
		%%	Toutefois, 
		%%	\todo[inline]{données intrinsèquement différentes : peuvent être caractérisées très rapidement}
		%%	\todo[inline]{données ambiguës : peuvent demander plus de temps, plus de réflexion}
		
		\todo[inline]{A REDIGER : TRANSISTION VERS PERTINENCE car la notion de v-measure est impossible à calculer en situation réelle}





    %%%%%--------------------------------------------------------------------
    %%%%% Section 4.4: Hypothèse de pertinence.
    %%%%%--------------------------------------------------------------------
	\newpage
	\section{Hypothèse de pertinence}
\label{section:4.4-HYPOTHESE-PERTINENCE}
% « \textit{est-ce le résultat est exploitable ?} »

	%%% Formulation des hypothèses:
	Nous aimerions vérifier l'hypothèse suivante :
	\todo{à compléter}

	\begin{tcolorbox}[
		title=\faVial~\textbf{Hypothèse de pertinence}~\faVial,
		colback=colorTcolorboxHypothesis!15,  % gray!20
		colframe=colorTcolorboxHypothesis!75,  % gray!50!black!75,
		width=\linewidth
	]
		« La vitesse de convergence du \textit{clustering} interactif \textbf{peut être optimisée} en réglant différents paramètres. Nous étudierons l'influence du prétraitement des données, de la vectorisation des données, de l'échantillonnage des contraintes à annoter et du \textit{clustering} sous contraintes (cf. figure~\ref{figure:4.4-HYPOTHESE-PERTINENCE}. »
		
		
		\begin{figure}[H]  % keep [H] to be in the tcolorbox.
			\centering
			\includegraphics[width=0.95\textwidth]{figures/hypotheses-04-pertinence}
			\caption{Illustration des études réalisées sur le \textit{clustering} interactif (\textit{étape 4/6}) en schématisant l'évolution de la performance (\textit{accord avec la vérité terrain calculé en v-measure}) d'une base d'apprentissage en cours de construction en fonction du nombre d'itérations de la méthode (\textit{nombre d'annotations par un expert métier}).}
			\label{figure:4.4-HYPOTHESE-PERTINENCE}
		\end{figure}

	\end{tcolorbox}
	
	%%%
	%%% Subsection 4.4.1: Étude de la cohérence statistique de la base d'apprentissage en cours de construction
	%%%
	\subsection{Étude de la cohérence statistique de la base d'apprentissage en cours de construction}
	
		%%% Protocole expérimental.
		\subsubsection{Protocole expérimental}
			\todo[inline]{Description succincte du protocole expérimental dans l'encadré d'hypothèse ?}

		%%% Résultats
		\subsubsection{Résultats obtenus}
			%
			\begin{figure}[!htb]
				\centering
				\includegraphics[width=0.95\textwidth]{figures/etude-pertinence-consistence}
				\caption{Évolution du score de cohérence moyen des tentatives en fonction de leur paramétrage : \textbf{(1)} meilleur paramétrage moyen une annotation partielle (\texttt{90}\% de \texttt{v-measure}), \textbf{(2)} meilleur paramétrage moyen une annotation suffisante (\texttt{100}\% de \texttt{v-measure}), \textbf{(3)} meilleur paramétrage moyen une annotation exhaustive (annoter toutes les contraintes possibles), et \textbf{(4)} paramétrage favori (\texttt{90}\% de \texttt{v-measure} avec un coût minimal). \\
				Note : \textit{Le score de cohérence de la vérité terrain peut varier en fonction des méthodes de prétraitements et de vectorisation utilisées.}}
				\label{figure:4.3.1-ETUDE-PERTINENCE-CONHERENCE-ANNOTATION}
			\end{figure}

		%%% Discussion
		\subsubsection{Discussion}
	
	%%%
	%%% Subsection 4.4.2: Étude de la pertinence sémentique de la base d'apprentissage en cours de construction
	%%%
	\subsection{Étude de la pertinence sémantique de la base d'apprentissage en cours de construction}
	
		%%% Protocole expérimental.
		\subsubsection{Protocole expérimental}
			\todo[inline]{Description succincte du protocole expérimental dans l'encadré d'hypothèse ?}

		%%% Résultats
		\subsubsection{Résultats obtenus}

		%%% Discussion
		\subsubsection{Discussion}


    %%%%%--------------------------------------------------------------------
    %%%%% Section 4.5: Hypothèse de rentabilité.
    %%%%%--------------------------------------------------------------------
	\newpage
	\section{Évaluation de l'hypothèse de rentabilité}
\label{section:4.5-HYPOTHESE-RENTABILITE}

	%%% Introduction / Transition.
	Dans les études précédentes, le cas d'arrêt de notre méthodologie d'annotation basée sur le \textit{clustering} interactif était conditionné à la vérité terrain.
	En effet, nous utilisions un seuil de $90$\% de \texttt{v-measure}, caractérisant une annotation dite "partielle" de la base d'apprentissage.
	Cependant, une telle référence n'est pas accessible en situation réelle car l'objectif de notre méthode est précisément de la construire cette vérité terrain.
	Nous devons donc nous intéresser à d'autres moyens pour estimer la rentabilité d'une itération supplémentaire et pouvoir ainsi définir de nouveaux cas d'arrêt pour le \textit{clustering} interactif.
	Pour cela, nous aimerions vérifier l'hypothèse suivante :
	
	%%% Formulation des hypothèses:
	\begin{tcolorbox}[
		title=\faVial~\textbf{Hypothèse de rentabilité}~\faVial,
		colback=colorTcolorboxHypothesis!15,
		colframe=colorTcolorboxHypothesis!75,
		width=\linewidth
	]
		« \textbf{
			Au cours d'une méthodologie d'annotation basée sur le \textit{clustering} interactif, il est possible d'estimer la rentabilité d'une itération supplémentaire de la méthode, et ainsi d'établir des cas d'arrêt indépendant d'une vérité terrain pour obtenir une base d’apprentissage satisfaisante.
		} » \\
		
		% Figure.
		La \textsc{Figure~\ref{figure:4.5-HYPOTHESE-RENTABILITE}} illustre cette hypothèse et l'espoir de pouvoir estimer le rapport entre le gain de pertinence obtenu et le coût nécessaire pour l'obtenir.
		%
		\begin{figure}[H]  % keep [H] to be in the tcolorbox.
			\centering
			\includegraphics[width=0.95\textwidth]{figures/hypotheses-05-rentabilite}
			\caption{
				Illustration des études réalisées sur le \textit{clustering} interactif (\textit{étape 5/6}) en schématisant l'évolution de la pertinence (\textit{valeur métier évaluée par l'expert et exprimé en nombre de clusters}) d'une base d'apprentissage en cours de construction en fonction du coût temporel de la méthode (\textit{temps nécessaire à l'expert métier et à la machine}), ainsi que la rentabilité de chaque itération de la méthode (\textit{rapport entre le gain potentiel de pertinence et le coût à investir}).
			}
			\label{figure:4.5-HYPOTHESE-RENTABILITE}
		\end{figure}
	\end{tcolorbox}
		
	% Résumé de l'étude.
	Afin de vérifier cette hypothèse, nous explorons deux approches :
	\begin{itemize}
		\item l'évolution de l'\textbf{accord entre l'annotation de l'expert et le \textit{clustering}} sur lequel est basé l'échantillon d'annotation, permettant d'estimer si la machine doit encore être corrigée par l'annotateur  (cf. \textsc{Section~\ref{section:4.5.1-ETUDE-RENTABILITE-ACCORD-ANNOTATION-CLUSTERING}}) ;
		\item et l'évolution de la \textbf{différence entre deux \textit{clusterings} successifs}, permettant de mesurer s'il y a eu des changements visibles dans le partitionnement des données après l'ajout des dernières contraintes (cf. \textsc{Section~\ref{section:4.5.2-ETUDE-RENTABILITE-SIMILARITE-CLUSTERING}}).
	\end{itemize}
	
	
	%%%
	%%% Subsection 4.5.1: Étude de l'évolution d'accord entre l'annotation et le \textit{clustering}.
	%%%
	\subsection{Étude de l'évolution d'accord entre l'annotation et le \textit{clustering}}
	\label{section:4.5.1-ETUDE-RENTABILITE-ACCORD-ANNOTATION-CLUSTERING}
		
		% Objectif de l'expérience.
		Nous cherchons à trouver un cas d'arrêt du \textit{clustering} interactif ne nécessitant pas de comparaison avec une vérité terrain, et notre première intuition concerne l'étude des annotations réalisées.
		En effet, à chaque itération, l'expert annote un échantillon de contraintes dans le but de confirmer ou de corriger le \textit{clustering} de l'itération précédente.
		Or, après un nombre suffisant d'itérations, le \textit{clustering} commence à se stabiliser : il devrait donc y avoir davantage d’annotations qui confirment le \textit{clustering} que d'annotations qui le corrigent, puis n'avoir que des accords entre les annotations et le \textit{clustering}.
		Ainsi, nous allons étudier l'évolution du nombre de contraintes annotées qui approuvent le partitionnement des données obtenu et essayer d'adapter cette analyse en cas d'arrêt pour notre méthode d'annotation.
	
		%%% Protocole expérimental.
		\subsubsection{Protocole expérimental}
			
			% Axiome.
			\begin{leftBarWarning}
				Dans le cadre de cette étude, nous supposons que l'expert métier connaît parfaitement le domaine traité dans ce jeu de données, et qu'il est capable de caractériser sans ambiguïté la similitude entre deux données issues de cet ensemble.
			\end{leftBarWarning}
			
			% Pseudo-code.
			Pour résumer le protocole expérimental que nous décrivons ci-dessous, vous pouvez vous référer au pseudo-code décrit dans \textsc{Algorithme~\ref{algorithm:4.5.1-ETUDE-RENTABILITE-ACCORD-ANNOTATION-CLUSTERING-PROTOCOLE}}.
			
			\begin{algorithm}
				\KwData{jeu de données annotés (vérité terrain)}
				%
				\ForEach{jeux de données à tester}{
					\textbf{initialisation (données)}: récupérer les données et la vérité terrain \;
					\textbf{initialisation (contraintes)}: créer une liste vide de contraintes \;
					\textbf{prétraitement}: supprimer le bruit dans les données avec \texttt{prep.simple} \;
					\textbf{vectorisation}: transformer les données en vecteurs avec \texttt{vect.tfidf} \;
					\textbf{clustering initial}: regrouper les données par similarité avec \texttt{clust.kmeans.cop} \;
					\Repeat{annotation de toutes les contraintes possibles}{
						\textbf{échantillonnage}: sélectionner des contraintes avec \texttt{samp.closest.diff} \;
						\textbf{simulation d'annotation}: caractériser les contraintes grâce à la vérité terrain \;
						\textbf{intégration}: ajouter les nouvelles contraintes au gestionnaire de contraintes \;
						\textbf{rentabilité}: calculer l'accord entre l'annotation et le \textit{clustering} précédent \;
						\textbf{clustering}: regrouper les données par similarité avec \texttt{clust.kmeans.cop} \;
					}
				}
				\textbf{analyse 1}: afficher l'évolution de l'accord entre annotation et \textit{clustering} \;
				\textbf{analyse 2}: calculer la corrélation entre le score d'accord et le score de performance \;
				%
				\KwResult{discussion sur la rentabilité d'après l'accord entre annotation et \textit{clustering}}
				%
				\caption{\textit{
					Description en pseudo-code du protocole expérimental de l'étude de l'évolution d'accord entre l'annotation et le \textit{clustering}.
				}}
				\label{algorithm:4.5.1-ETUDE-RENTABILITE-ACCORD-ANNOTATION-CLUSTERING-PROTOCOLE}
			\end{algorithm}
			
			% Description de la vérité terrain.
			Nous utilisons comme vérité terrain le jeu de données \texttt{Bank Cards (v1.0.0)} : ce dernier traite des demandes les plus fréquentes des clients en ce qui concerne la gestion de leur carte bancaire.
			Il est composé de $500$ questions rédigées en français et réparties en $10$ classes (\texttt{perte ou vol de carte}, \texttt{carte avalée}, \texttt{commande de carte}, ...).
			Pour plus de détails, consultez l'annexe~\ref{annex:C.1-DATASET-BANK-CARDS}.
			
			% Description des tentatives de la méthode et du calcul de rentabilité.
			Sur ce jeu de données, nous exécutons une tentative complète
			\footnote{Tentative complète : itérations d'échantillonnage, d'annotation et de \textit{clustering} jusqu'à annotation de toutes les contraintes possibles.}
			\todo{Utiliser 'footmisc' et 'footref' pour faire des notes de bas de pages communes ? ou lien vers des conclusions ?}
			de la méthode du \textit{clustering} interactif en utilisant notre paramétrage favori, et cette tentative est répétée $5$ fois pour contrer les aléas statistiques des exécutions.
			À chaque itération, un lot de $50$ contraintes est sélectionné puis annotés en simulant l'action d'un expert métier, et nous évaluons l'accord entre ces nouvelles annotations et la proposition de partitionnement des données réalisé par le \textit{clustering} à l'itération précédente :
			\begin{itemize}
				\item il y a \textbf{accord} lorsqu'une contrainte de deux données issues d'un même \textit{cluster} est annotée \texttt{MUST-LINK}, ou lorsqu'une contrainte de deux données issues de deux \textit{clusters} différents est annotée \texttt{CANNOT-LINK} (cf. \textsc{Figure~\ref{figure:4.5.1-ETUDE-RENTABILITE-ACCORD-ANNOTATION-CLUSTERING-EXEMPLE} (1)}) ;
				\item il y a \textbf{désaccord} lorsqu'une contrainte de deux données issues d'un même \textit{cluster} est annotée \texttt{CANNOT-LINK}, ou lorsqu'une contrainte de deux données issues de deux \textit{clusters} différents est annotée \texttt{MUST-LINK} (cf. \textsc{Figure~\ref{figure:4.5.1-ETUDE-RENTABILITE-ACCORD-ANNOTATION-CLUSTERING-EXEMPLE} (2)}).
			\end{itemize}
			Nous pouvons ainsi calculer un score d'accord défini par la ratio entre le nombre d'accords et le nombre de contraintes annotées.
			Pour nous permettre de discuter de l'utilité de ce score pour prédire la stabilisation du \textit{clustering} et ainsi définir un cas d'arrêt de notre méthodologie d'annotation, nous calculons aussi le score de corrélation entre cet accord et la performance obtenu à l'aide d'une vérité terrain (la corrélation \texttt{r} de \textit{Pearson} (\cite{kirch:2008:pearson-correlation-coefficient}) est utilisée).

			\begin{figure}[!htb]
				\centering
				\includegraphics[width=0.7\textwidth]{figures/example-accord-annotation-clustering}
				\caption{
					Exemples d'accords et de désaccord entre les annotations d'une itération et le résultat du \textit{clustering} de l'itération précédente.
					Des contraintes \texttt{MUST-LINK} (flèches vertes) et \texttt{CANNOT-LINK} (flèches rouges) sont représentées dans deux situations : \textbf{(1)} montre des cas d'accords (\texttt{MUST-LINK} dans un même \textit{cluster}, \texttt{CANNOT-LINK} entre deux \textit{clusters} différents), et \textbf{(2)} montre des cas de désaccords (\texttt{MUST-LINK} entre deux \textit{clusters} différents, \texttt{CANNOT-LINK} dans un même \textit{cluster}).
				}
				\label{figure:4.5.1-ETUDE-RENTABILITE-ACCORD-ANNOTATION-CLUSTERING-EXEMPLE}
			\end{figure}
			
			\begin{leftBarIdea}
				Nous concentrons l'étude sur notre paramétrage favori (voir \textsc{Section~\ref{section:4.4.3-ETUDE-PERTINENCE-RESUME-AUTOMATIQUE}}).
				Cependant, afin de compléter notre discussion avec d'autres points de comparaison, nous analysons aussi les autres paramétrages implémentés, notamment les meilleurs paramétrages moyens identifiés lors de l'hypothèse d'efficience (voir \textsc{Section~\ref{section:4.2-HYPOTHESE-EFFICIENCE}}).
			\end{leftBarIdea}
			
			% Référence scripts.
			\begin{leftBarInformation}
				Les scripts de l'expérience, réalisés avec des \textit{notebooks} Python (\cite{van-rossum-drake:2009:python-reference-manual}), sont disponibles dans un dossier dédié de~\cite{schild:2021:cognitivefactory-interactiveclusteringcomparativestudy}.
			\end{leftBarInformation}

		%%% Résultats
		\subsubsection{Résultats obtenus}
			
			% Figure : croissance générale.
			La \textsc{Figure~\ref{figure:4.5.1-ETUDE-RENTABILITE-ACCORD-ANNOTATION-CLUSTERING}} représente l'évolution moyenne du score d'accord entre annotation et \textit{clustering} pour les quatre paramétrages mis en avant lors de nos études.
			Nous pouvons constater une tendance générale à la croissance de ce score d'accord : pour le paramétrage favori \textbf{(4)}, l'accord est plutôt faible au début de la méthode (inférieur à $45$\% avant l'itération $15$), puis devient de plus en plus fort (dépassant les $60$\%) pour finalement atteider les $100$\% vers l'itération $45$.
			\begin{figure}[!htb]
				\centering
				\includegraphics[width=0.95\textwidth]{figures/etude-rentabilite-accord-annotation}
				\caption{
					Évolution au cours des itérations de l'accord entre l'annotation de contraintes d'un expert et le résultat de \textit{clustering} sur lequel est basé l'échantillonnage de contraintes.
					Ces accords sont exprimés grâce à des lots de $50$ contraintes annotées.
					Les évolutions moyennes de différents paramétrages de la méthode sont exposées :
					\textbf{(1)} meilleur paramétrage moyen pour atteindre une annotation partielle ;
					\textbf{(2)} meilleur paramétrage moyen pour atteindre une annotation suffisante ;
					\textbf{(3)} meilleur paramétrage moyen pour atteindre une annotation exhaustive ;
					et \textbf{(4)} paramétrage favori.
					À titre d'information, les courbes en noir représentent l'évolution de la \texttt{v-measure} entre le \textit{clustering} et la vérité terrain.
				} 
				\label{figure:4.5.1-ETUDE-RENTABILITE-ACCORD-ANNOTATION-CLUSTERING}
			\end{figure}
			
			% Tableau : corrélation modérée.
			La \textsc{Table~\ref{table:4.5.1-ETUDE-RENTABILITE-CORRELATION-ACCORD-PERFORMANCE}} contient le score de corrélation entre cet accord et la performance théoriques obtenue grâce à la vérité terrain.
			Cette corrélation est modérée : $0.49$ sur l'ensemble des tentatives, $0.69$ sur les tentatives utilisant notre paramétrage favori.
			\begin{table}[!htb]
				\begin{center}
				\begin{tabular}{|c|r|}
				
					\hline
					% ENTETE DU TABLEAU
					\multicolumn{1}{|c|}{\shortstack[c]{
						Paramétrage
					}}
						& \multicolumn{1}{c|}{\shortstack[c]{
							Corrélation \texttt{r}
						}}
						\tabularnewline
						\hline
					
					% Annotation partielle.
					Meilleur paramétrage moyen pour une annotation partielle \textbf{(1)}
						& $0.92$
						\tabularnewline
						\hline
					
					% Annotation suffisante.
					Meilleur paramétrage moyen pour une annotation suffisante \textbf{(2)}
						& $0.74$
						\tabularnewline
						\hline
					
					% Annotation exhaustive.
					Meilleur paramétrage moyen pour une annotation exhaustive \textbf{(3)}
						& $0.57$
						\tabularnewline
						\hline
					
					% Paramétrage favori
					Paramétrage favori \textbf{(4)}
						& $0.69$
						\tabularnewline
						\hline
					
					% Moyenne des $960$ tentatives.
					Moyenne des $960$ tentatives
						& $0.49$
						\tabularnewline
						\hline
					
				\end{tabular}
				\end{center}
				\caption{
					Score de corrélation \texttt{r} de \textit{Pearson} entre la performance du \textit{clustering} obtenu à l'aide d'une vérité terrain (\texttt{v-measure}) et le score d'accord entre annotation et \textit{clustering}.
				}
				\label{table:4.5.1-ETUDE-RENTABILITE-CORRELATION-ACCORD-PERFORMANCE}
			\end{table}
			
			% Description de la figure : croissance instable.
			Cependant, la tendance constatée est aussi saccadée par de nombreux pics pouvant faire perdre ou gagner jusqu'à $40$\% d'accord entre deux itérations.
			Des chutes d'accord peuvent intervenir à des itérations où la similarité du \textit{clustering} avec la vérité terrain est pourtant forte, comme c'est le cas autour des itérations $29$ et $36$ où l'accord chute de plus de $25$\% alors que la \texttt{v-measure} avec la vérité terrain est constamment au dessus de de $95$\%.
			
			% Description de la figure : autres paramétrages
			Les autres paramétrages représentés dans \textbf{(1)}, \textbf{(2)} et \textbf{(3)} comportent des tendances similaires (corrélation forte mais variations soudaines d'accord, chute d'accords malgré des \textit{clustering} aux performances élevées, ...).

		%%% Discussion
		\subsubsection{Discussion}
		
			% Rappel de l'objectif : trouver un cas d'arrêt en regardant l'accord entre l'annotation et le clustering.
			Dans cette étude, nous avons analysé l'évolution de l'accord entre les annotations et le partitionnement de données proposé par un \textit{clustering} dans l'espoir de définir un cas d'arrêt de notre méthodologie d'annotation qui soit indépendant d'une vérité terrain pré-établie.
			Cependant, en considérant les résultats obtenus, ce score d'accord ne semble pas répondre à cette objectif.
			
			% Trop instable pour définir un cas d'arrêt.
			Tout d'abord, malgré une corrélation acceptable avec la performance théorique du \textit{clustering} (moyenne à $0.49$, voir \textsc{Table~\ref{table:4.5.1-ETUDE-RENTABILITE-CORRELATION-ACCORD-PERFORMANCE}}), l'évolution du score d'accord reste instable.
			En effet, les nombreuses variations et saccades rendent toute analyse de rentabilité difficile voire impossible, ce qui ne permet pas de définir un cas d'arrêt pour notre méthode d'annotation.
			
			\begin{leftBarExamples}
				Concernant l'évolution du paramétrage favori (\textsc{Figure~\ref{figure:4.5.1-ETUDE-RENTABILITE-ACCORD-ANNOTATION-CLUSTERING}} \textbf{(4)}), nous ne pouvons pas précisément définir à partir de quelle itération les résultats semblent intéressant car le score d'accord oscille longuement entre $50$\% et $100$\% avec des pics de plus de $25$\% entre deux itérations. 
			\end{leftBarExamples}
			
			\begin{leftBarAuthorOpinion}
				% Rappel: l'objectif de notre méthode d'annotation est de corriger le plus rapidement un clustering.
				Après réflexion, ce score d'accord est probablement infructueux à cause du fonctionnement même de notre méthode, dont l'objectif est de corriger le partitionnement des données en utilisant un minimum de contraintes.
				En effet, dans le cadre de l'optimisation des paramètres réalisée en \textsc{Section~\ref{section:4.2-HYPOTHESE-EFFICIENCE}}, nous avons retenu dans notre paramétrage favori la sélection des contraintes les plus proches entre deux \textit{clusters} différents (\texttt{samp.closest.diff}) : cette sélection permet ainsi de décrire efficacement l'emplacement des frontières de \textit{clusters}.
				
				% Cet échantillonnage est non supervisé : il y a de nombreuses saccades.
				Or, cet échantillonnage reste une méthode non-supervisée : aux premières itérations, les contraintes sélectionnées ont de bonnes chances de mettre en avant une frontière mal positionnée, mais au fur et à mesure que des contraintes s'ajoutent, les nouvelles contraintes ont moins de chances de trouver des bordures de \textit{clusters} qui ne soient pas encore caractérisées.
				De ce fait, il se peut que les dernières sélections n'identifient aucune nouvelle frontière, qu'elles se concentrent sur des frontières déjà bien positionnées ou déjà décrites par d'autres contraintes, ou qu'elles nécessitent plusieurs itérations pour caractériser des frontières complexes (le comportement des autres méthodes de sélections représentées en \textsc{Figure~\ref{figure:4.5.1-ETUDE-RENTABILITE-ACCORD-ANNOTATION-CLUSTERING}} peut être illustré par des raisonnements similaires).
				L'ensemble de ces cas de figures peut ainsi expliquer les nombreuses saccades dans l'évolution du score d'accord : tantôt la sélection semble pertinente, tantôt la sélection semble inutile.
			\end{leftBarAuthorOpinion}
			
			% Trop instable pour caratériser une itération.
			Pour aller plus loin, nous pouvons aussi critiquer le score de corrélation qui ne semble pas montrer de lien fort entre les performances théoriques et les accords calculés, tant sur l'ensemble des tentatives que pour le paramétrage favori.
			Il est même rare d'observer des chutes importantes d'accords qui soient accompagnées d'une variation significative de \texttt{v-measure} avec la vérité terrain.
			Au final, ce score d'accord n'est donc pas vraiment représentatif de la rentabilité d'une itération ou de l'évolution de la pertinence du \textit{clustering}.
			\begin{leftBarAuthorOpinion}
				Pour expliquer cette absence de corrélation, il est possible que l'analyse des annotations réalisées ait été une idée infructueuse : les $50$ contraintes annotées peuvent peut-être exprimer un désaccord avec le précédent \textit{clustering}, mais ce n'est pas pour autant que l'ajout de ces nouvelles contraintes impacte significativement la pertinence globale du partitionnement des données.
			\end{leftBarAuthorOpinion}
			
			% Conclusions et suggestion.
			En conclusion, \textbf{le score d'accord entre l'annotation courante et le \textit{clustering} précédent n'est pas adéquat pour estimer un cas d'arrêt de notre méthode d'annotation}, principalement car il est trop instable et qu'il ne représente pas bien les bénéfices obtenus à chaque itération.
			Ainsi, si une analyse de l'annotation réalisée n'est pas fructueuse, nous nous tournerons vers l'analyse plus abstraite des différences entre deux résultats de \textit{clustering}
	
	%%%
	%%% Subsection 4.5.2: Étude de l'évolution de la différence entre deux \textit{clusterings} consécutifs.
	%%%
	\subsection{Étude de l'évolution de la différence entre deux \textit{clusterings} consécutifs}
	\label{section:4.5.2-ETUDE-RENTABILITE-SIMILARITE-CLUSTERING}
		
		% Objectif de l'expérience.
		Nous venons de conclure que l'analyse de l'accord entre l'annotation et le partitionnement des données ne permet pas d'estimer la rentabilité d'une itération de notre méthode d'annotation.
		Parmi les explications possibles, nous avons mis en cause l'analyse du lot de contraintes annotées : en effet, ce n'est pas parce que l'annotation de contraintes est en désaccord avec le précédent partitionnement des données que les correctifs associés auront un impact significatif sur le prochain partitionnement.
		Ainsi, nous voulons analyser l'évolution de la différence entre deux \textit{clusterings} successifs : en effet, si une itération apporte des correctifs ayant un impact, alors il devrait y avoir des différences visibles entre les deux itérations de \textit{clustering}.
	
		%%% Protocole expérimental.
		\subsubsection{Protocole expérimental}
			
			% Axiome.
			\begin{leftBarWarning}
				Dans le cadre de cette étude, nous supposons que l'expert métier connaît parfaitement le domaine traité dans ce jeu de données, et qu'il est capable de caractériser sans ambiguïté la similitude entre deux données issues de cet ensemble.
			\end{leftBarWarning}
			
			% Pseudo-code.
			Pour résumer le protocole expérimental que nous décrivons ci-dessous, vous pouvez vous référer au pseudo-code décrit dans \textsc{Algorithme~\ref{algorithm:4.5.1-ETUDE-RENTABILITE-SIMILARITE-CLUSTERING-PROTOCOLE}}.
			
			\begin{algorithm}
				\KwData{jeu de données annotés (vérité terrain)}
				%
				\ForEach{jeux de données à tester}{
					\textbf{initialisation (données)}: récupérer les données et la vérité terrain \;
					\textbf{initialisation (contraintes)}: créer une liste vide de contraintes \;
					\textbf{prétraitement}: supprimer le bruit dans les données avec \texttt{prep.simple} \;
					\textbf{vectorisation}: transformer les données en vecteurs avec \texttt{vect.tfidf} \;
					\textbf{clustering initial}: regrouper les données par similarité avec \texttt{clust.kmeans.cop} \;
					\Repeat{annotation de toutes les contraintes possibles}{
						\textbf{échantillonnage}: sélectionner des contraintes avec \texttt{samp.closest.diff} \;
						\textbf{simulation d'annotation}: caractériser les contraintes grâce à la vérité terrain \;
						\textbf{intégration}: ajouter les nouvelles contraintes au gestionnaire de contraintes \;
						\textbf{clustering}: regrouper les données par similarité avec \texttt{clust.kmeans.cop} \;
						\textbf{rentabilité}: calculer la différence entre les deux précédents \textit{clusterings} \;
					}
				}
				\textbf{analyse 1}: afficher l'évolution de la différence entre deux \textit{clustering} consécutifs \;
				\textbf{analyse 2}: calculer la corrélation entre le score de différence et le score de performance \;
				%
				\KwResult{discussion sur la rentabilité d'après la différence entre \textit{clusterings}}
				%
				\caption{\textit{
					Description en pseudo-code du protocole expérimental de l'étude de l'évolution de la différence entre deux \textit{clustering} consécutifs.
				}}
				\label{algorithm:4.5.1-ETUDE-RENTABILITE-SIMILARITE-CLUSTERING-PROTOCOLE}
			\end{algorithm}

			% Détails de l'expérience.
			Nous nous appuyons sur le même protocole que l'expérience précédente (cf. \textsc{Section~\ref{section:4.5.1-ETUDE-RENTABILITE-ACCORD-ANNOTATION-CLUSTERING}}) : nous utilisons comme vérité terrain le jeu de données \texttt{Bank Cards (v1.0.0)}, nous réalisons $5$ tentatives complètes de la méthode du \textit{clustering} interactif en utilisant notre paramétrage favori, et nous simulons l'annotation par un expert d'un lot de $50$ contraintes à chaque itération.
			
			% Ajout de la comparaison entre clustering.
			Cependant, au lieu de calculer un score d'accord entre annotation et \textit{clustering}, nous estimons la différence entre le \textit{clustering} précédent et le \textit{clustering} obtenu grâce aux dernières annotations.
			Cette différence entre deux \textit{clustering} $X$ et $Y$ est obtenue par la formule $1-\texttt{v-measure}(X,Y)$ où la \texttt{v-measure} caractérise la ressemblance entre deux partitionnements des données (\cite{rosenberg-hirschberg:2007:vmeasure-conditional-entropybased}).
			Pour nous permettre de discuter de l'utilité de ce score pour prédire la stabilisation du \textit{clustering} et ainsi définir un cas d'arrêt de notre méthodologie d'annotation, nous calculons aussi le score de corrélation entre cette différence et la performance obtenue à l'aide d'une vérité terrain (la corrélation \texttt{r} de \textit{Pearson} (\cite{kirch:2008:pearson-correlation-coefficient}) est utilisée).
			
			\begin{leftBarIdea}
				Comme précédemment, nous concentrons l'étude sur notre paramétrage favori (voir \textsc{Section~\ref{section:4.4.3-ETUDE-PERTINENCE-RESUME-AUTOMATIQUE}}).
				Cependant, afin de compléter notre discussion avec d'autres points de comparaison, nous analysons aussi les autres paramétrages implémentés, notamment les meilleurs paramétrages moyens identifiés lors de l'hypothèse d'efficience (voir \textsc{Section~\ref{section:4.2-HYPOTHESE-EFFICIENCE}}).
			\end{leftBarIdea}
			
			% Référence scripts.
			\begin{leftBarInformation}
				Les scripts de l'expérience, réalisés avec des \textit{notebooks} Python (\cite{van-rossum-drake:2009:python-reference-manual}), sont disponibles dans un dossier dédié de~\cite{schild:2021:cognitivefactory-interactiveclusteringcomparativestudy}.
			\end{leftBarInformation}

		%%% Résultats
		\subsubsection{Résultats obtenus}
			
			% Figure : décroissance générale.
			La \textsc{Figure~\ref{figure:4.5.2-ETUDE-RENTABILITE-SIMILARITE-CLUSTERING}} représente l'évolution moyenne du score de différence entre deux \textit{clusterings} pour les quatre paramétrages mis en avant lors de nos études.
			Nous pouvons constater une tendance générale à la décroissance vers $0$\% de ce score de différence : pour le paramétrage favori, la différence moyenne entre deux \textit{clustering} est initialement comprise entre $25$\% et $35$\% jusqu'à l'itération $10$, elle chute ensuite pour être inférieure à $5$\% après l'itération $20$, et elle termine enfin en oscillant très légèrement ($\pm1$\%) autour de $0$\% jusqu'à la fin des annotations.
			\begin{figure}[!htb]
				\centering
				\includegraphics[width=0.95\textwidth]{figures/etude-rentabilite-similarite-clustering}
				\caption{
					Évolution de la différence de résultats entre deux itérations de \textit{clustering}.
					Les évolutions moyennes de différents paramétrages de la méthode sont exposées :
					\textbf{(1)} meilleur paramétrage moyen pour atteindre une annotation partielle ;
					\textbf{(2)} meilleur paramétrage moyen pour atteindre une annotation suffisante ;
					\textbf{(3)} meilleur paramétrage moyen pour atteindre une annotation exhaustive ;
					et \textbf{(4)} paramétrage favori.
					À titre d'information, les courbes en noir représentent l'évolution de la \texttt{v-measure} entre le \textit{clustering} et la vérité terrain.
				}
				\label{figure:4.5.2-ETUDE-RENTABILITE-SIMILARITE-CLUSTERING}
			\end{figure}
			
			% Tableau : corrélation forte.
			La \textsc{Table~\ref{table:4.5.2-ETUDE-RENTABILITE-CORRELATION-SIMILARITE-PERFORMANCE}} contient le score de corrélation entre cette différence et la performance théoriques obtenue grâce à la vérité terrain.
			Cette corrélation est forte : $0.75$ sur l'ensemble des tentatives, $0.93$ sur les tentatives utilisant notre paramétrage favori.
			La \textsc{Figure~\ref{figure:4.5.2-ETUDE-RENTABILITE-SIMILARITE-CLUSTERING}} confirme cette corrélation :
			\begin{itemize}
				\item un score de \texttt{v-measure} avec la vérité terrain proche de $100$\% est accompagné d'un score de différence proche de $0$\% (après l'itération $20$ pour \textbf{(1)}, après l'itération $20$ pour \textbf{(2)}, après l'itération $30$ pour \textbf{(3)} et après l'itération $22$ pour \textbf{(4)}) ;
				\item une croissance de performance est généralement accompagnée d'un score non nul de différence (voir \textbf{(2)} et \textbf{(4)} entre les itérations $0$ et $20$), et plusieurs pics de performance sont accompagnés de scores forts de différence (particulièrement visible sur \textbf{(1)} vers l'itération $5$ et entre les itérations $10$ et $15$) ;
				\item il est toutefois à noter que l'inverse n'est pas vrai : un score non nul de différence n'accompagne pas forcément une croissance de performance, mais peut simplement caractériser un changement de partitionnement, comme c'est le cas dans \textbf{(3)} entre les itérations $0$ et $10$ où des modifications ont lieu (score de différence non nul) mais où la performance par rapport à la vérité terrain stagne.
			\end{itemize}
			\begin{table}[!htb]
				\begin{center}
				\begin{tabular}{|c|r|}
				
					\hline
					% ENTETE DU TABLEAU
					\multicolumn{1}{|c|}{\shortstack[c]{
						Paramétrage
					}}
						& \multicolumn{1}{c|}{\shortstack[c]{
							Corrélation
						}}
						\tabularnewline
						\hline
					
					% Annotation partielle.
					Meilleur paramétrage moyen pour une annotation partielle \textbf{(1)}
						& $0.96$
						\tabularnewline
						\hline
					
					% Annotation suffisante.
					Meilleur paramétrage moyen pour une annotation suffisante \textbf{(2)}
						& $0.92$
						\tabularnewline
						\hline
					
					% Annotation exhaustive.
					Meilleur paramétrage moyen pour une annotation exhaustive \textbf{(3)}
						& $0.85$
						\tabularnewline
						\hline
					
					% Paramétrage favori
					Paramétrage favori \textbf{(4)}
						& $0.93$
						\tabularnewline
						\hline
					
					% Moyenne des $960$ tentatives.
					Moyenne des $960$ tentatives
						& $0.75$
						\tabularnewline
						\hline
					
				\end{tabular}
				\end{center}
				\caption{
					Score de corrélation \texttt{r} de \textit{Pearson} entre la performance du \textit{clustering} obtenu à l'aide d'une vérité terrain (\texttt{v-measure}) et le score de différence entre deux \textit{clusterings} consécutifs.
				}
				\label{table:4.5.2-ETUDE-RENTABILITE-CORRELATION-SIMILARITE-PERFORMANCE}
			\end{table}
			
			% Description de la figure : autres paramétrages
			Les autres paramétrages représentés dans \textbf{(1)}, \textbf{(2)} et \textbf{(3)} comportent des tendances similaires (décroissance générale, forte corrélation avec la performance théorique) à quelques détails (\textbf{(1)} commence avec des scores de différence très forts avant décroître avec de nombreux pics ; \textbf{(3)} croît légèrement avant d'entamer sa décroissance, ...).
			
		%%% Discussion
		\subsubsection{Discussion}
		
			% Rappel de l'objectif : trouver un cas d'arrêt en regardant l'évolution de la différence entre deux clusterings.
			Dans cette étude, nous avons analysé l'évolution du score de différence entre deux itérations de \textit{clustering} dans l'espoir de définir un cas d'arrêt de notre méthodologie d'annotation qui soit indépendant d'une vérité terrain pré-établie.
			
			% Avantage 1 : Caractérise la rentabilité.
			Tout d'abord, nous pouvons affirmer qu'il y une forte corrélation entre l'évolution de ce score de différence et l'évolution du score de performance (voir \textsc{Table~\ref{table:4.5.2-ETUDE-RENTABILITE-CORRELATION-SIMILARITE-PERFORMANCE}} : \texttt{r} moyen de $0.75$ ; \texttt{r} supérieur à $0.85$ pour les paramétrages mis en avant).
			Cette corrélation est confirmée visuellement grâce à la \textsc{Figure~\ref{figure:4.5.2-ETUDE-RENTABILITE-SIMILARITE-CLUSTERING}} : plus les différences entre \textit{clusterings} sont faibles, plus les performances des \textit{clusterings} sont fortes.
			
			% Attention : Peut ne caractériser qu'un gros changement sans pour autant une amélioration.
			Un point d'attention est toutefois à retenir : une modification du partitionnement des données n'entraîne pas forcément un gain de performance (voir \textbf{(3)} entre les itérations $0$ et $10$ et \textbf{(4)} entre les itérations $0$ et $8$).
			Nous ne pouvons donc pas conclure que l'analyse de la différence entre eux itération de \textit{clustering} permet de caractériser totalement la rentabilité d'une itération.
			
			% Avantage 2 : Permet de définir un cas d'arrêts.
			Cependant, nous pouvons tout de même nous servir de ce score pour définir un cas d'arrêt pour notre méthodologie d'annotation lorsque la différence entre deux \textit{clusterings} est faible.
			Pour cela, il nous suffit de fixer un seuil bas du score de différence en dessous duquel il n'est plus rentable de faire de nouvelles itérations de la méthode car les performances devraient être suffisantes.
			Une analyse manuelle ou semi-manuelle (voir hypothèse de pertinence en \textsc{Section~\ref{section:4.4-HYPOTHESE-PERTINENCE}}) reste nécessaire pour confirmer la valeur métier du résultat obtenu.
			
			\begin{leftBarIdea}
				Si nous restons sur notre seuil théorique de $90$\% de \texttt{v-measure} (voir \textsc{Section~\ref{section:4.2-HYPOTHESE-EFFICIENCE}}) et que nous nous basons sur la \textsc{Figure~\ref{figure:4.5.2-ETUDE-RENTABILITE-SIMILARITE-CLUSTERING}} \textbf{(4)}, nous pouvons visuellement fixer ce seuil autour de $5$\% de différences.
				Le réglage fin de ce seuil pourra être le sujet de futures analyses complémentaires.
			\end{leftBarIdea}
			
			% Conclusions et suggestion.
			En conclusion, \textbf{le score de différences entre deux résultats de \textit{clustering} semble être un bon indicateur pour estimer un cas d'arrêt de notre méthodologie d'annotation}, et nous proposer d'utiliser un seuil de $5$\% pour implémenter ce cas d'arrêt.
			
	%%%
	%%% Subsection 4.5.3: Mise en commun des stratégies d'évaluation de la rentabilité d'une itération de la méthode et définition d'un cas d'arrêt indépendant d'une vérité terrain.
	%%%
	\subsection{Mise en commun des stratégies d'évaluation de la rentabilité d'une itération de la méthode et définition d'un cas d'arrêt indépendant d'une vérité terrain.}
	\label{section:4.5.3-ETUDE-RENTABILITE-MISE-EN-COMMUN}
			
		% Conclusion.
		\begin{leftBarSummary}
			Au cours de cette étude de rentabilité, nous avons pu voir que :
			\begin{itemize}
				\item[\itemko] l'analyse du score d'accord entre l'annotation courante et le \textit{clustering} précédent ne permet pas d'estimer la rentabilité d'une itération, ni de définir un cas d'arrêt de notre méthodologie d'annotation (cf. \textsc{Section~\ref{section:4.5.1-ETUDE-RENTABILITE-ACCORD-ANNOTATION-CLUSTERING}}) ;
				\item[\itemok] l'analyse des différences entre deux itérations de \textit{clusterings} est une approche prometteuse pour estimer la rentabilité d'une itération, bien qu'une modification significative entre deux résultats de \textit{clustering} n'implique pas forcément un gain de performance (les deux \textit{clustering} peuvent avoir une \texttt{v-measure} équivalente avec la vérité terrain) ;
				\item[\itemok] l'usage de différences entre deux itérations de \textit{clusterings} permet de définir un cas d'arrêt de de notre méthodologie d'annotation : si les différences sont faibles (par exemple : inférieures à $5$\%), alors les performances stagnent ou plafonnent, donc il peut être intéressant d'interrompre le \textit{clustering} interactif après avoir vérifier la manuellement pertinence des résultats obtenus (cf. (cf. \textsc{Section~\ref{section:4.5.2-ETUDE-RENTABILITE-SIMILARITE-CLUSTERING}}) et \textsc{Section~\ref{section:4.4.4-ETUDE-PERTINENCE-MISE-EN-COMMUN}}).
			\end{itemize}
		\end{leftBarSummary}
		
		% Transition: Vers Simulation d'erreurs.
		Pour terminer nos différentes analyses, il convient maintenant d'anticiper la présence d'erreurs d’annotation.
		En effet, nous avons fait jusqu'à présent l'hypothèse que l'annotateur ne se trompe jamais, mais cette hypothèse forte n'est pas toujours vérifiée en pratique.
		Pour estimer l'impact de ces erreurs ou incohérences d'annotation, nous devons donc réaliser une analyse de robustesse de notre méthode d'annotation : celle-ci sera réalisé en \textsc{Section~\ref{section:4.6-HYPOTHESE-ROBUSTESSE}}.


    %%%%%--------------------------------------------------------------------
    %%%%% Section 4.6: Hypothèse de robustesse.
    %%%%%--------------------------------------------------------------------
	\newpage
	\section{Évaluation de l'hypothèse de robustesse}
\label{section:4.6-HYPOTHESE-ROBUSTESSE}

	%%% Introduction / Transition.
	Dans les précédentes études, nous avons presque toujours analysé le \textit{clustering} interactif en supposant que l'annotateur connaît parfaitement le domaine traité par le jeu de données et qu'il est capable de caractériser sans ambiguïté la similitude entre deux données issues de cet ensemble.
	Bien entendu, cette hypothèse forte n'est pas toujours vérifiée en situation réelle : l'interprétation du langage peut contenir certaines ambiguïtés, l'opérateur peut faire des erreurs d’inattention, et deux annotateurs peuvent avoir des avis contraire sur un même sujet.
	Or, comme notre méthode d'annotation est itérative, elle est a priori sensible aux dérives fonctionnement liées à ce type de contradictions.
	Dans cette section, nous nous intéresserons donc à la robustesse du \textit{clustering} interactif en présence d'incohérences dans les contraintes et aux moyens de les contrer.
	Pour cela, nous aimerions donc vérifier l'hypothèse suivante :
	
	%%% Formulation des hypothèses:
	\begin{tcolorbox}[
		title=\faVial~\textbf{Hypothèse de robustesse}~\faVial,
		colback=colorTcolorboxHypothesis!15,
		colframe=colorTcolorboxHypothesis!75,
		width=\linewidth
	]
		% Hypothèse.
		« \textbf{
			Au cours d'une méthodologie d'annotation basée sur le \textit{clustering} interactif, il est possible d'estimer le taux d'incohérences dans les contraintes ainsi que leur impact sur les performances de la méthode.
		} » \\
		
		% Figure.
		La \textsc{Figure~\ref{figure:4.6-HYPOTHESE-ROBUSTESSE}} illustre cette hypothèse et l'espoir de estimer l'impact différences ou d'erreurs d'annotations sur le nombre d'itérations de la méthode.
		%
		\begin{figure}[H]  % keep [H] to be in the tcolorbox.
			\centering
			\includegraphics[width=0.95\textwidth]{figures/hypotheses-06-robustesse}
			\caption{
				Illustration des études réalisées sur le \textit{clustering} interactif (\textit{étape 6/6}) en schématisant l'évolution de la pertinence (\textit{valeur métier évaluée par l'expert et exprimé en nombre de clusters}) d'une base d'apprentissage en cours de construction en fonction du coût temporel de la méthode (\textit{temps nécessaire à l'expert métier et à la machine}), ainsi que les marges d'erreurs représentant l'impact de différences d'annotation sur le nombre d'itérations nécessaire à la méthode.
			}
			\label{figure:4.6-HYPOTHESE-ROBUSTESSE}
		\end{figure}
	\end{tcolorbox}
		
	% Résumé de l'étude.
	Afin de vérifier cette hypothèse, nous organisons trois expériences :
	\begin{itemize}
		\item une étude de cas sur la \textbf{correction des incohérences d'annotation} (cf. \textsc{Section~\ref{section:4.6.1-ETUDE-ROBUSTESSE-INTERETS-CORRECTION-ERREURS}}) ;
		\item une simulation de l'\textbf{impact des incohérences d'annotation} sur les performances de la méthode (cf. \textsc{Section~\ref{section:4.6.2-ETUDE-ROBUSTESSE-SIMULATION-IMPACT-ERREURS}}) ;
		\item une étude de cas sur le \textbf{score inter-annotateurs} obtenu lors d'une annotation de contraintes en situation réelle avec plusieurs opérateurs (cf. \textsc{Section~\ref{section:4.6.3-ETUDE-ROBUSTESSE-SCORE-INTER-ANNOTATEURS}}).
	\end{itemize}
	
	
	%%%
	%%% Subsection 4.6.1: Étude de l'intérêt de la correction des incohérences d'annotation.
	%%%
	\subsection{Étude de l'intérêt de la correction des incohérences d'annotation}
	\label{section:4.6.1-ETUDE-ROBUSTESSE-INTERETS-CORRECTION-ERREURS}
		
		% Objectif de l'expérience.
		Nous cherchons à estimer la robustesse du \textit{clustering} interactif face aux incohérences d'annotation.
		Dans cette étude, nous nous intéressons plus particulièrement à l'intérêt de la détection et de la correction des conflits présents dans les contraintes.
		En effet, deux approches de travail s'opposent :
		\begin{itemize}
			\item une approche naïve ignorant simplement les conflits : celle-ci n'engendre pas de coûts supplémentaires, mais elle s'expose aux risques de dérives de fonctionnement ;
			\item une seconde approche avec correction des conflits : celle-ci nécessite de revoir ou de ré-annoter des contraintes, impliquant un coût supplémentaire, mais permet de limiter l'impact de dérives potentielles.
		\end{itemize}
		Pour trancher entre ces deux options, nous simulons ces deux approches afin d'estimer si l'absence de correction induit une régression significative des performances de notre méthode d'annotation.
	
		%%% Protocole expérimental.
		\subsubsection{Protocole expérimental}
			
			% Pseudo-code.
			Pour résumer le protocole expérimental que nous décrivons ci-dessous, vous pouvez vous référer au pseudo-code décrit dans \textsc{Algorithme~\ref{algorithm:4.6.1-ETUDE-ROBUSTESSE-INTERETS-CORRECTION-ERREURS-PROTOCOLE}}.
			\todo{FORMAT: ajuster l’enchaînement pour ne pas avoir de page "blanche" avec l'algorithme.}
			
			\begin{algorithm}
				\KwData{jeu de données annoté (vérité terrain)}
				\KwIn{taux d'erreurs à tester}
				%
				\ForEach{arrangement de taux d'erreur à tester}{
					\textbf{initialisation (données)}: récupérer les données et la vérité terrain \;
					\textbf{initialisation (contraintes)}: créer une liste vide de contraintes \;
					\textbf{prétraitement}: supprimer le bruit dans les données avec \texttt{prep.simple} \;
					\textbf{vectorisation}: transformer les données en vecteurs avec \texttt{vect.tfidf} \;
					\textbf{clustering initial}: regrouper les données par similarité avec \texttt{clust.kmeans.cop} \;
					\textbf{évaluation}: estimer l'équivalence entre le \textit{clustering} obtenu et la vérité terrain \;
					\Repeat{annotation de toutes les contraintes possibles}{
						\textbf{échantillonnage}: sélectionner des contraintes avec \texttt{samp.closest.diff} \;
						\textbf{échantillonnage d'erreurs}: définir les contraintes qui seront erronées \;
						\textbf{simulation d'annotation}: caractériser les contraintes grâce à la vérité terrain \;
						\If{absence de correction des conflits d'annotation}{
							\textbf{intégration naïve}: ignorer les conflits avec le gestionnaire de contraintes \;
						}
						\ElseIf{détection et correction des conflits d'annotation}{
							\textbf{intégration corrective}: changer les annotations erronées en conflit \;
						}
						\textbf{clustering}: regrouper les données par similarité avec \texttt{clust.kmeans.cop} \;
						\textbf{évaluation}: estimer l'équivalence entre le \textit{clustering} obtenu et la vérité terrain \;
					}
				}
				\textbf{analyse}: déterminer l'impact par itération et par taux d'erreurs de la correction \;
				%
				\KwResult{discussion sur l'impact de la correction des incohérences}
				%
				\caption{\textit{
					Description en pseudo-code du protocole expérimental de l'étude d'intérêt de la correction des incohérences d'annotation.
				}}
				\label{algorithm:4.6.1-ETUDE-ROBUSTESSE-INTERETS-CORRECTION-ERREURS-PROTOCOLE}
			\end{algorithm}
			
			% Description de la vérité terrain.
			Nous utilisons comme vérité terrain le jeu de données \texttt{Bank Cards (v1.0.0)} : ce dernier traite des demandes les plus fréquentes des clients en ce qui concerne la gestion de leur carte bancaire.
			Il est composé de $500$ questions rédigées en français et réparties en $10$ classes (\texttt{perte ou vol de carte}, \texttt{carte avalée}, \texttt{commande de carte}, ...).
			Pour plus de détails, consultez l'annexe~\ref{annex:C.1-DATASET-BANK-CARDS}.
			
			% Description des tentatives de la méthode avec simulation d'erreurs.
			Sur ce jeu de données, nous exécutons une tentative complète
			\footnote{Tentative complète : itérations d'échantillonnage, d'annotation et de \textit{clustering} jusqu'à annotation de toutes les contraintes possibles.}
			de la méthode du \textit{clustering} interactif en utilisant notre paramétrage favori (voir \textsc{Section~\ref{section:4.4.3-ETUDE-PERTINENCE-RESUME-AUTOMATIQUE}}).
			Toutefois, contrairement aux précédents expériences, nous allons ajouter un pourcentage de contraintes erronées à chaque itération :
			\begin{itemize}
				\item Le taux d'erreurs insérées, variant de $0$\% à $50$\% par pas de $5$\%, reste fixe tout au long d'une même tentative de notre méthode : nous pouvons ainsi analyser l'impact d'un taux d'erreur fixe sur les performances au courant des itérations ;
				\item Les contraintes erronées à insérer sont tirées aléatoirement parmi le lot de contraintes qui aurait été échantillonné au cours d'une tentative sans erreur : ainsi, nous pouvons comparer itération par itération toutes ces simulations car elles partagent la même base de contraintes (aux valeurs de \texttt{MUST-LINK} et \texttt{CANNOT-LINK} près) ;
			\end{itemize}
			
			% Description des tentatives de la méthode avec gestion des conflits.
			Puisque nous introduisons des erreurs d'annotations, des conflits vont apparaître dans le gestionnaire de contraintes.
			Pour rappel, un conflit est détecté dans le cas où l'ajout d'une nouvelle contrainte annotée contredit ce qui a été précédemment déduit grâce aux propriétés de transitivité des contraintes de types \texttt{MUST-LINK}et \texttt{CANNOT-LINK} (voir \textsc{Figure~\ref{figure:3.3-CONTRAINTES-TRANSITIVITE}} en \textsc{Section~\ref{section:3.3.2-GESTION-DES-CONTRAINTES}}).
			Pour les traiter, nous testons deux approches :
			\begin{itemize}
				\item l'approche naïve ignorant simplement les conflits : si la prochaine contrainte à ajouter est incompatible avec la base de contraintes déjà intégrées au gestionnaire, alors nous ignorons simplement son existence sans remettre en question les précédentes annotations ;
				\item l'approche avec correction des conflits : pour simuler la correction d'un expert, nous recréons à chaque itération le gestionnaire de contraintes en intégrant d'abord les contraintes correctes puis les contraintes erronées ; ainsi, les conflits ne peuvent arrivent qu'à l'ajout d'une contrainte erronée, et il suffit d'ajouter sa version exacte pour simuler la correction de l'expert.
			\end{itemize}
			
			% Description des tentatives de la méthode avec les répétitions.
			Ainsi, il y a donc $11$ taux d'erreurs différents à simuler, chacun suivant $2$ approches de gestion de conflits différentes, et chacune de ces simulations d'erreurs seront répétées $10$ fois sur chaque tentative complète de la méthode pour limiter contrer les aléas statistiques des tirages de contraintes erronées, ce qui représente $220$ simulation par tentatives.
			Enfin, chaque tentative complète de \textit{clustering} interactif est répétée $5$ fois pour contrer les aléas statistiques des exécutions, ce qui représente un total de $1100$ tentatives complètes à simuler.

			% Description de l'analyse.
			Enfin, nous afficherons l'évolution de la performance moyenne du \textit{clustering} obtenu en fonction des divers taux d'erreurs simulés, et discuterons de l'impact au cours des itérations de la présence ou de l'absence de corrections des conflits d'annotations détectés.
			
			% Référence scripts.
			\begin{leftBarInformation}
				Les scripts de l'expérience, réalisés avec des \textit{notebooks} Python (\cite{van-rossum-drake:2009:python-reference-manual}), sont disponibles dans un dossier dédié de~\cite{schild:2021:cognitivefactory-interactiveclusteringcomparativestudy}.
			\end{leftBarInformation}

		%%% Résultats
		\subsubsection{Résultats obtenus}
		
			% Description statistiques.
			La \textsc{Figure~\ref{figure:4.6.1-ETUDE-ROBUSTESSE-INTERETS-CORRECTION-ERREURS}} représente l'évolution moyenne de la \texttt{v-measure} du \textit{clustering} en fonction du nombre d'itération de la méthode, déclinée avec les $11$ taux d'erreurs simulés et les $2$ approches de gestion des conflits.
			Les contraintes utilisées sont basées sur les échantillonnages réalisées au cours des tentatives sans erreurs : comme les mêmes contraintes sont donc utilisées (aux valeurs d'annotations près), toutes les courbes sont comparables point par point.
			
			\begin{leftBarWarning}
				Toutefois, il est important de noter que les tentatives sans contraintes ont besoin de maximum $3~000$ contraintes pour annoter toutes les contraintes possibles et leurs transitivités (moyenne: $2~488$, écart-type: $327$).
				Ainsi, toutes les courbes simulant les différents taux d'erreurs sont tronquées à $3~000$ contraintes, que les tentatives aient convergé ou non.
				Nous serons sensible à cette information pour ne pas faire de mauvaises interprétations, car le dernier point des différentes courbes ne représente pas forcément le point de convergence des tentatives associées.
			\end{leftBarWarning}
			
			\begin{figure}[!htb]
				\centering
				\includegraphics[width=0.95\textwidth]{figures/etude-erreur-simulation-impact-closest}
				\caption{
					Évolutions de la moyenne de la \texttt{v-measure} entre un résultat obtenu et la vérité terrain en fonction du nombre de contraintes annotées au cours des itérations du \textit{clustering} interactif.
					Les courbes en dégradé de couleur représentent les déclinaisons de cette évolution en intégrant un pourcentage d'annotations erronées (allant de $0$\% et $50$\%).
					\textbf{(1)} et \textbf{(2)} représente respectivement l'approche ignorant les conflits dans les contraintes et l'approche corrigeant les conflits détectés par le gestionnaire de contraintes.
					Toutes les courbes sont tronquées à $3~000$ contraintes.
				}
				\label{figure:4.6.1-ETUDE-ROBUSTESSE-INTERETS-CORRECTION-ERREURS}
			\end{figure}
			
			% Description de la figure.
			\todo[inline]{A REDIGER: description de la figure}

		%%% Discussion
		\subsubsection{Discussion}
		
			% Rappel de l'objectif : ...
			\todo[inline]{A REDIGER: rappel de l'objectif}
		
			% Remaques expérience utilisateur.
			\todo[inline]{A REDIGER: Super important de corriger !}
			\todo[inline]{A REDIGER: Besoin de mécanisme pour prédire et mettre de la redondance}
			
			% Conclusions et suggestion.
			\todo[inline]{A REDIGER: ouverture sur l'impact des incohérences}
	
	
	%%%
	%%% Subsection 4.6.2: Étude de l'impact des incohérences d'annotation sur les performances.
	%%%
	\subsection{Étude de l'impact des incohérences d'annotation sur les performances}
	\label{section:4.6.2-ETUDE-ROBUSTESSE-SIMULATION-IMPACT-ERREURS}
		
		% Objectif de l'expérience.
		\todo[inline]{A REDIGER: objectif de l'expérience}
	
		%%% Protocole expérimental.
		\subsubsection{Protocole expérimental}
			
			% Axiome.
			% Pseudo-code.
			% Détails de l'expérience.
			\todo[inline]{A REDIGER}
			
			% Référence scripts.
			\begin{leftBarInformation}
				Les scripts de l'expérience, réalisés avec des \textit{notebooks} Python (\cite{van-rossum-drake:2009:python-reference-manual}), sont disponibles dans un dossier dédié de~\cite{schild:2021:cognitivefactory-interactiveclusteringcomparativestudy}.
			\end{leftBarInformation}

		%%% Résultats
		\subsubsection{Résultats obtenus}
			\todo[inline]{A REDIGER}
		
			% Description statistiques.
			\todo[inline]{A REDIGER:}
			La \textsc{Table~\ref{table:4.6.2-ETUDE-ROBUSTESSE-SIMULATION-IMPACT-ERREURS}}
			
			\begin{table}[!htb]
				\begin{center}
				\begin{tabular}{|c|r|r|r|r|r|r|r|r|r|}
					% ENTETE DU TABLEAU
					% Erreur 0.00%
					% Erreur 0.05%
					% Erreur 0.10%
					% Erreur 0.15%
					% Erreur 0.20%
					% Erreur 0.25%
					% Erreur 0.30%
					% Erreur 0.35%
					% Erreur 0.40%
					% Erreur 0.45%
					% Erreur 0.50%
				\end{tabular}
				\end{center}
				\caption{
					Simulation ...
				}
				\label{table:4.6.2-ETUDE-ROBUSTESSE-SIMULATION-IMPACT-ERREURS}
			\end{table}

		%%% Discussion
		\subsubsection{Discussion}
		
			% Rappel de l'objectif : ...
			\todo[inline]{A REDIGER: rappel de l'objectif}
		
			% Remaques expérience utilisateur.
			\todo[inline]{A REDIGER: prédiction de retard}
			
			% Conclusions et suggestion.
			\todo[inline]{FIN}
			
			
	%%%
	%%% Subsection 4.6.3: Étude du score inter-annotateurs obtenu avec des opérateurs en situation réelle.
	%%%
	\subsection{Étude du score inter-annotateurs obtenu avec des opérateurs en situation réelle}
	\label{section:4.6.3-ETUDE-ROBUSTESSE-SCORE-INTER-ANNOTATEURS}
		
		% Objectif de l'expérience.
		Nous voulons étudier le score d'accord inter-annotateurs calculé lors d'une annotation de contraintes par plusieurs expert métiers en situation réelle.
		Pour cela, nous reprenons l'expérience de la \textsc{Section~\ref{section:4.3.1-ETUDE-COUTS-TEMPS-ANNOTATION}} visant à estimer le temps moyen d'annotation d'un lot de contraintes, et nous adaptons son protocole expérimental pour estimer l'accord inter-annotateurs.
		
	
		%%% Protocole expérimental.
		\subsubsection{Protocole expérimental}
			
			% Axiome.
			\begin{leftBarWarning}
				Dans cette étude, nous supposons que les annotateurs de l'expérience connaissent parfaitement le domaine traité dans le jeu de données, et qu'ils sont capables de caractériser sans ambiguïté la similitude entre deux données issues de cet ensemble.
				Afin de pourvoir faire cette hypothèse forte, et ainsi limiter les bruits dans l'analyse des résultats, le jeu de données devra traiter d'un sujet de culture générale (ne nécessitant donc pas de connaissance particulière) et des réviseurs supprimeront en amont et d'un commun accord les données trop spécifiques ou trop ambiguës.
			\end{leftBarWarning}
			
			% Pseudo-code.
			Pour résumer le protocole expérimental que nous décrivons ci-dessous, vous pouvez vous référer au pseudo-code décrit dans \textsc{Algorithme~\ref{algorithm:4.6.3-ETUDE-ROBUSTESSE-SCORE-INTER-ANNOTATEURS-PROTOCOLE}}.

			\begin{algorithm}
				\KwData{jeu de données annoté (vérité terrain)}
				\KwIn{plusieurs réviseurs, plusieurs annotateurs}
				%
				\textbf{initialisation}: définir et revoir le jeu de données entre réviseurs \;
				\textbf{échantillonnage}: sélectionner une base de contraintes équilibrée \;
				\ForEach{annotateur}{
					 \While{la base de contraintes n'a pas été entièrement annotée}{
						\textbf{annotation}: annoter une partie des contraintes \;
						\textbf{revue}: revue des contraintes en conflits d'annotation \;
					}
				}
				%
				\KwResult{modélisation du score inter-annotateurs sur le lot de contraintes}
				%
				\caption{\textit{
					Description en pseudo-code du protocole expérimental de l'étude du score inter-annotateurs d'annotation d'un lot de contraintes par plusieurs experts métiers en situation réelle.
				}}
				\label{algorithm:4.3.3-ETUDE-COUTS-TEMPS-ANNOTATION-PROTOCOLE}
			\end{algorithm}
			
			% Détails de l'expérience : préparation du jeu de données.
			Nous allons procéder en plusieurs étapes.
			D'abord, il faut choisir un jeu de données approprié : pour valider notre hypothèse forte sur les compétences de nos annotateurs, nous cherchons un jeu de données traitant d'un sujet de culture général.
			Pour cette expérience, nous avons donc choisi \texttt{MLSUM} : une collecte d'articles de journaux, classés par catégorie de publication et décrits par leur titre et leur résumé.
			Nous nous intéressons ici à la tâche de classification d'un titre d'article en fonction de sa catégorie de publication.
			Comme certains titres peuvent porter à confusion (un titre d'article n'étant pas toujours explicite sur son contenu), deux réviseurs sont chargés de choisir les données les plus explicites sur un échantillon d'un millier de données représentatives des catégories les plus communes.
			L'échantillon résultant, noté \texttt{MLSUM FR Train Subset (v1.0.0-schild)}, est composé de $744$ titres d'articles rédigés en français et répartis en $14$ classes (\textit{économie}, \textit{sport}, ...).
			Pour plus de détails, consultez l'annexe~\ref{annex:C.2-DATASET-MLSUM-SUBSET-SCHILD}.
		
			% Détails de l'expérience : sélection des contraintes à annoter.
			A partir de ces données, nous sélectionnons un lot de $400$ contraintes à annoter.
			Pour faciliter l'analyse, l'échantillonnage sera un aléatoire équilibré d'après la vérité terrain en $200$ \texttt{MUST-LINK} et en $200$ \texttt{CANNOT-LINK}.
			
			% Détails de l'expérience : annotations et consignes.
			Ensuite, un groupe de $3$ annotateurs vont annoter la sélection de $400$ contraintes en plusieurs sessions.
			Les directives données aux opérateurs sont les suivantes:
			\begin{itemize}
				\item \textbf{Contexte de l'opérateur} :
				« \textit{Vous êtes des \textbf{experts de la presse et de l’actualité} ; Vous voulez classer des articles dans des catégories en fonction de leur titre ; Vous ne savez pas précisément quelles catégories vous allez utiliser pour classer vos articles ; Mais vous savez \textbf{caractériser la similitude} de deux articles} » ;
				\item \textbf{Contexte sur le jeu de données} :
				« \textit{Le thème sont les catégories d’articles de presse ; La vérité terrain contient entre $10$ et $20$ catégories parmi les plus communes de la presse ; La vérité terrain contient entre $30$ et $100$ articles par catégorie ; Vous \textbf{pouvez regarder le jeu de données non annoté} autant que vous le voulez (disponible dans l'onglet \texttt{TEXTS} de l'application)} » ;
				\item \textbf{Consignes d'annotations} :
				« \textit{Faites des séries de \textbf{15 minutes minimum} pour avoir de la régularité ; Si possible, \textbf{isolez-vous} pour ne pas être dérangé et ne pas fausser les résultats ; Pour chaque série, \textbf{notez le temps et le nombre de contraintes annotés} ; Si vous ne savez pas quoi annoter (trop ambigu, vocabulaire inconnu, ...), \textbf{passez au suivant sans annoter} (vous êtes sensés être des experts de la presse !)} ».
			\end{itemize}
			%
			Pour réaliser l'annotation, les opérateurs auront accès à l'application web développée au cours de ce doctorat.
			Des captures d'écran sont disponibles en \textsc{Figure~\ref{figure:4.3.1-ETUDE-COUTS-TEMPS-ANNOTATION-APPLICATION-ANNOTATION}} et \textsc{Figure~\ref{figure:4.3.1-ETUDE-COUTS-TEMPS-ANNOTATION-APPLICATION-LISTE-CONTRAINTES}}.
			Une description plus détaillée de l'application et de ses fonctionnalités est disponible en \textsc{Section~\ref{section:3.3-DESCRIPTION-IMPLEMENTATION}}\todo{description à faire}.
			
			% Détails de l'expérience.
			\todo[inline]{A REDIGER / A COMPLETER}
			
			% Référence scripts.
			\begin{leftBarInformation}
				Les scripts de l'expérience, réalisés avec des \textit{notebooks} Python (\cite{van-rossum-drake:2009:python-reference-manual}), sont disponibles dans un dossier dédié de~\cite{schild:2021:cognitivefactory-interactiveclusteringcomparativestudy}.
			\end{leftBarInformation}
			
		%%% Résultats
		\subsubsection{Résultats obtenus}
		
			% Description statistiques.
			La \textsc{Table~\ref{table:4.6.3-ETUDE-ROBUSTESSE-SCORE-INTER-ANNOTATEURS}} expose les scores inter-annotateurs sur les $3$ opérateurs et le réviseur de cette expérience, ainsi que l'accord avec la vérité terrain.
			Le score d'accord moyen avec la vérité terrain est de $0.86$ (écart-type: $0.01$) ; ce score est de $0.81$ (écart-type: $0.05$) pour les contraintes de types \texttt{MUST-LINK} et est de $0.92$ (écart-type: $0.03$) pour les contraintes de types \texttt{CANNOT-LINK}.
			Le score inter-annotateurs moyen (sans le réviseur) est de $0.84$ (écart-type: $0.03$).
			
			\begin{table}[!htb]
				\begin{center}
				\begin{tabular}{|c|r|r|r|r|}
				
					\hline
					% ENTETE DU TABLEAU
					
						& \multicolumn{1}{c|}{\shortstack[c]{
							1 (Relecteur)
						}}
						& \multicolumn{1}{c|}{\shortstack[c]{
							7 (Annotateur)
						}}
						& \multicolumn{1}{c|}{\shortstack[c]{
							9 (Annotateur)
						}}
						& \multicolumn{1}{c|}{\shortstack[c]{
							12 (Annotateur)
						}}
						\tabularnewline
						\hline

					% Vérité terrain
					\multicolumn{1}{|c|}{\shortstack[c]{
						Vérité terrain
					}}
						& $0.95$
						& $0.86$
						& $0.84$
						& $0.87$
						\tabularnewline
						\hline

					% Vérité terrain
					\multicolumn{1}{|c|}{\shortstack[c]{
						1 (Relecteur)
					}}
						&
						& $0.91$
						& $0.86$
						& $0.89$
						\tabularnewline
						\hline

					% Vérité terrain
					\multicolumn{1}{|c|}{\shortstack[c]{
						7 (Annotateur)
					}}
						&
						&
						& $0.86$
						& $0.85$
						\tabularnewline
						\hline

					% Vérité terrain
					\multicolumn{1}{|c|}{\shortstack[c]{
						9 (Annotateur)
					}}
						&
						&
						&
						& $0.80$
						\tabularnewline
						\hline
					
				\end{tabular}
				\end{center}
				\caption{
					Score d'accord inter-annotateurs obtenu avec $1$ réviseur et $3$ annotateurs sur un lot commun de $400$ contraintes ($200$ \texttt{MUST-LINK}, $200$ \texttt{CANNOT-LINK}).
				}
				\label{table:4.6.3-ETUDE-ROBUSTESSE-SCORE-INTER-ANNOTATEURS}
			\end{table}
			
			% Note.
			\begin{leftBarInformation}
				Dans une autre expérience, où $14$ opérateurs devaient annoter une base de $1~000$ contraintes aléatoires, nous obtenons un accord moyen avec la vérité terrain de $0.93$ (écart-type: $0.02$) et un score inter-annotateurs moyen de $0.91$ (écart-type: $0.03$).
				Toutefois, nous ne mettons pas en avant ces résultats car le lot de contraintes à annoté est déséquilibré à cause de l'utilisation de l'échantillonnage aléatoire ($932$ \texttt{CANNOT-LINK}, $68$ \texttt{MUST-LINK}).
			\end{leftBarInformation}
	
		%%% Discussion
		\subsubsection{Discussion}
		
			% Rappel de l'objectif : ...
			\todo[inline]{A REDIGER: rappel de l'objectif}
		
			% Avantage 1 : peu d'erreurs
			\todo[inline]{A REDIGER: Peu d'erreurs (environ 16\% d'erreurs sans concertations)}
			
			% Remarque 1 : MUST-LINK < CANNOT-LINK
			\todo[inline]{A REDIGER: CANNOT-LINK légèrement plus facile que les MUST-LINK}
			
			% Conclusions et suggestion.
			\todo[inline]{A REDIGER: ouverture sur l'impact des non-corrections}
			
	%%%
	%%% Subsection 4.6.3: Bilan concernant la robustesse du \textit{clustering} interactif
	%%%
	\subsection{Bilan concernant la robustesse du \textit{clustering} interactif}
	\label{section:4.6.3-ETUDE-ROBUSTESSE-MISE-EN-COMMUN}
	
		\todo[inline]{A REDIGER}
	
		% Conclusion.
		\begin{leftBarSummary}
			Au cours de cette étude de pertinence, nous avons pu voir que :
			\begin{itemize}
				\item[\itemok] ...
				\item[\itemok] ...
				\item[\itemok] ...
			\end{itemize}
		\end{leftBarSummary}

	
    %%%%%--------------------------------------------------------------------
    %%%%% Section 4.7:
    %%%%%--------------------------------------------------------------------
	\newpage
    \section{Autres études à réaliser}
	\label{section:4.7-ETUDES-DIVERSES}
	\todo[inline]{SECTION À RÉDIGER}

        \subsection{Choix du nombre de clusters ==> problème de recherche complexe}
            o	Piste de résolution : plusieurs clusterings + vote collaboratif ? algorithmes sans le nombre de clusters en hyper-paramètres

        \subsection{Impact d'un modèle de langage ==> nécessite de nombreuses données spécifiques au domaine}
            o	Piste de résolution : script d'étude comparative déjà prêt, mais il manque les données opensources… 

        \subsection{Paradigme d’annotation (intention vs dialogue) ==> problème d'UX + objectif métier}
            o	Etude Ergo, sort de mon domaine d'expertise

        \subsection{(et plein d'autres que j'ajouterai au fur et à mesure de ma rédaction)}
            o	
