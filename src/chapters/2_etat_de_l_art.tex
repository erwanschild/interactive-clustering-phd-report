\chapter{Revue de littérature sur la tâche d'annotation en intelligence artificielle}
\label{chapter:2-REVUE-DE-LITTERATURE}
	\todo[inline]{CHAPITRE: TITRE À TROUVER: "\textit{(Revue de littérature)}"}
	
	% RÉSUMÉ DES ÉPISODES PRÉCÉDENTS: 
	\todo[inline]{CHAPITRE: INTRODUCTION À RÉDIGER}
	
	% ANNONCE DU BUT DU CHAPITRE: 
	\todo[inline]{CHAPITRE: INTRODUCTION À RÉDIGER}
	
	% TABLE DES MATIÈRES DU CHAPITRE
	\minitoc
	
	%%%%%--------------------------------------------------------------------
	%%%%% Section 2.1: Présentation théorique de l'annotation
	%%%%%--------------------------------------------------------------------
	%\newpage
	\section{Présentation théorique de l'annotation}
\label{section:2.1-PRESENTATION-ANNOTATION}

	%%%
	%%% Introduction: donner une définition de ce qu'est l'annotation.
	%%%
	Tout d'abord, introduisons quelques définitions pour appréhender le concept d'\textguillemets{\texttt{annotation}} et donnons quelques exemples pour comprendre les enjeux qui y sont associés.
	
	
	%%%
	%%% Subsection 2.1.1: Définition et objectifs de l'annotation de données.
	%%%
	\subsection{Définition et objectifs de l'annotation de données}
	\label{section:2.1.1-PRESENTATION-ANNOTATION-DEFINITION}
	
		%%% 2.1.1.A. Qu'est que l'\textguillemets{\texttt{apprentissage automatique}} ?
		\subsubsection{Qu'est que l'\textguillemets{\texttt{apprentissage automatique}} ?}
		\label{section:2.1.1.A-PRESENTATION-ANNOTATION-DEFINITION-MACHINE-LEARNING}
			
			% Définition.
			Nous proposons la définition suivante inspirée de l'\texttt{ACM} (\textit{Association for Computing Machinery}) : l'\textguillemets{\texttt{apprentissage automatique}} (ou \textguillemets{\texttt{Machine Learning}}) est une branche de l'intelligence artificielle dédiée au développement de méthodes permettant à l'ordinateur de \textbf{reproduire une tâche par l'exemple} : il n'est donc pas explicitement programmé pour réaliser cette tâche, mais il l'\textguillemets{apprend} à l'aide d'un modèle mathématique.
			Cet apprentissage peut être \textit{supervisé} (l'interprétation des exemples est fournie par un humain), \textit{non-supervisé} (la machine déduit l'interprétation des données sans intervention humaine) ou \textit{semi-supervisé} (mélange des deux précédentes approches).
			
			% Applications.
			Le \textit{Machine Learning} permet ainsi d'automatiser l'analyse et la manipulation de certains phénomènes complexes tels que le langage, l'observation visuelle, la détection d'anomalies, le traitement acoustique, ...
			
			% References.
			\begin{leftBarInformation}
				Si vous voulez revoir les bases de l'apprentissage automatique, des livres comme \cite{zhou:2021:machine-learning} ou \cite{raschka-mirjalili:2019:python-machine-learning} traitent des notions principales et de leur mise en application.
			\end{leftBarInformation}
			
		%%% 2.1.1.B. Qu'est qu'un \textguillemets{\texttt{corpus d'entraînement}}?
		\subsubsection{Qu'est qu'un \textguillemets{\texttt{corpus d'entraînement}} ?}
		\label{section:2.1.1.B-PRESENTATION-ANNOTATION-DEFINITION-BASE-APPRENTISSAGE}

			% Définitions.
			Pour concevoir un modèle en apprentissage automatique, il nous faut un ensemble d'exemples (textes, images, sons, vidéos, ou tout autre relevé d'informations) permettant de capturer le phénomène à appréhender : cela nous aide à la fois à le décrire et à mieux le comprendre.
			Nous utilisons alors les termes \textguillemets{\texttt{corpus d'entraînement}}, \textguillemets{\texttt{jeu de d'entraînement}} ou \textguillemets{\texttt{base d'apprentissage}} pour désigner cet ensemble de données.
			
			% Carcatéristique importante : la représentativité !
			Il est important de noter qu'un corpus n'est qu'un échantillon de taille finie d'un phénomène pouvant être infini ou indénombrable.
			Il est donc d'usage de valoriser cet échantillon s'il est \textguillemets{\texttt{représentatif}} du phénomène qu'il décrit, c'est-à-dire s'il capture bien le large panel de variations que peuvent prendre les données (\cite{biber:1993:representativeness-corpus-design}).
			
			% References.
			\begin{leftBarInformation}
				Nous discuterons davantage de cette notion de représentativité dans la \textsc{Section~\ref{section:2.3.1.A-DEFIS-ANNOTATION-ASPECT-DONNEES-REPRESENTATIVITE}}.
				D'autre part, si vous voulez mieux comprendre cette notion de corpus, vous pouvez vous référer à \cite{sinclair:2004:corpus-text-basic} issu du livre \textit{Developing Linguistic Corpora} (\cite{wynne:2004:developing-linguistic-corpora}).
			\end{leftBarInformation}
		
		%%% 2.1.1.C. Qu'est que l'\textguillemets{\texttt{annotation}} ?
		\subsubsection{Qu'est que l'\textguillemets{\texttt{annotation}} ?}
		\label{section:2.1.1.C-PRESENTATION-ANNOTATION-DEFINITION-ANNOTATION}
			
			% Définition.
			Les données d'un corpus manquent parfois d'information pour bien cerner un phénomène, il est alors nécessaire de faire intervenir un humain pour introduire des connaissances supplémentaires qui ne sont pas explicitement présentes dans ces données.
			Nous appelons \textguillemets{\texttt{annotation}} (ou \textguillemets{\texttt{étiquetage}}, \textguillemets{\texttt{labellisation}}) cette tâche consistant à décrire les données d'un corpus, et nous distinguons ainsi les données dites \textguillemets{\texttt{brutes}} (utilisées par les approches non-supervisées) des données dites \textguillemets{\texttt{annotées}} (utilisées par les approches supervisées) en fonction de l'absence ou de la présence d'un complément d'informations.
			
			% Valeurs associées: valeur ajoutée, information d'interprétation.
			Les informations renseignées peuvent porter sur la donnée entière ou seulement sur une partie seulement, peuvent concerner des variables catégorielles (ensemble fini) ou numériques (ensemble infini), et peuvent aussi être cumulatives ou mutuellement exclusives.
			Dans la littérature, \cite{garside-etal:1997:corpus-annotation-linguistic} présente l'annotation comme la tâche permettant de donner une \textguillemets{\texttt{valeur ajoutée}} aux données ; de son côté, \cite{leech:2004:adding-linguistic-annotation} précise que l'annotation est une action d'\textguillemets{\texttt{interprétation}} qui aide à la compréhension et à la reproduction d'un phénomène, mais aussi au contrôle du comportement des modèles d'apprentissage automatique.
	
	
	%%%
	%%% Subsection 2.1.2: Exemples de tâches d'annotations.
	%%%
	\subsection{Exemples de tâches d'annotations}
	\label{section:2.1.2-PRESENTATION-ANNOTATION-EXEMPLES}
		
		% Transition: définitions assez généralistes car champ d'application très vaste.
		Les définitions données dans la section précédente peuvent paraître abstraites car il est difficile de dépeindre la vaste diversité d'applications nécessitant des données labellisées.
		En effet, une tâche d'annotation répond toujours à un besoin précis, mais il y a une telle multiplicité de types de données (\textit{données tabulaires, textuelles, visuelles, auditives, ...}) et de cas d'usages (\textit{prédiction d'une valeur numérique (tâche de régression), prédiction d'une catégorie (tâche de classification), détection d'objets (tâche d'extraction), création de nouvelles données (tâche de génération), ... }) qu'une unique définition ne peut être que purement théorique.
		
		% Annonce: prendre des exemples sur l'univers de la bande dessinée.
		Ainsi, nous estimons qu'il est préférable de compléter ces définitions par quelques exemples concrets.
		Nous pourrons ainsi mieux dresser le portait d'une tâche d'annotation, avec ses intérêts et ses complications.
		Pour cela, nous allons prendre le thème de la bande dessinée et de ses dérivés, et explorer ensemble différents cas d'usage qui pourraient intéresser un auteur, un libraire ou un lecteur.
		
		
		%%% 2.1.2.A. Estimation du prix d'une bande dessinée.
		\subsubsection{Estimation du prix d'une bande dessinée.}
		\label{section:2.1.2.A-PRESENTATION-ANNOTATION-EXEMPLES-REGRESSION}
		
			% Cas d'usage: évaluer le juste prix.
			Les acheteurs et les vendeurs de bandes dessinées s'interrogent forcément sur le juste prix de l'oeuvre qu'ils veulent acquérir ou céder.
			Répondre à cette question avec précision nécessite diverses informations, à la fois sur l'oeuvre (comme son identification ou l'avis de ses lecteurs), sur le document en tant que tel (comme son état de conservation), mais aussi sur le prestige de son édition (éditions originales ou de collection).
			Sans un regard d'expert, il est possible de trouver certaines oeuvres rares vendues pour presque rien sur le marché d'occasion, ou à l'inverse voir certaines \texttt{BD} être achetées à prix d'or alors que le document est en piteux état.
			
			% Entrainer un modèle: régression.
			Afin d'aiguiller les acquéreurs, il est possible d'utiliser un modèle de \texttt{régression}\footnote{
				Pour plus de détails sur la régression: voir la revue de \cite{maalouf:2011:logistic-regression-data} ; voir un exemple basé sur la méthode des moindres carrés dans \cite{zdaniuk:2014:ordinary-leastsquares-ols}.
			} permettant de prédire le prix d'une \texttt{BD} à partir des différentes métadonnées à disposition.
			Mais pour entraîner un tel modèle, il est nécessaire d'avoir une base d'apprentissage contenant des exemples de transactions avec leur prix de vente.
			Nous pouvons structurer l'ensemble des informations nécessaire dans un tableau, et la tâche d'annotation consiste alors à renseigner pour chaque transaction :
			\begin{itemize}
				\item l'identification complète de la \texttt{BD} (titre, auteur, édition, ... ),
				\item l'état du document grâce à un regard d'expert (l'état peut par exemple être défini par une variable catégorielle dont les valeurs seraient "\texttt{Mauvais état}", "\texttt{Bon état}", "\texttt{Très bon état}", "\texttt{Neuf}") ;
				\item le prix de la \texttt{BD}, estimé ou réel (défini par une variable numérique).
			\end{itemize}
			
			% Citation de l'exemple.
			Un exemple de résultat d'annotation de ces données est disponible dans la \textsc{Table~\ref{table:2.1.2.A-PRESENTATION-ANNOTATION-EXEMPLES-REGRESSION}}.
			%
			\begin{leftBarExamples}
				\begin{table}[H]  % keep [H] to be in the tcolorbox.
					\begin{center}
					\def\arraystretch{0.8}  % interligne
					\begin{tabular}{|c|l|c|c|c|r|}
					
					\hline
					% ENTETE DU TABLEAU
					\rowcolor{colorLeftBarExamples!25}
					Collection
						& N°: Titre
						& Édition
						& Note
						& État
						& Prix (€)
						\tabularnewline
						\hline \hline
					% BD1.
					Lucky Luke
						& $01$: La mine d'or de Dick Digger
						& $1949$
						& $3.2/5$
						& Très bon
						& $5~000,00$
						\tabularnewline
						\hline
					% BD2.
					Lucky Luke
						& $12$: Les cousins Dalton
						& $1958$
						& $4.3/5$
						& Bon
						& $40,00$
						\tabularnewline
						\hline
					% BD3.
					Lucky Luke
						& $12$: Les cousins Dalton
						& $1962$
						& $4.3/5$
						& Très bon
						& $65,00$
						\tabularnewline
						\hline
					% BD4.
					Lucky Luke
						& $12$: Les cousins Dalton
						& $1985$
						& $4.3/5$
						& Très bon
						& $6,00$
						\tabularnewline
						\hline
					% BD5.
					Lucky Luke
						& $15$: L'évasion des Dalton
						& $1960$
						& $4.1/5$
						& Mauvais
						& $3,00$
						\tabularnewline
						\hline
					% ...
					\multicolumn{6}{|c|}{ \shortstack{ ... } }
						\tabularnewline
						\hline
				
					\end{tabular}
					\end{center}
					\caption{
						Exemple d'annotation du prix de vente de bandes dessinées en fonction de leur édition, de la note de leur lecteurs et de leur état (source: \url{https://www.bedetheque.com/serie-213-BD-Lucky-Luke.html}).
					}
					\label{table:2.1.2.A-PRESENTATION-ANNOTATION-EXEMPLES-REGRESSION}
				\end{table}
			\end{leftBarExamples}
			
			% Conclusion.
			Ainsi, si quelqu'un s'intéresse au prix d'une nouvelle bande dessinée pour lequel il n'y a pas de référence tarifaire, il peut interroger le modèle de régression qui proposera un prix en accord avec les exemples dont il dispose dans sa base d'apprentissage.
			
			% Autre cas d'usage similaire: Régression dans d'autres domaines.
			\begin{leftBarInformation}
				De manière équivalente, il est possible de faire de la régression dans d'autres domaines, notamment pour prédire un volume, une surface, une quantité, ...
				La tâche d'annotation consistera à chaque fois à renseigner la valeur numérique à prédire en fonction des différentes données à disposition.
			\end{leftBarInformation}
		
		
		%%% 2.1.2.B. Classification de l'état d'une bande dessinée à partir d'une photo.
		\subsubsection{Classification de l'état d'une bande dessinée à partir d'une photo.}
		\label{section:2.1.2.B-PRESENTATION-ANNOTATION-EXEMPLES-CLASSIFICATION}
			
			% Cas d'usage: classifier l'état.
			Il est d'usage d'adapter le prix de vente d'un produit en fonction de son état, et nous avons intégré ce facteur dans l'estimation du prix d'une bande dessinée (voir exemple précédent).
			Cependant, l'état de conservation n'est pas une notion objective et chacun peut avoir des références différentes.
			Au final, c'est souvent un libraire qui détermine si l'oeuvre est en bon ou en mauvais état, et, sans un regard d'expert, nous pouvons omettre un détail ou nous tromper lors de notre appréciation.
			
			% Entrainer un modèle: classification d'image.
			Afin de nous aider à estimer l'état d'une bande dessinée, il est possible d'utiliser un modèle de \texttt{classification}\footnote{
				Pour plus de détails sur la classification: voir les revues de \cite{aized-amin-soofi-arshad-awan:2017:classification-techniques-machine} ou de \cite{kotsiantis-etal:2006:machine-learning-review} ; voir un exemple basé sur les machine à vecteurs de support (\texttt{SVM}) dans \cite{cortes-vapnik:1995:supportvector-networks}.
			} permettant, à partir d'une image, d'affecter à chaque \texttt{BD} une catégorie prédéfinie (par exemple: "\texttt{Mauvais état}", "\texttt{Bon état}", "\texttt{Très bon état}", "\texttt{Neuf}").
			Pour entraîner un tel modèle, il est nécessaire d'avoir une base d'apprentissage contenant des exemples d'images de \texttt{BD} associées avec leur catégorie d'état.
			La tâche d'annotation peut alors consister à renseigner pour chaque couverture de bande dessinée la catégorie d'état qui lui correspond le plus.
			
			% Citation de l'exemple.
			Un exemple d'annotation de la classification de l'image est disponible dans la \textsc{Figure~\ref{figure:2.1.2.B-PRESENTATION-ANNOTATION-EXEMPLES-CLASSIFICATION}}.
			%
			\begin{leftBarExamples}
				\begin{figure}[H]
					\centering
					\includegraphics[width=0.80\textwidth]{figures/etatdelart-morris-1950-lucky-luke-2-1952-lucky-luke-4}
					\caption{
						Exemple d'annotation de l'état d'une \texttt{BD} (ici: \cite{morris-goscinny:1950:rodeo} et \cite{morris-goscinny:1952:sous-ciel-ouest}).
						La première est en très bon état (couverture comme neuve, tranches légèrement usées, pages intactes) tandis que la seconde est en mauvais état (couverture usée, dos abimée, traces sur les pages, ...).
					}
					\label{figure:2.1.2.B-PRESENTATION-ANNOTATION-EXEMPLES-CLASSIFICATION}
				\end{figure}
			\end{leftBarExamples}
			
			% Conclusion.
			Ainsi, si quelqu'un s'interroge sur l'état d'une bande dessinée en sa possession, ce modèle peut identifier la catégorie d'état la plus probable d'après les exemples disponibles dans sa base d'apprentissage.
			
			% Autre cas d'usage similaire: Identifier la langue de la BD.
			\begin{leftBarInformation}
				De manière équivalente, il est possible de faire de la classification sur d'autres données, comme par exemple la classification de textes pour identifier la langue de l'ouvrage.
				Dans l'exemple ci-dessous, les catégories proposées sont "\texttt{Français}", "\texttt{Anglais}" et "\texttt{Allemand}", et la tâche d'annotation consiste ici à associer à chaque texte une catégorie de langue.
				\begin{center}
				\begin{tabular}{ c l }
					\textguillemets{\textit{
						Les cousins Dalton ont dévalisé la diligence.
					}} & $\implies$ \textcolor{colorSilverLakeBlue}{\texttt{Français}} \\
					\textguillemets{\textit{
						The Dalton cousins robbed the stagecoach.
					}} & $\implies$ \textcolor{colorDarkPastelGreen}{\texttt{Anglais}} \\
					\textguillemets{\textit{
						Die Dalton-Cousins haben die Postkutsche ausgeraubt.
					}} & $\implies$ \textcolor{colorDarkPastelRed}{\texttt{Allemand}}
				\end{tabular}
				\end{center}
			\end{leftBarInformation}
		
		
		%%% 2.1.2.C. Identification d'une bande dessinée à partir de sa couverture.
		\subsubsection{Identification d'une bande dessinée à partir de sa couverture.}
		\label{section:2.1.2.C-PRESENTATION-ANNOTATION-EXEMPLES-EXTRACTION}
			
			% Cas d'usage: identifier une bande dessinée
			Identifier une bande dessinée n'est pas toujours facile, et recopier l'ensemble des informations l'identifiant peut prendre du temps.
			Les libraires ou les collectionneurs désirant faire l'inventaire des ouvrages en leur possession peuvent ainsi y passer de nombreuses heures, avec le risque de faire des erreurs lors de l'inscription des bande dessinée dans leur registre.
			
			% Entrainer un modèle: extraction de caractères.
			Afin d'aider les collectionneurs, il est possible d'utiliser un modèle de \texttt{reconnaissance optique des caractères} (\texttt{OCR})\footnote{
				Pour plus de détails sur l'\texttt{OCR}: voir la revue de \cite{berchmans-kumar:2014:optical-character-recognition} ou de \cite{awel-abidi:2019:review-optical-character}.
			} pour extraire automatiquement les informations importantes présentes sur les couvertures d'une \texttt{BD} à identifier.
			Pour entraîner un tel modèle, il est nécessaire d'avoir une base d'apprentissage contenant des exemples de pages de couverture avec la position et la valeur des informations pertinentes à extraire.
			La tâche d'annotation peut alors consister à renseigner pour chaque couverture de bande dessinée :
			\begin{itemize}
				\item la position des informations en l'encadrant sur l'image (avec un rectangle par exemple) ;
				\item la valeur écrite dans l'encadré sur l'image.
			\end{itemize}
			
			% Citation de l'exemple.
			Un exemple d'annotation de textes dans une image est disponible dans la \textsc{Figure~\ref{figure:2.1.2.C-PRESENTATION-ANNOTATION-EXEMPLES-EXTRACTION}}.
			%
			\begin{leftBarExamples}
				\begin{figure}[H]
					\centering
					\includegraphics[width=0.80\textwidth]{figures/etatdelart-morris-1958-lucky-luke-12}
					\caption{
						Exemple d'annotation de textes présents sur la couverture d'une bande dessinée (ici: \cite{morris-goscinny:1958:cousins-dalton}).
						Les informations essentielles telles que la collection, le numéro, le titre, l'auteur et l'éditeur y sont présentes. 
					}
					\label{figure:2.1.2.C-PRESENTATION-ANNOTATION-EXEMPLES-EXTRACTION}
				\end{figure}
			\end{leftBarExamples}
			
			% Conclusion.
			Ainsi, si quelqu'un veut identifier une nouvelle bande dessinée, il peut interroger le modèle d'extraction de caractères pour récupérer les informations textuelles présentes dans la couverture, à l'image des exemples disponibles dans sa base d'apprentissage.
			
			% Autres cas d'usage: Extraire les information d'un texte.
			\begin{leftBarInformation}
				De manière équivalente, il est possible de réaliser une \texttt{reconnaissance d'entités nommées (\texttt{NER})}\footnote{
					Pour plus de détails sur la reconnaissance d'entités nommées (\texttt{NER}): voir les revues de \cite{goyal-etal:2018:recent-named-entity} ou de \cite{li-etal:2022:survey-deep-learning}.
				} pour extraire les informations citées dans un texte.
				Dans l'exemple ci-dessous, les types d'entités présentes sont "\texttt{personnage}", "\texttt{métier}", "\texttt{argent}", "\texttt{lieu}" et "\texttt{date}".
				La tâche d'annotation consiste ici à identifier la position et le type de chaque entité présente.
				
				\begin{quote}
					\textguillemets{\textit{
						$\textbf{\text{Lucky Luke}}_{\textcolor{colorDarkPastelRed}{\texttt{(personnage)}}}$, le $\textbf{\text{cow-boy}}_{\textcolor{colorDarkPastelPurple}{\texttt{(métier)}}}$ solitaire, a attrapé les $\textbf{\text{Dalton}}_{\textcolor{colorDarkPastelRed}{\texttt{(personnage)}}}$ à $\textbf{\text{Coyote Gulch}}_{\textcolor{colorCarrotOrange}{\texttt{(lieu)}}}$ et a touché $\textbf{\text{50.000\$}}_{\textcolor{colorSilverLakeBlue}{\texttt{(argent)}}}$ en les livrant au $\textbf{\text{pénitencier}}_{\textcolor{colorCarrotOrange}{\texttt{(lieu)}}}$. Ils se sont évadés le $\textbf{\text{jeudi suivant}}_{\textcolor{colorDarkPastelGreen}{\texttt{(date)}}}$.
					}}
				\end{quote}
			\end{leftBarInformation}
		
		
		%%% 2.1.2.D. Interprétation audio d'une bande dessinée.
		\subsubsection{Interprétation audio d'une bande dessinée.}
		\label{section:2.1.2.D-PRESENTATION-ANNOTATION-EXEMPLES-TRANSCRIPTION}
		
			% Cas d'usage: générer une lecture audio.
			Il est de plus en plus commun de trouver des livres disponibles avec une lecture audio.
			Ces audio-livres, réalisés par une personne ou synthétisés par l'ordinateur, peuvent être à visée éducative ou simplement disponibles pour le loisir.
			Dans le cadre de notre exemple sur le thème des bandes dessinées, peu d'entre elles disposent d'une lecture audio.
			Une idée serait donc d'interpréter une lecture audio de ces bandes dessinées en synthétisant la voix des doubleurs de leurs adaptations télévisées (ou simplement d'un narrateur si l'oeuvre n'a pas été portée à l'écran).
			
			% Entrainer un modèle: synthétiseur vocal.
			Dans le but de créer ces audio-\texttt{BD}, nous pourrions envisager d'utiliser des modèles de \texttt{synthèse vocale} (\texttt{TTS})\footnote{
				Pour plus de détails sur la synthèse vocale: voir la revue de \cite{kothadiya-etal:2020:different-methods-review} ; voir un exemple d'architecture neuronale \texttt{end-to-end} dans \cite{mu-etal:2021:review-endtoend-speech}.
			} pour générer automatiquement la lecture des bulles d'une bande dessinée.
			Pour entraîner de tels modèles, il est nécessaire d'avoir une base d'apprentissage contenant des exemples d'audios prononcés par chacun des personnages (ici: les doubleurs de l'adaptation télévisée) avec la transcription de leur paroles pour chaque audio.
			La tâche d'annotation peut alors consister à renseigner le personnage et les paroles qu'il a prononcées.
			
			% Citation de l'exemple.
			Un exemple d'annotation phonétique est illustré dans la \textsc{Figure~\ref{figure:2.1.2.D-PRESENTATION-ANNOTATION-EXEMPLES-TRANSCRIPTION}}.
			%
			\begin{leftBarExamples}
				\begin{figure}[H]
					\centering
					\includegraphics[width=0.95\textwidth]{figures/etatdelart-thebault-transcription}
					\caption{
						Exemple de paroles prononcées dans un audio.
						Ici, la voix de \textit{Lucky Luke} est interprétée par Jacques THEBAULT.
						Le texte annoté, c'est-à-dire celui prononcé dans l'audio, est  \textguillemets{\texttt{Ils sont à vous chef, et j'vous s'rai reconnaissant de bien les garder cette fois.}}.
						Les phonèmes en alphabet phonétique international associés à chaque séquence de l'audio sont disponibles si besoin.
						%[il sɔ̃ a vu ʃɛf, e ʒvu sʁe ʁə.kɔ.nɛ.sɑ̃ də bjɛ̃ vu.lwaʁ le ɡaʁ.de sɛt fwa].
					}
					\label{figure:2.1.2.D-PRESENTATION-ANNOTATION-EXEMPLES-TRANSCRIPTION}
				\end{figure}
			\end{leftBarExamples}
			
			% Conclusion.
			Ainsi, nous pourrions consulter le modèle de synthèse vocale d'un personnage (ici: celui de \texttt{Lucky Luke}) avec un nouveau texte à prononcer pour en obtenir une lecture audio dont la voix se rapproche des enregistrements de la base d'apprentissage (ici: celle de Jacques THEBAULT).
			
			% Autres cas d'usage: comuler OCR + Classification + TTS
			\begin{leftBarInformation}
				Nous pourrions compléter le cas d'usage
				(1) en extrayant automatiquement le texte d'une planche de \texttt{BD} par \texttt{OCR},
				(2) en détectant automatiquement le personnage prononçant la bulle de \texttt{BD} par classification,
				puis (3) en générant du texte à prononcer par le personnage par synthèse vocale.
				Bien entendu, la conception et l'enchaînement de ces différents modèles sont plutôt complexes, et chaque tâche de \textit{Machine Learning} demande ses propres données annotées pour construire une base d'apprentissage.
			\end{leftBarInformation}
			
			
		%%% Exemple: Annotation génération musicale
		% Generation de musique: \cite{hernandez-olivan-beltran:2023:music-composition-deep}
		%\begin{leftBarExamples}
		%	Annotation des paroles d'une chanson (\cite{woods:1971:poor-lonesome-cowboy}). \\
		%
		%	\begin{guitar}
		%		\textbf{Im A Poor Lonesome Cowboy}
		%		\textit{
		%			~~~~from \texttt{Lucky Luke - Daisy Town OST} ($1971$)
		%			~~~~composed by \texttt{Claude Bolling}
		%			~~~~performed by \texttt{Pat Woods}
		%		}
		%		\texttt{Intro}
		%		\textit{
		%			[D]Lonesome [D7]cowboy, [G]Lonesome [G7]cowboy, [D]You're a [Bm]long [Bm7]long [E]way from [A]home.
		%			[D]Lonesome cowboy, [G]Lonesome cowboy, [D]You've a [Bm]long [Bm7]long [E]way [A7]to [D]roam.
		%		}
		%		\texttt{Couplet}
		%		\textit{
		%			I'm a [D]poor lonesome cowboy, I'm a long long way from home,
		%			And this poor lonesome [F\#m]cowboy, Has got a [Em]long long way [A]to roam.
		%			Over [D]mountains and over [D7]prairies, From [G]dawn 'til day is [Em]done,
		%			My [Bm]horse and me keep [F\#]ridin', [G]into the [A]settin' [D]sun.
		%		}
		%		{ \center \textbf{...} }
		%	\end{guitar}
		%\end{leftBarExamples}
		
		
		%%% Exemple: Annotation étiquette gramaticale
		%\begin{leftBarExamples}
		%	Annotation des étiquettes grammaticales dans un texte.
		%	\begin{quote}
		%		\textguillemets{\textit{
		%			$\text{Les}_{\textcolor{colorDarkPastelPurple}{\texttt{(DET)}}}$
		%			$\text{dangereux}_{\textcolor{colorMinionYellow}{\texttt{(ADJ)}}}$
		%			$\text{Dalton}_{\textcolor{colorDarkPastelRed}{\texttt{(PROPN)}}}$
		%			$\text{se}_{\textcolor{colorCarrotOrange}{\texttt{(PRON)}}}$
		%			$\text{sont}_{\textcolor{colorDarkPastelGreen}{\texttt{(AUX)}}}$
		%			$\text{encore}_{\textcolor{colorSilverLakeBlue}{\texttt{(ADV)}}}$
		%			$\text{évadés}_{\textcolor{colorDarkPastelGreen}{\texttt{(VERB)}}}$
		%			$\text{de}_{\textcolor{colorDimGray}{\texttt{(ADP)}}}$
		%			$\text{prison}_{\textcolor{colorDarkPastelRed}{\texttt{(NOUN)}}}$
		%			$\text{et}_{\textcolor{colorBlack}{\texttt{(CCONJ)}}}$
		%			$\text{ils}_{\textcolor{colorCarrotOrange}{\texttt{(PRON)}}}$
		%			$\text{ont}_{\textcolor{colorDarkPastelGreen}{\texttt{(AUX)}}}$
		%			$\text{déjà}_{\textcolor{colorSilverLakeBlue}{\texttt{(ADV)}}}$
		%			$\text{dévalisé}_{\textcolor{colorDarkPastelGreen}{\texttt{(VERB)}}}$
		%			$\text{une}_{\textcolor{colorDarkPastelPurple}{\texttt{(DET)}}}$
		%			$\text{banque}_{\textcolor{colorDarkPastelRed}{\texttt{(NOUN)}}}$
		%			$\text{à}_{\textcolor{colorDimGray}{\texttt{(ADP)}}}$
		%			$\text{Daisy Town}_{\textcolor{colorDarkPastelRed}{\texttt{(PROPN)}}}$.
		%		}}\\
		%		%{ \center \scriptsize (
		%		%	Adjectif: {\textcolor{colorMinionYellow}{\texttt{(ADJ)}}} ;
		%		%	Adverbe: {\textcolor{colorSilverLakeBlue}{\texttt{(ADV)}}} ;
		%		%	Conjonction: {\textcolor{colorBlack}{\texttt{(CCONJ)}}} ;
		%		%	Déterminant: {\textcolor{colorDarkPastelPurple}{\texttt{(DET)}}} ;
		%		%	Nom: {\textcolor{colorDarkPastelRed}{\texttt{(NOUN)}}}, {\textcolor{colorDarkPastelRed}{\texttt{(PROPN)}}} ;
		%		%	Préposition: {\textcolor{colorDimGray}{\texttt{(ADP)}}} ;
		%		%	Pronom: {\textcolor{colorCarrotOrange}{\texttt{(PRON)}}}, ;
		%		%	Verbe: {\textcolor{colorDarkPastelGreen}{\texttt{(AUX)}}}, {\textcolor{colorDarkPastelGreen}{\texttt{(VERB)}}}.
		%		%)}
		%	\end{quote}
		%\end{leftBarExamples}
	
	
	%%%
	%%% Subsection 2.1.3: Bilan concernant la présentation de l'annotation.
	%%%
	\subsection{Bilan concernant la présentation de l'annotation}
	\label{section:2.1.3-PRESENTATION-ANNOTATION-BILAN}
	
	%%%
	%%% Conclusion.
	%%%
	\begin{leftBarSummary}
		\begin{todolist}
			% Définition de l'annotation.
			\item[\itemok] \textguillemets{\texttt{Annoter}} une donnée consiste à \textbf{ajouter un complément d'information} pour pouvoir mieux interpréter puis reproduire un phénomène.
			% Type d'annotation.
			\item[\itemok] Le type d'annotation à réaliser \textbf{dépend du problème à traiter} : régression, classification, extraction d'information, génération ou synthèse de données, ...
			% Corpus d'entraînement.
			\item[\itemok] L'ensemble des données annotées peut être utilisé pour concevoir un modèle d'\textguillemets{\texttt{apprentissage automatique}}: il est alors appelé \textguillemets{\texttt{corpus d'entraînement}}.
		\end{todolist}
	\end{leftBarSummary}
	
	
	%%%%%--------------------------------------------------------------------
	%%%%% Section 2.2: Organisation usuelle d'un projet d'annotation
	%%%%%--------------------------------------------------------------------
	%\newpage
	\section{Organisation usuelle d'un projet d'annotation}
\label{section:2.2-ORGANISATION-ANNOTATION}
	
	%%%
	%%% Introduction: Présenter l'organisation usuelle, les acteurs et le besoin d'outils.
	%%%
	Dans la section précédente, nous avons présenté l'importance d'avoir des données annotées pour entraîner d'un modèle de \textit{Machine Learning}.
	Maintenant, nous allons détailler l'organisation de cette tâche d'annotation, identifier les compétences nécessaires aux intervenants du projet ainsi que les fonctionnalités essentielles des outils de labellisation.
	
	
	%%%
	%%% Subsection 2.2.1: Étapes clés du cycle d'annotation.
	%%%
	\subsection{Étapes clés du cycle d'annotation}
	\label{section:2.2.1-ORGANISATION-ANNOTATION-ETAPES-CLES}
		% \cite{pustejovsky-stubbs:2012:natural-language-annotation} et \cite{stubbs:2013:methodology-using-professional} formalisation MATTER
		
		%%% Introduction au cycle MATTER.
		Une référence en matière d'organisation de projet d'annotation est proposée par \cite{pustejovsky-stubbs:2012:natural-language-annotation} et est complétée dans \cite{stubbs:2013:methodology-using-professional}.
		Les auteurs y formalisent la conception et l'amélioration \textbf{cyclique} d'un modèle de \textit{Machine Learning}.
		Ce cycle est appelé cycle \texttt{MATTER} en référence aux six étapes de conception qui le composent : \textit{\textbf{M}odelize}, \textit{\textbf{A}nnotate}, \textit{\textbf{T}rain}, \textit{\textbf{T}est}, \textit{\textbf{E}valuate} et \textit{\textbf{R}evise}.
		Ces étapes sont schématisées en \textsc{Figure~\ref{figure:2.2.1-ORGANISATION-ANNOTATION-ETAPES-CLES-MATTER}} et nous détaillons chacune d'entre elles ci-dessous.
		%
		\begin{figure}[!htb]
			\centering
			\includegraphics[width=0.95\textwidth]{figures/etatdelart-pustejovsky-2012-cycle-matter-mama-tt}
			\caption{
				Cycle \texttt{MATTER} structurant un projet d'annotation en six étapes principales: \textit{\textbf{M}odelize}, \textit{\textbf{A}nnotate}, \textit{\textbf{T}rain}, \textit{\textbf{T}est}, \textit{\textbf{E}valuate} et \textit{\textbf{R}evise}.
			}
			\label{figure:2.2.1-ORGANISATION-ANNOTATION-ETAPES-CLES-MATTER}
		\end{figure}
		%
		\begin{leftBarAuthorOpinion}
			Nous conseillons \cite{finlayson-erjavec:2016:overview-annotation-creation} pour son excellente revue de littérature qui détaille pas à pas le cycle \texttt{MATTER} tout en dressant la liste des points importants importants de chacune des étapes.
		\end{leftBarAuthorOpinion}
		
		%%% 2.2.1.A. Concevoir la base d'apprentissage (\textbf{M}odelize}, \textit{\textbf{A}nnotate}).
		\subsubsection{Concevoir la base d'apprentissage (\textit{\textbf{M}odelize}, \textit{\textbf{A}nnotate}).}
		\label{section:2.2.1.A-ORGANISATION-ANNOTATION-ETAPES-CLES-MODELIZE-ANNOTATE}
		
			%%% a. Collecte de données.
			Pour obtenir un bon modèle de \textit{Machine Learning}, il faut avoir une base d’apprentissage de qualité.
			Comme nous l'avons dit précédemment, cela commence par disposer d'un ensemble de données d'exemples qui représente fidèlement les différentes facettes du problème à modéliser (voir \textsc{Section~\ref{section:2.3.1.A-DEFIS-ANNOTATION-ASPECT-DONNEES-REPRESENTATIVITE}}).
			Une phase de \texttt{collecte} de données est alors organisée : cette collecte peut se baser sur des extractions de bases de données ou de sites internet à disposition, sur des enquêtes réalisées après d'utilisateurs finaux, ou encore sur les avis éclairés d'experts du problème.
			Certaines données peuvent aussi être artificiellement créées afin de compléter la collecte pour les aspects du problème difficile à observer.
			Une fois la collecte terminée, ces données brutes ont besoin d'être annotées pour pouvoir être exploitées. \\
			
			%%% b. Modelisation des données.
			
			% Importance de la modélisation.
			Afin de garantir la qualité de cette labellisation, \textbf{il est important de ne pas précipiter la tâche d'annotation}.
			En effet, l'objectif de cette tâche ainsi que les informations à annoter peuvent considérablement changer en fonction du phénomène à décrire, des données à disposition et de la finalité du modèle de \textit{Machine Learning} à entraîner.
			Il est donc fortement conseillé de bien \textbf{modéliser le problème} pour clarifier les attendus et les modalités de cette annotation (voir \textsc{Figure~\ref{figure:2.2.1-ORGANISATION-ANNOTATION-ETAPES-CLES-MATTER}}, étape 1. \textit{\textbf{M}odelize}).
			
			% Modélisation vs Spécification, Guide d'annotation et exemple.
			\cite{pustejovsky-stubbs:2012:natural-language-annotation} précisent notamment deux concepts importants de cette phase :
			\begin{itemize}
				\item la \textbf{modélisation} du problème, représentation de manière abstraite l'objectif à atteindre et décrivant ainsi la logique générale de l'annotation dans un \textit{schéma d'annotation} ;
				\item les \textbf{spécifications}, compilant dans un \textit{guide d'annotation} l'ensemble des règles concrètes à respecter pour mettre en application la modélisation.
			\end{itemize}
			Pour résumer cette distinction, la modélisation représente \textit{quoi} annoter (\textit{objectif, définition, valeurs possibles, ...}) alors que les spécifications décrivent \textit{comment} annoter (\textit{règles d'attribution, exemples et contre-exemple, règles de format, ...}).
			\begin{leftBarExamples}
				% Exemple littérature.
				\cite{perrotin-etal:2018:annotation-actes-dialogue}, s'intéressant à la classification des conversations d'assistance en ligne en actes de dialogues, décrit son guide d'annotation dans \cite{asher-etal:2017:manuel-annotation-actes}.
				On y retrouve (1) la modélisation avec la présentation des étiquettes possibles à annoter, et (2) les spécifications avec les définitions concrètes, des exemples, des restrictions d'attribution, et la gestion des données non pertinentes. \\
				% Exemple BD.
				Dans nos exemples précédents (cf. \textsc{Section~\ref{section:2.1.2.B-PRESENTATION-ANNOTATION-EXEMPLES-CLASSIFICATION}}), nous avions modélisé le problème de classification de l'état d'une bande dessinée en quatre classes : "\texttt{Mauvais état}", "\texttt{Bon état}", "\texttt{Très bon état}", "\texttt{Neuf}".
				Il faudrait désormais rédiger les spécifications avec des définitions concrètes et quelques exemples  pour guider un annotateur, notamment pour l'aider à distinguer "\texttt{Bon état}" de "\texttt{Très bon état}".
			\end{leftBarExamples}
			
			% Quelques points importants sur la modélisation et exemples.
			Bien entendu, il n'est pas toujours facile de modéliser un problème ni de rédiger un guide d'annotation adéquat.
			Nous reviendrons plus tard sur les caractéristiques de cette tâche pouvant introduire de la complexité (voir \textsc{Section~\ref{section:2.3-DEFIS-ANNOTATION}}), mais il est important de souligner d'emblée les points élémentaires suivants :
			\begin{itemize}
				\item le besoin d'\textit{inter-opérabilité} et de \textit{ré-utilisabilité} : un projet d'annotation est toujours un investissement coûteux, il serait donc regrettable de perdre ou de ne pas pourvoir ré-utiliser ces données après ce projet.
				Par conséquent, il faut réfléchir au format des données ainsi qu'aux types de détails à fournir pour être sûr de pouvoir toujours exploiter les données si la modélisation évolue légèrement ou si un futur projet désire en bénéficier ;
				\item la balance entre \textit{généralité} et \textit{spécificité} : le niveau de détail requis dépend sans conteste du problème à modéliser : annoter trop peu de détail ne permet pas d'exploiter les données, mais en annoter trop peut rapidement complexifier la tâche et introduire des erreurs.
				Il faut donc trouver le juste milieu pour réaliser un travail de qualité qui ne soit pas trop pénible.
			\end{itemize}
			\begin{leftBarExamples}
				Dans la classification de langue exposée en \textsc{Section~\ref{section:2.1.2.B-PRESENTATION-ANNOTATION-EXEMPLES-CLASSIFICATION}}, nous y avons annoté chaque texte grâce à trois classes : "\texttt{Français}", "\texttt{Anglais}" et "\texttt{Allemand}".
				\begin{itemize}
					% Exemple inter-opérabilité et de ré-utilisabilité.
					\item par soucis d'\textit{inter-opérabilité}, nous pourrions plutôt utiliser la norme ISO 639-3 (\cite{international-organization-for-standardization:2007:codes-representation-names}), soit les code "\texttt{fra}", "\texttt{eng}" et "\texttt{deu}", afin de standardiser l'annotation et ainsi pouvoir partager plus facilement les données labellisées avec d'autres projets ;
					% Exemple généralité et spécificité.
					\item afin de présenter un cas simple, nous avions proposé un modèle avec trois langues communes pour une bande dessinée d'origine belge.
					Toutefois, nous aurions pu \textit{spécialiser} davantage notre modèle en fonction des variations régionales en prenant en compte le Corse ("\texttt{cos}") ou le Wallon ("\texttt{wln}").
					Cette distinction peut être essentielle pour certaines saga publiées dans ces langues (comme \texttt{Astérix \& Obélix}), mais peut simplement être une source de confusion pour les autres (comme \texttt{Lucky Luke}).
				\end{itemize}
			\end{leftBarExamples}
			
			% Aide à la formalisation.
			\begin{leftBarInformation}
				Pour aider à concevoir le guide d'annotation et afin de se poser les bonnes questions, \cite{dipper-etal:2004:useradaptive-annotation-guidelines} dresse une liste de définitions et de recommandations à prendre en considération.
				Bien que ces conseils soient issus du traitement de données linguistiques, ils permettent d'identifier les sections importantes d'un guide d'annotation en fonction des attentes des différents acteurs de l'annotation (\textit{l'auteur, l'annotateur, l'explorateur de données, ...}) et de les rédiger en suivants certaines règles simples (\textit{introduire les objectifs, ordonner les règles par complexité, traiter en premier les cas par défaut, trier les valeurs des variables catégorielles par ordre alphabétiques, ...}).
				Des exemples reconnus pour leur bonne conception y sont notamment cités si vous avez besoin de référence pour concevoir votre propre guide.
			\end{leftBarInformation}
			
			
			%%% c. Annotation
			
			% Annotation en tant que telle.
			Lorsque le guide d'annotation est rédigé, la \textbf{phase de labellisation} peut commencer (voir \textsc{Figure~\ref{figure:2.2.1-ORGANISATION-ANNOTATION-ETAPES-CLES-MATTER}}, étape 2. \textit{\textbf{A}nnotate}).
			Cette tâche est traditionnellement réalisée par un groupe d'experts choisi en fonction de leur connaissance sur problème à caractériser (dans nos exemples sur les bandes dessinées, ce serait plutôt des libraires ou des collectionneurs).
			Après leur avoir expliqué l'objectif de leur travail et partagé les règles de labellisation contenues dans le guide, les annotateurs se partagent les données et réalisent chacun une partie du corpus d'apprentissage.
			
			%%% d. Mini-cycle MAMA.
			
			\begin{leftBarInformation}
				% La théorie rencontre le réel.
				C'est généralement à ce stade que la théorie rencontre la pratique : certaines règles d'annotation peuvent difficilement être applicables, certains données peuvent être ambiguës ou hors-sujet, et deux annotateurs peuvent aussi avoir des avis différents sur l'annotation la plus adéquate.
				Il est aussi important de rappeler que l’annotation est un acte d'interprétation, et que les données sont donc labellisées par un humain dont l'avis n'est pas infaillible. 
				\cite{pustejovsky-stubbs:2012:natural-language-annotation} introduisent donc le premier sous-cycle \texttt{MAMA} en référence à la boucle entre \textit{\textbf{M}odelize} et \textit{\textbf{A}nnotate} qui peut avoir lieu tant que le guide d'annotation n'est pas adapté aux données manipulées ou que différents points de vues opposent les annotateurs.
				
				% Exemple.
				Par exemple, lors de l'annotation de la transcription audio en \textsc{Section~\ref{section:2.1.2.D-PRESENTATION-ANNOTATION-EXEMPLES-TRANSCRIPTION}}, il peut y avoir une voix principale accompagnée de plusieurs voix en arrière plan : une première adaptation du guide serait de clarifier si ces voix secondaires doivent être transcrites ou ignorées, voire si l'audio entier doit être considéré comme inexploitable.
				La réponse à cette question dépend bien entendu du phénomène à décrire et de l'objectif du modèle de \textit{Machine Learning} à entraîner : dans notre cas, nous pourrions probablement annoter uniquement la voix principale et ignorer l'audio si le bruit gène la compréhension.
			\end{leftBarInformation}
			
			%%% Finalité : la base d'apprentissage.
			À la fin de l'annotation (ou du cycle \texttt{MAMA}), le corpus d'entraînement est disponible pour concevoir un modèle de \textit{Machine Learning}.
		
		
		%%% 2.2.1.B. Concevoir le modèle (\textit{\textbf{T}rain}, \textit{\textbf{T}est}, \textit{\textbf{E}valuate}).
		\subsubsection{Concevoir le modèle (\textit{\textbf{T}rain}, \textit{\textbf{T}est}, \textit{\textbf{E}valuate}).}
		\label{section:2.2.1.B-ORGANISATION-ANNOTATION-ETAPES-CLES-TRAIN-TEST}
			
			%%% Apprentissage statistiques et importance du test.
			La phase d'entraînement du modèle est l'étape centrale de l'apprentissage automatique.
			Toutefois, comme l'apprentissage se base sur des méthodes statistiques, il est important d'introduire une phase de test et d'évaluation pour s'assurer des performances du modèle obtenu.
			Il est donc courant de considérer une boucle de raffinement du modèle tant que les performances n'ont pas atteint un seuil acceptable (voir \textsc{Figure~\ref{figure:2.2.1-ORGANISATION-ANNOTATION-ETAPES-CLES-MATTER}}, étapes 3. \textit{\textbf{T}rain}, 4. \textit{\textbf{T}est} et 5. \textit{\textbf{E}valuate}).
			
			%%% Train/Dev/Test.
			En pratique, il est d'usage de \textbf{créer trois jeux de données} à partir de la base d'apprentissage qui vient d'être annotée :
			\begin{itemize}
				\item le jeu d'\texttt{entraînement}: c'est sur cette partie des données que le modèle de \textit{Machine Learning} est conçu ;
				\item le jeu de \texttt{développement} (ou de validation): le modèle entraîné est évalué sur ce jeu de donnée pour étudier son comportement, identifier ses forces et ses faiblesses, et ainsi permettre de le comparer à d'autres modèles entraînés pour cette même tâche ;
				\item le jeu de \texttt{test} : le modèle retenu est évalué sur ce jeu de test pour déterminer ses performances réelles. Il est important que ce jeu de données 
			\end{itemize}
			% [van-der-goot:2021:we-need-talk]: définir un train/tune/dev/test
			
			%%% Evaluate.
			Ainsi, le modèle représente la connaissance présente dans le jeu \texttt{entraînement}, il est étudié puis affiné grâce au jeu de \texttt{développement}, et est finalement jugé en fonction de ses performances sur le jeu de \texttt{test}.
			Il est encore une fois difficile d'être exhaustif sur les analyses et les métriques à considérer car elles dépendent fortement du type de problème que le modèle tente de résoudre.
			Une métrique basique est l'\texttt{Accuracy} (ou taux de bonne prédiction), décrivant simplement le nombre de fois que le modèle a fait une bonne proposition sur l'ensemble du test.
			Suivant le problème et le type de données, d'autres métriques usuelles peuvent être utilisées comme le \texttt{MSE} (\textit{Mean Squared Error}) pour la prédiction de variables numérique (voir \cite{wallach-goffinet:1987:mean-squared-error}), le \texttt{f1-score} pour les variables catégorielles (voir \cite{sasaki:2007:truth-fmeasure}) ou le \texttt{WER} (\textit{Word Error Rate}) pour la transcription de textes (voir \cite{mccowan-etal:2005:use-information-retrieval}).
			Dans tous les cas, une règle d'or est de bien tenir à l'écart le jeu de test des deux autres jeux de données et qu'il ne soit pas utilisé dans la phase de développement pour éviter tout biais de sur-apprentissage\footnote{Pour plus de détails sur le sur-apprentissage: voir \cite{collins:2017:chapter-overfitting}}.

			%%% Finalité : le modèle et se sperformances.
			À la fin de ce cycle, le modèle de \textit{Machine Learning} est disposition à l'emploi, et ses performances théoriques sont celles obtenues sur le jeu de test.
		
		%%% 2.2.1.C. Revoir la base d'apprentissage (\textit{\textbf{R}evise}).
		\subsubsection{Revoir la base d'apprentissage (\textit{\textbf{R}evise}).}
		\label{section:2.2.1.C-ORGANISATION-ANNOTATION-ETAPES-CLES-REVISE}
		
			% Besoin de réviser.
			Pour terminer cette boucle, il est parfois nécessaire d'envisager de corriger son modèle en remettant en cause la modélisation du problème et l'annotation des données.
			\cite{voormann-gut:2008:agile-corpus-creationa} formalisait en effet ce besoin de réviser la conception d'une base d'apprentissage en observant les lacunes du modèle obtenu, et \cite{pustejovsky-stubbs:2012:natural-language-annotation} évoque certaines révisions nécessaires de la modélisation dès la phase d'annotation (voir sous-cycle \texttt{MAMA} dans la \textsc{Figure~\ref{figure:2.2.1-ORGANISATION-ANNOTATION-ETAPES-CLES-MATTER}}).
			
			% Identifier un besoin de réviser.
			Divers pistes peuvent mener à une évolution de la base d'apprentissage :
			\begin{itemize}
				\item le modèle de \textit{Machine Learning} peut avoir de mauvaise performances, malgré son affinage lors de la phase de développement, ou peut manquer d'adaptabilité sur des données réelles ;
				\item la modélisation ou l'annotation peuvent devenir obsolète car le phénomène modélisé évolue dans le temps ;
				\item un cas d'usage non identifié jusqu'à présent nécessite de nouvelles données pour être pris en compte ;
				\item ou encore, un nouvel algorithme de \textit{Machine Learning} a priori plus performant requiert une modélisation différente pour traiter le problème.
			\end{itemize}
			\begin{leftBarExamples}
				% Exemple Inflation prix
				Pour illustrer nos propos, prenons la tâche d'estimation du prix d'une bande dessinée (cf. \textsc{Section~\ref{section:2.1.2.A-PRESENTATION-ANNOTATION-EXEMPLES-REGRESSION}}) : il se peut que les prix annoté sur les transactions ne soient plus d'actualité à cause de l'inflation, et que les données doivent être ré-annotées pour prendre en compte les nouvelles valeurs du marché.
				
				% Exemple ajouter une classe.
				D'autre part, la modélisation en tant que telle peut aussi être impacté : par exemple, dans le cadre de la classification de l'état d'une bande dessinée à partir d'une photo (cf. \textsc{Section~\ref{section:2.1.2.B-PRESENTATION-ANNOTATION-EXEMPLES-CLASSIFICATION}}), on pourrait constater à l'usage qu'il manque une catégorie "\texttt{Très mauvais état}" nécessaire pour trier d’emblée toute \texttt{BD} indigne à la vente.
				
				% Exemple OCR.
				Enfin, il est possible que le modèle se comporte mal sur certaines données.
				Par exemple lors de l'identification d'une bande dessinée à partir de sa couverture (cf. \textsc{Section~\ref{section:2.1.2.C-PRESENTATION-ANNOTATION-EXEMPLES-EXTRACTION}}), certains textes du décors pourrait être extrait à tord (comme le texte de la pancarte \textguillemets{\textit{Saloon}} dans la \textsc{Figure~\ref{figure:2.1.2.C-PRESENTATION-ANNOTATION-EXEMPLES-EXTRACTION}}).
				Il faudra peut-être adapter l'annotation pour identifier les textes à ne pas extraire (avec une classe de rebus par exemple).
			\end{leftBarExamples}
			
			% Conclusion.
			Nous bouclons ainsi le cycle \texttt{MATTER} qui préfigure le besoin d'une amélioration continue d'un modèle de \textit{Machine Learning} pour que celui-ci soit le plus adapté à son environnement d'utilisation.
			
	
	%%%
	%%% Subsection 2.2.2: Portraits des acteurs intervenant sur un projet d'annotation.
	%%%
	\subsection{Portraits des acteurs intervenant sur un projet d'annotation}
	\label{section:2.2.2-ORGANISATION-ANNOTATION-ACTEURS}
	
		% Introduction: grande diversité de métiers.
		Au cours du cycle \texttt{MATTER}, nous pouvons constater que divers acteurs interviennent pour concevoir la base d'apprentissage et entraîner un modèle de \textit{Machine Learning}.
		Cette diversité de métiers qui gravitent autour du traitement automatique des données semble difficile à détailler, tant à cause de leur grand nombre que de leurs subtiles différences.
		Pour avoir un aperçu, vous pouvez consulter les offres d'emplois du marché actuel (voir \cite{team-datascientest:2022:metiers-data-mieux} ou \cite{databird:2023:10-metiers-data}) ou certaines formations professionnelles (voir \cite{isoz:2017:decouvrir-metiers-data}) pour pouvoir faire la distinction entre \textit{data scientist}, \textit{data analyst}, \textit{data librarian}, \textit{data journalist}, \textit{data architect}, \textit{data engineer}, \textit{data steward}, \textit{data archivist}, ou encore \textit{machine learning engineer}...
		
		% Approche par compétences.
		Afin d'avoir une approche moins commerciale de ces métiers, nous proposons plutôt de dresser les compétences requises au diverses phases du cycle, à l'image de \cite{radovilsky-etal:2018:skills-requirements-business} qui présente les acteurs de la science des données grâce à quatre groupes de compétences :
		\begin{enumerate}
			% Business expert.
			\item les compétences \textbf{métiers} : elles sont liées aux connaissances et à l'expertise sur le phénomène à modéliser ou le problème à résoudre.
			Ce sont grâce à ces compétences qu'un acteur peut être apte à annoter une donnée ou à qualifier la pertinence de la prédiction d'un modèle de \textit{Machine Learning}.
			Les métier(s) associé(s) sont : l'\texttt{expert métier} (\textit{business expert}) ;
			% Data analyst.
			\item les compétences \textbf{analytiques} : elles concernent entre autres la modélisation du problème, la gestion des données, et les analyses statistiques sur les biais et les performances.
			Ce sont grâce à ses compétences qu'un acteur peut concevoir le guide d'annotation, estimer le taux d'accord inter-annotateurs, ou encore réaliser l'évaluation statistique d'un modèle de \textit{Machine Learning}.
			Les métier(s) associé(s) sont : l'\texttt{analyste des données} (\textit{data analyst}) ou le \texttt{scientifique des données} (\textit{data scientist}) ;
			% Data scientist.
			\item les compétences \textbf{techniques} : elles portent sur l'ingénierie autour du modèle de \textit{Machine Learning}, comme le choix du meilleur algorithme d'entraînement et réglage fin des hyper-paramètres, l'archivage des différentes versions du modèle ainsi que son déploiement dans un environnement de production.
			Les métier(s) associé(s) sont : le \texttt{scientifique des données} (\textit{data scientist}), l'\texttt{ingénieur en Machine Learning} (\textit{Machine Learning Engineer}) ou l'\texttt{architecte des des données} (\textit{data architecte}) ;
			% Projet leader.
			\item et les compétences en \textbf{gestion} ou en \textbf{communication} : elles permettent d'aborder le cadrage du projet et la définition des objectifs, ainsi que diverses aptitudes transverses comme l'établissement de rapports, la gestion de projet, la vérification des normes, ...
			Les métier(s) associé(s) sont : le \texttt{chef de projet} (\textit{project leader}) ou le \texttt{responsable de la protection des données} (\textit{data protection officer}).
		\end{enumerate}
		\begin{leftBarInformation}
			On peut compléter cette vision par compétences avec la vision donnée par \cite{fort:2017:experts-ou-foule}, selon laquelle il y a trois types d'experts lors d'un projet d'annotation :
			\begin{itemize}
				\item les \textbf{experts du corpus} de données, ayant par exemple les connaissances sur les bandes dessinées, s'approchant donc de compétences \texttt{métiers} ;
				\item les \textbf{experts de l'annotation}, ayant par exemple les connaissances sur l'annotation de textes dans une image, s'approchant donc de compétences \texttt{analytiques} ;
				\item et les \textbf{experts de la tâche} de \textit{Machine Learning}, ayant par exemple les connaissances sur les techniques d'\texttt{OCR}, s'approchant donc des compétences \texttt{techniques}.
			\end{itemize}
		\end{leftBarInformation}
		
		% Application au cycle \texttt{MATER}.
		Ainsi, durant le cycle \texttt{MATTER}, nous pouvons voir les compétences ci-dessus se compléter :
		\begin{enumerate}
			% Concevoir la base d'apprentissage.
			\item la conception de la \textbf{base d'apprentissage} (étapes \textit{\textbf{M}odelize} et \textit{\textbf{A}nnotate}) nécessite :
			\begin{itemize}
				\item des compétences de \texttt{gestion} pour cadrer l'objectif du modèle à entraîner, et ainsi définir l'objectif auquel doit répondre l'annotation de données ;
				\item des compétences \texttt{analytiques} pour proposer une modélisation stable du phénomène et un guide d'annotation précis pour limiter les biais de conception ;
				\item des compétences \texttt{métiers} pour vérifier que la proposition de modélisation est pertinente vis-à-vis du cas d'usage, mais aussi pour réaliser l'annotation des données.
			\end{itemize}
			% Concevoir le modèle.
			\item la conception du \textbf{modèle de \textit{Machine Learning}} (étapes \textit{\textbf{T}rain}, \textit{\textbf{T}est}, \textit{\textbf{E}valuate}) nécessite :
			\begin{itemize}
				\item des compétences \texttt{analytiques} pour gérer les jeux de données (\textit{entraînement, développement, test}) et évaluer les performances statistiques du modèle ;
				\item des compétences \texttt{techniques} pour manipuler l'écosystème de développement du modèle, régler les hyper-paramètres, versionner les changements, et planifier la distribution du modèle ;
				\item des compétences de \texttt{gestion} pour s'assurer du respect des normes et des caractères privée ou confidentielle de certaines données.
			\end{itemize}
			% Revoir la base d'apprentissage
			\item la \textbf{révision} de la base d'apprentissage (étape \textit{\textbf{R}evise}) nécessite :
			\begin{itemize}
				\item des compétences \texttt{métiers} pour identifier le manque de pertinence du modèle vis-à-vis de certains cas d'usage ;
				\item des compétences \texttt{analytiques} pour disserter des performances réelles du modèle face à des données de production et remettre en question les précédents choix de modélisation pour espérer améliorer le modèle.
			\end{itemize}
		\end{enumerate}
		
		% Remarque sur la distance entre métier et technique.
		On notera que les compétences transverses de \textbf{gestion} ou \textbf{communication} ne sont pas spécifiques à une étape du cycle \texttt{MATTER} (\textit{le cadrage, la gestion de projet et l'établissement de rapports étant réalisés tout au long du projet}), alors que les compétences \texttt{métier} et \texttt{techniques} n'interviennent généralement pas au même moment du cycle : autrement dit, des experts métiers croisent rarement des experts techniques et ne partagent donc que très rarement leurs connaissances.
	
	
	%%%
	%%% Subsection 2.2.3: Choix du logiciel d'annotation.
	%%%
	\subsection{Choix du logiciel d'annotation}
	\label{section:2.2.3-ORGANISATION-ANNOTATION-LOGICIELS}
	
		% Introduction: besoin d'un outil d'annotation.
		Pour terminer la description de l'organisation d'un projet d'annotation, attardons nous sur le choix du logiciel à utiliser pour labelliser les données.
		En effet, une diversité d'applications existent pour répondre aux besoins des annotateurs, mais il est important de noter que l'absence de certaines fonctionnalités essentielles risque de gêner le projet d'annotation, soit par l'introduction de biais, soit à cause de son manque d'inter-opérabilité avec d'autres tâches d'analyses ou d'annotation.
		
		% Liste des fonctionnalités importantes.
		Nous faisons référence à \cite{finlayson-erjavec:2016:overview-annotation-creation} pour dresser ci-dessous une liste des fonctionnalités principales (voire essentielles) d'un logiciel d'annotation.
		Pour simplifier la lecture, nous proposons de regrouper ces fonctionnalités dans les catégories suivantes :
		\begin{itemize}
			% Répondre à la \textbf{besoin d'annotation}.
			\item répondre à la \textbf{besoin d'annotation} :
				cette fonctionnalité est bien entendu obligatoire, car un logiciel ne permettant pas d'annoter vos données ne vous sera d'aucune utilité.
				Cette remarque semble être une évidence, mais nous nous permettons aussi d'étendre l’avertissement aux logiciels n'étant pas destinés à votre besoin d'annotation mais qui peuvent être détournés pour y répondre indirectement : de telles contournements peuvent introduire des biais et offrent généralement une expérience utilisateur assez médiocre ;
			% Intégrer le \texbf{guide d'annotation}.
			\item intégrer le \textbf{guide d'annotation} :
				ce livrable issu de la phase de modélisation du cycle \texttt{MATTER} doit être facilement accessible par les annotateurs car il contient la documentation et les instructions à appliquer lors de la labellisation.
				Les logiciels permettant d'intégrer directement ces définitions (\textit{avec exemples et contre-exemples}) ainsi que les règles d'annotation (\textit{comme les labels mutuellement exclusifs, les détails obligatoires, ...}) ont donc un net avantage ergonomique pour respecter la modélisation définie et ainsi garantir la qualité de la base d'apprentissage ; 
			% Autoriser l'\textbf{annotation multiple}.
			\item autoriser l'\textbf{annotation multiple} et l'\textbf{annotation multi-modale}: 
				il est fréquent de devoir annoter plusieurs une même donnée suivant des modélisations ou des paradigmes différents pour répondre à plusieurs cas d'usage (\textit{en prenant l'exemple de l'annotation d'images, on peut détourer les objets présents, identifier les textes inscrits, proposer une ou plusieurs catégories générales, proposer une description textuelle, ...}) ou encore de devoir annoter des données de natures différentes (\textit{en combinant texte, image et voix comme dans l'annotation des sous-titres d'une vidéo}).
				Les logiciels permettant ainsi de labelliser plusieurs informations et de manipuler plusieurs types de données sont donc plus facilement réutilisables et permettent de centraliser les annotations ;
			% Evaluer la \textbf{qualité de l'annotation}.
			\item évaluer la \textbf{qualité de l'annotation} :
				les erreurs d'annotation et les divergences d'opinions sur la modélisation sont inévitables.
				Il est donc appréciable de pouvoir les identifier, soit sur la base d'une comparaison directe entre deux annotateurs, soit en comparant avec l'annotation la plus probable issue d'une base de référence.
				Il peut aussi être intéressant de pouvoir calculer les scores d'accord inter-annotateurs sur un même échantillon de données pour estimer la qualité de la base d'apprentissage, de pouvoir corriger les erreurs lors de revues ou encore de trancher les cas de conflits apparent lors d'avis discordants ;
			% Permettre l'\textbf{inter-opérabilité} technique.
			\item permettre l'\textbf{inter-opérabilité} technique :
				il est toujours frustrant de ne pas pouvoir réutiliser des annotations d'un projet à l'autre car le format de stockage n'est pas compatible.
				Les logiciels prenant donc en considération plusieurs format de données (\textit{\texttt{PNG}/\texttt{JPG}, \texttt{MP3}/\texttt{WAV}, \texttt{XLSX}/\texttt{XML}/\texttt{JSON}, ...}) et respectant les standards de la tâche d'annotation lors des imports et exports sont donc à privilégier.
				De plus, il est conseillé de ne pas écrire directement les annotation dans les données (\textguillemets{\textit{\textbf{[Lucky Luke]}{\textcolor{colorDarkPastelRed}{\texttt{/(personnage)}}}, le \textbf{[cow-boy]}{\textcolor{colorDarkPastelPurple}{\texttt{/(métier)}}} solitaire, a ...}}), mais de les stocker dans des fichiers séparés pour garder une meilleure gestion et permettre les annotations multiples ;
			% Gérer le \textbf{flux de travail} et le \textbf{suivi de projet}.
			\item gérer le \textbf{flux de travail} et le \textbf{suivi de projet} :
				certaines fonctionnalités simples sont nécessaires à l'organisation de l'équipe d'annotation.
				Cela peut comprendre la répartition de la charge de travail, l'historisation des changements pour permettre les retours arrières, la possibilité d'émettre des appels d'aide ou d'écrire des commentaires sur les choix d'annotation, ou encore l'accompagnement des nouveaux annotateurs lors de leur monté en compétence ;
			% Favoriser le \textbf{confort de l'annotateur}.
			\item favoriser le \textbf{confort de l'annotateur} :
				le logiciel choisi sera utilisé au quotidien par l'équipe d'annotation, il semble donc essentiel de leur offrir une expérience utilisateur agréable pour réaliser leur tâche.
				Cela peut passer par une customisation de l'interface utilisateur afin d'être adapter à l'objectif d'annotation et par le paramétrage de raccourcis claviers.
				Simplifier l'accès et l'installation du logiciel peut aussi s'avérer utile pour favoriser son adoption, en favorisant par exemple les applications web permettant plus facilement le travail collaboratif ;
			% Permettre des \texbf{annotations et d'analyses avancées}.
			\item permettre des \textbf{annotations et d'analyses avancées} :
				la littérature scientifique regorge de techniques pouvant assister un annotateur dans son travail (\textit{pré-annotation, apprentissage actif, visualisation, interaction, ...}).
				Nous détaillerons plusieurs de ces techniques dans la \textsc{Section~\ref{section:2.4-AVANCEES-ANNOTATION}}.
		\end{itemize}
		
		% Quelques exemples.
		Considérant la diversité de cas d'usage d'annotation, une liste exhaustive des outils de labellisation n'est bien entendu pas possible.
		Nous tenons toutefois à présenter quelques exemples illustrés dans la \textsc{Figure~\ref{figure:2.2.3-ORGANISATION-ANNOTATION-LOGICIELS}}.
		%
		\begin{figure}[!htb]
			\centering
			\includegraphics[width=0.95\textwidth]{figures/etatdelart-logiciel-exemples}
			\caption{
				Quatre exemples d'outils d'annotation :
				\textbf{(1)} \texttt{INCEpTION} pour le texte (\cite{klie-etal:2018:inception-platform-machineassisted}),
				\textbf{(2)} \texttt{prodigy} pour le texte ou l'image (\cite{montani-honnibal:2017:prodigy-modern-scriptable}),
				\textbf{(3)} \texttt{Audacity} pour l'audio (\cite{audacity-team:2000:audacity-free-audio}
				et \textbf{(4)} \texttt{CVAT} pour l'image (\cite{cvat.ai-corporation:2019:computer-vision-annotation}).
			}
			\label{figure:2.2.3-ORGANISATION-ANNOTATION-LOGICIELS}
		\end{figure}
		
		% Avis sur la sur-utilisation d'Excel.
		\begin{leftBarAuthorOpinion}
			De part notre expérience, nous constatons malheureusement que plusieurs projets industriels n'utilisent pas ou peu d'outils d'annotation dédiés, et se contentent plutôt d'outils rudimentaires comme des traitement de textes ou des tableurs tels que \texttt{Microsoft Excel} (\cite{microsoft-corporation:2018:microsoft-excel}).
			Une étude serait à mener pour étudier cette tendance et expliquer le manque d'intérêt porté aux outils spécialement conçus pour les tâches d'annotations.
			Peut-être que ces derniers s'adaptent mal aux particularités des divers projets industriels, expliquant ainsi l'utilisation d'outils simplistes mais flexibles. Ou alors est-ce par méconnaissance des difficultés et des bais potentiels de l'annotation que ces outils aux fonctionnalités avancées ne sont pas employés ?
		\end{leftBarAuthorOpinion}
	
	%%%
	%%% Conclusion.
	%%%
	\begin{leftBarSummary}
		\begin{todolist}
			\item[\itemok] Un projet d'annotation s'organise généralement en cycle (\texttt{MATTER}) au cours duquel nous réalisons une modélisation abstraite des données que nous formalisons dans un guide (\textit{\textbf{M}odelize}), nous appliquons ce guide pour labelliser notre base d'apprentissage (\textit{\textbf{A}nnotate}), puis nous entraînons et testons un modèle de \textit{Machine Learning} (\textit{\textbf{T}rain}, \textit{\textbf{T}est} et \textit{\textbf{E}valuate}). Ensuite, l'évaluation du modèle peut mener à une révision de la modélisation des données en fonction des performances obtenues (\textit{\textbf{R}evise}) ;
			\item[\itemok] Un tel projet d'annotation nécessite une diversité de connaissances et de compétences qui peuvent être réparties en quatre catégories : \texttt{métier}, \texttt{analytique}, \texttt{technique} et \texttt{gestion/communication} ;
			\item[\itemok] Un tel projet nécessite aussi un outil d'annotation dédié possédant certaines fonctionnalités essentielles comme la possibilité d'\texttt{intégrer le guide} d'annotation, le besoin de \texttt{contrôler la qualité} des annotations, la capacité à réaliser des \texttt{annotation multiples ou multimodales}, ou encore l'assimilation d'éléments de \texttt{gestion de projet}.
		\end{todolist}
	\end{leftBarSummary}
	
	
	%%%%%--------------------------------------------------------------------
	%%%%% Section 2.3: Les nombreux défis de l'annotation
	%%%%%--------------------------------------------------------------------
	%\newpage
	\section{Les nombreux défis de l'annotation}
\label{section:2.3-DEFIS-ANNOTATION}

	%%%
	%%% Introduction: annoncer la complexité due (1) aux données (2) à la tâche et (3) aux humains.
	%%%
	
	Comme nous avons pu l'apercevoir dans les sections précédentes, le cycle d'annotation recèle de zones d'ombres pouvant introduire des complications dans la conception d'une base d'apprentissage (\cite{baledent:2022:complexite-annotation-manuelle}).
	Pour aborder cette partie, nous alors voir :
	\begin{itemize}
		\item qu'il y a une forte pression sur la \textbf{qualité des données} devant constituer le corpus d'entraînement (cf. \textsc{Section~\ref{section:2.3.1-DEFIS-ANNOTATION-ASPECT-DONNEES}}) ;
		\item que ce standard de qualité entretient une \textbf{complexité inhérente} aux étapes de modélisation et d'annotation (cf. \textsc{Section~\ref{section:2.3.2-DEFIS-ANNOTATION-ASPECT-COMPLEXITE}}) ;
		\item et que cette complexité provoque des \textbf{différences de comportements} chez les annotateurs (cf. \textsc{Section~\ref{section:2.3.3-DEFIS-ANNOTATION-ASPECT-HUMAIN}}).
	\end{itemize}
	En détaillant chacun de ces trois points, nous discuterons de l'ensemble de techniques et bonnes pratiques mises en avant dans la littérature pour limiter les désagréments d'un projet d'annotation.
	Nous identifierons aussi les freins récurant pouvant intervenir dans les mises en application industrielles.
	
	
	%%%
	%%% Subsection 2.3.1: Défis concernant le besoin de qualité des données.
	%%%
	\subsection{Défis concernant le besoin de qualité des données}
	\label{section:2.3.1-DEFIS-ANNOTATION-ASPECT-DONNEES}
	
		% Introduction: Machine Learning = reproduire par l'exemple.
		Comme nous l'avons défini en \textsc{Section\ref{section:2.1.1.A-PRESENTATION-ANNOTATION-DEFINITION-MACHINE-LEARNING}}, l'\textguillemets{\texttt{apprentissage automatique}} regroupe un ensemble de techniques dont l'objectif est de reproduire une tâche \textbf{par l'exemple} : il est donc normal de porter une attention particulière aux données utilisées, car la qualité du modèle de \textit{Machine Learning} va fortement dépendre de la qualité de sa base d'apprentissage.
		Nous allons ici détailler trois défis actuels concernant cette création d'un jeu de données.
		
		
		%%% 2.3.1.A. Problèmes de représentativité.
		\subsubsection{Problèmes de représentativité}
		\label{section:2.3.1.A-DEFIS-ANNOTATION-ASPECT-DONNEES-REPRESENTATIVITE}
			
			% Introduction : Une collecte doit être "représentative" du problème.
			La phase de collecte de données est une étape importante du projet d'annotation.
			Malheureusement, la littérature scientifique associée à cette tâche est assez légère alors que c'est précisément à ce moment que ce joue une caractéristique cruciale de la future base d'apprentissage : la \textbf{représentativité} du phénomène à modéliser.
			
			% Définir la représentativité à partir de la méthode de l'échantillon.
			Cette notion est assez ambiguë, notamment car le terme technique \textguillemets{représentatif}fait écho à un mot de la vie courante qui peut avoir plusieurs sens.
			Dans \cite{kruskal-mosteller:1979:representative-sampling-nonscientific} et \cite{clemmensen-kjaersgaard:2022:data-representativity-machine}, plusieurs usages communs de ce terme sont recensés :
			\begin{itemize}
				\item \textguillemets{\textit{assertive claim}} :
				l'opérateur déclare que ses données sont représentatives du problème sans apporter d'arguments. Bien entendu, cette option est à bannir car elle n'apporte aucune information et peut cacher des vices de conception du modèle ;
				\item \textguillemets{\textit{absence or presence of selective force}} :
				la représentativité du phénomène est supposée en sélectionnant des données de \textbf{manière aléatoire} et en limitant le nombre de critères de sélection à ceux nécessaires pour l'étude réalisée ; 
				\item \textguillemets{\textit{miniature}} :
				aussi appelée \textbf{sélection stratifiée}, cette approche consiste à dire qu'un échantillon est représentatif d'un phénomène si la proportion de chacune de ses parties y est respectée
				(\textit{par exemple, un sondage peut respecter la répartition des tranches d'âge d'une population}) ;
				\item \textguillemets{\textit{typical/ideal case}} :
				cette définition consiste à représenter chaque partie du phénomène par un \textbf{exemple emblématique} ou un \textbf{exemple moyen}
				(\textit{par exemple, on peut illustrer l'univers de la bande dessinée française par un numéro de la saga \texttt{Astérix \& Obélix}}) ;
				\item \textguillemets{\textit{population coverage}} :
				ici, la représentativité est associée à la présence \textbf{exhaustive} de l'ensemble des caractéristiques importantes du phénomène avec au moins un exemple par caractéristique
				(\textit{par exemple, dans l'arche de Noé, il y devait y avoir au moins un couple de chaque animaux}).
			\end{itemize}
			
			% Définir la représentativité à partir de la méthode d'échantillonnage.
			Pour compléter ces définitions, \cite{kruskal-mosteller:1979:representative-sampling-ii} incitent sur le besoin de \textbf{définir avec précision la méthode d'\textit{échantillonnage}} plutôt que l'\textit{échantillon} lui-même : en effet, il est important de caractériser le phénomène à modéliser, de définir l'objectif de la collecte de données et de détailler comment cette collecte va être réalisée.
			\cite{clemmensen-kjaersgaard:2022:data-representativity-machine} introduisent à leur tour trois mesures pour aider à caractériser une collecte : la \textit{réflexion} (est-ce l'échantillon respecte la distribution de la population ?), la \textit{couverture} (est-ce que l'échantillon illustre la diversité de la population ?) et la \textit{présence de représentants} (est-ce que l'échantillon contient les exemples emblématiques de la population ?).
			De telles informations sont essentielles pour pouvoir juger de la valeur d'un échantillon par rapport à un cas d'usage et déterminer s'il est réutilisable dans pour une autre application.
			%
			\begin{leftBarExamples}
				% Cas d'usage : classification d'état.
				Illustrons nos propos avec la classification de l'état d'une bande dessinée à partir d'une photo (voir \textsc{Section~\ref{section:2.1.2.B-PRESENTATION-ANNOTATION-EXEMPLES-CLASSIFICATION}}).
				Afin de représenter correctement le cas d'usage, nous pourrions collecter des exemples de \texttt{BD} couvrant l'ensemble des dégradations fréquemment identifiées par les libraires (couvertures froissées, pages déchirées, couleurs délavées, ...) et les intégrer de manière proportionnelle dans la base d'apprentissage.
				Nous pourrions aussi nous assurer de la présence de cas emblématiques permettant de catégoriser les \texttt{BD} en "\texttt{Mauvais état}", "\texttt{Bon état}", "\texttt{Très bon état}", "\texttt{Neuf}".
				
				% Autre cas d'usage : pas représentatif.
				Toutefois, cette base d'apprentissage ne sera plus représentative si nous voulons détecter la langue de la bande dessinée (auquel cas, une répartition par langue sera a priori plus adéquate).
			\end{leftBarExamples}
			
			% Besoin de beaucoup d'exemples.
			La description d'un phénomène reste cependant une \textbf{tâche difficile}, notamment lorsque que celui-ci possède un ensemble vaste et abstrait de caractéristiques à décrire.
			Nous comptons généralement sur la loi des grands nombres pour espérer dresser un portrait fidèle du phénomène, mais cela impose parfois de traiter des volumes colossaux de données.
			%
			\begin{leftBarExamples}
				Considérons le traitement du langage : le vocabulaire employé peut concerner des dizaines de millier de mots, il existe des variantes régionales et divers jargons techniques, certains termes peuvent avoir plusieurs sens et des expressions peuvent dépendre de leur contexte (comme l'humour ou les critiques).
				Pour représenter ces spécificités (listées de manière non exhaustive), une base d'apprentissage devra contenir de nombreux exemples afin de capturer les différents aspects du langage à traiter.
				On peut citer par exemple \texttt{MLSUM: The Multilingual Summarization Corpus} (\cite{scialom-etal:2020:mlsum-multilingual-summarization}), une base de $1.5$ millions d'articles de journaux sur $5$ langues pour entraîner un modèle de résumé automatique, ou encore \texttt{The Multilingual Amazon Reviews Corpus} (\cite{keung-etal:2020:multilingual-amazon-reviewsa}), une base de $1.26$ millions de commentaires de produits sur $6$ langues pour entraîner un modèle de classification de la note d'un commentaire sur $5$ étoiles.
			\end{leftBarExamples}
			
			% Discussion sur les biais de sur-représentativités ou sous-représentativités.
			Toutefois, la masse de données ne résout pas toujours tous les problèmes de représentativité.
			Une des difficultés récurrentes concerne les aspects peu fréquents d'un phénomène qui se retrouvent ainsi \textbf{sous-représentés} : si l'enjeu du modèle à concevoir consiste justement à détecter ou reproduire ces aspects, il peut être intéressant de volontairement biaiser les proportions du corpus d'entraînement pour mieux les illustrer.
			À l'inverse, des cas communs ou fréquents peuvent être \textbf{sur-représentées} : il est parfois nécessaire de limiter leur occurrence dans le corpus d'apprentissage pour ne pas concevoir un modèle véhiculant des généralités ou des stéréotypes.
			Dans les deux cas, \textbf{toute intervention va introduire un biais} : l'opération doit donc être réfléchie et judicieusement réalisée pour contribuer à la finalité du modèle, d'où l'intérêt de bien la documenter pour faire entendre ce que vous voulez signifier par \textguillemets{échantillon représentatif}.
			%
			\begin{leftBarExamples}
				% Problème de sous-représentation: Détecter la langue latine ou la langue corse.
				D'une part, considérons le besoin de détecter la langue d'une bande dessinée (voir \textsc{Section~\ref{section:2.1.2.B-PRESENTATION-ANNOTATION-EXEMPLES-CLASSIFICATION}}).
				Il est fort probable que la base d'apprentissage contiennent peu de données sur les parutions en langues régionales (en Corse, en Wallon, en Alsacien, ...).
				Nous serons peut-être amener à ajouter des exemples supplémentaires pour espérer mieux les détecter et ainsi augmenter la \textit{couverture} de notre jeu de données.
				
				% Problème d'équilibrage: Stéréotypes de Stable Diffusion.
				D'autre part, regardons l'analyse du modèle de \texttt{Stable Diffusion} réalisée par \cite{nicoletti-bass:2023:generative-ai-takes} sur la génération de portraits de personnes fictives à partir d'une description textuelle de leur métier.
				L'étude montre que le modèle tant générer des portraits d'hommes à la peau blanche pour le métier d'architecte ou d'ingénieur, des femmes pour le rôle de concierge ou encore des personnes à la peau noire pour illustrer la classe ouvrière.
				Ici, ce modèle dépeint les inégalités de notre société en représentant ses stéréotypes, et cela ouvre la question suivante : veut-on vraiment reproduire à l'identique cette représentation ?
			\end{leftBarExamples}
			%
			\begin{leftBarIdea}
				% Équilibrage par la génération de données.
				Une piste pour équilibrer efficacement les corpus d'entraînement et permettre de corriger leurs biais consiste à utiliser des \textbf{données synthétiques} (\cite{jaipuria-etal:2020:deflating-dataset-bias}).
				Ces données peuvent être créées manuellement ou être générées automatiquement (voir \cite{shorten-etal:2021:text-data-augmentation} pour une revue de génération de textes et \cite{shorten-khoshgoftaar:2019:survey-image-data} pour la génération d'image).
				Bien entendue, une telle approche doit restée réfléchie pour ne pas introduire davantage de biais et pour répondre à un objectif précis d'équilibrage du jeu de données.
			\end{leftBarIdea}
		
		
		%%% 2.3.1.B. Problèmes de bruits.
		\subsubsection{Problèmes de bruits}
		\label{section:2.3.1.B-DEFIS-ANNOTATION-ASPECT-DONNEES-BRUITS}
		
			% Introduction : données bruités.
			La qualité d'une base d'apprentissage dépend fortement du bruit qu'elle contient.
			Ce bruit est inévitablement inséré lors de la collecte :
			d'une part, la méthode de collecte elle-même peut en introduire (\textit{instrument de mesure faillible, erreur humaine, ...}) ;
			d'autre part, par soucis de représentativité, le bruit intrinsèque du phénomène va être capturé (\textit{forte variabilité, présence d'irrégularité, ...}).
			Un échantillon de données va donc forcément devoir se confronter 

			% Classification des types de bruits.
			En s'inspirant de \cite{maharana-etal:2022:review-data-preprocessing} et de \cite{alasadi-bhaya:2017:review-data-preprocessing}, nous dressons la liste suivante de problèmes récurrents sur les données suite à une collecte :
			\begin{itemize}
				% données non pertinentes
				\item présence de \textbf{données non pertinentes} par rapport au cas d'usage ;
				% variations parasites
				\item dégradation des données par des \textbf{variations parasites} ;
				% absence de valeurs
				\item \textbf{absence de valeurs} descriptives essentielles ;
				% incohérences / ambiguïté
				\item présence d'\textbf{incohérences} ou d'\textbf{ambiguïté} entre les données ;
			\end{itemize}
			%
			\begin{leftBarExamples}
				% Problèmes: estimation du prix d'une BD.
				Pour illustrer ces problèmes, considérons la tâche d'estimation du prix d'une bande dessinée (voir \textsc{Section~\ref{section:2.1.2.A-PRESENTATION-ANNOTATION-EXEMPLES-REGRESSION}}) :
				\begin{itemize}
					% Exemple: données non pertinentes.
					\item une donnée concernant le prix d'un roman ou d'une encyclopédie avoir été inséré par mégarde et pourrait être considéré comme données non pertinentes pour ce cas d'usage ;
					% Exemple: variations parasites.
					\item un changement de typographie (\textit{majuscules, minuscules, accents, ponctuation}) dans l'écriture d'un titre pourrait mal identifier une bande dessinée ;
					% Exemple: absence de valeurs.
					\item une information peut ne pas avoir été renseigné lors d'une transaction (l'année d'édition par exemple), alors que c'est une caractéristique importante de la prise de décision ;
					% Exemple: incohérences / ambiguïté.
					\item une même bande dessinée (avec les mêmes caractéristiques) peut avoir été vendue à deux prix radicalement différent, introduisant ainsi une légère ambiguïté dans les données.
				\end{itemize}
				
				% Autres problèmes.
				Des problèmes similaires peuvent impacter le traitement du texte (\textit{fautes de grammaticales ou syntaxiques, ambiguïtés ou sémantiques, omissions, ...}), des images (\textit{flous, mauvais cadrages, colorimétries gênantes, ...}) et de l'audio (\textit{bruits en arrière plan, saturations, coupures inopportunes, ...}).
				% Problème d'image flou: classification de l'état d'une BD.
				Par exemple, dans la tâche de classification de l'état d'une bande dessinée à partir d'une photo (voir \textsc{Section~\ref{section:2.1.2.B-PRESENTATION-ANNOTATION-EXEMPLES-CLASSIFICATION}}), comment juger correctement de l'état d'une \texttt{BD} si la photo capturée est flou ?
			\end{leftBarExamples}
			
			% Correction des données par le prétraitement et le nettoyage.
			Pour limiter l'impact du bruit dans les données, \cite{alasadi-bhaya:2017:review-data-preprocessing} structurent les étapes de prétraitement de données entre quatre catégories :
			\begin{itemize}
				% 1. data cleaning: completer les données (mean, random, model, constante) et corriger le bruit des données (prétraitement, regroupement pour numérique).
				\item le \textbf{nettoyage des données} : cette étape consiste à compléter les données manquantes (\textit{en prenant la valeur moyenne par exemple}), à filtrer les données aberrantes ou inintéressantes, et surtout à lisser le bruit dans les données en gommant les variations parasites ;
				% 2. data integration: associer plusieurs sources de données pour les rendre consistente
				\item l'\textbf{intégration des données} : dans certains cas, plusieurs sources de données sont disponibles, il peut être donc intéressant de croiser ces sources de données pour augmenter la consistance de la base d'apprentissage et identifier les incohérences ;
				% 3. data transformation: transformer les données pour les exploiter plus facilement (en les lisant, les normalisant, ...)
				\item le \textbf{formatage des données} : pour exploiter facilement les données, certaines transformations sont parfois nécessaires pour limiter les ambiguïtés dues au leur format (\textit{par exemple: normaliser une valeur entre $0$ et $1$});
				% 4. data reduction: réduire la dimensionnalité et les features peu utiles
				\item la \textbf{réduction des données} : en réalisant une analyse approfondie des données, on peut quelques fois constater que certaines caractéristiques présentes sur les données sont peu utiles et peuvent être supprimer pour réduire la complexité de la base d'apprentissage.
			\end{itemize}
			\cite{baledent:2022:complexite-annotation-manuelle} rappelle néanmoins que le document source (ou ici : la donnée brute) doit rester accessible pour la phase d'annotation pour ne pas manquer d'information potentiellement intéressante.
			%
			\begin{leftBarExamples}
				% Correction: estimation du prix d'une BD.
				Reprenons les problèmes évoqués précédemment sur la tâche d'estimation du prix d'une bande dessinée :
				\begin{itemize}
					% Correction: données non pertinentes.
					\item une donnée inintéressante peut simplement être supprimée ;
					% Correction: variations parasites.
					\item normaliser les champs décrivant une bande dessinée en passant tout en minuscules limiterait les chances de mal l'identifier ;
					% Correction: absence de valeurs.
					\item une année d'édition manquante pourrait être identifiée par l'étiquette \texttt{inconnue} et un prix manquant pourrait être complété par la moyenne du prix des \texttt{BD} ayant les mêmes caractéristiques ;
					% Correction: incohérences / ambiguïté
					\item une prix faussé pourrait être identifié comme incohérent en analysant les prix des \texttt{BD} ayant les mêmes caractéristiques ;
				\end{itemize}
			\end{leftBarExamples}
		
		
		%%% 2.3.1.C. Problèmes d'exploitation et de diffusion.
		\subsubsection{Problèmes d'exploitation et de diffusion}
		\label{section:2.3.1.C-DEFIS-ANNOTATION-ASPECT-DONNEES-DROITS}
		
			% Introduction: toutes les données ne sont pas disponibles.
			En plus des difficultés techniques sur la réalisation d'une collecte de données, il y a aussi contraintes législatives et stratégiques à prendre en compte.
			
			% Contraintes sur la propriété intellectuelle des données.
			D'une part, il faut considérer le fait que certaines données sont protégées et ne peuvent pas être collectées ou exploitées librement.
			C'est le cas de données soumises aux droits de \textbf{propriétés intellectuelles} qui empêchent cet usage : on peut citer par exemple \cite{loignon:2023:ia-medias-francais} qui évoque le levé de bouclier des médias français contre l'utilisation de leur article pour entraîner des modèles de langues, mais aussi \cite{les-echos:2023:ia-auteur-game} qui questionne la violation du \textbf{droit d'auteur} lorsque qu'un modèle est entraîné sur l'oeuvre d'un artiste et qu'il est capable de la reproduire.
			
			% Contraintes sur le consentement.
			Ces limites concernent aussi la Réglementation Générale européenne sur la Protection des Données (\texttt{RGPD}, \cite{european-commission:2016:regulation-eu-2016}) restreignant les \textbf{collectes et usages non consenties} de données personnelles.
			Ainsi, il n'est pas possible d’entraîner n'importe quel modèle sur n'importe quelles données, et une telle contrainte impose de manipuler les données en garantissant l'anonymat et la confidentialité des personnes consentantes.
			
			% Contraintes stratégiques.
			Pour aller plus loin, cette notion de confidentialité touche les données personnelles, mais aussi le \textbf{caractère stratégique} d'une organisation.
			En effet, dans le monde académique, les données manipulées sont le plus souvent publiques et peuvent être employées pour contribuer à la recherche scientifique.
			Mais dans le secteur industriel, les jeux de données sont liés au domaine d'activité de l'entreprise : ils ont généralement requis un investissement conséquent en temps et en moyens, et ils représentent donc son avantage concurrentiel (\textit{par leur spécificité, leur caractère secret ou novateur, leur qualité compétitive, ...}).
			Il est donc rare de voir une entreprise partager ses jeux de données car elle pourrait perdre un de ses atouts stratégiques.
			
			% Exemple OpenSource, mais limitation commerciale, donc beaucoup de recréation.
			Un solution est de ce tourner les jeu de données \textbf{accessibles en \textit{Open Source}}.
			Plusieurs plateformes mettre en effet à disposition des données ou des modèles, comme \textit{Hugging Face} (\cite{hugging-face:2016:hugging-face-ai}) ou \texttt{Zenodo} (\cite{re3data.org:2013:zenodo}).
			Toutefois, deux limites subsistent à l'usage de ces données :
			\begin{itemize}
				\item les données mises à dispositions publiquement sont souvent assez générales et ne reflète pas la spécificité des cas d'usage de l'entreprise, limitant ainsi leur intérêt ;
				\item les données publiques ne sont pas forcément ouvertes à un usage commercial (\textit{elles peuvent par exemple employé la licence \texttt{CC BY-NC 4.0}, \cite{creative-commons:2013:cc-bync-legal}}), restreignant ainsi les seules applications au domaine de la recherche et veille scientifique.
			\end{itemize}
			Pour ne pas faire de faux pas juridique, \cite{rajbahadur-etal:2022:can-use-this} propose une approche pour vérifier si une licence permet d'exploiter un jeu de données.
			
			% Besoin de tracabilité.
			\begin{leftBarInformation}
				Pour terminer nous mentionnons aussi une proposition de législation européenne concernant la futures réglementation des modèles d'intelligence artificielle (\texttt{IA Act}, \cite{european-commission:2021:proposal-regulation-european}).
				Cette loi concerne les quatre objectifs suivants :
				\begin{itemize}
					\item \textguillemets{veiller à ce que les systèmes d'\texttt{IA} mis sur le marché de l'Union et utilisés soient sûrs et respectent la législation en vigueur en matière de droits fondamentaux et les valeurs de l'Union} ;
					\item \textguillemets{garantir la sécurité juridique pour faciliter les investissements et l'innovation dans le domaine de l'\texttt{IA}} ;
					\item \textguillemets{renforcer la gouvernance et l'application effective de la législation existante en matière de droits fondamentaux et des exigences de sécurité applicables aux systèmes d'\texttt{IA}} ;
					\item \textguillemets{faciliter le développement d'un marché unique pour des applications d'\texttt{IA} légales, sûres et dignes de confiance, et empêcher la fragmentation du marché}.
				\end{itemize}
				Un besoin de traçabilité des données et des modèles se fait donc sentir, renforçant les recommandations à documenter et détailler les traitement et choix pour garantir la représentativité et la qualité des données des bases d'apprentissage.
			\end{leftBarInformation}
		
		
	%%%
	%%% Subsection 2.3.2: Défis concernant la complexité inhérente à la tâche d'annotation.
	%%%
	\subsection{Défis concernant la complexité inhérente à la tâche d'annotation}
	\label{section:2.3.2-DEFIS-ANNOTATION-ASPECT-COMPLEXITE}
		\todo[inline]{SECTION: À RÉDIGER}
		
		%%% 2.3.2.A. Estimation de la complexité.
		\subsubsection{Estimation de la complexité}
		\label{section:2.3.2.A-DEFIS-ANNOTATION-ASPECT-COMPLEXITE-ESTIMATION}
		
			\todo[inline]{citer et détailler [gut-bayerl:2004:measuring-reliability-manual]: juste le score IAA}
			\todo[inline]{citer et détailler [fort-etal:2012:modeling-complexity-manual]: 6 points}
			
			\todo[inline]{quelques solutions: approche non supervisée ou semi-supervisée (apprentissage actif, approche itérative)}
		
		%%% 2.3.2.B. Problèmes de coûts.
		\subsubsection{Problèmes de coûts}
		\label{section:2.3.2.B-DEFIS-ANNOTATION-ASPECT-COMPLEXITE-COUTS}
		
			\todo[inline]{liste des coûts: temps, argents, recrutement, ...}
			\todo[inline]{conséquence: charge de travail élevée}
			\todo[inline]{quelques solutions: transfert d'apprentissage, pré-annotation, apprentissage actif}
		
		
	%%%
	%%% Subsection 2.3.3: Défis concernant les différences de comportements d'annotation.
	%%%
	\subsection{Défis concernant les différences de comportements d'annotation}
	\label{section:2.3.3-DEFIS-ANNOTATION-ASPECT-HUMAIN}
		\todo[inline]{SECTION: À RÉDIGER}
		
		%%% 2.3.3.A. Différences inter-annotateurs.
		\subsubsection{Différences inter-annotateurs}
		\label{section:2.3.3.A-DEFIS-ANNOTATION-ASPECT-HUMAIN-INTER-ANNOTATEURS}
		
			\todo[inline]{subjectivité}
			\todo[inline]{expertise et formation}
			
			\todo[inline]{quelques solutions palliatives: évaluation, redondances}
			\todo[inline]{quelques solutions de fonds: revues, subdiviser la tâche, guide, crowd-sourcing}
		
		%%% 2.3.3.B. Différences intra-annotateurs.
		\subsubsection{Différences intra-annotateurs}
		\label{section:2.3.3.B-DEFIS-ANNOTATION-ASPECT-HUMAIN-INTRA-ANNOTATEURS}
		
			\todo[inline]{régulation charge de travail}
			\todo[inline]{troubles}
			\todo[inline]{travail mal reconnu voire ingrat}
			\todo[inline]{quelques solutions: guide, revues, limiter le changement de contexte, gamification}
	
	
	%%%
	%%% Conclusion.
	%%%
	\begin{leftBarSummary}
		\begin{todolist}
			\item[\itemok] L'enjeu d'un projet d'annotation consiste à avoir des \textbf{données de qualité} qui soient représentatives du problème à traiter ;
			\item[\itemok] Or la tâche d'annotation et son exigence de qualité engendre de la \textbf{complexité}, et donc une \textbf{charge de travail élevée} ;
			\item[\itemok] Pour réguler cette charge de travail élevée, chaque opérateur va \textbf{adapter sa tâche} pour la rendre supportable, créant alors des \textbf{différences de comportement}.
		\end{todolist}
	\end{leftBarSummary}
	
	
	%%%%%--------------------------------------------------------------------
	%%%%% Section 2.4:
	%%%%%--------------------------------------------------------------------
	%\newpage
	\section{Techniques et organisations avancées d'annotation}
\label{section:2.4-AVANCEES-ANNOTATION}

	%%%
	%%% Introduction:
	%%%
	\todo[inline]{SECTION: À RÉDIGER: \\
		- Sur les données: ... \\
		- Sur la complexité: ... \\
		- Sur les annotateurs: ...
	}
	
	
	%%%
	%%% Subsection 2.4.1: Avancées concernant le besoin de qualité des données.
	%%%
	\subsection{Avancées concernant le besoin de qualité des données}
	\label{section:2.4.1-AVANCEES-ANNOTATION-ASPECT-DONNEES}
		\todo[inline]{SECTION: À RÉDIGER}
	
	
	%%%
	%%% Subsection 2.4.2: Avancées cla diminution de la complexité de la tâche d'annotation
	%%%
	\subsection{Avancées concernant la diminution de la complexité de la tâche d'annotation}
	\label{section:2.4.2-AVANCEES-ANNOTATION-ASPECT-COMPLEXITE}
		\todo[inline]{SECTION: À RÉDIGER}
	
	
	%%%
	%%% Subsection 2.4.3: Avancées concernant la réduction des différences de comportements intra- et inter-annotateurs.
	%%%
	\subsection{Avancées concernant la réduction des différences de comportements intra- et inter-annotateurs}
	\label{section:2.4.3-AVANCEES-ANNOTATION-ASPECT-HUMAIN}
		\todo[inline]{SECTION: À RÉDIGER}
	
	
	%%%
	%%% Conclusion.
	%%%
	\begin{leftBarSummary}
		\begin{todolist}
			\item[\itemok] (TODO)
		\end{todolist}
	\end{leftBarSummary}
	
	
	%%%%%--------------------------------------------------------------------
	%%%%% Section 2.5:
	%%%%%--------------------------------------------------------------------
	\section{(\textit{conception toujours difficile en entreprise})}
	\label{section:2.5-RETOUR-EXPERIENCES-INDUSTRIELLES}
		\todo[inline]{SECTION: TITRE À TROUVER: "\textit{REX entreprise}"}
		\todo[inline]{SECTION: À RÉDIGER: \\
			- Modélisation toujours compliquée ; \\
			- Expert métier "pas à leur place" ; \\
			- Peu de stratégies de notre revue mises en oeuvre...
		}