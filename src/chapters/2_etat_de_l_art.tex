\chapter{État de l'art : concevons un jeu de données}
    \label{chapter:2-ETAT-DE-L-ART}

    Dans cette partie, nous allons faire un état des lieux des méthodes pour créer le premier jeu de données nécessaire à l’entraînement d'un assistant conversationnel.
    Cela comprend une description des acteurs du projet, un rappel de l'organisation usuelle en fonction de leur compétence, et une énumération des problèmes et solutions les plus communs.
	\todo[inline]{TRANSITION À COMPLÉTER}
    \todo{Rappel des contraintes industrielles}

    \minitoc

    %%%%%--------------------------------------------------------------------
    %%%%% Section 2.1:
    %%%%%--------------------------------------------------------------------
    \section{Rappel sur le fonctionnement d'un chatbot}
    \todo{titre: Approche statistique vs symbolique}
		\todo[inline]{SECTION À RÉDIGER}
		\todo[inline]{
			Remarque Gautier 20/02/2023:
			Le "usuel" est clairement à discuter ici.
			Il y a deux approches à la connaissance, qui sont ici à discuter, je pense :
			- une approche statistique, qui cherche DIRECTEMENT à générer la connaissance à partir de la masse de données ingérée (on y retrouve les approches génératives, par exemple)
			- une approche symbolique, dans laquelle on décide de passer par des représentations symboliques intermédiaires (les intentions et entités) comme médiateur de la réponse qu'on apporte au client
			Il n'y a pas d'approche qui soit "usuelle", à mon sens, mais uniquement deux approches de la connaissance différentes, chacune à ses avantages, et en l'occurrence on peut apprécier le pragmatisme de l'approche symbolique, puisque ça a un côté très efficace et ça permet de garder le contrôle sur le vocabulaire (les symboles) qu'on souhaite couvrir.
			Quelle que soit ta position sur le sujet, je ne pense pas que tu puisses directement parler de fonctionnement usuel sans passer d'abord en revue les différentes approches qu'on peut choisir pour concevoir un chatbot 
		}

        •	Description du cas d'un chatbot \index{chatbot} supervisé / à base d'intention et d'entités
            o	 On se concentre sur ces implémentations car on peut y contrôler les réponses (image de marque en jeu)
            o	 Classification \index{chatbot!classification} d'intention (règles, classification supervisée, ...)
            o	 Extraction d'entités \index{chatbot!ner} (règles, ner, ...)
            o	 Mapping des réponses sur la base du couple $(intention, enites)$
            o	 \todo{citation}
            
        •	Description du cas d'un chatbot \index{chatbot} non-supervisé / à base d'un modèle de langage
            o	 \todo{citation}

    %%%%%--------------------------------------------------------------------
    %%%%% Section 2.2:
    %%%%%--------------------------------------------------------------------
    \section{Les étapes usuelles de conception d'un chatbot}
		\todo[inline]{SECTION À RÉDIGER}

        Préambule : l'organisation peut bien entendu varier suivant les contextes, mais la description qui suit est représentative des organisation principales
        
        \todo[inline]{
        	a distinguer suivant l'approche statistiques et l'approche symbolique
        }

        \subsection{Définition des acteurs}
		\todo[inline]{
			Remarque Gautier 20/02/2023:
        	Vu le chaos du monde du travail concernant la définition du data scientist, et en quoi il est différent d'un data engineer, analyst, etc..., ce sera important que tu livres ta définition et ton point de vue sur ce qu'est un DS.
        	En fait on pourrait imaginer trouver des experts métiers et des chefs de projets qui connaissent l'IA. On peut même les y former (c'est une des approches qu'on suit souvent). Mais c'est juste pas pratique à faire.
			Je me demande, à la lecture de cette section, si le problème n'est pas plutôt un problème de division des compétences ici, plutôt que de acteurs. On divise les compétences (connaissance des algorithmes, des données, du métier, de l'organisation d'un projet), et c'est de cette division que naissent les différents acteurs d'un projet. Ca serait intéressant de trouver un exemple d'un chatbot conçu par une seule personne qui prend en charge touts les aspects.
		}
		\todo[inline]{
			reformuler cette section par "compétences nécessaires" et montrer qu'elles sont en générales réparties entre plusieurs acteurs
		}

            •	Data scientistes :
                o	Experts en IA
                o	Peu de connaissance métier, i.e. peu de regarde critique sur la pertinence des résultats (autre que statistique)
            
            •	Expert métier :
                o	Peu de connaissance en IA, i.e. nécessitent des formations
                o	Connaissance métier forte, i.e. peuvent décrire la pertinence d’un résultat
            
            •	Chef de projet
                o	Peu de connaissance en IA
                o	Peu de connaissance métier
                o	Connaissance du besoin (hypothèse non vérifiée car parfois ils ne savent pas ce qu’ils veulent dû à la méconnaissance des capacités de l’IA)

        \subsection{Cadrage du projet}

            •	Objectifs :
                o	Clarification du besoin,
                o	Définition du périmètre couvert (i.e. les fonctionnalités et réponses à proposer),

            •   Livrable : un cahier des charges
            
            \todo[inline]{
				Remarque Gautier 20/02/2023:
				La aussi, ça mérite presque une digression (et ton point de vue perso) sur les méthodes de travail et l'agilité en particulier. Le cahier des charges et la spécification ont l'avantage de contractualiser le travail à faire, et lorque le travail est très divisé c'est important. Mais dans la pratique, aujourd'hui tout le monde dit qu'il est Agile. hors, dans l'agilité, on n'est pas sensé avoir de contractualisation. Pourquoi en faire une ici ?
            }

        \subsection{Collecte des données}

            •	Souvent pas de données à disposition :
                o	En R\&D, "80\%" sur la recherche d’algo sur des données publiques, d'où le besoin de datascientists,
                o	En entreprise, "80\%" sur la gestion des données privées/spécifiques sur des algo connus, d'où le besoin d'experts métiers ;
    
            •	Risque de biais dans les données :
                o	Biais d’échantillon : la collecte ne représente pas la réalité,
                o	Biais de sélection : le trie de la collecte ne représente plus la réalité,
                o	Biais de confirmation : on garde les données qui nous arrangent,
                o	Biais de valeur : les données ne sont pas éthiquement représentatives,
                o	Biais de contexte : les données d’un cas d’usage ne sont pas toujours réutilisables pour un autre cas d’usage (ex : différence entre les jargons des AV clients et celui des AV conseillers) ;
                o	\textbf{A COMPLETER}
            
            •   Livrable : une collecte de données brutes

        \subsection{Modélisation d’une structure et Labellisation des données}
            
            \todo[inline]{
				Remarque Gautier 20/02/2023:
				Au dela de ce que tu écris (avec lequel je suis d'accord), on a aussi un problème plus large. En choisissant une approche symbolique (cf mon commentaire plus haut), ça implique que la création et l'utilisation des chatbots fait se rencontrer deux mondes symboliques : 
				- le monde symbolique des experts travaillant dans le métier (i.e. les banquiers)
				- le monde symbolique des utilisateurs (i.e les clients)
				Il serait intéressant de discuter les raisons pour lesquels ces mondes symboliques peuvent converger (objectifs identiques et partagés, caractère humain...) et diverger (compétences et connaissances très inégales). Ca permet d'avoir un regard critique sur l'organisation du travail, et justement de prôner l'idée que l'on doit retirer le plus possible les facteurs de divergence durant la symbolisation de la connaissance.
			}

            •	Le coeur "métier" de la création du projet ;

            •	Objectif : Définition d'une modélisation sur la base des besoins attendus restreints au périmètre à couvrir ;

            •	En théorie :
                o	Intention: verbe d'actions,
                o	Entités: informations complémentaires, personnes, date, lieux, montants, noms de produits, ... ;

            •   Complexité de la tâche :
                o	Intention abstraite : définition difficile voir subjective, ...
                o	Annotation difficile :  différence entre théorie et pratique, données ambiguës, ...
                o	Plusieurs itérations car modélisation trop théorique / pas pratique
                o	Besoins de beaucoup de formation (pour donner la compétence aux experts) et d'atelier (pour se mettre d'accord)

            •   Livrable : un jeu de données annotées

        \subsection{Entraînement et tests}

            •	Le coeur "technique" de la création du projet ;

            •	Objectif : avoir un modèle qui soit adapté à son utilisation en production

            •	En théorie :
                o	Split en train et tests
                o	Entraînement et tests
                o	Association des réponses

            •   Complexité de la tâche :
                o	Modélisation précédente pas toujours adaptée : OK pour un métier, mais pas possible à entraîner à cause de déséquilibre, de manque de données, ...
                o	Algorithme fixe mais données variables : savoir quelle modélisation est la plus adaptée est compliqué à deviner
                o	Réponses pas toujours adaptées aux questions : décalage entre entraînement (modélisation théorique) et réponse (modélisation pratique)

        \subsection{Déploiement de la première version}

            •	RAS

            •	Parfois la modélisation est décalée par rapport à l'utilisation en production
                o	Comportement en moteur de recherche avec des questions courtes
                o	Vocabulaire non maitrisé par les utilisateurs
                o	problème d'ergo ou d'expérience utilisateur
            
            \todo[inline]{
				Remarque Gautier 20/02/2023:
				oui,cf mon commentaire plus haut sur la rencontre des mondes symboliques. C'est pour moi un désavantage de cette approche, et ça explique peut etre en partie le succès des approches non supervisées style ChatGPT
			}

        \subsection{Amélioration continue}

            •	Vérification du comportement ;

            •	Ajustement du modèle ;

            •	Déploiement des versions suivantes.
            
            \todo[inline]{
				Remarque Gautier 20/02/2023:
				Quels sont les objectifs de l'AC ? C'est seulement d'améliorer le tux de bonnes réponse ? Ou c'est plu large que ça ? (corriger les erreurs d'interprétation, faire converger les conceptions symboliques, éduquer les équipes, etc...)
			}

    %%%%%--------------------------------------------------------------------
    %%%%% Section 2.2:
    %%%%%--------------------------------------------------------------------
    \section{Zoom sur la partie Modélisation et Labellisation de la base d'apprentissage}
		\todo[inline]{SECTION À RÉDIGER}

        \subsection{Création « manuelle »}

            •	Enchaînement de plusieurs ateliers/cycles :
                o	Définition d’une structure en atelier et Annotation des données
                o	Premier conflit : La structure est trop théorique
                o	Redéfinition et Ré-annotation
                o	Second conflit : Les structure ou les données ne sont pas adaptées
                o	Collecte complémentaire, Redéfinition et Ré-annotation

            •	Avantages :
                o	Transmission progressive du savoir aux data scientists
                o	Test des modélisations potentielles

            •	Inconvénients :
                o	Nombreux ateliers
                o	Nombreuses remises en questions / aller-retour de conception
                o	L'avis initiale sur le périmètre à couvrir est flou quand cela concerne une centaine de demandes clients
                o	Se base sur de la connaissance que les experts métiers n’ont pas
                o	Comment les aider dans ce problème d’organisation ?

        \subsection{Création assistée par des regroupements non-supervisés}

            •	Constat :
                o	Pour des jeux de données à taille humaine (moins de 20.000 données), le premier tri est parfois "optimisé" manuellement sur la base des patterns commun (ordonnancement alphabétique)

            •	Solution :
                o	Un clustering pourrait simplifier cette tâche !
                \todo{clustering, topic modeling, ...}
                o   Rappel : grandes lignes du fonctionnement d'un algorithme de \gls{clustering} ?
                o	NB : une section ou une annexe détaillera les algorithmes de \index{clustering} les plus utilisés					o	KMeans : Classique, Incontournable, Rapide, Efficace
					o	Hiérarchique : Lent mais facile à implémenter
					o	Spectral : Permet des topologies complexes
					o	DBScan : Classique, Incontournable, Rapide, Efficace, Peu d'hyperparamètre
					o	Affinity propagation : 
					o	Metric learning : Lent mais plus adapté au corpus
					o	…

            •	Avantages :
                o	Regroupement automatique
                o	Découverte de la structure

            •	Inconvénients :
                o	Les résultats sont souvent peu pertinents
                o	Similarité par entités, et pas par intentions
                o	Nuances métiers non comprises
                o	Plusieurs soucis si le jeu de données est déséquilibré ou spécifique
                o	Absence d’un modèle de langue spécifique au contexte...
                o   parfois besoin d'hyperparamètres complexes à déterminer

        \subsection{Conception assistée par des regroupements semi-supervisés}

            •	Solution :
                o	On peut envisager ainsi de corriger le clustering en y insérant des contraintes métiers\hspace{2em}
                \cite{lampert-etal:2018:constrained-distance-based}
                o	Méthodes semi-supervisée
                o	NB : une section ou une annexe détaillera les algorithmes de clustering sous contraintes
					o	KMeans : Classique, Incontournable, Rapide, Efficace
					o	Hiérarchique : Lent mais facile à implémenter
					o	Spectral : Permet des topologies complexes
					o	DBScan : Classique, Incontournable, Rapide, Efficace, Peu d'hyperparamètre
					o	Affinity propagation : 
					o	Metric learning : Lent mais plus adapté au corpus
					o	…

            •   Interactions possibles avec le clustering (sur la base de proposition de l'humain)
                o	Sur les données / sur le résultat : ajouts de contraintes sur les données, suppressions ou modifications manuelles de données, réorganisation manuelles des clusters, …
                o	Sur les paramètres : modifier les hyper-paramètres, modifier le nombre de clusters, modifier les embeddings, utiliser d’autres algorithmes, …
                o	Besoin de visualisation : vue des contraintes, de la représentation vectorielle, …

            •	Avantage :
                o	On a réglé les problèmes de pertinence en ajoutant des contraintes

            •	Inconvénients : 
                o	Choisir comment modéliser ces contraintes peut être complexe
                o	Surtout énorme en ajoutant des contraintes
                o	Choisir les contraintes pertinentes est une tâche difficile

        \subsection{Conception basée sur des méthodes d’apprentissage actif}

            •	Solution :
                o	On peut demander à la machine de définir les contraintes dont elle a besoin pour s’améliorer / confirmer son comportement
                o	On peut séparer et cibler les tâches pour que le clustering se nourrissent des commentaires de l’expert et que l’expert corrige ce qui semble utile au clustering
                o	Sous-entendu : Préférer la collaboration à la supériorité (que ce soit celle de la machine ou celle de l’expert)
                o	NB : une section ou une annexe détaillera les interactions possibles entre homme et machine

            •   Interactions possibles avec le clustering (sur la base de propositions de la machine)
                o	Sur les données / sur le résultat : proposition de suppression de données aberrantes, proposition d'ajout de contraintes à des endroits stratégiques, …
                o	Sur les paramètres : réévaluation des paramètres, combiner plusieurs algorithmes et synthétiser le résultat, …

            •	Avantage :
                o	On a réglé les problèmes de pertinence et de coûts en ajoutant des contraintes

            •	Inconvénients / problème à résoudre : 
                o	Accepter de collaborer avec la machine (problème UX, ergo, accompagnement au changement)
                o	Il faut prouver cette méthode