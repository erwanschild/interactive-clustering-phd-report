\section[
		\texttt{cognitivefactory-features-maximization-metric}
	]{
		Implémentation de l'application web \\ \texttt{cognitivefactory-features-maximization-metric}
	}
\label{annex:C.3-DESCRIPTION-IMPLEMENTATION-FEATURES-MAXIMIZATION-METRIC}
	
	% Généralités.
	La librairie \texttt{cognitivefactory-features-maximization-metric} \footnote{
		\url{https://pypi.org/project/cognitivefactory-features-maximization-metric/}
	} (\cite{schild:2023:cognitivefactory-featuresmaximizationmetric}) a été implémentée au cours de ce doctorat pour pouvoir utiliser la \texttt{Maximisation des traits} (\textit{Features Maximization} notée \texttt{FMC}).
	Cette technique, proposée par \cite{lamirel-etal:2017:novel-approach-feature}, permet de sélectionner les composantes vectorielles pertinentes (\textit{features}) d'une base d'apprentissage.
	Nous allons détailler cette librairie comme suit :
	\begin{itemize}
		\item le calcul du score de \textit{Features F-Measure} associé à cette méthode (cf. \textsc{Section~\ref{annex:C.3.1-DESCRIPTION-IMPLEMENTATION-FEATURES-MAXIMIZATION-METRIC-CALCUL-FMEASURE}}) ;
		\item la sélection des composantes vectorielles pertinentes à l'aide du score de \textit{Features F-Measure} (cf. \textsc{Section~\ref{annex:C.3.2-DESCRIPTION-IMPLEMENTATION-FEATURES-MAXIMIZATION-METRIC-SELECTION-FEATURES}}) ;
		\item l'activation des composantes vectorielles pertinentes pour chaque classe de la base d'apprentissage (cf. \textsc{Section~\ref{annex:C.3.3-DESCRIPTION-IMPLEMENTATION-FEATURES-MAXIMIZATION-METRIC-ACTIVATION-FEATURES}}) ;
		\item l'application de cette méthode à l'analyse des patterns linguistiques pertinents d'une base d'apprentissage utilisée pour de la classification de textes en intentions (cf. \textsc{Section~\ref{annex:C.3.4-DESCRIPTION-IMPLEMENTATION-FEATURES-MAXIMIZATION-METRIC-APPLICATION-TEXTES}}).
	\end{itemize}
	
	% Information : comme y accéder.
	\begin{leftBarInformation}
		La documentation technique de cette librairie est accessible au lien suivant : \url{https://cognitivefactory.github.io/features-maximization-metric/}.
	\end{leftBarInformation}
	
	% Notations.
	Pour la suite de l'exposé, nous allons utiliser les notations suivantes :
	\begin{itemize}
		% Data.
		\item $\mathcal{D}$ représente l'ensemble des données $d$ de la base d'apprentissage,
		et $\texttt{vecteur}[d]$ représente la description vectorielle d'une donnée $d \in \mathcal{D}$ ;
		% Features.
		\item $\mathcal{F}$ représente l'ensemble des composantes vectorielles $f$ (\textit{features}),
		et $\texttt{vecteur}[d][f]$ représente le poids de la composante $f \in \mathcal{F}$ du vecteur décrivant la donnée $d \in \mathcal{D}$ ;
		% Classes.
		\item $\mathcal{C}$ représente l'ensemble des classes $c$ possibles de la base d'apprentissage,
		et $\texttt{classe}[d]$ représente la classe d'une donnée $d \in \mathcal{D}$ ;
	\end{itemize}
	
	
	%%%
	%%% Subsection C.3.1: Calcul du score de \texttt{F-Measure}.
	%%%
	\subsection{Calcul du score de \textit{Features F-Measure}}
	\label{annex:C.3.1-DESCRIPTION-IMPLEMENTATION-FEATURES-MAXIMIZATION-METRIC-CALCUL-FMEASURE}
	
		% Introduction.
		D'abord, il faut calculer le score de \textit{Features F-Measure} (\texttt{FM}) à partir de deux termes : la \textit{Features Recall} (\texttt{FR}) et la \textit{Features Predominance} (\texttt{FP}).
		\newline
		
		
		%%% FEATURES RECALL.
		
		% Description du calcul.
		La \textbf{\textit{Features Recall}} d'une composante vectorielle $f \in \mathcal{F}$ et d'une classe $c \in \mathcal{C}$, notée $\texttt{FR}[f][c]$, est un score compris entre $0$ et $1$ qui est obtenu par le ratio entre :
		\begin{itemize}
			\item la somme des poids des vecteurs pour la composante $f$ et pour les données de la classe $c$, et
			\item la somme des poids des vecteurs pour la composante $f$ et pour toutes les données.
		\end{itemize}
		
		% Equation.
		\begin{equation}
			\label{equation:C.3.1-DESCRIPTION-IMPLEMENTATION-FEATURES-MAXIMIZATION-METRIC-FEATURES-RECALL}
			\texttt{FR}[f][c]~=~\frac{
				\sum\limits_{
					\substack{
						d \in \mathcal{D} \\
						\texttt{classe}[d]=c
					}
				} \texttt{vector}[d][f]
			}{
				\sum\limits_{
					c' \in \mathcal{C}
				}
				\sum\limits_{
					\substack{
						d' \in \mathcal{D} \\
						\texttt{classe}[d']=c'
					}
				} \texttt{vector}[d'][f]
			}
		\end{equation}
		
		% Interprétation.
		\begin{leftBarAuthorOpinion}
			Ce score répond à la question :
			\textguillemets{\textit{est-ce que la \textit{feature} $f$ permet de distinguer la classe $c$ des autres classes $c'$ ?}}
		\end{leftBarAuthorOpinion}
		
		
		%%% FEATURES PREDOMINANCE.
		
		% Description du calcul.
		La \textbf{\textit{Features Predominance}} d'une composante vectorielle $f \in \mathcal{F}$ et d'une classe $c \in \mathcal{C}$, notée $\texttt{FP}[f][c]$, est un score compris entre $0$ et $1$ qui est obtenu par le ratio entre :
		\begin{itemize}
			\item la somme des poids des vecteurs pour la composante $f$ et pour les données de la classe $c$, et
			\item la somme des poids des vecteurs pour toutes les composantes et pour les données de la classe $c$.
		\end{itemize}
		
		% Equation.
		\begin{equation}
			\label{equation:C.3.1-DESCRIPTION-IMPLEMENTATION-FEATURES-MAXIMIZATION-METRIC-FEATURES-PREDOMINANCE}
			\texttt{FP}[f][c]~=~\frac{
				\sum\limits_{
					\substack{
						d \in \mathcal{D} \\
						\texttt{classe}[d]=c
					}
				} \texttt{vector}[d][f]
			}{
				\sum\limits_{
					f' \in \mathcal{F}
				}
				\sum\limits_{
					\substack{
						d \in \mathcal{D} \\
						\texttt{classe}[d]=c
					}
				} \texttt{vector}[d][f']
			}
		\end{equation}
		
		% Interprétation.
		\begin{leftBarAuthorOpinion}
			Ce score répond à la question :
			\textguillemets{\textit{est-ce que la composante $f$ identifie mieux la classe $c$ que les autres composantes $f'$ ?}}
		\end{leftBarAuthorOpinion}
		
		%%% FEATURES F-MEASURE.
		
		% Description du calcul.
		La \textbf{\textit{Features F-Measure}} d'une composante vectorielle $f \in \mathcal{F}$ et d'une classe $c \in \mathcal{C}$, notée $\texttt{FM}[f][c]$, est un score entre $0$ et $1$ qui est obtenu par la moyenne harmonique entre la \textit{Features Recall} et la \textit{Features Predominance}.
		
		% Equation.
		\begin{equation}
			\label{equation:C.3.1-DESCRIPTION-IMPLEMENTATION-FEATURES-MAXIMIZATION-METRIC-FEATURES-FMEASURE}
			\texttt{FM}[f][c]~=~2 \cdot \frac{
				\texttt{FR}[f][c] \cdot \texttt{FP}[f][c]
			}{
				\texttt{FR}[f][c] + \texttt{FP}[f][c]
			}
		\end{equation}
		
		% Interprétation.
		\begin{leftBarAuthorOpinion}
			Ce score répond à la question :
			\textguillemets{\textit{combien d'informations contient la \textit{feature} $f$ au sujet de la classe $c$ ?}}
		\end{leftBarAuthorOpinion}
	
	
	%%%
	%%% Subsection C.3.2: Sélection de \textit{features} à l'aide de la \texttt{F-Measure}.
	%%%
	\subsection{Sélection de \textit{features} à l'aide de la \texttt{F-Measure}}
	\label{annex:C.3.2-DESCRIPTION-IMPLEMENTATION-FEATURES-MAXIMIZATION-METRIC-SELECTION-FEATURES}
	
		% Introduction.
		L'objectif de cette seconde étape est de supprimer les composantes vectorielles qui n'apportent pas d'information de manière générale.
		Pour cela, un seuil de sélection est défini grâce à la moyenne globale des scores de \textit{Features F-Measure}.
		\newline
		
		%%% FEATURES FMEASURE OVERALL AVERAGE.
		
		% Description du calcul.
		La \textbf{\textit{Features Overall Average}}, notée $\overline{\overline{\texttt{FM}}}$, est un score entre $0$ et $1$ qui est obtenu par la moyenne de la \texttt{Features F-Measure} pour toutes les composantes et pour toutes les classes.
		
		% Equation.
		\begin{equation}
			\label{equation:C.3.2-DESCRIPTION-IMPLEMENTATION-FEATURES-MAXIMIZATION-METRIC-FEATURES-OVERALL-AVERAGE}
			\overline{\overline{\texttt{FM}}}~=~\frac{
				\sum\limits_{
					f \in \mathcal{F}
				}
				\sum\limits_{
					c \in \mathcal{C}
				} \texttt{FM}[f][c]
			}{
				|\mathcal{F}| \cdot |\mathcal{C}|
			}
		\end{equation}
		
		% Interprétation.
		\begin{leftBarAuthorOpinion}
			Ce seuil répond à la question :
			\textguillemets{\textit{quelle est la moyenne d'information contenue dans cette représentation vectorielle ?}}
		\end{leftBarAuthorOpinion}
		
		%%% FEATURES SELECTION.
		
		% Description du calcul.
		La \textbf{sélection de \textit{features}} se fait en comparant les valeurs de \texttt{Features F-Measures} à cette moyenne globale : si une composante $f \in \mathcal{F}$ a un score \texttt{FM}[f][c'] supérieur à la moyenne $\overline{\overline{\texttt{FM}}}$ pour au moins une classe $c' \in \mathcal{C}$, alors la composante $f$ est sélectionnée ; sinon, elle est supprimée.
		
		% Equation.
		\begin{equation}
			\label{equation:C.3.2-DESCRIPTION-IMPLEMENTATION-FEATURES-MAXIMIZATION-METRIC-FEATURES-SELECTION}
			\begin{array}{ll}
				\mathcal{F}_{\texttt{SELECTED}}
					& :=~\bigl\{~
						f \in \mathcal{F}
						~\big|~
						(\exists c' \in \mathcal{C})~\texttt{FM}[f][c'] \geq \overline{\overline{\texttt{FM}}}
					~\bigr\} \\
				\mathcal{F}_{\texttt{DELETED}}
					& :=~\bigl\{~
						f \in \mathcal{F}
						~\big|~
						(\forall c' \in \mathcal{C})~\texttt{FM}[f][c'] < \overline{\overline{\texttt{FM}}}
					~\bigr\}
			\end{array}
		\end{equation}
		
		% Interprétation.
		\begin{leftBarAuthorOpinion}
			Cette sélection répond à la question :
			\textguillemets{\textit{quelles sont les \textit{features} qui apportent plus d'information que la moyenne d'information contenue dans cette représentation vectorielle ?}}
		\end{leftBarAuthorOpinion}
	
	
	%%%
	%%% Subsection C.3.3: Activation des \textit{features} à l'aide de la \texttt{F-Measure}.
	%%%
	\subsection{Activation des \textit{features} à l'aide de la \texttt{F-Measure}}
	\label{annex:C.3.3-DESCRIPTION-IMPLEMENTATION-FEATURES-MAXIMIZATION-METRIC-ACTIVATION-FEATURES}
	
		% Introduction.
		L'objectif de cette dernière étape est de vérifier l'activation des composantes vectorielles pour chaque classe $c \in \mathcal{C}$.
		Pour cela, un seuil d'activation est défini pour chaque \textit{feature} $f \in \mathcal{F}$ grâce à la moyenne locale des scores de \textit{Features F-Measure}.
		\newline
		
		
		%%% FEATURES FMEASURE MARGINAL AVERAGE.
		
		% Description du calcul.
		La \textbf{\textit{Features Marginal Average}} d'une composante sélectionnée $f \in \mathcal{F}_{\texttt{SELECTED}}$, notée $\overline{\texttt{FM}[f]}$, est un score entre $0$ et $1$ qui est obtenu par la moyenne locale de la \texttt{Features F-Measure} pour la composante $f$ et pour toutes les classes.
		
		% Equation.
		\begin{equation}
			\label{equation:C.3.3-DESCRIPTION-IMPLEMENTATION-FEATURES-MAXIMIZATION-METRIC-FEATURES-MARGINAL-AVERAGE}
			\overline{\texttt{FM}[f]}~=~\frac{
				\sum\limits_{
					c' \in \mathcal{C}
				} \texttt{FM}[f][c']
			}{
				|\mathcal{F_{\texttt{SELECTED}}}|
			}
		\end{equation}
		
		% Interprétation.
		\begin{leftBarAuthorOpinion}
			Ce seuil répond à la question :
			\textguillemets{\textit{quelle est la moyenne d'information contenue par la feature $f$ dans cette représentation vectorielle ?}}
		\end{leftBarAuthorOpinion}
		
		%%% FEATURES ACTIVATION.
		
		% Description du calcul.
		L'\textbf{activation de \textit{features}} se fait en comparant les valeurs de \texttt{Features F-Measures} à cette moyenne locale : si une composante sélectionnée $f \in \mathcal{F}_{\texttt{SELECTED}}$ a un score \texttt{FM}[f][c] supérieur à la moyenne locale $\overline{\texttt{FM}[f]}$ pour une classe $c \in \mathcal{C}$, alors cette composante $f$ est activée pour cette classe $c$.
		
		% Equation.
		\begin{equation}
			\label{equation:C.3.3-DESCRIPTION-IMPLEMENTATION-FEATURES-MAXIMIZATION-METRIC-FEATURES-ACTIVATION}
			\begin{cases}
				f~\text{active}~c
					& \text{si}~\texttt{FM}[f][c] \geq \overline{\texttt{FM}[f]} \\
				f~\text{n'active pas}~c
					& \text{sinon}
			\end{cases}
		\end{equation}
		
		% Interprétation.
		\begin{leftBarAuthorOpinion}
			Cette activation répond à la question :
			\textguillemets{\textit{pour quelle(s) classe(s) la \textit{feature} $f$ est pertinente ?}}
		\end{leftBarAuthorOpinion}
	
	
	%%%
	%%% Subsection C.3.4: Application à l'analyse de la classification de textes.
	%%%
	\subsection{Application à l'analyse de la classification de textes}
	\label{annex:C.3.4-DESCRIPTION-IMPLEMENTATION-FEATURES-MAXIMIZATION-METRIC-APPLICATION-TEXTES}
	
		% Description.
		Plaçons-nous dans le cas d'une classification de textes en intentions :
		\begin{itemize}
			% Data.
			\item $\mathcal{D}$ représente l'ensemble des textes de la base d'apprentissage ;
			% Features.
			\item $\mathcal{F}$ représente l'ensemble des composantes vectorielles permettant de décrire les textes ;
			% Classes.
			\item $\mathcal{C}$ représente l'ensemble des intentions possibles de la base d'apprentissage.
		\end{itemize}
		
		% Application au TF-IDF
		Si nous utilisons une représentation vectorielle basée sur une description statistique du vocabulaire (comme \texttt{Bag of Words} (\cite{harris:1954:distributional-structure}) ou \texttt{TF-IDF} (\cite{ramos:2003:using-tfidf-determine})), nous pouvons alors déduire quels sont les mots du vocabulaire qui décrivent le mieux chaque intention grâce à la \texttt{Maximisation des traits}.
		Il est possible de compléter l'analyse en considérant que :
		\begin{itemize}
			\item un pattern linguistique $f$ est caractéristique d'une classe $c$ s'il ne s'active que pour cette classe. La liste des patterns linguistiques représentatifs d'une classe $c$ se notera $\mathcal{F}_{\texttt{ACTIVATED}}[c]$ ;
			\item un pattern linguistique $f$ est ambigu s'il s'active pour plusieurs classes.
		\end{itemize}
		
		% Equation.
		\begin{equation}
			\label{equation:C.3.4-DESCRIPTION-IMPLEMENTATION-FEATURES-MAXIMIZATION-METRIC-PATTERN-ACTIVATION}
			\mathcal{F}_{\texttt{ACTIVATED}}[c]
				~:=~\bigl\{~
					f \in \mathcal{F}_{\texttt{SELECTED}}
					~\big|~
					\texttt{FM}[f][c] \geq \overline{\texttt{FM}[f]}
				~\bigr\} \\
		\end{equation}
		
		% Référence à l'étude de pertinence.
		\begin{leftBarAuthorOpinion}
			Nous utilisons cette technique dans la \textsc{Section~\ref{section:4.4.2-ETUDE-PERTINENCE-PATTERNS-LINGUISTIQUES}} pour déterminer la pertinence sémantique d'un \textit{cluster} :
			\begin{itemize}
				\item en vérifiant la valeur métier du vocabulaire caractéristique du \textit{cluster} ;
				\item en vérifiant que les textes de ce \textit{cluster} contiennent peu de mots des vocabulaires caractéristiques des autres \textit{clusters}.
			\end{itemize}
		\end{leftBarAuthorOpinion}