%%
%% * DOCUMENT:
%%   - title: `PhD report on Interactive Clustering`
%%   - autor: Erwan SCHILD
%%   - date: 02/09/2022
%% * LATEX TEMPLATE:
%%   - title: thesul v0.15
%%   - autor: Denis ROEGEL
%%   - date: 30 mars 2013, update 12 mars 2022
%%   - url: [thesul](https://members.loria.fr/DRoegel/TeX/TUL.html)
%%

%%%%%%%%%%%%%%%%%%%%%%%%%%%%%%%%%%%%%%%%%%%%%%%%%%%%%%%%%%%%%%%%%%%%%%%
%%%%% PARAMÈTRES DU DOCUMENT
%%%%%%%%%%%%%%%%%%%%%%%%%%%%%%%%%%%%%%%%%%%%%%%%%%%%%%%%%%%%%%%%%%%%%%%

%%% Pour la classe du document:
\documentclass[11pt]{thesul}

%%% Pour déclarer un brouillon:
% \ThesisDraft

%%% Pour les vraies marges:
% \SetRealMargins{1mm}{1mm}


%%%%%--------------------------------------------------------------------
%%%%% Chargement des packages
%%%%%--------------------------------------------------------------------

%%% Pour un PDF avec des hyperliens:
\usepackage[pageanchor=false]{tulhypref}

%%% Pour une bibliographie au styme 'named':
% \usepackage{named}

%%% Pour les tableaux:
% \usepackage{array}
% \usepackage{multirow}

%%% Pour les figures:
\usepackage{graphicx}

%%% Pour les algorithmes:
\usepackage{algorithm}

%%% Pour les mini-tables des matières:
\usepackage[french]{minitoc}

%%% Pour le glossaire:
\usepackage{glossaries}


%%%%%--------------------------------------------------------------------
%%%%% Paramètres des en-têtes
%%%%%--------------------------------------------------------------------

%%% Exemples d'en-têtes:
% \UppercaseHeadings 
% \UnderlineHeadings
% \newcommand\bfheadings[1]{{\bf #1}}
% \FormatHeadingsWith{\bfheadings}
% \FormatHeadingsWith{\uppercase}
% \FormatHeadingsWith{\underline}

\newcommand\upun[1]{\uppercase{\underline{\underline{#1}}}}
\FormatHeadingsWith\upun
\newcommand\itheadings[1]{\textit{#1}}
\FormatHeadingsWith{\itheadings}

% Pour un trait sous une en-tete:
\setlength{\HeadRuleWidth}{0.4pt}


%%%%%--------------------------------------------------------------------
%%%%% Paramètres des référence, du glossaire et de l'index
%%%%%--------------------------------------------------------------------

%%% Enlever les numéro et préfix des chapitres:
\NoChapterNumberInRef
\NoChapterPrefix

%%% Faire un glossaire:
\makeglossary

%%% Faire un index:
\makeindex


%%%%%%%%%%%%%%%%%%%%%%%%%%%%%%%%%%%%%%%%%%%%%%%%%%%%%%%%%%%%%%%%%%%%%%%
%%%%% DÉBUT DU DOCUMENT
%%%%%%%%%%%%%%%%%%%%%%%%%%%%%%%%%%%%%%%%%%%%%%%%%%%%%%%%%%%%%%%%%%%%%%%

\begin{document}

%%% Pour définir le style de pages:
\OddHead={{\leftmark\rightmark}{\hfil\slshape\rightmark}}
\EvenHead={{\leftmark}{{\slshape\leftmark}\hfil}}
\OddFoot={\hfil\thepage}
\EvenFoot={\thepage\hfil}
\pagestyle{ThesisHeadingsII}

%%%%%--------------------------------------------------------------------
%%%%% Paramétrage de la table des matières (encadrement, numérotation, ...)
%%%%%--------------------------------------------------------------------

%%% Ecrire "Partie" dans la table des matières:
% \WritePartLabelInToc
%%% Ecrire "Chapitre" dans la table des matières:
% \WriteChapterLabelInToc
%%% Ajouter ou non le prochain titre à la table des matières:
% \WriteThisInToc
% \DontWriteThisInToc

%%% Ajouter une ligne ou un saut de page dans la table des matières:
% \PutLineInToc
% \PutNewPageInToc

%%% Pour encadrer encadre les parties dans la table des matières:
%%% (nb: ces commandes doivent figurer apres \begin{document})
% \FramePartsInToc
\DontFramePartsInToc
%%% Pour encadrer encadre les chapitres dans la table des matières:
%%% (nb: ces commandes doivent figurer apres \begin{document})
\FrameChaptersInToc
% \DontFrameChaptersInToc
%%% Encadrer ou non le prochain titre:
% \FrameThisInToc
% \DontFrameThisInToc

%%% Pour réinitialiser la numérotation des chapitres à chaque partie:
%%% (nb: ces commandes doivent figurer apres \begin{document})
\ResetChaptersAtParts
%%% Numéroter ou non le prochain titre dans la table des matières:
% \NumberThisInToc
% \DontNumberThisInToc

%%%%%--------------------------------------------------------------------
%%%%% Mini-table des matières à chaque chapitre
%%%%%--------------------------------------------------------------------

%%% Préparer les mini-tables des matières par chapitre (commande de minitoc.sty):
\dominitoc

%%% Créer une mini table du chapitre à cet endroit:
% \minitoc

%%%%%--------------------------------------------------------------------
%%%%% Page de titre
%%%%%--------------------------------------------------------------------

%%% Titre:
\ThesisTitle{
    Faciliter la conception d'un assistant conversationnel avec le clustering interactif
}

%%% Date:
\ThesisDate{09 setpembre 2022}

%%% Auteur:
\ThesisAuthor{Erwan SCHILD}

%%% Type de these:
\ThesisUL

%%% Membres du jury:
\President={
M. Le président de la soutenance
}
\Rapporteurs ={
Le rapporteur 1\\
Le rapporteur 2
}
\Examinateurs={
L'examinateur 1\\
L'examinateur 2
}
\Encadrants={
Dr. Jean-Charles LAMIREL\\
Dr. Florian MICONI
}

%%% Création de la page de titre:
\MakeThesisTitlePage

%%%%%--------------------------------------------------------------------
%%%%% Abstract
%%%%%--------------------------------------------------------------------

%%% (si le résumé apparaît sur une colonne étroite, avec la bibliographie à gauche, c'est sans doute parce que vous avez oublié de générer les fichiers d'index et de glossaire...)

\NumberAbstractPages
\begin{ThesisAbstract}

    %%% Abstract en français
    \begin{FrenchAbstract}
        Le résumé.
        \KeyWords{chat, chien, puces.}
    \end{FrenchAbstract}

    %%% Abstract en anglais
    \begin{EnglishAbstract}
        In computational geometry many search problems and range queries can be solved by performing an iterative search for the same key in separate ordered lists.  In Part I of this report we show that, if these ordered lists can be put in a one-to-one correspondence with the nodes of a graph of degree  d  so that the iterative search always proceeds along edges of that graph, then we can do much better than the obvious sequence of binary searches. Without expanding the storage by more than a constant factor, we can build a data-structure, called a fractional cascading structure, in which all original searches after the first can be carried out at only  log d  extra cost per search.  Several results related to the dynamization of this structure are also presented. Part II gives  numerous applications of this technique to geometric problems.
        Examples include intersecting a polygonal path with a line, slanted range search, orthogonal range search, computing locus functions, and others. Some results on the optimality of fractional cascading, and certain extensions of the technique for retrieving additional information are also included.
        \KeyWords{cat, dog, flees.}
    \end{EnglishAbstract}
\end{ThesisAbstract}

%%%%%--------------------------------------------------------------------
%%%%% Page de remerciements
%%%%%--------------------------------------------------------------------

% \WriteThisInToc
\begin{ThesisAcknowledgments}
    Les remerciements.\\
    Ma femme.\\
    Mes amis.\\
    Mon chien.\\
\end{ThesisAcknowledgments}

%%%%%--------------------------------------------------------------------
%%%%% Page de dédicaces
%%%%%--------------------------------------------------------------------

% \WriteThisInToc
\begin{ThesisDedication}
    Je dédie cette thèse\\
    à ma machine.\\
    Oui, à Pandore,\\
    qui fut la première de toutes.
\end{ThesisDedication}

%%%%%--------------------------------------------------------------------
%%%%% Table des matières
%%%%%--------------------------------------------------------------------

%%% Génération de la table des matières:
\tableofcontents

%%% Génère la liste des figures:
\listoffigures

%%% Génère la liste des tableaux:
\listoftables

%%%%%%%%%%%%%%%%%%%%%%%%%%%%%%%%%%%%%%%%%%%%%%%%%%%%%%%%%%%%%%%%%%%%%%%
%%%%% CORPS DU DOCUMENT
%%%%%%%%%%%%%%%%%%%%%%%%%%%%%%%%%%%%%%%%%%%%%%%%%%%%%%%%%%%%%%%%%%%%%%%

%%% Commencer la numérotation arabe (cf. '\pagenumbering{arabic}') avec la page 1 sur une page impaire:
\mainmatter

%%%%%--------------------------------------------------------------------
%%%%% Introduction
%%%%%--------------------------------------------------------------------
\chapter*{Introduction}
    \label{chapter:INTRODUCTION}
	\todo[inline]{CHAPITRE À REFORMULER FAÇON SWALES}

    %%%%%--------------------------------------------------------------------
    %%%%% Section I.1:
    %%%%%--------------------------------------------------------------------
    \section{"\textit{Asset centrality}"}
	\todo[inline]{SECTION À RÉDIGER}

    \begin{todolist}
    
        \item Des enjeux ou problèmes actuels

            •	Accessibilité à l'information :
                o	Grosses bases documentaires, pas toujours ordonnées ;

            •	Relations client à distance
                o   Besoin d'un accessibilité h24 ;

        \item Utilisation de plus en plus fréquente des chatbots
        
            •   Description succinte ;

            •   Cas d'usage usuels ;

            •   Tous les canaux d'utilisation ;

            •   Avantages et Dérives potentiels de l'utilisation (emploi, biais, pertinence, ergonomie, ...) ;

        \item Révolution techniques fréquentes (règles, classification, modèles)

            •	Moteurs de règles :
                o	Basé sur la détecté de mots clés,
                o	(+) facile à mettre en œuvre,
                o	(-) peu robuste au langage naturel,
                o	Paramétrage des réponses ;

            •	Paramétrage intentions-entités :
                o	Classification d’intention et/ou détection d’entités,
                o	(+) plus robuste au langage naturel, facile à paramétrer, réponses contrôlées,
                o	(-) demande de l’entrainement, des données, …,
                o	Paramétrage des réponses ;

            •	Génération de réponse :
                o	Réseau de neurones avec attention,
                o	Transformers,
                o	(+) plus robuste,
                o	(-) plus complexe à mettre en œuvre, réponses non contrôlées,
                o	Réponses non paramétrées ;

            •	Approche hybride :
                o   Cumul des trois approches pour cumuler certains avantages suivant les besoins ;
	\end{todolist}

    %%%%%--------------------------------------------------------------------
    %%%%% Section I.2:
    %%%%%--------------------------------------------------------------------
	
	\section{"\textit{Estabilishing a Niche}"}
	\todo[inline]{SECTION À RÉDIGER}

    \begin{todolist}

        \item Cadre industriel

            •	Algorithme fixe

            •	Données spécifiques

        \item GAP: Besoins de données

            •	Collecte de données spécifiqus au domaine traité :
                o   extraction de base de données (solution simple),
                o   collecte manuelle (organisation complexe, biais de collecte),
                o   scraping (pas toujours fiable) ;

            •	Nombreux biais :
                o   Biais,,
                o   Réglementation,
                o   Compétences (NOTRE COEUR DU SUJET),
                o   ...
    \end{todolist}

    %%%%%--------------------------------------------------------------------
    %%%%% Section I.3:
    %%%%%--------------------------------------------------------------------
	
	\section{"\textit{Occupying the Niche}"}
	\todo[inline]{SECTION À RÉDIGER}

    \begin{todolist}
        \item Etude de l'organisation d'une entreprise pour concevoir ses jeux de données
        \item Etude de l'état de l'art pour concevoir des jeux de données
        \item proposition/contribution : une méthode adaptée pour un cadre industriel
    \end{todolist}

%%%%%--------------------------------------------------------------------
%%%%% Chapitre 1Table des matières
%%%%%--------------------------------------------------------------------
\chapter{État de l'art}
    \label{CHAP_1_ETAT_DE_L_ART}

    \minitoc

    \section{Dans la fiction}

        \subsection{Rêvons un peu...}
            Une « autre » page avec « plein » de texte « et » très varié.
            Une autre page avec plein de texte très varié. \cite{ibm:2016, rasa:2017, wagstaff:2000}
            Une autre page avec plein de texte très varié. 

        \subsection{Dans les films...}
            Une « autre » page avec « plein » de texte « et » très varié.
            Une autre page avec plein de texte très varié .
            Une autre page avec plein de texte très varié.

        \subsection{Dans les comics...}
            Un \gls{svm4} avec « plein » de texte « et » très varié.
            Une autre \gls{svm4} avec plein de texte très varié. \cite{gancarski:2007} et \cite{lampert:2019}
            Une autre \gls{svm4} avec plein de texte très varié.

    \section{Dans la recherche}

        \subsection{Depuis 1970...}
            Une page de \gls{comics} avec plein de texte très varié.
            Une autre page de \gls{comics}  avec plein de texte très varié.

        \subsection{Depuis 1980...}
            Les musiques des années \gls{1980} avec plein de texte très varié.
            Les musiques des années \gls{1980} avec plein de texte très varié.

        \subsection{Depuis 1990...}
            Les musiques des années \gls{1990} avec plein de texte très varié.
            Les musiques des années \gls{1990} avec plein de texte très varié.

            \begin{figure}[htb]
                \includegraphics{tulul}
                \caption{xxxxxxxxx xxxxxxx\label{rintintin1}}
            \end{figure}

        \subsection{Depuis 2010...}
            Une autre page avec plein de texte très varié.
            Une autre page avec plein de texte très varié.

            \begin{figure}[htb]
                \includegraphics{tulloria}
                \caption{yyyyy yyyyy yyyy\label{rintintin2}}
            \end{figure}

        \subsection{Depuis 2022...}
            Une autre page avec plein de texte très varié.
            Une autre page avec plein de texte très varié.

    \section{Dans l'industrie}

        \subsection{Avec XXXXX...}
            Une autre page avec plein de texte très varié.
            Une autre page avec plein de texte très varié.

        \subsection{Avec YYYY...}
            Une autre page avec plein de texte très varié.
            Une autre page avec plein de texte très varié.

        \subsection{Depuis ZZZZZ...}

            \begin{itemize}
                \item Une autre page avec plein de texte très varié.
                \item Une autre page avec plein de texte très varié.
                    \begin{itemize}
                        \item Une autre page.
                        \item Une autre page.
                    \end{itemize}
                \item Une autre page avec plein de texte très varié.
            \end{itemize}

%%%%%--------------------------------------------------------------------
%%%%% Chapitre 2
%%%%%--------------------------------------------------------------------
\chapter{Travaux}

    \label{CHAP_2_TRAVAUX}

    \minitoc

    \section{Dans la fiction}

        \subsection{Rêvons un peu...}
            Une « autre » page avec « plein » de texte « et » très varié.
            Une autre page avec plein de texte très varié .
            Une autre page avec plein de texte très varié.

        \subsection{Dans les films...}
            Une « autre » page avec « plein » de texte « et » très varié.
            Une autre page avec plein de texte très varié .
            Une autre page avec plein de texte très varié.

        \subsection{Dans les comics...}
            Une « autre » page avec « plein » de texte « et » très varié.
            Une autre page avec plein de texte très varié .
            Une autre page avec plein de texte très varié.

    \section{Dans la recherche}

        \subsection{Depuis 1980...}
            Une autre page avec plein de texte très varié.
            Une autre page avec plein de texte très varié.

        \subsection{Depuis 1990...}
            Une autre page avec plein de texte très varié.
            Une autre page avec plein de texte très varié.

        \subsection{Depuis 2000...}
            Une autre page avec plein de texte très varié.
            Une autre page avec plein de texte très varié.

            \begin{table}[htbp]
                \center\small
                \begin{tabular*}{0.8\linewidth}{cc}
                    \hline
                    \hline
                    INTENTION 
                        & EXEMPLE
                    \tabularnewline
                    \hline
                    \textit{carte avalée}
                        & Ma carte a été avalée par un distributeur !
                        \tabularnewline
                    \textit{commande de carte}
                        & Est-ce que je peux avoir une carte sans frais ?
                        \tabularnewline
                    \textit{consultation du solde}
                        & Comment savoir si je suis à découvert ?
                        \tabularnewline
                \end{tabular*}
                \caption{Exemple de questions pour chaque intention du jeu de données utilisé.}
                \label{TABLEAU_DATASET_TUTO_CARTE_BANCAIRE_V1}
            \end{table}

        \subsection{Depuis 2010...}
            Une autre page avec plein de texte très varié.
            Une autre page avec plein de texte très varié.

        \subsection{Depuis 2022...}
            Une autre page avec plein de texte très varié.
            Une autre page avec plein de texte très varié.

    \section{Dans l'industrie}
    \index{industrie}

        \subsection{Avec XXXXX...}
            \index{industrie!XXXXX}
            Une autre page avec plein de texte très varié.
            Une autre page avec plein de texte très varié.

        \subsection{Avec YYYY...}
        \index{industrie!YYYY}
            Une autre page avec plein de texte très varié.
            Une autre page avec plein de texte très varié.

        \subsection{Depuis ZZZZZ...}
        \index{industrie!ZZZZZ}
            Une autre page avec plein de texte très varié.
            Une autre page avec plein de texte très varié.

%%%%%--------------------------------------------------------------------
%%%%% Conclusions
%%%%%--------------------------------------------------------------------
\chapter*{Conclusion}

    \label{CONCLUSION}

    Une autre page avec plein de texte très varié.
    Une autre page avec plein de texte très varié.
    Une autre page avec plein de texte très varié.
    Une autre page avec plein de texte très varié.
    Une autre page avec plein de texte très varié.
    Une autre page avec plein de texte très varié.
    Une autre page avec plein de texte très varié.
    Une autre page avec plein de texte très varié.
    Une autre page avec plein de texte très varié.
    Une autre page avec plein de texte très varié.
    Une autre page avec plein de texte très varié.
    Une autre page avec plein de texte très varié.

    Une autre page avec plein de texte très varié.
    Une autre page avec plein de texte très varié.
    Une autre page avec plein de texte très varié.
    Une autre page avec plein de texte très varié.
    Une autre page avec plein de texte très varié.
    Une autre page avec plein de texte très varié.
    Une autre page avec plein de texte très varié.
    Une autre page avec plein de texte très varié.
    Une autre page avec plein de texte très varié.
    Une autre page avec plein de texte très varié.
    Une autre page avec plein de texte très varié.
    Une autre page avec plein de texte très varié.
    Une autre page avec plein de texte très varié.
    Une autre page avec plein de texte très varié.


    Une autre page avec plein de texte très varié.
    Une autre page avec plein de texte très varié.
    Une autre page avec plein de texte très varié.
    Une autre page avec plein de texte très varié.
    Une autre page avec plein de texte très varié.
    Une autre page avec plein de texte très varié.

%%%%%%%%%%%%%%%%%%%%%%%%%%%%%%%%%%%%%%%%%%%%%%%%%%%%%%%%%%%%%%%%%%%%%%%
%%%%% ANNEXES DU DOCUMENT
%%%%%%%%%%%%%%%%%%%%%%%%%%%%%%%%%%%%%%%%%%%%%%%%%%%%%%%%%%%%%%%%%%%%%%%

%%% Préparer les annexes et la bibliographie:
\PutLineInToc
% \PutNewPageInToc
\DontFrameChaptersInToc

%%%%%--------------------------------------------------------------------
%%%%% Annexes
%%%%%--------------------------------------------------------------------
\label{ANNEXES}
\Annexes

\Annex{Annexe théorique}

    \section{Les algorithmes de clustering }

        \subsection{Kmeans}
        \subsection{Hierarchique}
        \subsection{Spectral}
        \subsection{DBScan}
        \subsection{Affinity Propagation}

    \section{Evaluation d’une clustering (??ANNEXE??)}

        \subsection{Homogénéité – Complétude – Vmeasure}
        \subsection{FMC}

\Annex{Annexe technique}

    \section{package pypi interactive-clustering}
    \section{package pypi interactive-clustering-gui}
    \section{package pypi features-maximization-metrics}
    \section{experimentations jupyter notebook}

\Annex{Datasets}

    \section{french bank cards}
    \section{DNA press title}

%%%%%--------------------------------------------------------------------
%%%%% Bibliographie
%%%%%--------------------------------------------------------------------
%%% Définition de la bibliographie:
\bibliographystyle{apalike}
\bibliography{references/bibliography}

%%%%%--------------------------------------------------------------------
%%%%% glossaire
%%%%%--------------------------------------------------------------------
%%% Définition du glossaire:
\newglossaryentry{comics}{
    name=comics,
    description={une bande dessinée à parution régulière}}
\newglossaryentry{1980}{
    name={Années 1980},
    description={une decénie de musiques chouettes}}
\newglossaryentry{1990}{
    name={Années 1990},
    description={une decénie de musiques discutables}}

%%% Définition des accronymes:
\newacronym{svm4}{SVM}{Support Vector Machine}

\WriteThisInToc
\printglossaries

%%%%%--------------------------------------------------------------------
%%%%% Index
%%%%%--------------------------------------------------------------------
\WriteThisInToc
\printindex

\end{document}