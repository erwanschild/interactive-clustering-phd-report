%% * DOCUMENT:
%%   - title: `PhD report on Interactive Clustering`
%%   - autor: Erwan SCHILD
%%   - date: 02/09/2022
%%   - url: [erwanschild/interactive-clustering-phd-report](https://github.com/erwanschild/interactive-clustering-phd-report)
%% * LATEX TEMPLATE:
%%   - title: thesul v0.15
%%   - autor: Denis ROEGEL
%%   - date: 30 mars 2013, update 12 mars 2022
%%   - url: [thesul](https://members.loria.fr/DRoegel/TeX/TUL.html)

%%%%%%%%%%%%%%%%%%%%%%%%%%%%%%%%%%%%%%%%%%%%%%%%%%%%%%%%%%%%%%%%%%%%%%%
%%%%% PARAMÈTRES DU DOCUMENT
%%%%%%%%%%%%%%%%%%%%%%%%%%%%%%%%%%%%%%%%%%%%%%%%%%%%%%%%%%%%%%%%%%%%%%%

%%% Pour la classe du document:
\documentclass[11pt]{template/thesul}

%%% Pour déclarer un brouillon:
% \ThesisDraft

%%% Pour les vraies marges:
% \SetRealMargins{1mm}{1mm}

%%% Magie noire..
\usepackage{morewrites}  % for "! No room for a new \write."
%\morewritessetup{ allocate = 32 }

%%%%%--------------------------------------------------------------------
%%%%% Chargement des packages
%%%%%--------------------------------------------------------------------

%%% Pour la langue:
\usepackage[french]{babel}
\usepackage[T1]{fontenc}
\usepackage [utf8]{inputenc}
\usepackage{xspace}
\usepackage{lettrine}

%%% Pour un PDF avec des hyperliens:
\usepackage[pageanchor=true]{template/tulhypref}

%%% Pour une bibliographie au styme 'named':
% \usepackage{named}

%%% Pour les notes à faire:
\usepackage{todonotes}

%%% Pour les styles:
\usepackage{
	amssymb,
	colortbl,
	pifont,
	xcolor,  % http://latexcolor.com/
}
\usepackage{
	soul,
}

\definecolor{colorDarkPastelRed}{HTML}{C23B21}  % Dark Pastel Red (#C23B21)
\definecolor{colorCarrotOrange}{HTML}{F39A27}  % Carrot Orange (#F39A27)
\definecolor{colorMinionYellow}{HTML}{EADA52}  % Minion Yellow (#EADA52)
\definecolor{colorDarkPastelGreen}{HTML}{03C03C}  % Dark Pastel Green (#03C03C)
\definecolor{colorSilverLakeBlue}{HTML}{579ABE}  % Silver Lake Blue (#579ABE)
\definecolor{colorDarkPastelPurple}{HTML}{976ED7}  % Dark Pastel Purple (#976ED7)
\definecolor{colorPastelBrown}{HTML}{836953}  % Pastel Brown (#836953)
\definecolor{colorDimGray}{HTML}{737373}  % Dim Gray (#737373)
\definecolor{colorBlack}{HTML}{000000}  % Black (#000000)

\definecolor{colorApplicationMUSTLINK}{HTML}{2D8817}
\definecolor{colorApplicationCANNOTLINK}{HTML}{942E1C}
\definecolor{colorApplicationDELETE}{HTML}{C06000}
\definecolor{colorApplicationSKIP}{HTML}{1BD5E6}
\definecolor{colorApplicationREVIEW}{HTML}{0075FF}
\definecolor{colorApplicationNOTAVAILABLE}{HTML}{808080}
\definecolor{colorApplicationAVAILABLE}{HTML}{10A549}
\definecolor{colorApplicationWORKING}{HTML}{1BD5E6}
\definecolor{colorApplicationDONE}{HTML}{57B17B}
\definecolor{colorApplicationERROR}{HTML}{FF0000}


%%% Pour les symbols:
\usepackage{fontawesome5}  % https://www.ipgp.fr/~moguilny/LaTeX/fontawesome5Icons.pdf
	% Signet: \faBookmark \faBook
	% Warning: \faExclamation \faExclamationTriangle
	% Commentaire personnel: \faCommentDots
	% Idées: \faLightbulb
	% Check: \faCheckSquare
	% MUST-LINK: \faCheck \faCheckCircle \faEquals
	% CANNOT-LINK: \faTimes \faTimesCircle \faNotEqual
	% Information, Questions ou Plus d'info: \faInfoCircle \faQuestionCircle % \faPlusCircle
	% Reminder: \faBell
	% Eprouvette: \faVial
	% Github: \faGit \faGithub
	% Exemples: \faEye \faSearchPlus \faSearch
	% Divers : \faCheck, \faQuestion, \faExclamation, \faPen, \faCheckSquare, \faTrash, \faDownload, \faAngleLeft, \faAngleRight, \faHome, \faList, \faCog, \faEquals, \faNotEqual
	
%%% Pour les guillemets:
\newcommand{\textguillemets}[1]{\og #1 \fg}

%%% Pour les listes:
\usepackage{enumitem}
\setlist{topsep=6pt, itemsep=6pt}
\newlist{todolist}{itemize}{2}
\setlist[todolist]{label=$\square$}
\newcommand{\cmark}{\ding{51}}
\newcommand{\xmark}{\ding{55}}
\newcommand{\itemok}{\rlap{$\square$}{\raisebox{2pt}{\large\hspace{1pt}\cmark}}\hspace{-2.5pt}}
\newcommand{\itemko}{\rlap{$\square$}{\large\hspace{1pt}\xmark}}

%%% Pour les tableaux:
\usepackage{array}
%\usepackage{arydshln}
\usepackage{hhline}
\usepackage{multirow}
\definecolor{colorTableCaption}{HTML}{737373}  % Dim Gray (#737373)
\definecolor{colorTableHeader}{HTML}{737373}  % Dim Gray (#737373)
\usepackage[
	labelsep=endash,
	font={color=colorTableCaption},
	labelfont={sc, bf},
	textfont={it},
]{caption}

%%% Pour les figures:
\usepackage{float}
\usepackage{graphicx}

%%% Pour les schémas:
\usepackage{tikz}
\usetikzlibrary{
	arrows,
	shapes,
	automata,
	petri,
	positioning,
	calc
}


%%% Pour les encadrés:
\usepackage{tcolorbox}
\definecolor{colorTcolorboxHypothesis}{HTML}{87A86B}  % Asparagus (#87A86B)

%%% Pour les paragraphes indentés:
\usepackage{framed}  % provide \leftbar

% Renommer la commande pour accepter la gestion des couleurs.
\renewenvironment{leftbar}[2][15]
{
    \def\FrameCommand
    {
        {\color{#2}\vrule width 3pt}
        \hspace{0pt}
        \fboxsep=\FrameSep\colorbox{#2!#1}
    }
    \MakeFramed{\hsize\hsize\advance\hsize-\width\FrameRestore}
}
{\endMakeFramed}

% Définition des encadrés importants.
\definecolor{colorLeftBarImportantGreen}{HTML}{03C03C}  % Dark Pastel Green (#03C03C)
\newenvironment{leftBarImportantGreen}{
	\begin{leftbar}{colorLeftBarImportantGreen}
	\noindent
}{
    \end{leftbar}
}  % \begin{leftBarImportantGreen} \lipsum[1] \end{leftBarImportantGreen}
\definecolor{colorLeftBarImportantRed}{HTML}{C23B21}  % Dark Pastel Red (#C23B21)
\newenvironment{leftBarImportantRed}{
	\begin{leftbar}{colorLeftBarImportantRed}
	\noindent
}{
    \end{leftbar}
}  % \begin{leftBarImportantRed} \lipsum[1] \end{leftBarImportantRed}
\definecolor{colorLeftBarImportantGray}{HTML}{737373}  % Dim Gray (#737373)
\newenvironment{leftBarImportantGray}{
	\begin{leftbar}{colorLeftBarImportantGray}
	\noindent
}{
    \end{leftbar}
}  % \begin{leftBarImportantGray} \lipsum[1] \end{leftBarImportantGray}

% Définition de l'encadré d'attention.
\definecolor{colorLeftBarWarning}{HTML}{F39A27}  % Carrot Orange (#F39A27)
\newenvironment{leftBarWarning}{
	\begin{leftbar}{colorLeftBarWarning}
	\noindent
	\textbf{
		\textcolor{colorLeftBarWarning}{\faExclamationTriangle}
		~Attention :
	}
}{
    \end{leftbar}
}  % \begin{leftBarWarning} \lipsum[1] \end{leftBarWarning}

% Définition de l'encadré des points à retenir.
\definecolor{colorLeftBarSummary}{HTML}{03C03C}  % Dark Pastel Green (#03C03C)
\newenvironment{leftBarSummary}{
	\begin{leftbar}{colorLeftBarSummary}
	\noindent
	\textbf{
		\textcolor{colorLeftBarSummary}{\faBookmark}
		~Points à retenir :
	}
}{
    \end{leftbar}
}  % \begin{leftBarSummary} \lipsum[1] \end{leftBarSummary}

% Définition de l'encadré d'idée.
\definecolor{colorLeftBarIdea}{HTML}{EADA52}  % Minion Yellow (#EADA52)
\newenvironment{leftBarIdea}{
	\begin{leftbar}{colorLeftBarIdea}
	\noindent
	\textbf{
		\textcolor{colorLeftBarIdea}{\faLightbulb}
		~Idées :
	}
}{
    \end{leftbar}
}  % \begin{leftBarIdea} \lipsum[1] \end{leftBarIdea}

% Définition de l'encadré de notes de l'auteur.
\definecolor{colorLeftBarAuthorOpinion}{HTML}{579ABE}  % Silver Lake Blue (#579ABE)
\newenvironment{leftBarAuthorOpinion}{
	\begin{leftbar}{colorLeftBarAuthorOpinion}
	\noindent
	\textbf{
		\textcolor{colorLeftBarAuthorOpinion}{\faCommentDots}
		~Notes de l'auteur :
	}
}{
    \end{leftbar}
}  % \begin{leftBarAuthorOpinion} \lipsum[1] \end{leftBarAuthorOpinion}

% Définition de l'encadré d'informations
\definecolor{colorLeftBarInformation}{HTML}{737373}  % Dim Gray (#737373)
\newenvironment{leftBarInformation}{
	\begin{leftbar}{colorLeftBarInformation}
	\noindent
	\textbf{
		\textcolor{colorLeftBarInformation}{\faInfoCircle}
		~Pour information :
	}
}{
    \end{leftbar}
}  % \begin{leftBarInformation} \lipsum[1] \end{leftBarInformation}

% Définition de l'encadré de rappel
\definecolor{colorLeftBarReminder}{HTML}{737373}  % Dim Gray (#737373)
\newenvironment{leftBarReminder}{
	\begin{leftbar}{colorLeftBarReminder}
	\noindent
	\textbf{
		\textcolor{colorLeftBarReminder}{\faBell}
		~Rappel :
	}
}{
    \end{leftbar}
}  % \begin{leftBarReminder} \lipsum[1] \end{leftBarReminder}

% Définition de l'encadré d'exemples
\definecolor{colorLeftBarExamples}{HTML}{976ED7}  % Dark Pastel Purple (#976ED7)
\newenvironment{leftBarExamples}{
	\begin{leftbar}{colorLeftBarExamples}
	\noindent
	\textbf{
		\textcolor{colorLeftBarExamples}{\faSearch}
		Exemples :
	}
}{
    \end{leftbar}
}  % \begin{leftBarExamples} \lipsum[1] \end{leftBarExamples}

% Définition d'un compteur pour les notes de bas de page dans les encadrés (footnote)
\newcounter{localCounterOfFootnoteValue}
	% \setcounter{localCounterOfFootnoteValue}{\value{footnote}}  % initailiser le compteur local au compteur de 'footnote' actuel
	% \footnotemark  % note A, compteur 'footnote' augmenté de un
	% \footnotemark  % note B, compteur 'footnote' augmenté de un
	% \stepcounter{localCounterOfFootnoteValue}  % augmenter le compteur local de 1.
	% \footnotetext[\value{localCounterOfFootnoteValue}]{ texte de la 'footnotemark' A}
	% \stepcounter{localCounterOfFootnoteValue}  % augmenter le compteur local de 1.
	% \footnotetext[\value{localCounterOfFootnoteValue}]{ texte de la 'footnotemark' B}

%%% Pour les algorithmes:
\usepackage{algpseudocode}
\usepackage[
	vlined,
	boxed,
	algochapter,
	linesnumbered,
	french,
	onelanguage,
]{algorithm2e} % algorithm
\SetAlgoCaptionSeparator{--}
\SetAlCapFnt{ \bfseries\scshape }
\definecolor{colorAlgoCaption}{HTML}{737373}  % Dim Gray (#737373)
\renewcommand\AlCapSty{ \color{colorAlgoCaption} }

%%% Pour le code Python:
\usepackage{listings}
\definecolor{colorCodeString}{HTML}{990000}  % Dark red (#990000)
\definecolor{colorCodeComment}{HTML}{008000}  % Dark Green (#008000)
\definecolor{colorCodeKeyword}{HTML}{0000B3}  % Medium Blue (#0000B3)
\definecolor{colorCodeEmphasize}{HTML}{F72673}  % Deep Pink (#F72673)
\definecolor{colorCodeBackground}{HTML}{91A3B0}  % Cadet grey (#91A3B0)
\renewcommand{\lstlistingname}{Code}
\lstset{
  inputencoding=utf8,
  breaklines=true,
  captionpos=b,
  escapeinside={\%*}{*)},
  frame=lines,
  numbers=left,
  showstringspaces=false,
  morekeywords=[1]{,as,assert,nonlocal,with,yield,self,True,False,None,}
  basicstyle=\tiny,%\ttfamily,
  backgroundcolor=\color{colorCodeBackground!15},
  commentstyle=\color{colorCodeComment},
  keywordstyle=\color{colorCodeKeyword}\bfseries,
  stringstyle=\color{colorCodeString},
  emphstyle=\color{colorCodeEmphasize}\underbar,
  tabsize=2,
  literate=
    {é}{{\'e}}{1}%
    {è}{{\`e}}{1}%
	{à}{{\`a}}{1}%
}

%%% Pour les mini-tables des matières:
\usepackage[french]{minitoc}

%%% Pour les références bibliographiques:
\usepackage[style=apa, natbib=true, backend=biber]{biblatex}
\addbibresource{references/bibliography.bib}
\usepackage{csquotes}

%%% Pour le glossaire:
\usepackage[automake]{glossaries}

%%% Pour le brouillon:
\usepackage{lipsum}

%%% Pour les partitions:
\usepackage{guitar}

%%%%%--------------------------------------------------------------------
%%%%% Paramètres des en-têtes
%%%%%--------------------------------------------------------------------

%%% Exemples d'en-têtes:
% \UppercaseHeadings 
% \UnderlineHeadings
% \newcommand\bfheadings[1]{{\bf #1}}
% \FormatHeadingsWith{\bfheadings}
% \FormatHeadingsWith{\uppercase}
% \FormatHeadingsWith{\underline}

\newcommand\upun[1]{\uppercase{\underline{\underline{#1}}}}
\FormatHeadingsWith\upun
\newcommand\itheadings[1]{\textit{#1}}
\FormatHeadingsWith{\itheadings}

% Pour un trait sous une en-tete:
\setlength{\HeadRuleWidth}{0.4pt}


%%%%%--------------------------------------------------------------------
%%%%% Paramètres des référence, du glossaire et de l'index
%%%%%--------------------------------------------------------------------

%%% Enlever les numéro et préfix des chapitres:
\NoChapterNumberInRef
\NoChapterPrefix

%%% Faire un glossaire:
\makeglossaries

%%% Définition du glossaire:
%%% Definition des termes du glossaire:
\newglossaryentry{clustering}{
    name={clustering},
    description={!!TODO!!}
}
\newglossaryentry{contrainte}{
    name={contrainte},
    description={!!TODO!!}
}
\newglossaryentry{mustlink}{
    name={MUST-LINK},
    description={!!TODO!!}
	parent=contrainte
}
\newglossaryentry{cannotlink}{
    name={CANNOT-LINK},
    description={!!TODO!!}
	parent=contrainte
}
\newglossaryentry{annotation}{
    name={annotation},
    description={!!TODO!!}
}
\newglossaryentry{vmeasure}{
    name={vmeasure},
    description={!!TODO!!}
}

%%% Définition des accronymes:
\newacronym{svm}{SVM}{Support Vector Machine}
\newacronym{fmc}{FMC}{Feature Maximizaton Contrast}

%%% Faire un index:
\makeindex


%%%%%%%%%%%%%%%%%%%%%%%%%%%%%%%%%%%%%%%%%%%%%%%%%%%%%%%%%%%%%%%%%%%%%%%
%%%%% DÉBUT DU DOCUMENT
%%%%%%%%%%%%%%%%%%%%%%%%%%%%%%%%%%%%%%%%%%%%%%%%%%%%%%%%%%%%%%%%%%%%%%%

\begin{document}

%%% Pour définir le style de pages:
\OddHead={{\leftmark\rightmark}{\hfil\slshape\rightmark}}
\EvenHead={{\leftmark}{{\slshape\leftmark}\hfil}}
\OddFoot={\hfil\thepage}
\EvenFoot={\thepage\hfil}
\pagestyle{ThesisHeadingsII}

%%% Pour aider à aligner le texte.
\emergencystretch=3em

%%%%%--------------------------------------------------------------------
%%%%% Paramétrage de la table des matières (encadrement, numérotation, ...)
%%%%%--------------------------------------------------------------------

%%% Ecrire "Partie" dans la table des matières:
% \WritePartLabelInToc
%%% Ecrire "Chapitre" dans la table des matières:
% \WriteChapterLabelInToc
%%% Ajouter ou non le prochain titre à la table des matières:
% \WriteThisInToc
% \DontWriteThisInToc

%%% Ajouter une ligne ou un saut de page dans la table des matières:
% \PutLineInToc
% \PutNewPageInToc

%%% Pour encadrer encadre les parties dans la table des matières:
%%% (nb: ces commandes doivent figurer apres \begin{document})
% \FramePartsInToc
\DontFramePartsInToc
%%% Pour encadrer encadre les chapitres dans la table des matières:
%%% (nb: ces commandes doivent figurer apres \begin{document})
\FrameChaptersInToc
% \DontFrameChaptersInToc
%%% Encadrer ou non le prochain titre:
% \FrameThisInToc
% \DontFrameThisInToc

%%% Pour réinitialiser la numérotation des chapitres à chaque partie:
%%% (nb: ces commandes doivent figurer apres \begin{document})
\ResetChaptersAtParts
%%% Numéroter ou non le prochain titre dans la table des matières:
% \NumberThisInToc
% \DontNumberThisInToc

%%% Pour numéroter les subsubsection avec une lettre en minuscule.
\setcounter{secnumdepth}{3}
\renewcommand{\thesubsubsection}{\thesubsection.\alph{subsubsection}}


%%%%%--------------------------------------------------------------------
%%%%% Mini-table des matières à chaque chapitre
%%%%%--------------------------------------------------------------------

%%% Préparer les mini-tables des matières par chapitre (commande de minitoc.sty):
\dominitoc

%%% Créer une mini table du chapitre à cet endroit:
% \minitoc

%%%%%--------------------------------------------------------------------
%%%%% Page de titre
%%%%%--------------------------------------------------------------------

%%% Titre:
\ThesisTitle{
	% Techniques de Valorisation de l'Expertise Humaine dans l'Annotation : \\
	% Application à la Modélisation de Textes en Intentions à l'aide d'un Clustering Interactif
	De l'Importance de Valoriser l'Expertise Humaine dans l'Annotation :
	Application à la Modélisation de Textes en Intentions à l'aide d'un Clustering Interactif
}

%%% Date:
\ThesisDate{
    20 décembre 2023
}  % 8 semaine après la date du rendu

%%% Auteur:
\ThesisAuthor{
    Erwan SCHILD
}

%%% Type de these:
\ThesisUL

%%% Membres du jury:
\President={
    Dr. Pascale KUNTZ-COSPEREC
}
\Rapporteurs={
	% rapporteurs = membre du jury évaluant le manuscrit
    Dr. Pascale KUNTZ-COSPEREC\\
    Dr. Thomas LAMPERT
}
\Examinateurs={
	% examinateurs = membre du jury présents à la soutenance
    Dr. Adrien COULET
}
\Encadrants={
    Dr. Jean-Charles LAMIREL\\
    Dr. Florian MICONI
}
\Invites={
    Dr. Gautier DURANTIN\\
    Dr. Mathieu POWALKA
}

%%% Création de la page de titre:
\MakeThesisTitlePage

%%%%%--------------------------------------------------------------------
%%%%% Abstract
%%%%%--------------------------------------------------------------------

%%% (si le résumé apparaît sur une colonne étroite, avec la bibliographie à gauche, c'est sans doute parce que vous avez oublié de générer les fichiers d'index et de glossaire...)

\NumberAbstractPages
\begin{ThesisAbstract}

	%%% Abstract en français
	\begin{FrenchAbstract}
		Le résumé à \st{faire}.
		\KeyWords{
			\texttt{Apprentissage automatique} ;
			\texttt{Traitement automatique du langage naturel} ;
			\texttt{Annotation de contraintes} ;
			\texttt{Clustering sous contraintes} ;
			\texttt{Clustering interactif} ;
			\texttt{Assistant conversationnel}.
		}
	\end{FrenchAbstract}
	
	%%% Abstract en anglais
	\begin{EnglishAbstract}
		The abstract to do
		\KeyWords{
			\texttt{Machine Learning} ;
			\texttt{Natural Language Processing} ;
			\texttt{Constraints annotation} ;
			\texttt{Constrained clustering} ;
			\texttt{Interactive clustering} ;
			\texttt{Chatbot}.
		}
	\end{EnglishAbstract}
\end{ThesisAbstract}


%%%%%--------------------------------------------------------------------
%%%%% Page de remerciements
%%%%%--------------------------------------------------------------------

% \WriteThisInToc
\begin{ThesisAcknowledgments}

	Par la présente, je souhaite remercier:
	% L'entreprise \texttt{Euro Information}, le laboratoire du \texttt{LORIA} et l'\texttt{ANRT} pour m'avoir permis de réaliser ce doctorat dans de bonnes conditions ;
	% Mes encadrants, Jean-Charles et Florian,
	% Mes collègues, notamment Amélie et Gautier,
	% Ma relectrice, s'en ki ce manuscry serez truffé de photes d'autaugraff !
	% Ma famille et mes amis qui m'ont \st{supporté} soutenu et encouragé tout au long de ces quatre années ;
	% Ma tortue,
	% Ma femme, qui a été un amour et une oreille attentive
	\begin{todolist}
		\item \st{test}
	\end{todolist}

\end{ThesisAcknowledgments}


%%%%%--------------------------------------------------------------------
%%%%% Page de dédicaces
%%%%%--------------------------------------------------------------------

% \WriteThisInToc
\begin{ThesisDedication}

	% Je dédie cette thèse
	% à quelqu'un de bien. \\
	% Si vous lisez cela, \\
	% alors vous êtes concerné !
	
	Je dédie cette thèse à Joséphine, notre tortue, \\
	qui, dans son enclos, faisait la même chose que moi \\
	pendant ma rédaction : tourner en rond...

\end{ThesisDedication}

%%%%%--------------------------------------------------------------------
%%%%% Table des matières
%%%%%--------------------------------------------------------------------

%%% Génération de la table des matières:
\renewcommand{\contentsname}{Table des matières}
\tableofcontents

%%%%%%%%%%%%%%%%%%%%%%%%%%%%%%%%%%%%%%%%%%%%%%%%%%%%%%%%%%%%%%%%%%%%%%%
%%%%% CORPS DU DOCUMENT
%%%%%%%%%%%%%%%%%%%%%%%%%%%%%%%%%%%%%%%%%%%%%%%%%%%%%%%%%%%%%%%%%%%%%%%

%%% Commencer la numérotation arabe (cf. '\pagenumbering{arabic}') avec la page 1 sur une page impaire:
\mainmatter

%%%%%--------------------------------------------------------------------
%%%%% Introduction
%%%%%--------------------------------------------------------------------
\chapter{Introduction}
\label{chapter:1-INTRODUCTION}


	%%%%%--------------------------------------------------------------------
	%%%%% ASSET CENTRALITY: Idéalisation de l'\texttt{IA} aux yeux du grand public.
	%%%%%--------------------------------------------------------------------
	\section*{Idéalisation de l'\texttt{IA} aux yeux du grand public}
	\addcontentsline{toc}{section}{
		\protect\numberline{}
		Idéalisation de l'\texttt{IA} aux yeux du grand public
	}
		
		%%% Démocratisation de l'utilisation d'IA.
		L'Intelligence Artificielle (\texttt{IA}) a connu une démocratisation massive ces dernières années.
		Elle est considérée comme une révolution majeure de notre société, à tel point qu'il devient presque impossible de s'en passer.
		\begin{itemize}
			\item Vous avez besoin de trouver votre chemin ? utilisez votre \texttt{GPS}.
			\item Vous avez un problème avec une commande ou besoin d'un service après-vente ? un bot informatique est disponible jour et nuit pour traiter votre demande.
			\item Vous ne savez plus quelle série regarder ? \texttt{Netflix} peut faire des suggestions personnalisées.
			\item Vous avez les mains pleines de farine et vous voulez lancer un minuteur ou écouter de la musique ? Demandez-le \texttt{OK Google} ou \texttt{Alexa}.
			\item Il vous manque une belle image pour votre présentation ? \texttt{DALL-E} peut la générer.
			\item Vous devez rédiger une dissertation en histoire-géographie ? \texttt{ChatGPT} s'en occupera.
			\item La classe \LaTeX{} proposée par votre école doctorale ne compile pas ? \texttt{ChatGPT} peut aussi identifier l'erreur et même la corriger...
		\end{itemize}
		
		%%% Mais des mythes se créent.
		Les modèles d'\texttt{IA} s'immiscent ainsi dans la plupart des activités de notre quotidien.
		Cependant, cette omniprésence est aussi source de confusion et d'incompréhension pour le grand public.
		En effet :
		\begin{itemize}
			% Craintes.
			\item L'\texttt{IA} peut être perçue comme une menace, soit parce qu'elle vole des emplois, soit parce qu'elle risquerait de devenir incontrôlable.
			Ces craintes sont notamment véhiculées par la culture populaire, à l'image de \texttt{Terminator} ou d'\texttt{Ultron} qui se sont retournés contre leur créateur.
			% Attentes trop hautes.
			\item Les attentes des utilisateurs sont parfois trop élevées par rapport aux capacités réelles du modèle.
			Il en résulte alors un sentiment de frustration, en particulier lorsque l'utilisateur exprime un besoin urgent, mais que le modèle se contente de répondre qu'il n'a pas compris la question et que vous devriez reformuler votre demande.
			% Confiance aveugle.
			\item Le crédit accordé aux modèles d'\texttt{IA} est parfois excessif, au point que l'esprit critique des utilisateurs s'efface petit à petit.
			Les capacités spectaculaires des derniers modèles génératifs accentue davantage cette confiance aveugle, et il devient même difficile d'identifier les fausses informations générées tant elles semblent crédibles (\textit{par exemple, \texttt{ChatGPT} peut vous mentir avec conviction en inventant certains détails dans ses réponses}). 
		\end{itemize}
		
		%%% Idée reçu : pas complexe à faire.
		L'idée reçue selon laquelle il est facile de concevoir un modèle d'\texttt{IA} explique en partie ces confusions.
		Encore une fois, la culture populaire véhicule cette image d'une conception accessible à tous et à moindres coûts.
		En reprenant l'exemple d'\textit{Ultron}, il suffit au \texttt{Dr. Hank Pym} de \textguillemets{calquer ses schémas mentaux} pour créer le robot, comme si cela était un acte banal ; plus récemment, dans la série \texttt{Black Mirror} (S2 Ep1), le personnage de \textit{Martha} se procure un avatar de son conjoint décédé sans trop de difficultés en communiquant simplement les messages et les photos de ce dernier, mais aucune mention n'est faite sur le processus de conception, excepté qu'il est \textguillemets{expérimental}.
		\newline
		
		%%% TR: annotation de données.
		\textbf{Toutes ces illusions masquent ainsi un point pourtant fondamental à la conception des modèles d'\texttt{IA} : la qualité des données sur lesquelles ils sont entraînés.}
		
		
	%%%%%--------------------------------------------------------------------
	%%%%% NICHE: Désillusion quant à la simplicité de l'annotation de données.
	%%%%%--------------------------------------------------------------------
	\section*{Désillusion quant à la simplicité de l'annotation de données}
	\addcontentsline{toc}{section}{
		\protect\numberline{}
		Désillusion quant à la simplicité de l'annotation de données
	}
		
		% Modèle d'IA => besoins de données.
		Nous pouvons résumer l'apprentissage automatique comme étant un ensemble de méthodes permettant de reproduire un phénomène par l'exemple.
		En d'autres termes, il faut des données en qualité et en quantité suffisante pour concevoir la base d'apprentissage d'un modèle d'\texttt{IA}.
		C'est là qu'intervient la tâche d'annotation : celle-ci consiste à demander à un expert du métier, c'est-à-dire à un spécialiste du phénomène, d'enrichir les données pour leur attribuer une signification, de leur fournir de la valeur ajoutée, et ainsi permettre à la machine de comprendre et reproduire le phénomène.
		\newline
		
		% Exemple assistant conversationnel.
		Pour illustrer nos propos, prenons l'exemple des assistants conversationnels (\textit{chatbot}).
		Ces assistants ont pour objectif de traiter automatiquement des requêtes exprimées en langage naturel.
		Pour ce faire, il est possible de réaliser une modélisation de textes en intentions de dialogue.
		À ce titre, des requêtes comme \textguillemets{\textit{joue moi du jazz s'il te plaît !}} ou \textguillemets{\textit{peux-tu lancer une playlist de Noël sur l'enceinte du salon !}} peuvent être modélisées par l'intention \texttt{jouer\_musique}.
		La base d'apprentissage d'un assistant conversationnel est alors constitué d'un ensemble de textes qui ont été annotée en intention.
		
		% Complexité cachée.
		L'aspect élémentaire de notre exemple cache cependant toute la difficulté de cette tâche :
		\begin{itemize}
			\item Il est possible d'avoir une vaste diversité d'intentions (\textit{jouer de la musique, allumer la lumière, consulter la météo, démarrer un minuteur, appeler sa maman, ...}) : la complexité grandit d'autant qu'il y a de cas d'usage modélisables ;
			\item L'interprétation du langage est complexe par essence : le vocabulaire employé peut être spécifique, une requête peut être ambiguë ou un double sens, une erreur grammaticale peut gênée la compréhension de la phrase, ...
			\item La modélisation en intention est un exercice subjectif durant laquelle deux annotateurs peuvent avoir des avis différents : dans notre exemple, et avec les mots \textguillemets{\textit{peux-tu lancer ...}}, est-ce que l'assistant devrait effectuer une action, ou devrait-il simplement exprimer s'il en est capable ?
		\end{itemize}
		
		% Bilan sur cette complexité.
		Ainsi, l'annotation de textes en intentions met en évidence la complexité de cette tâche.
		Ben Hamner, cofondateur et directeur technique de l'entreprise \texttt{Kaggle}, résume ainsi cette problématique : \textbf{l'\texttt{IA} c'est \textguillemets{1\% d'écriture de code informatique, 9\% d'analyse de ce qui ne va pas dans le code informatique, et 90\% d'analyse de ce qui ne va pas dans les données d'entraînement}}.
		
		
	%%%%%--------------------------------------------------------------------
	%%%%% GAP: Intervention difficile des experts métiers dans les projets de conception de base d'apprentissage.
	%%%%%--------------------------------------------------------------------
	\section*{Intervention difficile des experts métiers dans les projets de conception de base d'apprentissage}
	\addcontentsline{toc}{section}{
		\protect\numberline{}
		Intervention difficile des experts métiers dans les projets de conception de base d'apprentissage
	}
		
		% Organisation MATTER.
		Pour absorber la complexité liée à la tâche d'annotation, un projet de conception d'une base d'apprentissage s'organise généralement de manière cyclique.
		\begin{itemize}
			\item En premier lieu, les experts métiers sont consultés lors d'atelier d'idéation pour définir une première version de la modélisation (\textit{dans le cadre de notre exemple sur les assistants conversationnel, ce serait la définition de la liste des intention possible et de leur définition}).
			\item Dans un second temps, les experts métiers parcourent l'ensemble des données pour les annoter en fonction de la modélisation définie ;
			\item si certaines frictions apparaissent (\textit{modélisation pas adaptée, différence d'avis en annotateurs, ...}), le projet retourne à l'étape de modélisation pour proposer une nouvelle version révisée, et le cycle recommence...
		\end{itemize}
		
		% Complexité d'organisation.
		Une telle approche à l'avantage de régler la complexité de la tâche par de petits ajustements (\textit{un héritage de la gestion de projet en mode agile}), mais elle possède les inconvénients d'être chronophage et onéreuse.
		En effet, à chaque remise en cause de la modélisation, toutes les données annotées sont potentiellement à revoir pour s'assurer de leur compatibilité avec la nouvelle modélisation proposée.
		D'autre part, la discussion sur une remise en cause potentielle de cette modélisation relève du domaine analytique, qui n'est pas le domaine principale d'expertise des annotateurs intervenant dans le projet.
		Il est donc nécessaire de former les experts métiers à la tâche d'analyse de données pour que leurs remarques soient le plus pertinentes possibles.
		\newline
		
		% BAM : ça va pas !
		Nous touchons alors du doigt un problème philosophique : \textbf{comment sommes-nous arrivé à la conclusion que la meilleure manière de faire intervenir un expert métier sur une tâche d'annotation, c'est en lui demandant une tâche pour laquelle il n'est pas expert ?}
		En effet, si nous voulions par exemple faire un assistant conversationnel sur un domaine gastronomique, nous engagerions a priori un chef cuisinier ou un restaurateur ; toutefois, il semble incongru d'engager ces profils dans le but de réaliser des analyses statistiques ou d'être compétents sur des questions linguistiques.
		
		% Constat : on a reposer la complexité sur l'annotateur.
		En conclusion, l'organisation traditionnelle des projets d'annotations ne semble pas résoudre la complexité de cette tâche, mais semble plutôt l'ignorer en espérant trouver ou former des annotateurs ayant à la fois des compétences métiers et des compétences analytiques.
		
		
	%%%%%--------------------------------------------------------------------
	%%%%% OCCUPYING THE NICHE: Recherche d'une méthode alternative pour modéliser le texte en intentions tout en valorisant l'intervention de l'expert
	%%%%%--------------------------------------------------------------------
	\section*{Recherche d'une méthode alternative pour modéliser le texte en intentions tout en valorisant l'intervention de l'expert}
	\addcontentsline{toc}{section}{
		\protect\numberline{}
		Recherche d'une méthode alternative pour modéliser le texte en intentions tout en valorisant l'intervention de l'expert
	}
		
		
		\todo[inline]{A REDIGER}
		
			%- Besoin de recentrer l'activité des experts métiers ;
			%- Besoin d'assister la conception d'un jeu de données ;
			%- Nous proposons donc une méthode itérative et semi-supervisée.
		
	%%%%%--------------------------------------------------------------------
	%%%%% PLAN: Annonce du plan de ce manuscrit.
	%%%%%--------------------------------------------------------------------
	\section*{Annonce du plan de ce manuscrit}
	\addcontentsline{toc}{section}{
		\protect\numberline{}
		Annonce du plan de ce manuscrit
	}
		
		% Introduction.
		Afin de traiter la problématique que nous venons d'exposer, nous organisons la discussion de ce manuscrit de la manière suivante :
		
		% Plan.
		\begin{itemize}
			% Chapitre 2: Revue de littérature.
			\item Au cours du \textsc{Chapitre~\ref{chapter:2-REVUE-DE-LITTERATURE}}, nous présentons en détails la tâche d'annotation, son organisation traditionnelle ainsi que les nombreux défis qu'elle comporte.
			Pour mieux illustrer nos propos, nous utilisons des exemples inspirés de l'univers de la bande dessinée.
			% Section 2.4: Contexte du doctorat.
			\item Nous complétons la revue de littérature en expliquant le contexte de ce doctorat en \textsc{Section~\ref{section:2.4-CONTEXTE-DOCTORAT}} : ce complément nous permet de mettre en évidence la difficulté d'intervention des experts métiers dans un projet traditionnel d'annotation.
			% Chapitre 3 : Présentation de la méthode.
			\item Le \textsc{Chapitre~\ref{chapter:3-CLUSTERING-INTERACTIF}} est dédié à la présentation de notre méthodologie d'annotation alternative basée sur un \texttt{Clustering Interactif}.
			La description de l'implémentation technique est consultable dans l'\textsc{Annexe~\ref{annex:C-ANNEXE-IMPLEMENTATIONS}}.
			% Chapitre 4 : Etude de la méthode
			\item Dans le \textsc{Chapitre~\ref{chapter:4-ETUDES}}, nous décrivons les six hypothèses que nous voulions vérifier sur notre méthodologie d'annotation : efficacité, efficience, coûts, pertinence, rentabilité et robustesse.
			\item Le \textsc{Chapitre~\ref{chapter:5-GUIDE}} fait le point sur l'ensemble des discussions et découvertes contenues des précédents chapitres, et comporte différents avis et conseils pratiques.
			Le chapitre entier est prévu pour être un guide d'utilisation synthétique de notre méthodologie d'annotation.
		\end{itemize}
		
		% Conclusion.
		Le \textsc{Chapitre~\ref{chapter:6-CONCLUSION}} dresse la conclusion et clôt la discussion en abordant des thématiques et perspectives plus générales.


%%%%%--------------------------------------------------------------------
%%%%% Chapitre: Etat de l'art
%%%%%--------------------------------------------------------------------
\chapter{Revue de littérature sur la tâche d'annotation en intelligence artificielle}
\label{chapter:2-REVUE-DE-LITTERATURE}
	
	% ANNONCE DU BUT DU CHAPITRE: ETAT DE L'ART DE L'ANNOTATION.
	L'annotation (ou labellisation) de données est une tâche essentielle de l'apprentissage automatique, permettant de décrire des jeux de données nécessaires à l'entraînement et à l'évaluation de modèles d'intelligence artificielle.
	Dans ce chapitre, nous allons traiter les points suivants :
	%
	\begin{leftBarImportantRed}
		\begin{todolist}
			\item Définir à quoi correspond une tâche de labellisation, et détailler la vaste organisation technique et méthodologique autour d'un projet d'annotation de données ;
			\item Présenter un aperçu des difficultés principales que peut rencontrer un projet d'annotation, ainsi que certaines techniques conçues pour s'en prévenir ;
			\item Expliquer le contexte industriel dans lequel ce doctorat s'inscrit et montrer l'intérêt d'assister la conception des jeux de données d'entraînement.
		\end{todolist}
	\end{leftBarImportantRed}
	
	% TABLE DES MATIÈRES DU CHAPITRE
	\minitoc
	
	%%%%%--------------------------------------------------------------------
	%%%%% Section 2.1: Présentation théorique de l'annotation
	%%%%%--------------------------------------------------------------------
	%\newpage
	%\vspace{2cm}
	\section{Présentation théorique de l'annotation}
\label{section:2.1-PRESENTATION-ANNOTATION}

	%%%
	%%% Introduction: donner une définition de ce qu'est l'annotation.
	%%%
	Tout d'abord, introduisons quelques définitions pour appréhender le concept d'\textguillemets{\texttt{annotation}} et donnons quelques exemples pour comprendre les enjeux qui y sont associés.
	
	
	%%%
	%%% Subsection 2.1.1: Définition et objectifs de l'annotation de données.
	%%%
	\subsection{Définition et objectifs de l'annotation de données}
	\label{section:2.1.1-PRESENTATION-ANNOTATION-DEFINITION}
	
		%%% 2.1.1.A. Qu'est que l'\textguillemets{\texttt{apprentissage automatique}} ?
		\subsubsection{Qu'est que l'\textguillemets{\texttt{apprentissage automatique}} ?}
		\label{section:2.1.1.A-PRESENTATION-ANNOTATION-DEFINITION-MACHINE-LEARNING}
			
			% Définition.
			Nous proposons la définition suivante inspirée de l'\texttt{ACM} (\textit{Association for Computing Machinery}) : l'\textguillemets{\texttt{apprentissage automatique}} (ou \textguillemets{\texttt{Machine Learning}}) est une branche de l'intelligence artificielle dédiée au développement de méthodes permettant à l'ordinateur de \textbf{reproduire une tâche par l'exemple} : il n'est donc pas explicitement programmé pour réaliser cette tâche, mais il l'\textguillemets{apprend} à l'aide d'un modèle mathématique.
			Cet apprentissage peut être \textit{supervisé} (l'interprétation des exemples est fournie par un humain), \textit{non-supervisé} (la machine déduit l'interprétation des données sans intervention humaine) ou \textit{semi-supervisé} (mélange des deux précédentes approches).
			
			% Applications.
			Le \textit{Machine Learning} permet ainsi d'automatiser l'analyse et la manipulation de certains phénomènes complexes tels que le langage, l'observation visuelle, la détection d'anomalies, le traitement acoustique, ...
			
			% References.
			\begin{leftBarInformation}
				Si vous voulez revoir les bases de l'apprentissage automatique, des livres comme \cite{zhou:2021:machine-learning} ou \cite{raschka-mirjalili:2019:python-machine-learning} traitent des notions principales et de leur mise en application.
			\end{leftBarInformation}
			
		%%% 2.1.1.B. Qu'est qu'un \textguillemets{\texttt{corpus d'entraînement}}?
		\subsubsection{Qu'est qu'un \textguillemets{\texttt{corpus d'entraînement}} ?}
		\label{section:2.1.1.B-PRESENTATION-ANNOTATION-DEFINITION-BASE-APPRENTISSAGE}

			% Définitions.
			Pour concevoir un modèle en apprentissage automatique, il nous faut un ensemble d'exemples (textes, images, sons, vidéos, ou tout autre relevé d'informations) permettant de capturer le phénomène à appréhender : cela nous aide à la fois à le décrire et à mieux le comprendre.
			Nous utilisons alors les termes \textguillemets{\texttt{corpus d'entraînement}}, \textguillemets{\texttt{jeu de d'entraînement}} ou \textguillemets{\texttt{base d'apprentissage}} pour désigner cet ensemble de données.
			
			% Carcatéristique importante : la représentativité !
			Il est important de noter qu'un corpus n'est qu'un échantillon de taille finie d'un phénomène pouvant être infini ou indénombrable.
			Il est donc d'usage de valoriser cet échantillon s'il est \textguillemets{\texttt{représentatif}} du phénomène qu'il décrit, c'est-à-dire s'il capture bien le large panel de variations que peuvent prendre les données (\cite{biber:1993:representativeness-corpus-design}).
			
			% References.
			\begin{leftBarInformation}
				Nous discuterons davantage de cette notion de représentativité dans la \textsc{Section~\ref{section:2.3.1.A-DEFIS-ANNOTATION-ASPECT-DONNEES-REPRESENTATIVITE}}.
				D'autre part, si vous voulez mieux comprendre cette notion de corpus, vous pouvez vous référer à \cite{sinclair:2004:corpus-text-basic} issu du livre \textit{Developing Linguistic Corpora} (\cite{wynne:2004:developing-linguistic-corpora}).
			\end{leftBarInformation}
		
		%%% 2.1.1.C. Qu'est que l'\textguillemets{\texttt{annotation}} ?
		\subsubsection{Qu'est que l'\textguillemets{\texttt{annotation}} ?}
		\label{section:2.1.1.C-PRESENTATION-ANNOTATION-DEFINITION-ANNOTATION}
			
			% Définition.
			Les données d'un corpus manquent parfois d'information pour bien cerner un phénomène, il est alors nécessaire de faire intervenir un humain pour introduire des connaissances supplémentaires qui ne sont pas explicitement présentes dans ces données.
			Nous appelons \textguillemets{\texttt{annotation}} (ou \textguillemets{\texttt{étiquetage}}, \textguillemets{\texttt{labellisation}}) cette tâche consistant à décrire les données d'un corpus, et nous distinguons ainsi les données dites \textguillemets{\texttt{brutes}} (utilisées par les approches non-supervisées) des données dites \textguillemets{\texttt{annotées}} (utilisées par les approches supervisées) en fonction de l'absence ou de la présence d'un complément d'informations.
			
			% Valeurs associées: valeur ajoutée, information d'interprétation.
			Les informations renseignées peuvent porter sur la donnée entière ou seulement sur une partie seulement, peuvent concerner des variables catégorielles (ensemble fini) ou numériques (ensemble infini), et peuvent aussi être cumulatives ou mutuellement exclusives.
			Dans la littérature, \cite{garside-etal:1997:corpus-annotation-linguistic} présente l'annotation comme la tâche permettant de donner une \textguillemets{\texttt{valeur ajoutée}} aux données ; de son côté, \cite{leech:2004:adding-linguistic-annotation} précise que l'annotation est une action d'\textguillemets{\texttt{interprétation}} qui aide à la compréhension et à la reproduction d'un phénomène, mais aussi au contrôle du comportement des modèles d'apprentissage automatique.
	
	
	%%%
	%%% Subsection 2.1.2: Exemples de tâches d'annotations.
	%%%
	\subsection{Exemples de tâches d'annotations}
	\label{section:2.1.2-PRESENTATION-ANNOTATION-EXEMPLES}
		
		% Transition: définitions assez généralistes car champ d'application très vaste.
		Les définitions données dans la section précédente peuvent paraître abstraites car il est difficile de dépeindre la vaste diversité d'applications nécessitant des données labellisées.
		En effet, une tâche d'annotation répond toujours à un besoin précis, mais il y a une telle multiplicité de types de données (\textit{données tabulaires, textuelles, visuelles, auditives, ...}) et de cas d'usages (\textit{prédiction d'une valeur numérique (tâche de régression), prédiction d'une catégorie (tâche de classification), détection d'objets (tâche d'extraction), création de nouvelles données (tâche de génération), ... }) qu'une unique définition ne peut être que purement théorique.
		
		% Annonce: prendre des exemples sur l'univers de la bande dessinée.
		Ainsi, nous estimons qu'il est préférable de compléter ces définitions par quelques exemples concrets.
		Nous pourrons ainsi mieux dresser le portait d'une tâche d'annotation, avec ses intérêts et ses complications.
		Pour cela, nous allons prendre le thème de la bande dessinée et de ses dérivés, et explorer ensemble différents cas d'usage qui pourraient intéresser un auteur, un libraire ou un lecteur.
		
		
		%%% 2.1.2.A. Estimation du prix d'une bande dessinée.
		\subsubsection{Estimation du prix d'une bande dessinée.}
		\label{section:2.1.2.A-PRESENTATION-ANNOTATION-EXEMPLES-REGRESSION}
		
			% Cas d'usage: évaluer le juste prix.
			Les acheteurs et les vendeurs de bandes dessinées s'interrogent forcément sur le juste prix de l'oeuvre qu'ils veulent acquérir ou céder.
			Répondre à cette question avec précision nécessite diverses informations, à la fois sur l'oeuvre (comme son identification ou l'avis de ses lecteurs), sur le document en tant que tel (comme son état de conservation), mais aussi sur le prestige de son édition (éditions originales ou de collection).
			Sans un regard d'expert, il est possible de trouver certaines oeuvres rares vendues pour presque rien sur le marché d'occasion, ou à l'inverse voir certaines \texttt{BD} être achetées à prix d'or alors que le document est en piteux état.
			
			% Entrainer un modèle: régression.
			Afin d'aiguiller les acquéreurs, il est possible d'utiliser un modèle de \texttt{régression}\footnote{
				Pour plus de détails sur la régression: voir la revue de \cite{maalouf:2011:logistic-regression-data} ; voir un exemple basé sur la méthode des moindres carrés dans \cite{zdaniuk:2014:ordinary-leastsquares-ols}.
			} permettant de prédire le prix d'une \texttt{BD} à partir des différentes métadonnées à disposition.
			Mais pour entraîner un tel modèle, il est nécessaire d'avoir une base d'apprentissage contenant des exemples de transactions avec leur prix de vente.
			Nous pouvons structurer l'ensemble des informations nécessaire dans un tableau, et la tâche d'annotation consiste alors à renseigner pour chaque transaction :
			\begin{itemize}
				\item l'identification complète de la \texttt{BD} (titre, auteur, édition, ... ),
				\item l'état du document grâce à un regard d'expert (l'état peut par exemple être défini par une variable catégorielle dont les valeurs seraient "\texttt{Mauvais état}", "\texttt{Bon état}", "\texttt{Très bon état}", "\texttt{Neuf}") ;
				\item le prix de la \texttt{BD}, estimé ou réel (défini par une variable numérique).
			\end{itemize}
			
			% Citation de l'exemple.
			Un exemple de résultat d'annotation de ces données est disponible dans la \textsc{Table~\ref{table:2.1.2.A-PRESENTATION-ANNOTATION-EXEMPLES-REGRESSION}}.
			%
			\begin{leftBarExamples}
				\begin{table}[H]  % keep [H] to be in the tcolorbox.
					\begin{center}
					\def\arraystretch{0.8}  % interligne
					\begin{tabular}{|c|l|c|c|c|r|}
					
					\hline
					% ENTETE DU TABLEAU
					\rowcolor{colorLeftBarExamples!25}
					Collection
						& N°: Titre
						& Édition
						& Note
						& État
						& Prix (€)
						\tabularnewline
						\hline \hline
					% BD1.
					Lucky Luke
						& $01$: La mine d'or de Dick Digger
						& $1949$
						& $3.2/5$
						& Très bon
						& $5~000,00$
						\tabularnewline
						\hline
					% BD2.
					Lucky Luke
						& $12$: Les cousins Dalton
						& $1958$
						& $4.3/5$
						& Bon
						& $40,00$
						\tabularnewline
						\hline
					% BD3.
					Lucky Luke
						& $12$: Les cousins Dalton
						& $1962$
						& $4.3/5$
						& Très bon
						& $65,00$
						\tabularnewline
						\hline
					% BD4.
					Lucky Luke
						& $12$: Les cousins Dalton
						& $1985$
						& $4.3/5$
						& Très bon
						& $6,00$
						\tabularnewline
						\hline
					% BD5.
					Lucky Luke
						& $15$: L'évasion des Dalton
						& $1960$
						& $4.1/5$
						& Mauvais
						& $3,00$
						\tabularnewline
						\hline
					% ...
					\multicolumn{6}{|c|}{ \shortstack{ ... } }
						\tabularnewline
						\hline
				
					\end{tabular}
					\end{center}
					\caption{
						Exemple d'annotation du prix de vente de bandes dessinées en fonction de leur édition, de la note de leur lecteurs et de leur état (source: \url{https://www.bedetheque.com/serie-213-BD-Lucky-Luke.html}).
					}
					\label{table:2.1.2.A-PRESENTATION-ANNOTATION-EXEMPLES-REGRESSION}
				\end{table}
			\end{leftBarExamples}
			
			% Conclusion.
			Ainsi, si quelqu'un s'intéresse au prix d'une nouvelle bande dessinée pour lequel il n'y a pas de référence tarifaire, il peut interroger le modèle de régression qui proposera un prix en accord avec les exemples dont il dispose dans sa base d'apprentissage.
			
			% Autre cas d'usage similaire: Régression dans d'autres domaines.
			\begin{leftBarInformation}
				De manière équivalente, il est possible de faire de la régression dans d'autres domaines, notamment pour prédire un volume, une surface, une quantité, ...
				La tâche d'annotation consistera à chaque fois à renseigner la valeur numérique à prédire en fonction des différentes données à disposition.
			\end{leftBarInformation}
		
		
		%%% 2.1.2.B. Classification de l'état d'une bande dessinée à partir d'une photo.
		\subsubsection{Classification de l'état d'une bande dessinée à partir d'une photo.}
		\label{section:2.1.2.B-PRESENTATION-ANNOTATION-EXEMPLES-CLASSIFICATION}
			
			% Cas d'usage: classifier l'état.
			Il est d'usage d'adapter le prix de vente d'un produit en fonction de son état, et nous avons intégré ce facteur dans l'estimation du prix d'une bande dessinée (voir exemple précédent).
			Cependant, l'état de conservation n'est pas une notion objective et chacun peut avoir des références différentes.
			Au final, c'est souvent un libraire qui détermine si l'oeuvre est en bon ou en mauvais état, et, sans un regard d'expert, nous pouvons omettre un détail ou nous tromper lors de notre appréciation.
			
			% Entrainer un modèle: classification d'image.
			Afin de nous aider à estimer l'état d'une bande dessinée, il est possible d'utiliser un modèle de \texttt{classification}\footnote{
				Pour plus de détails sur la classification: voir les revues de \cite{aized-amin-soofi-arshad-awan:2017:classification-techniques-machine} ou de \cite{kotsiantis-etal:2006:machine-learning-review} ; voir un exemple basé sur les machine à vecteurs de support (\texttt{SVM}) dans \cite{cortes-vapnik:1995:supportvector-networks}.
			} permettant, à partir d'une image, d'affecter à chaque \texttt{BD} une catégorie prédéfinie (par exemple: "\texttt{Mauvais état}", "\texttt{Bon état}", "\texttt{Très bon état}", "\texttt{Neuf}").
			Pour entraîner un tel modèle, il est nécessaire d'avoir une base d'apprentissage contenant des exemples d'images de \texttt{BD} associées avec leur catégorie d'état.
			La tâche d'annotation peut alors consister à renseigner pour chaque couverture de bande dessinée la catégorie d'état qui lui correspond le plus.
			
			% Citation de l'exemple.
			Un exemple d'annotation de la classification de l'image est disponible dans la \textsc{Figure~\ref{figure:2.1.2.B-PRESENTATION-ANNOTATION-EXEMPLES-CLASSIFICATION}}.
			%
			\begin{leftBarExamples}
				\begin{figure}[H]
					\centering
					\includegraphics[width=0.80\textwidth]{figures/etatdelart-morris-1950-lucky-luke-2-1952-lucky-luke-4}
					\caption{
						Exemple d'annotation de l'état d'une \texttt{BD} (ici: \cite{morris-goscinny:1950:rodeo} et \cite{morris-goscinny:1952:sous-ciel-ouest}).
						La première est en très bon état (couverture comme neuve, tranches légèrement usées, pages intactes) tandis que la seconde est en mauvais état (couverture usée, dos abimée, traces sur les pages, ...).
					}
					\label{figure:2.1.2.B-PRESENTATION-ANNOTATION-EXEMPLES-CLASSIFICATION}
				\end{figure}
			\end{leftBarExamples}
			
			% Conclusion.
			Ainsi, si quelqu'un s'interroge sur l'état d'une bande dessinée en sa possession, ce modèle peut identifier la catégorie d'état la plus probable d'après les exemples disponibles dans sa base d'apprentissage.
			
			% Autre cas d'usage similaire: Identifier la langue de la BD.
			\begin{leftBarInformation}
				De manière équivalente, il est possible de faire de la classification sur d'autres données, comme par exemple la classification de textes pour identifier la langue de l'ouvrage.
				Dans l'exemple ci-dessous, les catégories proposées sont "\texttt{Français}", "\texttt{Anglais}" et "\texttt{Allemand}", et la tâche d'annotation consiste ici à associer à chaque texte une catégorie de langue.
				\begin{center}
				\begin{tabular}{ c l }
					\textguillemets{\textit{
						Les cousins Dalton ont dévalisé la diligence.
					}} & $\implies$ \textcolor{colorSilverLakeBlue}{\texttt{Français}} \\
					\textguillemets{\textit{
						The Dalton cousins robbed the stagecoach.
					}} & $\implies$ \textcolor{colorDarkPastelGreen}{\texttt{Anglais}} \\
					\textguillemets{\textit{
						Die Dalton-Cousins haben die Postkutsche ausgeraubt.
					}} & $\implies$ \textcolor{colorDarkPastelRed}{\texttt{Allemand}}
				\end{tabular}
				\end{center}
			\end{leftBarInformation}
		
		
		%%% 2.1.2.C. Identification d'une bande dessinée à partir de sa couverture.
		\subsubsection{Identification d'une bande dessinée à partir de sa couverture.}
		\label{section:2.1.2.C-PRESENTATION-ANNOTATION-EXEMPLES-EXTRACTION}
			
			% Cas d'usage: identifier une bande dessinée
			Identifier une bande dessinée n'est pas toujours facile, et recopier l'ensemble des informations l'identifiant peut prendre du temps.
			Les libraires ou les collectionneurs désirant faire l'inventaire des ouvrages en leur possession peuvent ainsi y passer de nombreuses heures, avec le risque de faire des erreurs lors de l'inscription des bande dessinée dans leur registre.
			
			% Entrainer un modèle: extraction de caractères.
			Afin d'aider les collectionneurs, il est possible d'utiliser un modèle de \texttt{reconnaissance optique des caractères} (\texttt{OCR})\footnote{
				Pour plus de détails sur l'\texttt{OCR}: voir la revue de \cite{berchmans-kumar:2014:optical-character-recognition} ou de \cite{awel-abidi:2019:review-optical-character}.
			} pour extraire automatiquement les informations importantes présentes sur les couvertures d'une \texttt{BD} à identifier.
			Pour entraîner un tel modèle, il est nécessaire d'avoir une base d'apprentissage contenant des exemples de pages de couverture avec la position et la valeur des informations pertinentes à extraire.
			La tâche d'annotation peut alors consister à renseigner pour chaque couverture de bande dessinée :
			\begin{itemize}
				\item la position des informations en l'encadrant sur l'image (avec un rectangle par exemple) ;
				\item la valeur écrite dans l'encadré sur l'image.
			\end{itemize}
			
			% Citation de l'exemple.
			Un exemple d'annotation de textes dans une image est disponible dans la \textsc{Figure~\ref{figure:2.1.2.C-PRESENTATION-ANNOTATION-EXEMPLES-EXTRACTION}}.
			%
			\begin{leftBarExamples}
				\begin{figure}[H]
					\centering
					\includegraphics[width=0.80\textwidth]{figures/etatdelart-morris-1958-lucky-luke-12}
					\caption{
						Exemple d'annotation de textes présents sur la couverture d'une bande dessinée (ici: \cite{morris-goscinny:1958:cousins-dalton}).
						Les informations essentielles telles que la collection, le numéro, le titre, l'auteur et l'éditeur y sont présentes. 
					}
					\label{figure:2.1.2.C-PRESENTATION-ANNOTATION-EXEMPLES-EXTRACTION}
				\end{figure}
			\end{leftBarExamples}
			
			% Conclusion.
			Ainsi, si quelqu'un veut identifier une nouvelle bande dessinée, il peut interroger le modèle d'extraction de caractères pour récupérer les informations textuelles présentes dans la couverture, à l'image des exemples disponibles dans sa base d'apprentissage.
			
			% Autres cas d'usage: Extraire les information d'un texte.
			\begin{leftBarInformation}
				De manière équivalente, il est possible de réaliser une \texttt{reconnaissance d'entités nommées (\texttt{NER})}\footnote{
					Pour plus de détails sur la reconnaissance d'entités nommées (\texttt{NER}): voir les revues de \cite{goyal-etal:2018:recent-named-entity} ou de \cite{li-etal:2022:survey-deep-learning}.
				} pour extraire les informations citées dans un texte.
				Dans l'exemple ci-dessous, les types d'entités présentes sont "\texttt{personnage}", "\texttt{métier}", "\texttt{argent}", "\texttt{lieu}" et "\texttt{date}".
				La tâche d'annotation consiste ici à identifier la position et le type de chaque entité présente.
				
				\begin{quote}
					\textguillemets{\textit{
						$\textbf{\text{Lucky Luke}}_{\textcolor{colorDarkPastelRed}{\texttt{(personnage)}}}$, le $\textbf{\text{cow-boy}}_{\textcolor{colorDarkPastelPurple}{\texttt{(métier)}}}$ solitaire, a attrapé les $\textbf{\text{Dalton}}_{\textcolor{colorDarkPastelRed}{\texttt{(personnage)}}}$ à $\textbf{\text{Coyote Gulch}}_{\textcolor{colorCarrotOrange}{\texttt{(lieu)}}}$ et a touché $\textbf{\text{50.000\$}}_{\textcolor{colorSilverLakeBlue}{\texttt{(argent)}}}$ en les livrant au $\textbf{\text{pénitencier}}_{\textcolor{colorCarrotOrange}{\texttt{(lieu)}}}$. Ils se sont évadés le $\textbf{\text{jeudi suivant}}_{\textcolor{colorDarkPastelGreen}{\texttt{(date)}}}$.
					}}
				\end{quote}
			\end{leftBarInformation}
		
		
		%%% 2.1.2.D. Interprétation audio d'une bande dessinée.
		\subsubsection{Interprétation audio d'une bande dessinée.}
		\label{section:2.1.2.D-PRESENTATION-ANNOTATION-EXEMPLES-TRANSCRIPTION}
		
			% Cas d'usage: générer une lecture audio.
			Il est de plus en plus commun de trouver des livres disponibles avec une lecture audio.
			Ces audio-livres, réalisés par une personne ou synthétisés par l'ordinateur, peuvent être à visée éducative ou simplement disponibles pour le loisir.
			Dans le cadre de notre exemple sur le thème des bandes dessinées, peu d'entre elles disposent d'une lecture audio.
			Une idée serait donc d'interpréter une lecture audio de ces bandes dessinées en synthétisant la voix des doubleurs de leurs adaptations télévisées (ou simplement d'un narrateur si l'oeuvre n'a pas été portée à l'écran).
			
			% Entrainer un modèle: synthétiseur vocal.
			Dans le but de créer ces audio-\texttt{BD}, nous pourrions envisager d'utiliser des modèles de \texttt{synthèse vocale} (\texttt{TTS})\footnote{
				Pour plus de détails sur la synthèse vocale: voir la revue de \cite{kothadiya-etal:2020:different-methods-review} ; voir un exemple d'architecture neuronale \texttt{end-to-end} dans \cite{mu-etal:2021:review-endtoend-speech}.
			} pour générer automatiquement la lecture des bulles d'une bande dessinée.
			Pour entraîner de tels modèles, il est nécessaire d'avoir une base d'apprentissage contenant des exemples d'audios prononcés par chacun des personnages (ici: les doubleurs de l'adaptation télévisée) avec la transcription de leur paroles pour chaque audio.
			La tâche d'annotation peut alors consister à renseigner le personnage et les paroles qu'il a prononcées.
			
			% Citation de l'exemple.
			Un exemple d'annotation phonétique est illustré dans la \textsc{Figure~\ref{figure:2.1.2.D-PRESENTATION-ANNOTATION-EXEMPLES-TRANSCRIPTION}}.
			%
			\begin{leftBarExamples}
				\begin{figure}[H]
					\centering
					\includegraphics[width=0.95\textwidth]{figures/etatdelart-thebault-transcription}
					\caption{
						Exemple de paroles prononcées dans un audio.
						Ici, la voix de \textit{Lucky Luke} est interprétée par Jacques THEBAULT.
						Le texte annoté, c'est-à-dire celui prononcé dans l'audio, est  \textguillemets{\texttt{Ils sont à vous chef, et j'vous s'rai reconnaissant de bien les garder cette fois.}}.
						Les phonèmes en alphabet phonétique international associés à chaque séquence de l'audio sont disponibles si besoin.
						%[il sɔ̃ a vu ʃɛf, e ʒvu sʁe ʁə.kɔ.nɛ.sɑ̃ də bjɛ̃ vu.lwaʁ le ɡaʁ.de sɛt fwa].
					}
					\label{figure:2.1.2.D-PRESENTATION-ANNOTATION-EXEMPLES-TRANSCRIPTION}
				\end{figure}
			\end{leftBarExamples}
			
			% Conclusion.
			Ainsi, nous pourrions consulter le modèle de synthèse vocale d'un personnage (ici: celui de \texttt{Lucky Luke}) avec un nouveau texte à prononcer pour en obtenir une lecture audio dont la voix se rapproche des enregistrements de la base d'apprentissage (ici: celle de Jacques THEBAULT).
			
			% Autres cas d'usage: comuler OCR + Classification + TTS
			\begin{leftBarInformation}
				Nous pourrions compléter le cas d'usage
				(1) en extrayant automatiquement le texte d'une planche de \texttt{BD} par \texttt{OCR},
				(2) en détectant automatiquement le personnage prononçant la bulle de \texttt{BD} par classification,
				puis (3) en générant du texte à prononcer par le personnage par synthèse vocale.
				Bien entendu, la conception et l'enchaînement de ces différents modèles sont plutôt complexes, et chaque tâche de \textit{Machine Learning} demande ses propres données annotées pour construire une base d'apprentissage.
			\end{leftBarInformation}
			
			
		%%% Exemple: Annotation génération musicale
		% Generation de musique: \cite{hernandez-olivan-beltran:2023:music-composition-deep}
		%\begin{leftBarExamples}
		%	Annotation des paroles d'une chanson (\cite{woods:1971:poor-lonesome-cowboy}). \\
		%
		%	\begin{guitar}
		%		\textbf{Im A Poor Lonesome Cowboy}
		%		\textit{
		%			~~~~from \texttt{Lucky Luke - Daisy Town OST} ($1971$)
		%			~~~~composed by \texttt{Claude Bolling}
		%			~~~~performed by \texttt{Pat Woods}
		%		}
		%		\texttt{Intro}
		%		\textit{
		%			[D]Lonesome [D7]cowboy, [G]Lonesome [G7]cowboy, [D]You're a [Bm]long [Bm7]long [E]way from [A]home.
		%			[D]Lonesome cowboy, [G]Lonesome cowboy, [D]You've a [Bm]long [Bm7]long [E]way [A7]to [D]roam.
		%		}
		%		\texttt{Couplet}
		%		\textit{
		%			I'm a [D]poor lonesome cowboy, I'm a long long way from home,
		%			And this poor lonesome [F\#m]cowboy, Has got a [Em]long long way [A]to roam.
		%			Over [D]mountains and over [D7]prairies, From [G]dawn 'til day is [Em]done,
		%			My [Bm]horse and me keep [F\#]ridin', [G]into the [A]settin' [D]sun.
		%		}
		%		{ \center \textbf{...} }
		%	\end{guitar}
		%\end{leftBarExamples}
		
		
		%%% Exemple: Annotation étiquette gramaticale
		%\begin{leftBarExamples}
		%	Annotation des étiquettes grammaticales dans un texte.
		%	\begin{quote}
		%		\textguillemets{\textit{
		%			$\text{Les}_{\textcolor{colorDarkPastelPurple}{\texttt{(DET)}}}$
		%			$\text{dangereux}_{\textcolor{colorMinionYellow}{\texttt{(ADJ)}}}$
		%			$\text{Dalton}_{\textcolor{colorDarkPastelRed}{\texttt{(PROPN)}}}$
		%			$\text{se}_{\textcolor{colorCarrotOrange}{\texttt{(PRON)}}}$
		%			$\text{sont}_{\textcolor{colorDarkPastelGreen}{\texttt{(AUX)}}}$
		%			$\text{encore}_{\textcolor{colorSilverLakeBlue}{\texttt{(ADV)}}}$
		%			$\text{évadés}_{\textcolor{colorDarkPastelGreen}{\texttt{(VERB)}}}$
		%			$\text{de}_{\textcolor{colorDimGray}{\texttt{(ADP)}}}$
		%			$\text{prison}_{\textcolor{colorDarkPastelRed}{\texttt{(NOUN)}}}$
		%			$\text{et}_{\textcolor{colorBlack}{\texttt{(CCONJ)}}}$
		%			$\text{ils}_{\textcolor{colorCarrotOrange}{\texttt{(PRON)}}}$
		%			$\text{ont}_{\textcolor{colorDarkPastelGreen}{\texttt{(AUX)}}}$
		%			$\text{déjà}_{\textcolor{colorSilverLakeBlue}{\texttt{(ADV)}}}$
		%			$\text{dévalisé}_{\textcolor{colorDarkPastelGreen}{\texttt{(VERB)}}}$
		%			$\text{une}_{\textcolor{colorDarkPastelPurple}{\texttt{(DET)}}}$
		%			$\text{banque}_{\textcolor{colorDarkPastelRed}{\texttt{(NOUN)}}}$
		%			$\text{à}_{\textcolor{colorDimGray}{\texttt{(ADP)}}}$
		%			$\text{Daisy Town}_{\textcolor{colorDarkPastelRed}{\texttt{(PROPN)}}}$.
		%		}}\\
		%		%{ \center \scriptsize (
		%		%	Adjectif: {\textcolor{colorMinionYellow}{\texttt{(ADJ)}}} ;
		%		%	Adverbe: {\textcolor{colorSilverLakeBlue}{\texttt{(ADV)}}} ;
		%		%	Conjonction: {\textcolor{colorBlack}{\texttt{(CCONJ)}}} ;
		%		%	Déterminant: {\textcolor{colorDarkPastelPurple}{\texttt{(DET)}}} ;
		%		%	Nom: {\textcolor{colorDarkPastelRed}{\texttt{(NOUN)}}}, {\textcolor{colorDarkPastelRed}{\texttt{(PROPN)}}} ;
		%		%	Préposition: {\textcolor{colorDimGray}{\texttt{(ADP)}}} ;
		%		%	Pronom: {\textcolor{colorCarrotOrange}{\texttt{(PRON)}}}, ;
		%		%	Verbe: {\textcolor{colorDarkPastelGreen}{\texttt{(AUX)}}}, {\textcolor{colorDarkPastelGreen}{\texttt{(VERB)}}}.
		%		%)}
		%	\end{quote}
		%\end{leftBarExamples}
	
	
	%%%
	%%% Subsection 2.1.3: Bilan concernant la présentation de l'annotation.
	%%%
	\subsection{Bilan concernant la présentation de l'annotation}
	\label{section:2.1.3-PRESENTATION-ANNOTATION-BILAN}
	
	%%%
	%%% Conclusion.
	%%%
	\begin{leftBarSummary}
		\begin{todolist}
			% Définition de l'annotation.
			\item[\itemok] \textguillemets{\texttt{Annoter}} une donnée consiste à \textbf{ajouter un complément d'information} pour pouvoir mieux interpréter puis reproduire un phénomène.
			% Type d'annotation.
			\item[\itemok] Le type d'annotation à réaliser \textbf{dépend du problème à traiter} : régression, classification, extraction d'information, génération ou synthèse de données, ...
			% Corpus d'entraînement.
			\item[\itemok] L'ensemble des données annotées peut être utilisé pour concevoir un modèle d'\textguillemets{\texttt{apprentissage automatique}}: il est alors appelé \textguillemets{\texttt{corpus d'entraînement}}.
		\end{todolist}
	\end{leftBarSummary}
	
	
	%%%%%--------------------------------------------------------------------
	%%%%% Section 2.2: Organisation usuelle d'un projet d'annotation
	%%%%%--------------------------------------------------------------------
	%\newpage
	%%% TODO\vspace{2cm}
	\section{Organisation usuelle d'un projet d'annotation}
\label{section:2.2-ORGANISATION-ANNOTATION}
	
	%%%
	%%% Introduction: Présenter l'organisation usuelle, les acteurs et le besoin d'outils.
	%%%
	Dans la section précédente, nous avons présenté l'importance d'avoir des données annotées pour entraîner d'un modèle de \textit{Machine Learning}.
	Maintenant, nous allons détailler l'organisation de cette tâche d'annotation, identifier les compétences nécessaires aux intervenants du projet ainsi que les fonctionnalités essentielles des outils de labellisation.
	
	
	%%%
	%%% Subsection 2.2.1: Étapes clés du cycle d'annotation.
	%%%
	\subsection{Étapes clés du cycle d'annotation}
	\label{section:2.2.1-ORGANISATION-ANNOTATION-ETAPES-CLES}
		% \cite{pustejovsky-stubbs:2012:natural-language-annotation} et \cite{stubbs:2013:methodology-using-professional} formalisation MATTER
		
		%%% Introduction au cycle MATTER.
		Une référence en matière d'organisation de projet d'annotation est proposée par \cite{pustejovsky-stubbs:2012:natural-language-annotation} et est complétée dans \cite{stubbs:2013:methodology-using-professional}.
		Les auteurs y formalisent la conception et l'amélioration \textbf{cyclique} d'un modèle de \textit{Machine Learning}.
		Ce cycle est appelé cycle \texttt{MATTER} en référence aux six étapes de conception qui le composent : \textit{\textbf{M}odelize}, \textit{\textbf{A}nnotate}, \textit{\textbf{T}rain}, \textit{\textbf{T}est}, \textit{\textbf{E}valuate} et \textit{\textbf{R}evise}.
		Ces étapes sont schématisées en \textsc{Figure~\ref{figure:2.2.1-ORGANISATION-ANNOTATION-ETAPES-CLES-MATTER}} et nous détaillons chacune d'entre elles ci-dessous.
		%
		\begin{figure}[!htb]
			\centering
			\includegraphics[width=0.95\textwidth]{figures/etatdelart-pustejovsky-2012-cycle-matter-mama-tt}
			\caption{
				Cycle \texttt{MATTER} structurant un projet d'annotation en six étapes principales: \textit{\textbf{M}odelize}, \textit{\textbf{A}nnotate}, \textit{\textbf{T}rain}, \textit{\textbf{T}est}, \textit{\textbf{E}valuate} et \textit{\textbf{R}evise}.
			}
			\label{figure:2.2.1-ORGANISATION-ANNOTATION-ETAPES-CLES-MATTER}
		\end{figure}
		%
		\begin{leftBarAuthorOpinion}
			Nous conseillons \cite{finlayson-erjavec:2016:overview-annotation-creation} pour son excellente revue de littérature qui détaille pas à pas le cycle \texttt{MATTER} tout en dressant la liste des points importants importants de chacune des étapes.
		\end{leftBarAuthorOpinion}
		
		%%% 2.2.1.A. Concevoir la base d'apprentissage (\textbf{M}odelize}, \textit{\textbf{A}nnotate}).
		\subsubsection{Concevoir la base d'apprentissage (\textit{\textbf{M}odelize}, \textit{\textbf{A}nnotate}).}
		\label{section:2.2.1.A-ORGANISATION-ANNOTATION-ETAPES-CLES-MODELIZE-ANNOTATE}
		
			%%% a. Collecte de données.
			Pour obtenir un bon modèle de \textit{Machine Learning}, il faut avoir une base d’apprentissage de qualité.
			Comme nous l'avons dit précédemment, cela commence par disposer d'un ensemble de données d'exemples qui représente fidèlement les différentes facettes du problème à modéliser (voir \textsc{Section~\ref{section:2.3.1.A-DEFIS-ANNOTATION-ASPECT-DONNEES-REPRESENTATIVITE}}).
			Une phase de \texttt{collecte} de données est alors organisée : cette collecte peut se baser sur des extractions de bases de données ou de sites internet à disposition, sur des enquêtes réalisées après d'utilisateurs finaux, ou encore sur les avis éclairés d'experts du problème.
			Certaines données peuvent aussi être artificiellement créées afin de compléter la collecte pour les aspects du problème difficile à observer.
			Une fois la collecte terminée, ces données brutes ont besoin d'être annotées pour pouvoir être exploitées. \\
			
			%%% b. Modelisation des données.
			
			% Importance de la modélisation.
			Afin de garantir la qualité de cette labellisation, \textbf{il est important de ne pas précipiter la tâche d'annotation}.
			En effet, l'objectif de cette tâche ainsi que les informations à annoter peuvent considérablement changer en fonction du phénomène à décrire, des données à disposition et de la finalité du modèle de \textit{Machine Learning} à entraîner.
			Il est donc fortement conseillé de bien \textbf{modéliser le problème} pour clarifier les attendus et les modalités de cette annotation (voir \textsc{Figure~\ref{figure:2.2.1-ORGANISATION-ANNOTATION-ETAPES-CLES-MATTER}}, étape 1. \textit{\textbf{M}odelize}).
			
			% Modélisation vs Spécification, Guide d'annotation et exemple.
			\cite{pustejovsky-stubbs:2012:natural-language-annotation} précisent notamment deux concepts importants de cette phase :
			\begin{itemize}
				\item la \textbf{modélisation} du problème, représentation de manière abstraite l'objectif à atteindre et décrivant ainsi la logique générale de l'annotation dans un \textit{schéma d'annotation} ;
				\item les \textbf{spécifications}, compilant dans un \textit{guide d'annotation} l'ensemble des règles concrètes à respecter pour mettre en application la modélisation.
			\end{itemize}
			Pour résumer cette distinction, la modélisation représente \textit{quoi} annoter (\textit{objectif, définition, valeurs possibles, ...}) alors que les spécifications décrivent \textit{comment} annoter (\textit{règles d'attribution, exemples et contre-exemple, règles de format, ...}).
			\begin{leftBarExamples}
				% Exemple littérature.
				\cite{perrotin-etal:2018:annotation-actes-dialogue}, s'intéressant à la classification des conversations d'assistance en ligne en actes de dialogues, décrit son guide d'annotation dans \cite{asher-etal:2017:manuel-annotation-actes}.
				On y retrouve (1) la modélisation avec la présentation des étiquettes possibles à annoter, et (2) les spécifications avec les définitions concrètes, des exemples, des restrictions d'attribution, et la gestion des données non pertinentes. \\
				% Exemple BD.
				Dans nos exemples précédents (cf. \textsc{Section~\ref{section:2.1.2.B-PRESENTATION-ANNOTATION-EXEMPLES-CLASSIFICATION}}), nous avions modélisé le problème de classification de l'état d'une bande dessinée en quatre classes : "\texttt{Mauvais état}", "\texttt{Bon état}", "\texttt{Très bon état}", "\texttt{Neuf}".
				Il faudrait désormais rédiger les spécifications avec des définitions concrètes et quelques exemples  pour guider un annotateur, notamment pour l'aider à distinguer "\texttt{Bon état}" de "\texttt{Très bon état}".
			\end{leftBarExamples}
			
			% Quelques points importants sur la modélisation et exemples.
			Bien entendu, il n'est pas toujours facile de modéliser un problème ni de rédiger un guide d'annotation adéquat.
			Nous reviendrons plus tard sur les caractéristiques de cette tâche pouvant introduire de la complexité (voir \textsc{Section~\ref{section:2.3-DEFIS-ANNOTATION}}), mais il est important de souligner d'emblée les points élémentaires suivants :
			\begin{itemize}
				\item le besoin d'\textit{inter-opérabilité} et de \textit{ré-utilisabilité} : un projet d'annotation est toujours un investissement coûteux, il serait donc regrettable de perdre ou de ne pas pourvoir ré-utiliser ces données après ce projet.
				Par conséquent, il faut réfléchir au format des données ainsi qu'aux types de détails à fournir pour être sûr de pouvoir toujours exploiter les données si la modélisation évolue légèrement ou si un futur projet désire en bénéficier ;
				\item la balance entre \textit{généralité} et \textit{spécificité} : le niveau de détail requis dépend sans conteste du problème à modéliser : annoter trop peu de détail ne permet pas d'exploiter les données, mais en annoter trop peut rapidement complexifier la tâche et introduire des erreurs.
				Il faut donc trouver le juste milieu pour réaliser un travail de qualité qui ne soit pas trop pénible.
			\end{itemize}
			\begin{leftBarExamples}
				Dans la classification de langue exposée en \textsc{Section~\ref{section:2.1.2.B-PRESENTATION-ANNOTATION-EXEMPLES-CLASSIFICATION}}, nous y avons annoté chaque texte grâce à trois classes : "\texttt{Français}", "\texttt{Anglais}" et "\texttt{Allemand}".
				\begin{itemize}
					% Exemple inter-opérabilité et de ré-utilisabilité.
					\item par soucis d'\textit{inter-opérabilité}, nous pourrions plutôt utiliser la norme ISO 639-3 (\cite{international-organization-for-standardization:2007:codes-representation-names}), soit les code "\texttt{fra}", "\texttt{eng}" et "\texttt{deu}", afin de standardiser l'annotation et ainsi pouvoir partager plus facilement les données labellisées avec d'autres projets ;
					% Exemple généralité et spécificité.
					\item afin de présenter un cas simple, nous avions proposé un modèle avec trois langues communes pour une bande dessinée d'origine belge.
					Toutefois, nous aurions pu \textit{spécialiser} davantage notre modèle en fonction des variations régionales en prenant en compte le Corse ("\texttt{cos}") ou le Wallon ("\texttt{wln}").
					Cette distinction peut être essentielle pour certaines saga publiées dans ces langues (comme \texttt{Astérix \& Obélix}), mais peut simplement être une source de confusion pour les autres (comme \texttt{Lucky Luke}).
				\end{itemize}
			\end{leftBarExamples}
			
			% Aide à la formalisation.
			\begin{leftBarInformation}
				Pour aider à concevoir le guide d'annotation et afin de se poser les bonnes questions, \cite{dipper-etal:2004:useradaptive-annotation-guidelines} dresse une liste de définitions et de recommandations à prendre en considération.
				Bien que ces conseils soient issus du traitement de données linguistiques, ils permettent d'identifier les sections importantes d'un guide d'annotation en fonction des attentes des différents acteurs de l'annotation (\textit{l'auteur, l'annotateur, l'explorateur de données, ...}) et de les rédiger en suivants certaines règles simples (\textit{introduire les objectifs, ordonner les règles par complexité, traiter en premier les cas par défaut, trier les valeurs des variables catégorielles par ordre alphabétiques, ...}).
				Des exemples reconnus pour leur bonne conception y sont notamment cités si vous avez besoin de référence pour concevoir votre propre guide.
			\end{leftBarInformation}
			
			
			%%% c. Annotation
			
			% Annotation en tant que telle.
			Lorsque le guide d'annotation est rédigé, la \textbf{phase de labellisation} peut commencer (voir \textsc{Figure~\ref{figure:2.2.1-ORGANISATION-ANNOTATION-ETAPES-CLES-MATTER}}, étape 2. \textit{\textbf{A}nnotate}).
			Cette tâche est traditionnellement réalisée par un groupe d'experts choisi en fonction de leur connaissance sur problème à caractériser (dans nos exemples sur les bandes dessinées, ce serait plutôt des libraires ou des collectionneurs).
			Après leur avoir expliqué l'objectif de leur travail et partagé les règles de labellisation contenues dans le guide, les annotateurs se partagent les données et réalisent chacun une partie du corpus d'apprentissage.
			
			%%% d. Mini-cycle MAMA.
			
			\begin{leftBarInformation}
				% La théorie rencontre le réel.
				C'est généralement à ce stade que la théorie rencontre la pratique : certaines règles d'annotation peuvent difficilement être applicables, certains données peuvent être ambiguës ou hors-sujet, et deux annotateurs peuvent aussi avoir des avis différents sur l'annotation la plus adéquate.
				Il est aussi important de rappeler que l’annotation est un acte d'interprétation, et que les données sont donc labellisées par un humain dont l'avis n'est pas infaillible. 
				\cite{pustejovsky-stubbs:2012:natural-language-annotation} introduisent donc le premier sous-cycle \texttt{MAMA} en référence à la boucle entre \textit{\textbf{M}odelize} et \textit{\textbf{A}nnotate} qui peut avoir lieu tant que le guide d'annotation n'est pas adapté aux données manipulées ou que différents points de vues opposent les annotateurs.
				
				% Exemple.
				Par exemple, lors de l'annotation de la transcription audio en \textsc{Section~\ref{section:2.1.2.D-PRESENTATION-ANNOTATION-EXEMPLES-TRANSCRIPTION}}, il peut y avoir une voix principale accompagnée de plusieurs voix en arrière plan : une première adaptation du guide serait de clarifier si ces voix secondaires doivent être transcrites ou ignorées, voire si l'audio entier doit être considéré comme inexploitable.
				La réponse à cette question dépend bien entendu du phénomène à décrire et de l'objectif du modèle de \textit{Machine Learning} à entraîner : dans notre cas, nous pourrions probablement annoter uniquement la voix principale et ignorer l'audio si le bruit gène la compréhension.
			\end{leftBarInformation}
			
			%%% Finalité : la base d'apprentissage.
			À la fin de l'annotation (ou du cycle \texttt{MAMA}), le corpus d'entraînement est disponible pour concevoir un modèle de \textit{Machine Learning}.
		
		
		%%% 2.2.1.B. Concevoir le modèle (\textit{\textbf{T}rain}, \textit{\textbf{T}est}, \textit{\textbf{E}valuate}).
		\subsubsection{Concevoir le modèle (\textit{\textbf{T}rain}, \textit{\textbf{T}est}, \textit{\textbf{E}valuate}).}
		\label{section:2.2.1.B-ORGANISATION-ANNOTATION-ETAPES-CLES-TRAIN-TEST}
			
			%%% Apprentissage statistiques et importance du test.
			La phase d'entraînement du modèle est l'étape centrale de l'apprentissage automatique.
			Toutefois, comme l'apprentissage se base sur des méthodes statistiques, il est important d'introduire une phase de test et d'évaluation pour s'assurer des performances du modèle obtenu.
			Il est donc courant de considérer une boucle de raffinement du modèle tant que les performances n'ont pas atteint un seuil acceptable (voir \textsc{Figure~\ref{figure:2.2.1-ORGANISATION-ANNOTATION-ETAPES-CLES-MATTER}}, étapes 3. \textit{\textbf{T}rain}, 4. \textit{\textbf{T}est} et 5. \textit{\textbf{E}valuate}).
			
			%%% Train/Dev/Test.
			En pratique, il est d'usage de \textbf{créer trois jeux de données} à partir de la base d'apprentissage qui vient d'être annotée :
			\begin{itemize}
				\item le jeu d'\texttt{entraînement}: c'est sur cette partie des données que le modèle de \textit{Machine Learning} est conçu ;
				\item le jeu de \texttt{développement} (ou de validation): le modèle entraîné est évalué sur ce jeu de donnée pour étudier son comportement, identifier ses forces et ses faiblesses, et ainsi permettre de le comparer à d'autres modèles entraînés pour cette même tâche ;
				\item le jeu de \texttt{test} : le modèle retenu est évalué sur ce jeu de test pour déterminer ses performances réelles. Il est important que ce jeu de données 
			\end{itemize}
			% [van-der-goot:2021:we-need-talk]: définir un train/tune/dev/test
			
			%%% Evaluate.
			Ainsi, le modèle représente la connaissance présente dans le jeu \texttt{entraînement}, il est étudié puis affiné grâce au jeu de \texttt{développement}, et est finalement jugé en fonction de ses performances sur le jeu de \texttt{test}.
			Il est encore une fois difficile d'être exhaustif sur les analyses et les métriques à considérer car elles dépendent fortement du type de problème que le modèle tente de résoudre.
			Une métrique basique est l'\texttt{Accuracy} (ou taux de bonne prédiction), décrivant simplement le nombre de fois que le modèle a fait une bonne proposition sur l'ensemble du test.
			Suivant le problème et le type de données, d'autres métriques usuelles peuvent être utilisées comme le \texttt{MSE} (\textit{Mean Squared Error}) pour la prédiction de variables numérique (voir \cite{wallach-goffinet:1987:mean-squared-error}), le \texttt{f1-score} pour les variables catégorielles (voir \cite{sasaki:2007:truth-fmeasure}) ou le \texttt{WER} (\textit{Word Error Rate}) pour la transcription de textes (voir \cite{mccowan-etal:2005:use-information-retrieval}).
			Dans tous les cas, une règle d'or est de bien tenir à l'écart le jeu de test des deux autres jeux de données et qu'il ne soit pas utilisé dans la phase de développement pour éviter tout biais de sur-apprentissage\footnote{Pour plus de détails sur le sur-apprentissage: voir \cite{collins:2017:chapter-overfitting}}.

			%%% Finalité : le modèle et se sperformances.
			À la fin de ce cycle, le modèle de \textit{Machine Learning} est disposition à l'emploi, et ses performances théoriques sont celles obtenues sur le jeu de test.
		
		%%% 2.2.1.C. Revoir la base d'apprentissage (\textit{\textbf{R}evise}).
		\subsubsection{Revoir la base d'apprentissage (\textit{\textbf{R}evise}).}
		\label{section:2.2.1.C-ORGANISATION-ANNOTATION-ETAPES-CLES-REVISE}
		
			% Besoin de réviser.
			Pour terminer cette boucle, il est parfois nécessaire d'envisager de corriger son modèle en remettant en cause la modélisation du problème et l'annotation des données.
			\cite{voormann-gut:2008:agile-corpus-creationa} formalisait en effet ce besoin de réviser la conception d'une base d'apprentissage en observant les lacunes du modèle obtenu, et \cite{pustejovsky-stubbs:2012:natural-language-annotation} évoque certaines révisions nécessaires de la modélisation dès la phase d'annotation (voir sous-cycle \texttt{MAMA} dans la \textsc{Figure~\ref{figure:2.2.1-ORGANISATION-ANNOTATION-ETAPES-CLES-MATTER}}).
			
			% Identifier un besoin de réviser.
			Divers pistes peuvent mener à une évolution de la base d'apprentissage :
			\begin{itemize}
				\item le modèle de \textit{Machine Learning} peut avoir de mauvaise performances, malgré son affinage lors de la phase de développement, ou peut manquer d'adaptabilité sur des données réelles ;
				\item la modélisation ou l'annotation peuvent devenir obsolète car le phénomène modélisé évolue dans le temps ;
				\item un cas d'usage non identifié jusqu'à présent nécessite de nouvelles données pour être pris en compte ;
				\item ou encore, un nouvel algorithme de \textit{Machine Learning} a priori plus performant requiert une modélisation différente pour traiter le problème.
			\end{itemize}
			\begin{leftBarExamples}
				% Exemple Inflation prix
				Pour illustrer nos propos, prenons la tâche d'estimation du prix d'une bande dessinée (cf. \textsc{Section~\ref{section:2.1.2.A-PRESENTATION-ANNOTATION-EXEMPLES-REGRESSION}}) : il se peut que les prix annoté sur les transactions ne soient plus d'actualité à cause de l'inflation, et que les données doivent être ré-annotées pour prendre en compte les nouvelles valeurs du marché.
				
				% Exemple ajouter une classe.
				D'autre part, la modélisation en tant que telle peut aussi être impacté : par exemple, dans le cadre de la classification de l'état d'une bande dessinée à partir d'une photo (cf. \textsc{Section~\ref{section:2.1.2.B-PRESENTATION-ANNOTATION-EXEMPLES-CLASSIFICATION}}), on pourrait constater à l'usage qu'il manque une catégorie "\texttt{Très mauvais état}" nécessaire pour trier d’emblée toute \texttt{BD} indigne à la vente.
				
				% Exemple OCR.
				Enfin, il est possible que le modèle se comporte mal sur certaines données.
				Par exemple lors de l'identification d'une bande dessinée à partir de sa couverture (cf. \textsc{Section~\ref{section:2.1.2.C-PRESENTATION-ANNOTATION-EXEMPLES-EXTRACTION}}), certains textes du décors pourrait être extrait à tord (comme le texte de la pancarte \textguillemets{\textit{Saloon}} dans la \textsc{Figure~\ref{figure:2.1.2.C-PRESENTATION-ANNOTATION-EXEMPLES-EXTRACTION}}).
				Il faudra peut-être adapter l'annotation pour identifier les textes à ne pas extraire (avec une classe de rebus par exemple).
			\end{leftBarExamples}
			
			% Conclusion.
			Nous bouclons ainsi le cycle \texttt{MATTER} qui préfigure le besoin d'une amélioration continue d'un modèle de \textit{Machine Learning} pour que celui-ci soit le plus adapté à son environnement d'utilisation.
			
	
	%%%
	%%% Subsection 2.2.2: Portraits des acteurs intervenant sur un projet d'annotation.
	%%%
	\subsection{Portraits des acteurs intervenant sur un projet d'annotation}
	\label{section:2.2.2-ORGANISATION-ANNOTATION-ACTEURS}
	
		% Introduction: grande diversité de métiers.
		Au cours du cycle \texttt{MATTER}, nous pouvons constater que divers acteurs interviennent pour concevoir la base d'apprentissage et entraîner un modèle de \textit{Machine Learning}.
		Cette diversité de métiers qui gravitent autour du traitement automatique des données semble difficile à détailler, tant à cause de leur grand nombre que de leurs subtiles différences.
		Pour avoir un aperçu, vous pouvez consulter les offres d'emplois du marché actuel (voir \cite{team-datascientest:2022:metiers-data-mieux} ou \cite{databird:2023:10-metiers-data}) ou certaines formations professionnelles (voir \cite{isoz:2017:decouvrir-metiers-data}) pour pouvoir faire la distinction entre \textit{data scientist}, \textit{data analyst}, \textit{data librarian}, \textit{data journalist}, \textit{data architect}, \textit{data engineer}, \textit{data steward}, \textit{data archivist}, ou encore \textit{machine learning engineer}...
		
		% Approche par compétences.
		Afin d'avoir une approche moins commerciale de ces métiers, nous proposons plutôt de dresser les compétences requises au diverses phases du cycle, à l'image de \cite{radovilsky-etal:2018:skills-requirements-business} qui présente les acteurs de la science des données grâce à quatre groupes de compétences :
		\begin{enumerate}
			% Business expert.
			\item les compétences \textbf{métiers} : elles sont liées aux connaissances et à l'expertise sur le phénomène à modéliser ou le problème à résoudre.
			Ce sont grâce à ces compétences qu'un acteur peut être apte à annoter une donnée ou à qualifier la pertinence de la prédiction d'un modèle de \textit{Machine Learning}.
			Les métier(s) associé(s) sont : l'\texttt{expert métier} (\textit{business expert}) ;
			% Data analyst.
			\item les compétences \textbf{analytiques} : elles concernent entre autres la modélisation du problème, la gestion des données, et les analyses statistiques sur les biais et les performances.
			Ce sont grâce à ses compétences qu'un acteur peut concevoir le guide d'annotation, estimer le taux d'accord inter-annotateurs, ou encore réaliser l'évaluation statistique d'un modèle de \textit{Machine Learning}.
			Les métier(s) associé(s) sont : l'\texttt{analyste des données} (\textit{data analyst}) ou le \texttt{scientifique des données} (\textit{data scientist}) ;
			% Data scientist.
			\item les compétences \textbf{techniques} : elles portent sur l'ingénierie autour du modèle de \textit{Machine Learning}, comme le choix du meilleur algorithme d'entraînement et réglage fin des hyper-paramètres, l'archivage des différentes versions du modèle ainsi que son déploiement dans un environnement de production.
			Les métier(s) associé(s) sont : le \texttt{scientifique des données} (\textit{data scientist}), l'\texttt{ingénieur en Machine Learning} (\textit{Machine Learning Engineer}) ou l'\texttt{architecte des des données} (\textit{data architecte}) ;
			% Projet leader.
			\item et les compétences en \textbf{gestion} ou en \textbf{communication} : elles permettent d'aborder le cadrage du projet et la définition des objectifs, ainsi que diverses aptitudes transverses comme l'établissement de rapports, la gestion de projet, la vérification des normes, ...
			Les métier(s) associé(s) sont : le \texttt{chef de projet} (\textit{project leader}) ou le \texttt{responsable de la protection des données} (\textit{data protection officer}).
		\end{enumerate}
		\begin{leftBarInformation}
			On peut compléter cette vision par compétences avec la vision donnée par \cite{fort:2017:experts-ou-foule}, selon laquelle il y a trois types d'experts lors d'un projet d'annotation :
			\begin{itemize}
				\item les \textbf{experts du corpus} de données, ayant par exemple les connaissances sur les bandes dessinées, s'approchant donc de compétences \texttt{métiers} ;
				\item les \textbf{experts de l'annotation}, ayant par exemple les connaissances sur l'annotation de textes dans une image, s'approchant donc de compétences \texttt{analytiques} ;
				\item et les \textbf{experts de la tâche} de \textit{Machine Learning}, ayant par exemple les connaissances sur les techniques d'\texttt{OCR}, s'approchant donc des compétences \texttt{techniques}.
			\end{itemize}
		\end{leftBarInformation}
		
		% Application au cycle \texttt{MATER}.
		Ainsi, durant le cycle \texttt{MATTER}, nous pouvons voir les compétences ci-dessus se compléter :
		\begin{enumerate}
			% Concevoir la base d'apprentissage.
			\item la conception de la \textbf{base d'apprentissage} (étapes \textit{\textbf{M}odelize} et \textit{\textbf{A}nnotate}) nécessite :
			\begin{itemize}
				\item des compétences de \texttt{gestion} pour cadrer l'objectif du modèle à entraîner, et ainsi définir l'objectif auquel doit répondre l'annotation de données ;
				\item des compétences \texttt{analytiques} pour proposer une modélisation stable du phénomène et un guide d'annotation précis pour limiter les biais de conception ;
				\item des compétences \texttt{métiers} pour vérifier que la proposition de modélisation est pertinente vis-à-vis du cas d'usage, mais aussi pour réaliser l'annotation des données.
			\end{itemize}
			% Concevoir le modèle.
			\item la conception du \textbf{modèle de \textit{Machine Learning}} (étapes \textit{\textbf{T}rain}, \textit{\textbf{T}est}, \textit{\textbf{E}valuate}) nécessite :
			\begin{itemize}
				\item des compétences \texttt{analytiques} pour gérer les jeux de données (\textit{entraînement, développement, test}) et évaluer les performances statistiques du modèle ;
				\item des compétences \texttt{techniques} pour manipuler l'écosystème de développement du modèle, régler les hyper-paramètres, versionner les changements, et planifier la distribution du modèle ;
				\item des compétences de \texttt{gestion} pour s'assurer du respect des normes et des caractères privée ou confidentielle de certaines données.
			\end{itemize}
			% Revoir la base d'apprentissage
			\item la \textbf{révision} de la base d'apprentissage (étape \textit{\textbf{R}evise}) nécessite :
			\begin{itemize}
				\item des compétences \texttt{métiers} pour identifier le manque de pertinence du modèle vis-à-vis de certains cas d'usage ;
				\item des compétences \texttt{analytiques} pour disserter des performances réelles du modèle face à des données de production et remettre en question les précédents choix de modélisation pour espérer améliorer le modèle.
			\end{itemize}
		\end{enumerate}
		
		% Remarque sur la distance entre métier et technique.
		On notera que les compétences transverses de \textbf{gestion} ou \textbf{communication} ne sont pas spécifiques à une étape du cycle \texttt{MATTER} (\textit{le cadrage, la gestion de projet et l'établissement de rapports étant réalisés tout au long du projet}), alors que les compétences \texttt{métier} et \texttt{techniques} n'interviennent généralement pas au même moment du cycle : autrement dit, des experts métiers croisent rarement des experts techniques et ne partagent donc que très rarement leurs connaissances.
	
	
	%%%
	%%% Subsection 2.2.3: Choix du logiciel d'annotation.
	%%%
	\subsection{Choix du logiciel d'annotation}
	\label{section:2.2.3-ORGANISATION-ANNOTATION-LOGICIELS}
	
		% Introduction: besoin d'un outil d'annotation.
		Pour terminer la description de l'organisation d'un projet d'annotation, attardons nous sur le choix du logiciel à utiliser pour labelliser les données.
		En effet, une diversité d'applications existent pour répondre aux besoins des annotateurs, mais il est important de noter que l'absence de certaines fonctionnalités essentielles risque de gêner le projet d'annotation, soit par l'introduction de biais, soit à cause de son manque d'inter-opérabilité avec d'autres tâches d'analyses ou d'annotation.
		
		% Liste des fonctionnalités importantes.
		Nous faisons référence à \cite{finlayson-erjavec:2016:overview-annotation-creation} pour dresser ci-dessous une liste des fonctionnalités principales (voire essentielles) d'un logiciel d'annotation.
		Pour simplifier la lecture, nous proposons de regrouper ces fonctionnalités dans les catégories suivantes :
		\begin{itemize}
			% Répondre à la \textbf{besoin d'annotation}.
			\item répondre à la \textbf{besoin d'annotation} :
				cette fonctionnalité est bien entendu obligatoire, car un logiciel ne permettant pas d'annoter vos données ne vous sera d'aucune utilité.
				Cette remarque semble être une évidence, mais nous nous permettons aussi d'étendre l’avertissement aux logiciels n'étant pas destinés à votre besoin d'annotation mais qui peuvent être détournés pour y répondre indirectement : de telles contournements peuvent introduire des biais et offrent généralement une expérience utilisateur assez médiocre ;
			% Intégrer le \texbf{guide d'annotation}.
			\item intégrer le \textbf{guide d'annotation} :
				ce livrable issu de la phase de modélisation du cycle \texttt{MATTER} doit être facilement accessible par les annotateurs car il contient la documentation et les instructions à appliquer lors de la labellisation.
				Les logiciels permettant d'intégrer directement ces définitions (\textit{avec exemples et contre-exemples}) ainsi que les règles d'annotation (\textit{comme les labels mutuellement exclusifs, les détails obligatoires, ...}) ont donc un net avantage ergonomique pour respecter la modélisation définie et ainsi garantir la qualité de la base d'apprentissage ; 
			% Autoriser l'\textbf{annotation multiple}.
			\item autoriser l'\textbf{annotation multiple} et l'\textbf{annotation multi-modale}: 
				il est fréquent de devoir annoter plusieurs une même donnée suivant des modélisations ou des paradigmes différents pour répondre à plusieurs cas d'usage (\textit{en prenant l'exemple de l'annotation d'images, on peut détourer les objets présents, identifier les textes inscrits, proposer une ou plusieurs catégories générales, proposer une description textuelle, ...}) ou encore de devoir annoter des données de natures différentes (\textit{en combinant texte, image et voix comme dans l'annotation des sous-titres d'une vidéo}).
				Les logiciels permettant ainsi de labelliser plusieurs informations et de manipuler plusieurs types de données sont donc plus facilement réutilisables et permettent de centraliser les annotations ;
			% Evaluer la \textbf{qualité de l'annotation}.
			\item évaluer la \textbf{qualité de l'annotation} :
				les erreurs d'annotation et les divergences d'opinions sur la modélisation sont inévitables.
				Il est donc appréciable de pouvoir les identifier, soit sur la base d'une comparaison directe entre deux annotateurs, soit en comparant avec l'annotation la plus probable issue d'une base de référence.
				Il peut aussi être intéressant de pouvoir calculer les scores d'accord inter-annotateurs sur un même échantillon de données pour estimer la qualité de la base d'apprentissage, de pouvoir corriger les erreurs lors de revues ou encore de trancher les cas de conflits apparent lors d'avis discordants ;
			% Permettre l'\textbf{inter-opérabilité} technique.
			\item permettre l'\textbf{inter-opérabilité} technique :
				il est toujours frustrant de ne pas pouvoir réutiliser des annotations d'un projet à l'autre car le format de stockage n'est pas compatible.
				Les logiciels prenant donc en considération plusieurs format de données (\textit{\texttt{PNG}/\texttt{JPG}, \texttt{MP3}/\texttt{WAV}, \texttt{XLSX}/\texttt{XML}/\texttt{JSON}, ...}) et respectant les standards de la tâche d'annotation lors des imports et exports sont donc à privilégier.
				De plus, il est conseillé de ne pas écrire directement les annotation dans les données (\textguillemets{\textit{\textbf{[Lucky Luke]}{\textcolor{colorDarkPastelRed}{\texttt{/(personnage)}}}, le \textbf{[cow-boy]}{\textcolor{colorDarkPastelPurple}{\texttt{/(métier)}}} solitaire, a ...}}), mais de les stocker dans des fichiers séparés pour garder une meilleure gestion et permettre les annotations multiples ;
			% Gérer le \textbf{flux de travail} et le \textbf{suivi de projet}.
			\item gérer le \textbf{flux de travail} et le \textbf{suivi de projet} :
				certaines fonctionnalités simples sont nécessaires à l'organisation de l'équipe d'annotation.
				Cela peut comprendre la répartition de la charge de travail, l'historisation des changements pour permettre les retours arrières, la possibilité d'émettre des appels d'aide ou d'écrire des commentaires sur les choix d'annotation, ou encore l'accompagnement des nouveaux annotateurs lors de leur monté en compétence ;
			% Favoriser le \textbf{confort de l'annotateur}.
			\item favoriser le \textbf{confort de l'annotateur} :
				le logiciel choisi sera utilisé au quotidien par l'équipe d'annotation, il semble donc essentiel de leur offrir une expérience utilisateur agréable pour réaliser leur tâche.
				Cela peut passer par une customisation de l'interface utilisateur afin d'être adapter à l'objectif d'annotation et par le paramétrage de raccourcis claviers.
				Simplifier l'accès et l'installation du logiciel peut aussi s'avérer utile pour favoriser son adoption, en favorisant par exemple les applications web permettant plus facilement le travail collaboratif ;
			% Permettre des \texbf{annotations et d'analyses avancées}.
			\item permettre des \textbf{annotations et d'analyses avancées} :
				la littérature scientifique regorge de techniques pouvant assister un annotateur dans son travail (\textit{pré-annotation, apprentissage actif, visualisation, interaction, ...}).
				Nous détaillerons plusieurs de ces techniques dans la \textsc{Section~\ref{section:2.4-AVANCEES-ANNOTATION}}.
		\end{itemize}
		
		% Quelques exemples.
		Considérant la diversité de cas d'usage d'annotation, une liste exhaustive des outils de labellisation n'est bien entendu pas possible.
		Nous tenons toutefois à présenter quelques exemples illustrés dans la \textsc{Figure~\ref{figure:2.2.3-ORGANISATION-ANNOTATION-LOGICIELS}}.
		%
		\begin{figure}[!htb]
			\centering
			\includegraphics[width=0.95\textwidth]{figures/etatdelart-logiciel-exemples}
			\caption{
				Quatre exemples d'outils d'annotation :
				\textbf{(1)} \texttt{INCEpTION} pour le texte (\cite{klie-etal:2018:inception-platform-machineassisted}),
				\textbf{(2)} \texttt{prodigy} pour le texte ou l'image (\cite{montani-honnibal:2017:prodigy-modern-scriptable}),
				\textbf{(3)} \texttt{Audacity} pour l'audio (\cite{audacity-team:2000:audacity-free-audio}
				et \textbf{(4)} \texttt{CVAT} pour l'image (\cite{cvat.ai-corporation:2019:computer-vision-annotation}).
			}
			\label{figure:2.2.3-ORGANISATION-ANNOTATION-LOGICIELS}
		\end{figure}
		
		% Avis sur la sur-utilisation d'Excel.
		\begin{leftBarAuthorOpinion}
			De part notre expérience, nous constatons malheureusement que plusieurs projets industriels n'utilisent pas ou peu d'outils d'annotation dédiés, et se contentent plutôt d'outils rudimentaires comme des traitement de textes ou des tableurs tels que \texttt{Microsoft Excel} (\cite{microsoft-corporation:2018:microsoft-excel}).
			Une étude serait à mener pour étudier cette tendance et expliquer le manque d'intérêt porté aux outils spécialement conçus pour les tâches d'annotations.
			Peut-être que ces derniers s'adaptent mal aux particularités des divers projets industriels, expliquant ainsi l'utilisation d'outils simplistes mais flexibles. Ou alors est-ce par méconnaissance des difficultés et des bais potentiels de l'annotation que ces outils aux fonctionnalités avancées ne sont pas employés ?
		\end{leftBarAuthorOpinion}
	
	%%%
	%%% Conclusion.
	%%%
	\begin{leftBarSummary}
		\begin{todolist}
			\item[\itemok] Un projet d'annotation s'organise généralement en cycle (\texttt{MATTER}) au cours duquel nous réalisons une modélisation abstraite des données que nous formalisons dans un guide (\textit{\textbf{M}odelize}), nous appliquons ce guide pour labelliser notre base d'apprentissage (\textit{\textbf{A}nnotate}), puis nous entraînons et testons un modèle de \textit{Machine Learning} (\textit{\textbf{T}rain}, \textit{\textbf{T}est} et \textit{\textbf{E}valuate}). Ensuite, l'évaluation du modèle peut mener à une révision de la modélisation des données en fonction des performances obtenues (\textit{\textbf{R}evise}) ;
			\item[\itemok] Un tel projet d'annotation nécessite une diversité de connaissances et de compétences qui peuvent être réparties en quatre catégories : \texttt{métier}, \texttt{analytique}, \texttt{technique} et \texttt{gestion/communication} ;
			\item[\itemok] Un tel projet nécessite aussi un outil d'annotation dédié possédant certaines fonctionnalités essentielles comme la possibilité d'\texttt{intégrer le guide} d'annotation, le besoin de \texttt{contrôler la qualité} des annotations, la capacité à réaliser des \texttt{annotation multiples ou multimodales}, ou encore l'assimilation d'éléments de \texttt{gestion de projet}.
		\end{todolist}
	\end{leftBarSummary}
	
	
	%%%%%--------------------------------------------------------------------
	%%%%% Section 2.3: Aperçu des nombreux défis de l'annotation
	%%%%%--------------------------------------------------------------------
	%\newpage
	%\vspace{2cm}
	\section{Les nombreux défis de l'annotation}
\label{section:2.3-DEFIS-ANNOTATION}

	%%%
	%%% Introduction: annoncer la complexité due (1) aux données (2) à la tâche et (3) aux humains.
	%%%
	
	Comme nous avons pu l'apercevoir dans les sections précédentes, le cycle d'annotation recèle de zones d'ombres pouvant introduire des complications dans la conception d'une base d'apprentissage (\cite{baledent:2022:complexite-annotation-manuelle}).
	Pour aborder cette partie, nous alors voir :
	\begin{itemize}
		\item qu'il y a une forte pression sur la \textbf{qualité des données} devant constituer le corpus d'entraînement (cf. \textsc{Section~\ref{section:2.3.1-DEFIS-ANNOTATION-ASPECT-DONNEES}}) ;
		\item que ce standard de qualité entretient une \textbf{complexité inhérente} aux étapes de modélisation et d'annotation (cf. \textsc{Section~\ref{section:2.3.2-DEFIS-ANNOTATION-ASPECT-COMPLEXITE}}) ;
		\item et que cette complexité provoque des \textbf{différences de comportements} chez les annotateurs (cf. \textsc{Section~\ref{section:2.3.3-DEFIS-ANNOTATION-ASPECT-HUMAIN}}).
	\end{itemize}
	En détaillant chacun de ces trois points, nous discuterons de l'ensemble de techniques et bonnes pratiques mises en avant dans la littérature pour limiter les désagréments d'un projet d'annotation.
	Nous identifierons aussi les freins récurant pouvant intervenir dans les mises en application industrielles.
	
	
	%%%
	%%% Subsection 2.3.1: Défis concernant le besoin de qualité des données.
	%%%
	\subsection{Défis concernant le besoin de qualité des données}
	\label{section:2.3.1-DEFIS-ANNOTATION-ASPECT-DONNEES}
	
		% Introduction: Machine Learning = reproduire par l'exemple.
		Comme nous l'avons défini en \textsc{Section\ref{section:2.1.1.A-PRESENTATION-ANNOTATION-DEFINITION-MACHINE-LEARNING}}, l'\textguillemets{\texttt{apprentissage automatique}} regroupe un ensemble de techniques dont l'objectif est de reproduire une tâche \textbf{par l'exemple} : il est donc normal de porter une attention particulière aux données utilisées, car la qualité du modèle de \textit{Machine Learning} va fortement dépendre de la qualité de sa base d'apprentissage.
		Nous allons ici détailler trois défis actuels concernant cette création d'un jeu de données.
		
		
		%%% 2.3.1.A. Problèmes de représentativité.
		\subsubsection{Problèmes de représentativité}
		\label{section:2.3.1.A-DEFIS-ANNOTATION-ASPECT-DONNEES-REPRESENTATIVITE}
			
			% Introduction : Une collecte doit être "représentative" du problème.
			La phase de collecte de données est une étape importante du projet d'annotation.
			Malheureusement, la littérature scientifique associée à cette tâche est assez légère alors que c'est précisément à ce moment que ce joue une caractéristique cruciale de la future base d'apprentissage : la \textbf{représentativité} du phénomène à modéliser.
			
			% Définir la représentativité à partir de la méthode de l'échantillon.
			Cette notion est assez ambiguë, notamment car le terme technique \textguillemets{représentatif}fait écho à un mot de la vie courante qui peut avoir plusieurs sens.
			Dans \cite{kruskal-mosteller:1979:representative-sampling-nonscientific} et \cite{clemmensen-kjaersgaard:2022:data-representativity-machine}, plusieurs usages communs de ce terme sont recensés :
			\begin{itemize}
				\item \textguillemets{\textit{assertive claim}} :
				l'opérateur déclare que ses données sont représentatives du problème sans apporter d'arguments. Bien entendu, cette option est à bannir car elle n'apporte aucune information et peut cacher des vices de conception du modèle ;
				\item \textguillemets{\textit{absence or presence of selective force}} :
				la représentativité du phénomène est supposée en sélectionnant des données de \textbf{manière aléatoire} et en limitant le nombre de critères de sélection à ceux nécessaires pour l'étude réalisée ; 
				\item \textguillemets{\textit{miniature}} :
				aussi appelée \textbf{sélection stratifiée}, cette approche consiste à dire qu'un échantillon est représentatif d'un phénomène si la proportion de chacune de ses parties y est respectée
				(\textit{par exemple, un sondage peut respecter la répartition des tranches d'âge d'une population}) ;
				\item \textguillemets{\textit{typical/ideal case}} :
				cette définition consiste à représenter chaque partie du phénomène par un \textbf{exemple emblématique} ou un \textbf{exemple moyen}
				(\textit{par exemple, on peut illustrer l'univers de la bande dessinée française par un numéro de la saga \texttt{Astérix \& Obélix}}) ;
				\item \textguillemets{\textit{population coverage}} :
				ici, la représentativité est associée à la présence \textbf{exhaustive} de l'ensemble des caractéristiques importantes du phénomène avec au moins un exemple par caractéristique
				(\textit{par exemple, dans l'arche de Noé, il y devait y avoir au moins un couple de chaque animaux}).
			\end{itemize}
			
			% Définir la représentativité à partir de la méthode d'échantillonnage.
			Pour compléter ces définitions, \cite{kruskal-mosteller:1979:representative-sampling-ii} incitent sur le besoin de \textbf{définir avec précision la méthode d'\textit{échantillonnage}} plutôt que l'\textit{échantillon} lui-même : en effet, il est important de caractériser le phénomène à modéliser, de définir l'objectif de la collecte de données et de détailler comment cette collecte va être réalisée.
			\cite{clemmensen-kjaersgaard:2022:data-representativity-machine} introduisent à leur tour trois mesures pour aider à caractériser une collecte : la \textit{réflexion} (est-ce l'échantillon respecte la distribution de la population ?), la \textit{couverture} (est-ce que l'échantillon illustre la diversité de la population ?) et la \textit{présence de représentants} (est-ce que l'échantillon contient les exemples emblématiques de la population ?).
			De telles informations sont essentielles pour pouvoir juger de la valeur d'un échantillon par rapport à un cas d'usage et déterminer s'il est réutilisable dans pour une autre application.
			%
			\begin{leftBarExamples}
				% Cas d'usage : classification d'état.
				Illustrons nos propos avec la classification de l'état d'une bande dessinée à partir d'une photo (voir \textsc{Section~\ref{section:2.1.2.B-PRESENTATION-ANNOTATION-EXEMPLES-CLASSIFICATION}}).
				Afin de représenter correctement le cas d'usage, nous pourrions collecter des exemples de \texttt{BD} couvrant l'ensemble des dégradations fréquemment identifiées par les libraires (couvertures froissées, pages déchirées, couleurs délavées, ...) et les intégrer de manière proportionnelle dans la base d'apprentissage.
				Nous pourrions aussi nous assurer de la présence de cas emblématiques permettant de catégoriser les \texttt{BD} en "\texttt{Mauvais état}", "\texttt{Bon état}", "\texttt{Très bon état}", "\texttt{Neuf}".
				
				% Autre cas d'usage : pas représentatif.
				Toutefois, cette base d'apprentissage ne sera plus représentative si nous voulons détecter la langue de la bande dessinée (auquel cas, une répartition par langue sera a priori plus adéquate).
			\end{leftBarExamples}
			
			% Besoin de beaucoup d'exemples.
			La description d'un phénomène reste cependant une \textbf{tâche difficile}, notamment lorsque que celui-ci possède un ensemble vaste et abstrait de caractéristiques à décrire.
			Nous comptons généralement sur la loi des grands nombres pour espérer dresser un portrait fidèle du phénomène, mais cela impose parfois de traiter des volumes colossaux de données.
			%
			\begin{leftBarExamples}
				Considérons le traitement du langage : le vocabulaire employé peut concerner des dizaines de millier de mots, il existe des variantes régionales et divers jargons techniques, certains termes peuvent avoir plusieurs sens et des expressions peuvent dépendre de leur contexte (comme l'humour ou les critiques).
				Pour représenter ces spécificités (listées de manière non exhaustive), une base d'apprentissage devra contenir de nombreux exemples afin de capturer les différents aspects du langage à traiter.
				On peut citer par exemple \texttt{MLSUM: The Multilingual Summarization Corpus} (\cite{scialom-etal:2020:mlsum-multilingual-summarization}), une base de $1.5$ millions d'articles de journaux sur $5$ langues pour entraîner un modèle de résumé automatique, ou encore \texttt{The Multilingual Amazon Reviews Corpus} (\cite{keung-etal:2020:multilingual-amazon-reviewsa}), une base de $1.26$ millions de commentaires de produits sur $6$ langues pour entraîner un modèle de classification de la note d'un commentaire sur $5$ étoiles.
			\end{leftBarExamples}
			
			% Discussion sur les biais de sur-représentativités ou sous-représentativités.
			Toutefois, la masse de données ne résout pas toujours tous les problèmes de représentativité.
			Une des difficultés récurrentes concerne les aspects peu fréquents d'un phénomène qui se retrouvent ainsi \textbf{sous-représentés} : si l'enjeu du modèle à concevoir consiste justement à détecter ou reproduire ces aspects, il peut être intéressant de volontairement biaiser les proportions du corpus d'entraînement pour mieux les illustrer.
			À l'inverse, des cas communs ou fréquents peuvent être \textbf{sur-représentées} : il est parfois nécessaire de limiter leur occurrence dans le corpus d'apprentissage pour ne pas concevoir un modèle véhiculant des généralités ou des stéréotypes.
			Dans les deux cas, \textbf{toute intervention va introduire un biais} : l'opération doit donc être réfléchie et judicieusement réalisée pour contribuer à la finalité du modèle, d'où l'intérêt de bien la documenter pour faire entendre ce que vous voulez signifier par \textguillemets{échantillon représentatif}.
			%
			\begin{leftBarExamples}
				% Problème de sous-représentation: Détecter la langue latine ou la langue corse.
				D'une part, considérons le besoin de détecter la langue d'une bande dessinée (voir \textsc{Section~\ref{section:2.1.2.B-PRESENTATION-ANNOTATION-EXEMPLES-CLASSIFICATION}}).
				Il est fort probable que la base d'apprentissage contiennent peu de données sur les parutions en langues régionales (en Corse, en Wallon, en Alsacien, ...).
				Nous serons peut-être amener à ajouter des exemples supplémentaires pour espérer mieux les détecter et ainsi augmenter la \textit{couverture} de notre jeu de données.
				
				% Problème d'équilibrage: Stéréotypes de Stable Diffusion.
				D'autre part, regardons l'analyse du modèle de \texttt{Stable Diffusion} réalisée par \cite{nicoletti-bass:2023:generative-ai-takes} sur la génération de portraits de personnes fictives à partir d'une description textuelle de leur métier.
				L'étude montre que le modèle tant générer des portraits d'hommes à la peau blanche pour le métier d'architecte ou d'ingénieur, des femmes pour le rôle de concierge ou encore des personnes à la peau noire pour illustrer la classe ouvrière.
				Ici, ce modèle dépeint les inégalités de notre société en représentant ses stéréotypes, et cela ouvre la question suivante : veut-on vraiment reproduire à l'identique cette représentation ?
			\end{leftBarExamples}
			%
			\begin{leftBarIdea}
				% Équilibrage par la génération de données.
				Une piste pour équilibrer efficacement les corpus d'entraînement et permettre de corriger leurs biais consiste à utiliser des \textbf{données synthétiques} (\cite{jaipuria-etal:2020:deflating-dataset-bias}).
				Ces données peuvent être créées manuellement ou être générées automatiquement (voir \cite{shorten-etal:2021:text-data-augmentation} pour une revue de génération de textes et \cite{shorten-khoshgoftaar:2019:survey-image-data} pour la génération d'image).
				Bien entendue, une telle approche doit restée réfléchie pour ne pas introduire davantage de biais et pour répondre à un objectif précis d'équilibrage du jeu de données.
			\end{leftBarIdea}
		
		
		%%% 2.3.1.B. Problèmes de bruits.
		\subsubsection{Problèmes de bruits}
		\label{section:2.3.1.B-DEFIS-ANNOTATION-ASPECT-DONNEES-BRUITS}
		
			% Introduction : données bruités.
			La qualité d'une base d'apprentissage dépend fortement du bruit qu'elle contient.
			Ce bruit est inévitablement inséré lors de la collecte :
			d'une part, la méthode de collecte elle-même peut en introduire (\textit{instrument de mesure faillible, erreur humaine, ...}) ;
			d'autre part, par soucis de représentativité, le bruit intrinsèque du phénomène va être capturé (\textit{forte variabilité, présence d'irrégularité, ...}).
			Un échantillon de données va donc forcément devoir se confronter 

			% Classification des types de bruits.
			En s'inspirant de \cite{maharana-etal:2022:review-data-preprocessing} et de \cite{alasadi-bhaya:2017:review-data-preprocessing}, nous dressons la liste suivante de problèmes récurrents sur les données suite à une collecte :
			\begin{itemize}
				% données non pertinentes
				\item présence de \textbf{données non pertinentes} par rapport au cas d'usage ;
				% variations parasites
				\item dégradation des données par des \textbf{variations parasites} ;
				% absence de valeurs
				\item \textbf{absence de valeurs} descriptives essentielles ;
				% incohérences / ambiguïté
				\item présence d'\textbf{incohérences} ou d'\textbf{ambiguïté} entre les données ;
			\end{itemize}
			%
			\begin{leftBarExamples}
				% Problèmes: estimation du prix d'une BD.
				Pour illustrer ces problèmes, considérons la tâche d'estimation du prix d'une bande dessinée (voir \textsc{Section~\ref{section:2.1.2.A-PRESENTATION-ANNOTATION-EXEMPLES-REGRESSION}}) :
				\begin{itemize}
					% Exemple: données non pertinentes.
					\item une donnée concernant le prix d'un roman ou d'une encyclopédie avoir été inséré par mégarde et pourrait être considéré comme données non pertinentes pour ce cas d'usage ;
					% Exemple: variations parasites.
					\item un changement de typographie (\textit{majuscules, minuscules, accents, ponctuation}) dans l'écriture d'un titre pourrait mal identifier une bande dessinée ;
					% Exemple: absence de valeurs.
					\item une information peut ne pas avoir été renseigné lors d'une transaction (l'année d'édition par exemple), alors que c'est une caractéristique importante de la prise de décision ;
					% Exemple: incohérences / ambiguïté.
					\item une même bande dessinée (avec les mêmes caractéristiques) peut avoir été vendue à deux prix radicalement différent, introduisant ainsi une légère ambiguïté dans les données.
				\end{itemize}
				
				% Autres problèmes.
				Des problèmes similaires peuvent impacter le traitement du texte (\textit{fautes de grammaticales ou syntaxiques, ambiguïtés ou sémantiques, omissions, ...}), des images (\textit{flous, mauvais cadrages, colorimétries gênantes, ...}) et de l'audio (\textit{bruits en arrière plan, saturations, coupures inopportunes, ...}).
				% Problème d'image flou: classification de l'état d'une BD.
				Par exemple, dans la tâche de classification de l'état d'une bande dessinée à partir d'une photo (voir \textsc{Section~\ref{section:2.1.2.B-PRESENTATION-ANNOTATION-EXEMPLES-CLASSIFICATION}}), comment juger correctement de l'état d'une \texttt{BD} si la photo capturée est flou ?
			\end{leftBarExamples}
			
			% Correction des données par le prétraitement et le nettoyage.
			Pour limiter l'impact du bruit dans les données, \cite{alasadi-bhaya:2017:review-data-preprocessing} structurent les étapes de prétraitement de données entre quatre catégories :
			\begin{itemize}
				% 1. data cleaning: completer les données (mean, random, model, constante) et corriger le bruit des données (prétraitement, regroupement pour numérique).
				\item le \textbf{nettoyage des données} : cette étape consiste à compléter les données manquantes (\textit{en prenant la valeur moyenne par exemple}), à filtrer les données aberrantes ou inintéressantes, et surtout à lisser le bruit dans les données en gommant les variations parasites ;
				% 2. data integration: associer plusieurs sources de données pour les rendre consistente
				\item l'\textbf{intégration des données} : dans certains cas, plusieurs sources de données sont disponibles, il peut être donc intéressant de croiser ces sources de données pour augmenter la consistance de la base d'apprentissage et identifier les incohérences ;
				% 3. data transformation: transformer les données pour les exploiter plus facilement (en les lisant, les normalisant, ...)
				\item le \textbf{formatage des données} : pour exploiter facilement les données, certaines transformations sont parfois nécessaires pour limiter les ambiguïtés dues au leur format (\textit{par exemple: normaliser une valeur entre $0$ et $1$});
				% 4. data reduction: réduire la dimensionnalité et les features peu utiles
				\item la \textbf{réduction des données} : en réalisant une analyse approfondie des données, on peut quelques fois constater que certaines caractéristiques présentes sur les données sont peu utiles et peuvent être supprimer pour réduire la complexité de la base d'apprentissage.
			\end{itemize}
			\cite{baledent:2022:complexite-annotation-manuelle} rappelle néanmoins que le document source (ou ici : la donnée brute) doit rester accessible pour la phase d'annotation pour ne pas manquer d'information potentiellement intéressante.
			%
			\begin{leftBarExamples}
				% Correction: estimation du prix d'une BD.
				Reprenons les problèmes évoqués précédemment sur la tâche d'estimation du prix d'une bande dessinée :
				\begin{itemize}
					% Correction: données non pertinentes.
					\item une donnée inintéressante peut simplement être supprimée ;
					% Correction: variations parasites.
					\item normaliser les champs décrivant une bande dessinée en passant tout en minuscules limiterait les chances de mal l'identifier ;
					% Correction: absence de valeurs.
					\item une année d'édition manquante pourrait être identifiée par l'étiquette \texttt{inconnue} et un prix manquant pourrait être complété par la moyenne du prix des \texttt{BD} ayant les mêmes caractéristiques ;
					% Correction: incohérences / ambiguïté
					\item une prix faussé pourrait être identifié comme incohérent en analysant les prix des \texttt{BD} ayant les mêmes caractéristiques ;
				\end{itemize}
			\end{leftBarExamples}
		
		
		%%% 2.3.1.C. Problèmes d'exploitation et de diffusion.
		\subsubsection{Problèmes d'exploitation et de diffusion}
		\label{section:2.3.1.C-DEFIS-ANNOTATION-ASPECT-DONNEES-DROITS}
		
			% Introduction: toutes les données ne sont pas disponibles.
			En plus des difficultés techniques sur la réalisation d'une collecte de données, il y a aussi contraintes législatives et stratégiques à prendre en compte.
			
			% Contraintes sur la propriété intellectuelle des données.
			D'une part, il faut considérer le fait que certaines données sont protégées et ne peuvent pas être collectées ou exploitées librement.
			C'est le cas de données soumises aux droits de \textbf{propriétés intellectuelles} qui empêchent cet usage : on peut citer par exemple \cite{loignon:2023:ia-medias-francais} qui évoque le levé de bouclier des médias français contre l'utilisation de leur article pour entraîner des modèles de langues, mais aussi \cite{les-echos:2023:ia-auteur-game} qui questionne la violation du \textbf{droit d'auteur} lorsque qu'un modèle est entraîné sur l'oeuvre d'un artiste et qu'il est capable de la reproduire.
			
			% Contraintes sur le consentement.
			Ces limites concernent aussi la Réglementation Générale européenne sur la Protection des Données (\texttt{RGPD}, \cite{european-commission:2016:regulation-eu-2016}) restreignant les \textbf{collectes et usages non consenties} de données personnelles.
			Ainsi, il n'est pas possible d’entraîner n'importe quel modèle sur n'importe quelles données, et une telle contrainte impose de manipuler les données en garantissant l'anonymat et la confidentialité des personnes consentantes.
			
			% Contraintes stratégiques.
			Pour aller plus loin, cette notion de confidentialité touche les données personnelles, mais aussi le \textbf{caractère stratégique} d'une organisation.
			En effet, dans le monde académique, les données manipulées sont le plus souvent publiques et peuvent être employées pour contribuer à la recherche scientifique.
			Mais dans le secteur industriel, les jeux de données sont liés au domaine d'activité de l'entreprise : ils ont généralement requis un investissement conséquent en temps et en moyens, et ils représentent donc son avantage concurrentiel (\textit{par leur spécificité, leur caractère secret ou novateur, leur qualité compétitive, ...}).
			Il est donc rare de voir une entreprise partager ses jeux de données car elle pourrait perdre un de ses atouts stratégiques.
			
			% Exemple OpenSource, mais limitation commerciale, donc beaucoup de recréation.
			Un solution est de ce tourner les jeu de données \textbf{accessibles en \textit{Open Source}}.
			Plusieurs plateformes mettre en effet à disposition des données ou des modèles, comme \textit{Hugging Face} (\cite{hugging-face:2016:hugging-face-ai}) ou \texttt{Zenodo} (\cite{re3data.org:2013:zenodo}).
			Toutefois, deux limites subsistent à l'usage de ces données :
			\begin{itemize}
				\item les données mises à dispositions publiquement sont souvent assez générales et ne reflète pas la spécificité des cas d'usage de l'entreprise, limitant ainsi leur intérêt ;
				\item les données publiques ne sont pas forcément ouvertes à un usage commercial (\textit{elles peuvent par exemple employé la licence \texttt{CC BY-NC 4.0}, \cite{creative-commons:2013:cc-bync-legal}}), restreignant ainsi les seules applications au domaine de la recherche et veille scientifique.
			\end{itemize}
			Pour ne pas faire de faux pas juridique, \cite{rajbahadur-etal:2022:can-use-this} propose une approche pour vérifier si une licence permet d'exploiter un jeu de données.
			
			% Besoin de tracabilité.
			\begin{leftBarInformation}
				Pour terminer nous mentionnons aussi une proposition de législation européenne concernant la futures réglementation des modèles d'intelligence artificielle (\texttt{IA Act}, \cite{european-commission:2021:proposal-regulation-european}).
				Cette loi concerne les quatre objectifs suivants :
				\begin{itemize}
					\item \textguillemets{veiller à ce que les systèmes d'\texttt{IA} mis sur le marché de l'Union et utilisés soient sûrs et respectent la législation en vigueur en matière de droits fondamentaux et les valeurs de l'Union} ;
					\item \textguillemets{garantir la sécurité juridique pour faciliter les investissements et l'innovation dans le domaine de l'\texttt{IA}} ;
					\item \textguillemets{renforcer la gouvernance et l'application effective de la législation existante en matière de droits fondamentaux et des exigences de sécurité applicables aux systèmes d'\texttt{IA}} ;
					\item \textguillemets{faciliter le développement d'un marché unique pour des applications d'\texttt{IA} légales, sûres et dignes de confiance, et empêcher la fragmentation du marché}.
				\end{itemize}
				Un besoin de traçabilité des données et des modèles se fait donc sentir, renforçant les recommandations à documenter et détailler les traitement et choix pour garantir la représentativité et la qualité des données des bases d'apprentissage.
			\end{leftBarInformation}
		
		
	%%%
	%%% Subsection 2.3.2: Défis concernant la complexité inhérente à la tâche d'annotation.
	%%%
	\subsection{Défis concernant la complexité inhérente à la tâche d'annotation}
	\label{section:2.3.2-DEFIS-ANNOTATION-ASPECT-COMPLEXITE}
		\todo[inline]{SECTION: À RÉDIGER}
		
		%%% 2.3.2.A. Estimation de la complexité.
		\subsubsection{Estimation de la complexité}
		\label{section:2.3.2.A-DEFIS-ANNOTATION-ASPECT-COMPLEXITE-ESTIMATION}
		
			\todo[inline]{citer et détailler [gut-bayerl:2004:measuring-reliability-manual]: juste le score IAA}
			\todo[inline]{citer et détailler [fort-etal:2012:modeling-complexity-manual]: 6 points}
			
			\todo[inline]{quelques solutions: approche non supervisée ou semi-supervisée (apprentissage actif, approche itérative)}
		
		%%% 2.3.2.B. Problèmes de coûts.
		\subsubsection{Problèmes de coûts}
		\label{section:2.3.2.B-DEFIS-ANNOTATION-ASPECT-COMPLEXITE-COUTS}
		
			\todo[inline]{liste des coûts: temps, argents, recrutement, ...}
			\todo[inline]{conséquence: charge de travail élevée}
			\todo[inline]{quelques solutions: transfert d'apprentissage, pré-annotation, apprentissage actif}
		
		
	%%%
	%%% Subsection 2.3.3: Défis concernant les différences de comportements d'annotation.
	%%%
	\subsection{Défis concernant les différences de comportements d'annotation}
	\label{section:2.3.3-DEFIS-ANNOTATION-ASPECT-HUMAIN}
		\todo[inline]{SECTION: À RÉDIGER}
		
		%%% 2.3.3.A. Différences inter-annotateurs.
		\subsubsection{Différences inter-annotateurs}
		\label{section:2.3.3.A-DEFIS-ANNOTATION-ASPECT-HUMAIN-INTER-ANNOTATEURS}
		
			\todo[inline]{subjectivité}
			\todo[inline]{expertise et formation}
			
			\todo[inline]{quelques solutions palliatives: évaluation, redondances}
			\todo[inline]{quelques solutions de fonds: revues, subdiviser la tâche, guide, crowd-sourcing}
		
		%%% 2.3.3.B. Différences intra-annotateurs.
		\subsubsection{Différences intra-annotateurs}
		\label{section:2.3.3.B-DEFIS-ANNOTATION-ASPECT-HUMAIN-INTRA-ANNOTATEURS}
		
			\todo[inline]{régulation charge de travail}
			\todo[inline]{troubles}
			\todo[inline]{travail mal reconnu voire ingrat}
			\todo[inline]{quelques solutions: guide, revues, limiter le changement de contexte, gamification}
	
	
	%%%
	%%% Conclusion.
	%%%
	\begin{leftBarSummary}
		\begin{todolist}
			\item[\itemok] L'enjeu d'un projet d'annotation consiste à avoir des \textbf{données de qualité} qui soient représentatives du problème à traiter ;
			\item[\itemok] Or la tâche d'annotation et son exigence de qualité engendre de la \textbf{complexité}, et donc une \textbf{charge de travail élevée} ;
			\item[\itemok] Pour réguler cette charge de travail élevée, chaque opérateur va \textbf{adapter sa tâche} pour la rendre supportable, créant alors des \textbf{différences de comportement}.
		\end{todolist}
	\end{leftBarSummary}
	
	
	%%%%%--------------------------------------------------------------------
	%%%%% Section 2.4:
	%%%%%--------------------------------------------------------------------
	%\newpage
	%\vspace{2cm}
	\section[
	Contexte du doctorat: comment assister la conception d'une base d'apprentissage ?
]{
	Contexte du doctorat: comment assister la conception d'une base d'apprentissage pour un agent conversationnel bancaire en français ?
}
\label{section:2.4-CONTEXTE-DOCTORAT}

	% Introduction: Domaine d'application au NLP / chatbot.
	Durant ce doctorat, nous nous sommes intéressé à la \textbf{conception d'assistants conversationnels} (\textit{chatbot}).
	En effet, leur utilisation en entreprise est de plus en plus courante (\cite{goasduff:2019:chatbots-will-appeal}, \cite{costello-lodolce:2022:gartner-predicts-chatbots}), notamment pour l'automatisation de certaines tâches simples et l'accès aux informations de bases documentaires.
	La popularité de ces assistants vient entre autres de la possibilité de dialoguer directement avec la machine grâce à des requêtes exprimées en langage naturel, offrant ainsi un gain de confort, de disponibilité et de performance.
	
	% Comment créer un chatbot ?
	Par expérience, nous avons constaté que plusieurs critères sont nécessaires pour déployer un assistant conversationnel dans un contexte industriel :
	\begin{itemize}
		\item Il faut être capable de gérer le dialogue entre l'utilisateur et l'assistant (\textit{comprendre la requête initiale, demander ou confirmer des informations complémentaires, ...}) ;
		\item Il faut être capable de contrôler le contenu des réponses de l'assistant et de s'assurer de ses performances (\textit{répondre ou agir de manière adaptée, ne pas fournir de réponses contenant des informations confidentielles, ne pas répondre de manière indécente, ...}) ;
		\item Il est possible de donner à l'assistant l'accès à certaines ressources (\textit{lire et écrire en base de données, exécuter des programmes tiers, ...}) mais il faut alors en garantir un certain niveau de sécurité (\textit{se prémunir contre les requêtes malveillantes et les erreurs de manipulations}).
	\end{itemize}
	
	Pour toute ces raisons, \textbf{l'architecture traditionnelle des \textit{chatbot} est plutôt orientée par tâches} (\textit{task-oriented}), c'est-à-dire qu'elle manipule une abstraction du dialogue en intentions \footnote{
		Intention de dialogue : en traitement automatique du langage naturel, une intention représente la compréhension de la demande formulée par un utilisateur au cours de la conversation.
		Elle est généralement définie par le verbe d'action de la demande, et est représentée par une étiquette.
		Par exemple, les requêtes \textguillemets{\textit{joue moi du jazz s'il te plaît !}} ou \textguillemets{\textit{peux-tu lancer une playlist de Noël sur l'enceinte du salon !}} peut être modélisées par l'intention \texttt{jouer\_musique}.
		Pour plus d'information, consulter l'\textsc{Annexe~\ref{annex:B-ANNEXE-CHATBOTS}}.
	} et gère un paramétrage des réponses dépendant des intentions détectées (voir \cite{chen-etal:2017:survey-dialogue-systems} et \cite{brabra-etal:2022:dialogue-management-conversational}).
	
	\begin{leftBarInformation}
		L'\textsc{Annexe~\ref{annex:B-ANNEXE-CHATBOTS}} détaille plus amplement les différences entre les assistants \textit{task-oriented} (approches symboliques) et les assistants \textit{chat-oriented} (approches numériques ou génératives).
		Cette annexe se base notamment sur une revue des architectures de conceptions publiée par \cite{chen-etal:2017:survey-dialogue-systems}.
	\end{leftBarInformation}
	
	
	% Conception difficile, besoin d'experts !
	Néanmoins, l'élaboration de tels assistants reste un \textbf{défi difficile à relever dans le monde industriel} :
	\begin{itemize}
		\item Le traitement du langage en tant que tel est un problème complexe : il faut traiter une grande variété de bruits et d'ambiguïtés de dialogue en plus d'un vocabulaire souvent spécifique au domaine de l'assistant (voir les problèmes de bruits et de représentativité en \textsc{Section~\ref{section:2.3.1-DEFIS-ANNOTATION-ASPECT-DONNEES}}) ;
		\item La base d'apprentissage ainsi que les réponses de l'assistant peuvent contenir des données privées ou confidentielles : ainsi, il y a peu de données réutilisables à partir de sources publiques, et de fortes pressions sont exercées sur le contrôle du comportement du \textit{chatbot} (voir les problèmes de droits d'utilisation et de confidentialité en \textsc{Section~\ref{section:2.3.1-DEFIS-ANNOTATION-ASPECT-DONNEES}}) ;
		\item L'assistant doit parfois pouvoir manipuler un grand nombre de cas d'usages : sa modélisation peut alors représenter des dizaines d'intentions pour lesquelles des centaines de branches de dialogues peuvent être paramétrées, introduisant ainsi une grande complexité aux tâches de modélisation et d'annotation de sa base d'apprentissage (voir \textsc{Section~\ref{section:2.3.2-DEFIS-ANNOTATION-ASPECT-COMPLEXITE}}) ;
	\end{itemize}
	
	Pour surmonter ces difficultés, les entreprises font alors intervenir des experts aux compétences diverses (voir \textsc{Section~\ref{section:2.2.2-ORGANISATION-ANNOTATION-ACTEURS}}), notamment des experts analytiques pour concevoir une modélisation stable des textes en intentions, puis des experts métiers pour valider la pertinence de la modélisation proposée et annoter les données suivant cette modélisation.
	
	
	% Ces experts doivent résoudre les différences de comportement dus à la complexité de tâche.
	Or au vu de la complexité d'un tel projet, des différences de comportements entre opérateurs, telles que des erreurs d'annotation ou des divergences d'opinion, sont inévitables (voir \textsc{Section~\ref{section:2.3.3-DEFIS-ANNOTATION-ASPECT-HUMAIN}}).
	Il est alors nécessaire de former les experts métiers aux tâches d'annotation et à certaines tâches d'analyse afin d'encadrer les discussions autour de certaines différences de comportements pouvant mener à des remises en cause de la modélisation abstraite de textes en intentions (voir étape \textit{Revise} du cyle \texttt{MATTER}, \textsc{Section~\ref{section:2.2.1.A-ORGANISATION-ANNOTATION-ETAPES-CLES-MODELIZE-ANNOTATE}}).
	Au final, \textbf{cette organisation devient très coûteuse} car elle demande des formations analytiques à des experts métiers, nécessite l'organisation d'ateliers de modélisation en mode essai-erreur pour trouver une base d'apprentissage stable et pertinente, et fait intervenir des experts métiers sur une abstraction de leurs connaissances du quotidien.
	
	% Exemples.
	\begin{leftBarExamples}
		Au cours de ce doctorat, nous avons entre autres travaillé sur des assistants conversationnels à destination de conseillers bancaires et de leur clients.
		Ces assistants doivent traiter une large variété de sujets (banque, assurance, finance, ...) et peuvent donc rapidement contenir une centaine d'intentions pour plus d'un millier de branches de dialogue.
		La conception de la base d'apprentissage de tels assistants représente ainsi un réel défi d'organisation, sur plusieurs semaines, notamment pour faire intervenir des experts de la banque-assurance dans un projet d'intelligence artificielle, domaine dans lequel ils n'ont pas ou peu de connaissances...
	\end{leftBarExamples}
	
	
	% Objectif de ce doctorat : assister la phase de modélisation d'une base d'apprentissage.
	Cependant, il pourrait être intéressant de remettre en question cette organisation des projets d'annotation où les experts métiers sont interrogés sur des compétences qui ne sont pas les leurs.
	Ainsi, dans le but de trouver une solution à cette problématique, \textbf{nous nous sommes alors posé la question suivante} :
	\begin{leftBarImportantGreen}
		\begin{center}
		\textbf{
			Comment assister la phase de modélisation de textes en intentions \\
			pour concevoir la base d'apprentissage d'un assistant conversationnel \\
			en impliquant des experts métiers pour leurs vraies compétences \\
			et en leur demandant un minimum de bagages analytiques ou techniques ?
		}
		\end{center}
	\end{leftBarImportantGreen}
	
	
	% Transition : idée d'un \texttt{clustering interactif} !
	\begin{leftBarIdea}
		Pour répondre à cette problématique, nous nous sommes intéressé à trois concepts intéressants de la littérature :
		\begin{itemize}
			\item aux techniques de \textit{clustering}, permettant de déléguer à la machine la tâche de modélisation grâce à une segmentation des données sur la base de leurs similarités (\cite{xu-tian:2015:comprehensive-survey-clustering}) ;
			\item à l'annotation de contraintes binaires sur les données, permettant de corriger le fonctionnement d'un algorithme de \textit{clustering} en y introduisant de la connaissance métier (\cite{lampert-etal:2018:constrained-distance-based}) ;
			\item aux techniques d'apprentissage actif, favorisant les interactions entre l'Homme et la Machine pour atteindre un objectif (\cite{settles:2010:active-learning-literature}).
		\end{itemize}
		Ces trois concepts seront détaillés au début du chapitre suivant, et seront assemblés dans le but de concevoir une nouvelle méthode d'annotation basée sur un \texttt{Clustering Interactif}.
	\end{leftBarIdea}

%%%%%--------------------------------------------------------------------
%%%%% Chapitre: Clustering Interactif
%%%%%--------------------------------------------------------------------
\chapter{Proposition d'un \textit{Clustering Interactif} pour assister la modélisation d'un jeu de données}
\label{chapter:3-CLUSTERING-INTERACTIF}
	
	% RÉSUMÉ DES ÉPISODES PRÉCÉDENTS: LA MODELISATION C'EST COMPLIQUÉ !
	Dans le chapitre précédent, nous avons vu les points essentiels suivants :
	%
	\begin{leftBarImportantGreen}
		\begin{todolist}
			% 1. Importance du jeu de données.
			\item[\itemok] Dans un cadre industriel, le choix de l'algorithme utilisé pour l'entraînement d'un modèle est déterminé à l'avance, donc la qualité de l'assistant repose principalement sur la fiabilité et la pertinence de son jeu de données ;
			% 2. Experts métiers avec connaissance métiers, pas de connaissance en datascience.
			\item[\itemok] Pour concevoir ce jeu de données, il est nécessaire de faire appel à des experts maîtrisant le domaine à couvrir par l'assistant car les données sont en général spécifiques ou privées ;
			% 3. Experts métiers ayant peu de donnaissance en data science.
			\item[\itemok] L'intervention de ces experts métiers au sein du projet est en général laborieuse :
			d'une part à cause de leur manque de connaissances en data science (ce n'est pas leur domaine d'expertise),
			d'autre part à cause de la complexité inhérente des tâches de modélisation et d'annotation des données.
			% 4. Tache manuelle avec peu d'assistance.
			\item[\itemok] Par manque de compétences, de connaissances ou d'ergonomie, la tâche de conception d'un jeu de données reste manuelle et est encore mal assistée par ordinateur.
		\end{todolist}
	\end{leftBarImportantGreen}
	\todo[inline]{À REFORMULER APRÈS L'ÉCRITURE DE L'ÉTAT DE L'ART}
	
	% ANNONCE DU BUT DU CHAPITRE: MA CONTRIBUTION !
	Dans cette partie, nous proposons une alternative à l'organisation manuelle destinée à la conception d'un jeu de données.
	Notre proposition vise à remplir un double objectif :
	%
	\begin{leftBarImportantRed}
		\begin{todolist}
			% 1. Efficacité de création du trainset.
			\item Proposer une méthode permettant d'assister la modélisation et l'annotation des données pour créer plus efficacement une base d'apprentissage pour la classification d'intention d'un assistant conversationnel ;
			% 2. Efficacité d'intervention d'un expert.
			\item Redéfinir les tâches et les objectifs des différents acteurs afin de rester au plus proche de leurs compétences réelles, particulièrement en ce qui concerne les experts métiers intervenants dans le projet.
		\end{todolist}
	\end{leftBarImportantRed}
			
	% Référence articles.
	\begin{leftBarInformation}
		Cette proposition de méthode a été l'objet d'une présentation à la conférence \texttt{EGC (Extraction et Gestion des Connaissances)}~(\cite{schild-etal:2021:conception-iterative-semisupervisee}), et d'une extension dans le journal \texttt{IJDWM (International Journal of Data Warehousing and Mining)}~(\cite{schild-etal:2022:iterative-semisupervised-design}).
		Nous reprenons ici certains des éléments présentés avec quelques détails supplémentaires.
	\end{leftBarInformation}

	% TABLE DES MATIÈRES DU CHAPITRE
	\minitoc


	%%%%%--------------------------------------------------------------------
	%%%%% Section 3.1: Intuitions à l'origine d'un \textit{clustering} interactif.
	%%%%%--------------------------------------------------------------------
	%\newpage
	\section{Intuitions à l'origine d'un \textit{clustering} interactif}
\label{section:3.1-INTUITIONS-ORIGINES}


	
	% Avantages/Limites: xu-tian:2015:comprehensive-survey-clustering
	% Limite clustering: agarwal-etal:2011:issues-challenges-tools (hyperparamètres) ; steinbach-etal:2004:challenges-clustering-high (high data)
	% Plusieurs types d'interactions possibles : bae-etal:2021:interactive-clustering-comprehensive


	%%%
	%%% Subsection 3.1.1. Utiliser une approche non-supervisées pour créer une modélisation des données.
	%%%
	\subsection{Utiliser une approche non-supervisées pour créer une modélisation des données}
	\label{section:3.1.1-INTUITIONS-ORIGINES-NON-SUPERVISEES}
	
		\todo[inline]{A REDIGER}
	
	%%%
	%%% Subsection 3.1.2. Corriger l'approche non-supervisée à l'aide d'une annotation de contraintes.
	%%%
	\subsection{Corriger l'approche non-supervisée à l'aide d'une annotation de contraintes}
	\label{section:3.1.2-INTUITIONS-ORIGINES-SEMI-SUPERVISEES}
	
		\todo[inline]{A REDIGER}
	
	%%%
	%%% Subsection 3.1.3. Tirer parti des avantages de l'apprentissage actif pour optimiser les interactions Homme/machine.
	%%%
	\subsection{Tirer parti des avantages de l'apprentissage actif pour optimiser les interactions Homme/machine}
	\label{section:3.1.3-INTUITIONS-ORIGINES-APPRENTISSAGE-ACTIF}
	
		\todo[inline]{A REDIGER}


	\todo[inline]{ANCIENNE PARTIE (début)}
	% 1. L'annotation de contraintes est plus intuitif pour un expert métier.
	La pierre angulaire de notre méthode repose sur le fait qu'il est difficile pour un expert métier de classer une question suivant une modélisation abstraite prédéfinie :
	cela l'éloigne de ses compétences métiers initiales, nécessite en contre-partie de nombreuses formations, et introduit de nombreuses erreurs d'annotations.
	De fait, il semble plus adéquat de demander à l'expert métier de discriminer deux questions sur la base de leurs réponses :
	une telle approche demande a priori une charge de travail plus faible et semble plus intuitive car elle se rapproche des compétences réelles de l'annotateur (\textguillemets{\textit{est-ce les deux données traite du même cas d'usage ?}}).
	Ainsi, nous basons notre méthode sur l'annotation de contraintes sur les données.
	
	% 2. Un expert métier seul ne peut trouver une modélisation adéquate, il faut se reposer sur l'interaction homme-machine.
	Toutefois, l'annotation de contraintes semble elle aussi fastidieuse.
	En effet, pour faire émerger une base d'apprentissage, il faut annoter un grand nombre de contraintes et être attentifs aux éventuelles incohérences pour ne pas introduire de contraintes contradictoires.
	Pour assister l'expert dans cette tâche, nous avons donc décidé de l'intégrer dans une stratégie d'apprentissage actif en essayant de tirer parti des interactions possibles avec la machine.
	Ce choix est motivé entre autres par l'intuition qu'il est possible de coopérer avec la machine pour obtenir plus efficacement un résultat pertinent.

	% TR:
	C'est sur la combinaison de ces deux éléments que repose notre méthode d'annotation pour concevoir le jeu d'entraînement de notre assistant conversationnel.
	\todo[inline]{ANCIENNE PARTIE (fin)}
	
	%%%
	%%% Conclusion.
	%%%
	\begin{leftBarSummary}
		... :
		\begin{todolist}
			% Pression sur la qualité.
			\item[\itemok] ...
			\item[\itemok] ...
			\item[\itemok] ...
		\end{todolist}
	\end{leftBarSummary}


	%%%%%--------------------------------------------------------------------
	%%%%% Section 3.2: Description théorique de notre \textit{clustering} interactif.
	%%%%%--------------------------------------------------------------------
	%\newpage
	\section{Description de notre \textit{clustering} interactif.}
\label{section:3.2-DESCRIPTION-THEORIQUE}

	Sur la base des intuitions que nous venons de détailler, nous proposons la méthode suivante dans le but d'assister la modélisation et l'annotation d'une collecte de données brutes en une base d'apprentissage nécessaire à l'entraînement d'un assistant conversationnel.
	
	%%%
	%%% Subsection 3.2.1. Description générale.
	%%%
	\subsection{Description générale}
	\label{section:3.2.1-DESCRIPTION-THEORIQUE-GENERALE}
	
		% Présentation succinte.
		\textbf{Notre méthode d'annotation}, que nous appelons \textguillemets{\texttt{Clustering Interactif}}, \textbf{repose sur l'alternance successive entre deux phases clefs} (voir \textsc{Figure~\ref{figure:3.2-CLUSTERING-INTERACTIF}}) :
		\begin{itemize}
			\item[\(\bullet\)] une phase d'\textbf{annotation de contraintes}
			par un expert permettant de caractériser la similarité entre deux données suivant leur cas d'usage métier ;
			\item[\(\bullet\)] une phase de \textbf{segmentation automatique} des données
			par une machine sur la base de la proximité sémantique des données et des contraintes précédemment annotées.
		\end{itemize}
		
		% Objectif recherché
		L'objectif recherché en associant ces deux phases est de \textbf{créer un cercle vertueux pour améliorer itérativement la qualité de la base d'apprentissage} en cours de construction.
		En effet, à chaque itération, l'expert métier obtiendra une proposition de segmentation des données qu'il pourra raffiner dans le but de corriger le fonctionnement de la machine et ainsi d'obtenir une segmentation plus pertinente à l'itération suivante.
		
		% Figure.
		\begin{figure}[!htb]
			\centering
			\includegraphics[width=0.95\textwidth]{figures/interactive-clustering-architecture-sequentielle}
			\caption{
				Schéma illustrant l'architecture du \textit{clustering} interactif.
				La boucle principale enchaîne un échantillonnage de couples de données, une annotation de contraintes, et un \textit{clustering} sous contraintes.
			}
			\label{figure:3.2-CLUSTERING-INTERACTIF}
		\end{figure}
	
	
	%%%
	%%% Subsection 3.2.2. Description détaillée.
	%%%
	\subsection{Description détaillée}
	\label{section:3.2.2-DESCRIPTION-THEORIQUE-DETAILLEE}
	
		% Pseudo code.
		L'\textsc{Algorithme~\ref{algorithm:3.2-CLUSTERING-INTERACTIF}} décrit formellement notre proposition de \texttt{Clustering Interactif} que nous détaillons ci-dessous.
		\begin{algorithm}
			\KwData{données non segmentées}
			\KwIn{budget à disposition}
			%
			\textbf{initialisation}: créer une liste vide de contraintes \;
			\textit{optionnel}: évaluer les hyper-paramètres de la segmentation automatique \;
			\textbf{segmentation initial}: regrouper les données par similarité \;
			\Repeat{segmentation satisfaisante OU budget épuisé}{
				\textit{optionnel}: évaluer les hyper-paramètres de l'échantillonnage \;
				\textbf{échantillonnage}: sélectionner une partie de la segmentation à corriger \;
				\textbf{annotation}: corriger la segmentation en ajoutant des contraintes sur l'échantillon \;
				\textit{optionnel}: ré-évaluer les hyper-paramètres de la segmentation automatique \;
				\textbf{segmentation}: regrouper les données par similarité avec les contraintes \;
				\textbf{validation}: estimer la pertinence et la stabilité de la segmentation \;
				\textbf{coûts}: estimer le budget restant et les coûts restant à investir \;
			}
			\textbf{interprétation}: trier et nommer les clusters pour les exploiter \;
			%
			\KwResult{données segmentées (i.e. base d'apprentissage)}
			%
			\caption{\textit{
				Description en pseudo-code de la méthode d'annotation proposée employant le \textit{clustering} interactif.
			}}
			\label{algorithm:3.2-CLUSTERING-INTERACTIF}
		\end{algorithm}
		
		% Description de l'initialisation.
		Pour l'\textbf{initialisation} de la méthode (cf. \textsc{Algorithme~\ref{algorithm:3.2-CLUSTERING-INTERACTIF}}, \textit{lignes 1 à 3}), nous définissons une liste vide de contraintes : tout au long du processus, nous y ajoutons les contraintes annotées par l'expert grâce à ses connaissances métiers (nous entrerons en détails en décrivant la phase d'annotation).
		Il faut aussi une première segmentation des données par la machine : celle-ci se réalise par l'exécution d'un algorithme de \textit{clustering}.
		Nous estimons qu'il n'est pas du ressort de l'expert métier de choisir le réglage de l'algorithme de \textit{clustering} et ses hyper-paramètres.
		Ces derniers pourront être déterminés par un \textit{data scientist} en fonction du problème à traiter ou laissés par défaut.
		Il est à noter que cette première segmentation des données est réalisée sans bénéficier de la connaissance de l'expert, il est donc peu probable que le résultat soit pertinent à ce stade.
		
		% Description de l'échantillonnage.
		Nous entrons dans le coeur de la boucle itérative par la phase d'\textbf{échantillonnage} (cf. \textsc{Algorithme~\ref{algorithm:3.2-CLUSTERING-INTERACTIF}}, \textit{lignes 5 et 6}).
		Comme mentionné au préalable, savoir quelles contraintes ajouter pour corriger efficacement le \textit{clustering} est un problème NP-difficile (le nombre de possibilités croît proportionnellement au carré du nombre de données).
		De plus, l'intervention d'experts est chiffrée et représente en général une partie des coûts à investir dans un projet (voir \textsc{Section~\ref{section:2.3.2.C-DEFIS-ANNOTATION-ASPECT-COMPLEXITE-COUTS}}).
		Il est donc inconcevable de laisser un expert métier annoter des contraintes "seul" et "au hasard".
		Ainsi, pour optimiser ses interventions, il convient de déterminer là où l'expert aura le plus d'impact lors de sa transmission de connaissances.
		C'est pourquoi la phase d'échantillonnage est primordiale dans la méthode proposée : Nous proposons d'y sélectionner des couples de données sur la base de leur similarité, de leur segmentation ou encore de leurs relations avec d'autres données déjà liées par des contraintes.
		
		% Description de l'annotation.
		Sur la base de cet échantillon, l'expert peut entamer son étape d'\textbf{annotation de contraintes} (cf. \textsc{Algorithme~\ref{algorithm:3.2-CLUSTERING-INTERACTIF}}, \textit{ligne 7}).
		Pour alléger la charge d'annotation, nous avons décidé de discriminer les données de l'échantillon par des contraintes binaires simples : \texttt{MUST-LINK} et \texttt{CANNOT-LINK}.
		Ces contraintes représentent respectivement la similitude ou la différence entre deux données, et seront utilisées pour regrouper ou séparer certaines données dans la prochaine segmentation.
		En fonction de l'orientation du projet et afin de rester au plus proche des compétences réelles de l'expert, la formulation de l'énoncer d'annotation doit être judicieusement définie : par exemple, les contraintes peuvent représenter une similitude
		sur la thématique concernée \footnote{
			Exemples de thématiques : \textit{crédit} vs. \textit{assurance} ; \textit{sport} vs. \textit{culture}, ...
		}, sur l'action désirée \footnote{
			Exemples d'actions : \textit{souscrire} vs. \textit{résilier} ; \textit{activer} vs. \textit{désactiver} ; \textit{s'informer} vs. \textit{réaliser}, ...
		}, ou encore sur le besoin de l'utilisateur \footnote{
			Exemple de besoins : \textit{souscrire un crédit} vs. \textit{souscrire une assurance} ; \textit{s'informer en sport} vs. \textit{s'informer en culture}, ...
		}.
		On notera que des incohérences peuvent s'introduire, ayant pour conclusions de devoir à la fois considérer comme similaires et différentes deux données : ces incohérences peuvent être détecter grâce à aux propriétés de transitivités des contraintes (voir la gestion des conflits en \textsc{Annexe~\ref{annex:C.2.2-DESCRIPTION-IMPLEMENTATION-INTERACTIVE-CLUSTERING-GESTION-DES-CONTRAINTES}}).
		
		% Description du clustering.
		Pour finir, la dernière phase de cette boucle est composée d'une nouvelle \textbf{segmentation} des données (cf. \textsc{Algorithme~\ref{algorithm:3.2-CLUSTERING-INTERACTIF}}, \textit{lignes 8 et 9}).
		Cette segmentation devra respecter les contraintes préalablement définies par l'expert, nous nous tournons donc vers l'utilisation d'un \textit{clustering} sous contraintes.
		Au fur et à mesure des itérations, de plus en plus de contraintes seront ajoutées pour corriger le \textit{clustering}. ainsi, au bout d'un certain nombre d'itérations, la segmentation des données reflétera la vision que l'expert aura voulu transmettre.
		Comme précédemment, nous estimons qu'il n'est pas du ressort de l'expert métier de choisir de l'algorithme de \textit{clustering} et ses hyper-paramètres.
		Ces derniers pourront être déterminés par un \textit{data scientist} en fonction du problème à traiter, estimés en fonction de l'itération et des contraintes disponibles, ou laissés par défaut.
		
		% Description de l'évaluation.
		Comme la méthode est itérative, il faut pouvoir estimer des \textbf{cas d'arrêt} (cf. \textsc{Algorithme~\ref{algorithm:3.2-CLUSTERING-INTERACTIF}}, \textit{lignes 10 à 12}).
		Le cas d'arrêt le plus évident n'est pas technique mais relatif aux coûts investis dans l'opération : si le projet n'a plus de budget dédié à l'annotation, il faudra créer la base d'apprentissage avec le résultat à disposition, quel que soit la pertinence de la segmentation obtenue sur les données.
		Ce cas d'arrêt par défaut peut malheureusement être synonyme d'échec pour le projet si les résultats sont inexploitables.
		D'autres cas d'arrêts peuvent être envisagés en fonction de la qualité ou de la pertinence de la segmentation.
		D'une part, nous pouvons comparer l'évolution de la segmentation des données : si les segmentations sont similaires sur plusieurs itérations, il est possible que la modélisation atteint un optimum local ou un palier de performance.
		D'autre part, nous pouvons aussi comparer l'évolution de l'accord entre la segmentation obtenue et l'annotation de l'expert : en effet, si l'expert ne contredit plus la répartition proposée des données, il est probable que sa vision et la vision de la machine aient convergé.
		Dans les deux cas, l'analyse de l'expert métier reste nécessaire pour valider si la modélisation des données est pertinente ou si elle comporte encore des incohérences à corriger.

		% Description de l'évaluation.
		Lorsque la boucle itérative est finie, nous avons à disposition une segmentation des données qui a été corrigé par un expert et qui reflète ses connaissances métier.
		La dernière étape consiste alors à \textbf{interpréter} ces \textit{clusters} pour pouvoir les exploiter (cf. \textsc{Algorithme~\ref{algorithm:3.2-CLUSTERING-INTERACTIF}}, \textit{ligne 13}).
		Cela commence par leur attribuer un nom au lieu de leur identifiant technique, de les définir en les rapprochant d'un cas d'usage métier, et éventuellement de les raffiner manuellement en supprimant certaines données aberrantes.
		
		% Exemple abstrait.
		\begin{leftBarExamples}
			La \textsc{Figure~\ref{figure:3.2.2-DESCRIPTION-THEORIQUE-DETAILLEE-EXEMPLE}} déroule l'initialisation et la première itération de la méthode sur un exemple fictif.
			Nous pouvons constater qu'entre les images \textbf{(2)} et \textbf{(5)}, la segmentation des données à évoluée grâce à l'introduction de contraintes.
		
			\begin{figure}[H]
				\centering
				\includegraphics[width=0.95\textwidth]{figures/example-iteration-clustering-interatif}
				\caption{
					Exemple d'une itération de \textit{clustering} interactif. \\
					Lors de l'initialisation,
					\textbf{(1)} correspond au jeu de données brut,
					et \textbf{(2)} correspond à une première segmentation des données en $3$ \textit{clusters}.
					Lors de l'itération $1$ :
					\textbf{(3)} correspond à un exemple d'échantillonnage de $6$ contraintes représentées par les flèches en pointillées,
					\textbf{(4)} correspond à la caractérisation de ces $6$ contraintes par des liens \texttt{MUST-LINK} en vert et \texttt{CANNOT-LINK} en rouge,
					et \textbf{(5)} correspond à la nouvelle segmentation des données en $3$ \textit{clusters} respectant les $6$ contraintes annotées.
					La prochaine itération se poursuivra par un nouvel échantillonnage de contraintes.
				}
				\label{figure:3.2.2-DESCRIPTION-THEORIQUE-DETAILLEE-EXEMPLE}
			\end{figure}
		\end{leftBarExamples}
	
	
	%%%
	%%% Subsection 3.2.3. Description techniques et implémentation.
	%%%
	\subsection{Description techniques et implémentation}
	\label{section:3.2.3-DESCRIPTION-TECHNIQUE-IMPLEMENTATION}
	
		% Généralités.
		Au cours de ce doctorat, nous avons réalisé un ensemble d'implémentations en \texttt{Python} afin de mettre en oeuvre notre méthodologie de \textit{clustering interactif}.
		Celle-ci est répartie en trois librairies :
		\begin{enumerate}
			% cognitivefactory-interactive-clustering
			\item \texttt{cognitivefactory-interactive-clustering} \footnote{
				\url{https://pypi.org/project/cognitivefactory-interactive-clustering/}
			} (\cite{schild:2022:cognitivefactory-interactiveclustering}), regroupant les gestions de données et des contraintes, les algorithmes de \textit{clustering} et d'échantillonnage ;
			% cognitivefactory-features-maximization-metric
			\item \texttt{cognitivefactory-features-maximization-metric} \footnote{
				\url{https://pypi.org/project/cognitivefactory-features-maximization-metric/}
			} (\cite{schild:2023:cognitivefactory-featuresmaximizationmetric}), disposant d'une méthode de sélection des patterns linguistiques pertinents d'un jeu de données labellisées, permettant ainsi d'analyser la pertinence d'un résultat de \textit{clustering} ;
			% cognitivefactory-interactive-clustering-gui
			\item \texttt{cognitivefactory-interactive-clustering-gui} \footnote{
				\url{https://pypi.org/project/cognitivefactory-interactive-clustering-gui/}
			} (\cite{schild-etal:2022:cognitivefactory-interactiveclusteringgui}), intégrant la logique de la méthodologie dans une application web.
		\end{enumerate}
		
		
		% Exemple de visuels de l'application.
		\begin{leftBarExamples}
			La \textsc{Figure~\ref{figure:3.2.3-DESCRIPTION-TECHNIQUE-IMPLEMENTATION-EXEMPLE}} représente une capture d'écran de la page d'annotation de contraintes de l'application web intégrant notre méthodologie de \textit{clustering interactif}.
			
			% Capture d'écran: annotation.
			\begin{figure}[H]
				\centering
				\includegraphics[width=0.95\textwidth]{figures/interactive-clustering-application-annotation-0small}
				\caption{
					Capture d'écran de l'application web implémentant notre méthodologie de \textit{clustering} interactif : \textbf{page d'annotation d'une contrainte}.
					Parmi les éléments importants, nous retrouvons les deux textes à annoter (disposés à gauche et droite de l'écran) et les boutons d'annotation (bouton vert pour un \texttt{MUST-LINK}, bouton rouge pour un \texttt{CANNOT-LINK}.
					Les autres fonctionnalités sont détaillées en \textsc{Annexe~\ref{annex:C.4-DESCRIPTION-IMPLEMENTATION-INTERACTIVE-CLUSTERING-GUI}}.
				}
				\label{figure:3.2.3-DESCRIPTION-TECHNIQUE-IMPLEMENTATION-EXEMPLE}
			\end{figure}
		\end{leftBarExamples}
		
		% Information : voir en annexe.
		\begin{leftBarInformation}
			Ces implémentations sont présentées dans l'\textsc{Annexe~\ref{annex:C-ANNEXE-IMPLEMENTATIONS}}.
			L'ensemble des détails techniques et des explications sur les choix d'implémentation y sont décrits.
		\end{leftBarInformation}


	%%%%%--------------------------------------------------------------------
	%%%%% Section 3.3: Espoirs portés sur la méthode proposée.
	%%%%%--------------------------------------------------------------------
	%\newpage
	\section{Espoirs portés sur la méthode proposée.}
\label{section:3.3-ESPOIRS-METHODE}

	\todo[inline]{
		SECTION À RÉDIGER; \\
		- Moins de formations, d'ateliers, \\
		- Se concentrer sur son domaine de compétence (i.e. pas de datascience pour les experts métiers) \\
		- Permettre de trouver la base d'apprentissage \\
		- Méthode réaliste / pas trop coûteuse  \\
		et transition vers chapitre 4 et annonce chapitre 5
	}

%%%%%--------------------------------------------------------------------
%%%%% Chapitre: Etude de la méthode
%%%%%--------------------------------------------------------------------
\chapter{Étude de six hypothèses sur le \textit{Clustering Interactif}}
\label{chapter:4-ETUDES}
	
	% RÉSUMÉ DES ÉPISODES PRÉCÉDENTES:
	Dans le chapitre précédent, nous avons présenté une méthode de création d'un jeu de données d'entraînement pour un assistant conversationnel, que nous appelons "\textit{clustering interactif}" :
	%
	\begin{leftBarImportantGreen}
		\begin{todolist}
			% 1. Structure de la méthode.
			\item[\itemok] La méthode proposée repose sur la combinaison entre un regroupement automatique des données par la machine et l'annotation de contraintes binaires par un expert métier pour corriger le regroupement proposé ;
			% 2. Enjeu 1 de la méthode : moins de technique.
			\item[\itemok] Une telle approche devrait limiter les pré-requis techniques actuellement exigés à un expert métier en les déléguant à la machine.
			% 2. Enjeu 2 de la méthode : plus de connaissance métier.
			\item[\itemok] En échange, l'expert se concentre d'avantage sur la transmission de ses connaissances avec une annotation caractérisant la similitude métier entre deux données.
			% 3. Divers.
			\item[\itemok] ...
		\end{todolist}
	\end{leftBarImportantGreen}
	\todo[inline]{divers à compléter (technique ? méthode ? ...).}
	
	% ANNONCE DU BUT DU CHAPITRE: TEST DE LA MÉTHODE.
	Comme nous l'avons détaillé dans le \textsc{Chapitre~\ref{chapter:2-REVUE-DE-LITTERATURE}}\todo{a revoir}, des procédés d'annotation similaires existent pour des données facilement visualisables, comme dans le cadre du traitement d'images.
	Cependant, l'application d'une telle approche dans le cadre de la classification de données textuelles est peu détaillée dans la littérature.
	Ainsi, dans cette partie, nous étudions la faisabilité d'un tel \textit{clustering interactif} pour des données textuelles en explorant les questions suivantes :
	%
	\begin{leftBarImportantRed}
		\begin{todolist}
			% 1. Efficacité.
			\item Peut-on obtenir une base d'apprentissage à l'aide de notre proposition d'implémentation de la méthodologie du \textit{clustering} interactif ? (cf. hypothèse d'\textbf{efficacité} en \textsc{Section~\ref{section:4.1-HYPOTHESE-EFFICACITE}}) ;
			% 2. Efficience.
			\item Peut-on déterminer un paramétrage optimal de cette implémentation pour obtenir plus rapidement une base d'apprentissage ? (cf. hypothèse d'\textbf{efficience} en \textsc{Section~\ref{section:4.2-HYPOTHESE-EFFICIENCE}}) ;
			% 3. Coûts.
			\item D'après les données initiales, peut-on approximer l'investissement nécessaire pour obtenir une base d'apprentissage exploitable ? (cf. hypothèse sur les \textbf{coûts} en \textsc{Section~\ref{section:4.3-HYPOTHESE-COUTS}}) ;
			% 4. Pertinence.
			\item A un instant donné, peut-on estimer la pertinence métier d'une base d'apprentissage en cours de construction ? (cf. hypothèse de \textbf{pertinence} en \textsc{Section~\ref{section:4.4-HYPOTHESE-PERTINENCE}}) ;
			% 5. Impact.
			\item Au cours du processus de construction de la base d'apprentissage, peut-on aisément estimer les potentiels d'une étape de raffinage supplémentaire ? (cf. hypothèse de \textbf{rentabilité} en \textsc{Section~\ref{section:4.5-HYPOTHESE-RENTABILITE}}) ;
			% 6. Robustesse.
			\item Peut-on estimer l'influence d'une différence d'annotation dans la construction de la base d'apprentissage ? (cf. hypothèse de \textbf{robustesse} en \textsc{Section~\ref{section:4.6-HYPOTHESE-ROBUSTESSE}}).
		\end{todolist}
	\end{leftBarImportantRed}
	
	% ILLUSTRATION: SCHEMA DES HYPOTHESES
	Afin d'illustrer ces interrogations, nous vous proposons de considérer de la \textsc{Figure~\ref{figure:4.0-HYPOTHESE-00-DEFAULT}}. Dans les sections suivantes, cette figure évoluera pour résumer les études réalisées.
	%
	\begin{figure}[!htb]
		\centering
		\includegraphics[width=0.95\textwidth]{figures/hypotheses-00-default}
		\caption{
			Illustration des études réalisées sur le \textit{clustering} interactif (\textit{étape 0/6}) en schématisant l'évolution de la performance (\textit{accord avec la vérité terrain calculé en v-measure}) d'une base d'apprentissage en cours de construction en fonction du nombre d'itérations de la méthode (\textit{nombre d'annotations par un expert métier}).
		}
		\label{figure:4.0-HYPOTHESE-00-DEFAULT}
	\end{figure}
	
	\todo[inline]{
		Pour ces études, nous allons (1) faire des analyses théoriques (2) des analyses empiriques (car dans la vrai vie on n'a pas de vérité terrain).
		Nous utilisons aussi la vmeasure.
	}
	\todo[inline]{
		Table des nomenclatures.
	}
	
	% PRÉAMBULE TECHNIQUE : CPU + scrips + datasets.
	\begin{leftBarInformation}
		Pour ces études, l'exécution des différentes expériences a été réalisée sur des CPU \texttt{Intel(R) Xeon(R) CPU E5-2660 v4 \@ 2.00GHz} et parallélisé avec la librairie Python \textit{multiprocessing} (un worker par CPU).
		Les scripts d'exécution et d'analyse de ces expériences, rédigés au sein de \textit{notebooks} Python (\cite{van-rossum-drake:2009:python-reference-manual}) ou de script R (\cite{r-core-team:2017:language-environment-statistical}), sont disponibles dans \cite{schild:2021:cognitivefactory-interactiveclusteringcomparativestudy}.
		Enfin, les jeux de données utilisés pour ces études sont détaillés en Annexe~\ref{annex:A-ANNEXE-DATASET}.
	\end{leftBarInformation}
	
	
	% TABLE DES MATIÈRES DU CHAPITRE
	\minitoc
	
	%%%%%--------------------------------------------------------------------
	%%%%% Section 4.1: Hypothèse d'efficacité.
	%%%%%--------------------------------------------------------------------
	\newpage
	\section{Évaluation de l'hypothèse d'efficacité}
\label{section:4.1-HYPOTHESE-EFFICACITE}
%  : « \textit{est-ce que la méthode permet d'annoter un jeu de données ?} »
	
	%%% Formulation des hypothèses:
	Nous aimerions vérifier l'hypothèse suivante :

	\begin{tcolorbox}[
		title=\faVial~\textbf{Hypothèse d'efficacité}~\faVial,
		colback=colorTcolorboxHypothesis!15,
		colframe=colorTcolorboxHypothesis!75,
		width=\linewidth
	]
		% Hypothèse.
		« \textbf{
			Une méthodologie d'annotation basée sur le \textit{clustering} interactif permet d'obtenir une base d'apprentissage pour un assistant conversationnel qui respecte la vision donnée par l'expert métier au cours de l'annotation.
		} » \\
		
		% Résumé de l'étude.
		Afin de vérifier cette hypothèse, nous mettrons en place une expérience de ré-annotation d'une base d'apprentissage (qui servira ici de vérité terrain) à l'aide de notre méthode, en simulant l'annotation d'un expert, et nous critiquerons l'évolution de la nouvelle base d'apprentissage obtenue et sa similitude avec la base d'apprentissage initiale.
		
		% Figure.
		La figure~\ref{figure:4.1-HYPOTHESE-EFFICACITE} illustre cette hypothèse et l'espoir de convergence d'une base d'apprentissage en cours de construction vers sa vérité terrain.
		%
		\begin{figure}[H]  % keep [H] to be in the tcolorbox.
			\centering
			\includegraphics[width=0.8\textwidth]{figures/hypotheses-01-efficacite}
			\caption{Illustration des études réalisées sur le \textit{clustering} interactif (\textit{étape 1/6}) en schématisant l'évolution de la performance (\textit{accord avec la vérité terrain calculé en v-measure}) d'une base d'apprentissage en cours de construction en fonction du nombre d'itérations de la méthode (\textit{nombre d'annotations par un expert métier}).}
			\label{figure:4.1-HYPOTHESE-EFFICACITE}
		\end{figure}

	\end{tcolorbox}
	
	%%%
	%%% Subsection 4.1.1: Étude de convergence vers une vérité terrain pré-établie en simulant l'annotation d'une base d'apprentissage et mesurant la vitesse de sa création
	%%%
	\subsection{Étude de convergence vers une vérité terrain pré-établie en simulant l'annotation d'une base d'apprentissage et mesurant la vitesse de sa création}
	\label{section:4.1.1-ETUDE-CONVERGENCE}
			
		% Référence articles.
		\begin{leftBarInformation}
			Cette étude a été l'objet d'une présentation à la conférence \texttt{EGC (Extraction et Gestion des Connaissances)}~\citep{schild:conception-interactive-clustering:2021}, et d'une extension dans le journal \texttt{IJDWM (International Journal of Data Warehousing and Mining)}~\citep{schild:extension-interactive-clustering:2022}.
			\footnote{Les résultats et la discussion de ces articles ont été mis à jour et réécrits pour mieux s'intégrer au discours ce manuscrit.}
		\end{leftBarInformation}

		%%% Protocole expérimental.
		\subsubsection{Protocole expérimental}
		
			% Objectif de l'expérience.
			Nous voulons vérifier qu'une méthodologie d'annotation basée sur notre implémentation du \textit{clustering} interactif permet de créer une base d'apprentissage pour un assistant conversationnel.
			Pour cela, nous prenons une base d'apprentissage employée pour entraîner un modèle de classification de textes\todo{référence, lien vers ANNEXE, + description conditions de création du JDD}, et nous utilisons ce jeu de données comme vérité terrain.
			L'objectif de cette expérience est de simuler la création de cette base d'apprentissage et de nous assurer que le résultat obtenu correspond à la vérité terrain.
			
			% Axiome.
			\begin{leftBarWarning}
				Dans le cadre de cette étude, nous supposons que l'expert métier connaît parfaitement le domaine traité dans ce jeu de données, et qu'il est capable de caractériser sans ambiguïté la similitude entre deux données issues de cet ensemble.
			\end{leftBarWarning}
			
			% Pseudo-code.
			Pour résumer le protocole expérimental que nous décrivons ci-dessous, vous pouvez vous référer au pseudo-code décrit dans Alg.~\ref{algorithm:4.1.1-ETUDE-CONVERGENCE-PROTOCOLE}.
			%
			\begin{algorithm}[!htb]
				\begin{algorithmic}[1]
					\Require jeu de données annoté (vérité terrain)
					\ForAll{arrangement d'algorithmes et de paramètres à tester}
						\State \textbf{initialisation (données)}: récupérer les données et la vérité terrain
						\State \textbf{initialisation (contraintes)}: créer une liste vide de contraintes
						\State \textbf{prétraitement}: supprimer le bruit dans les données
						\State \textbf{vectorisation}: transformer les données en vecteurs
						\State \textbf{clustering initial}: regrouper les données par similarité des vecteurs
						\State \textbf{évaluation}: estimer l'équivalence entre le clustering obtenu et la vérité terrain
						\Repeat
							\State \textbf{échantillonnage}: sélectionner de nouvelles contraintes à annoter
							\State \textbf{simulation d'annotation}: ajouter des contraintes en utilisant la vérité terrain
							\State \textbf{clustering}: regrouper les données par similarité avec les contraintes
							\State \textbf{évaluation}: estimer l'équivalence entre le clustering obtenu et la vérité terrain
						\Until{annotation de toutes les contraintes possibles}
						\State \textbf{évaluation finale}: espérer avoir un score d'équivalence de $100$\% entre le clustering obtenu et la vérité terrain
					\EndFor
					\Ensure arrangements d'algorithmes et de paramètres ayant un score d'équivalence de $100$\%
				\end{algorithmic}
				\caption{Description en pseudo-code du protocole expérimental de l'étude de convergence du \textit{clustering} interactif vers une vérité terrain pré-établie.}
				\label{algorithm:4.1.1-ETUDE-CONVERGENCE-PROTOCOLE}
			\end{algorithm}
			
			% Détails de l'expérience.
			\todo[inline]{description JDD}
			Lors de cette expérience, chaque tentative de la méthode commencera sur la version non labellisée de la vérité terrain à disposition, sans aucune contrainte connue à l'avance.
			Au fur et à mesure des itérations de la méthode, nous simulerons l'annotation de l'expert métier en comparant les labels de la vérité terrain : ainsi, deux données ont une contrainte \texttt{MUST-LINK} si elles ont le même label, et une contrainte \texttt{CANNOT-LINK} sinon.
			Cela traduit le prérequis d'avoir un annotateur qui soit capable, dans son domaine d'expertise, de différencier deux données selon leur ressemblance.
			Une tentative de l'application de notre méthode s'arrête lorsque toutes les contraintes possibles entre les données ont été annotés par l'expert.

			% Description implémentation de l'interactive clustering.
			Pour cette étude, nous essayons une tentative pour chaque combinaison de paramètre de notre implémentation du clustering interactif (cf. section~\ref{section:3.3-DESCRIPTION-IMPLEMENTATION}). Cela comprend les tâches et leurs paramètres respectifs suivants :
			%
			\begin{enumerate}
				\item le \textbf{prétraitement} des données, avec les niveaux suivants : \textbf{absent} (noté \texttt{prep.no}), \textbf{simple} (noté \texttt{prep.simple}), \textbf{avec lemmatisation} (noté \texttt{prep.lemma}) et \textbf{avec filtres} (noté \texttt{prep.filter}) ;
				\item la \textbf{vectorisation} des données, avec les niveaux suivants : \textbf{TF-IDF} (noté \texttt{vect.tfidf}) et \textbf{SpaCy} (noté \texttt{vect.frcorenewsmd}) ;
				\item le \textbf{clustering sous contraintes} des données, avec les niveaux suivants : \textbf{KMeans} (modèle \textit{COP} noté \texttt{clust.kmeans.cop}), \textbf{Hiérarchique} (lien \textit{single} noté \texttt{clust.hier.sing} ; lien \textit{complete} noté \texttt{clust.hier.comp} ; lien \textit{average} noté \texttt{clust.hier.avg} ; lien \textit{ward} noté \texttt{clust.hier.ward}) et \textbf{Spectral} (modèle \textit{SPEC} noté \texttt{clust.spec}). Le choix du nombre de clusters n'est pas étudié ici, et ce nombre est fixé au nombre de classes présentes dans la vérité terrain ;
				\item l'\textbf{échantillonnage} des contraintes à annoter, avec les niveaux suivants : \textbf{purement aléatoire} (noté \texttt{samp.random.full}), \textbf{pseudo-aléatoire} (noté \texttt{samp.random.same}), \textbf{même cluster et étant les plus éloignées} (noté \texttt{samp.farhtest.same}) et \textbf{clusters différents et étant les plus proches} (noté \texttt{samp.closest.diff}). Le choix de la taille d'échantillon n'est pas étudié ici, et cette taille est arbitrairement fixé à $50$.
			\end{enumerate}
			
			Il y a donc $192$ combinaisons testées, et chaque tentative est répétée $5$ fois pour contrer les aléas statistiques de certains algorithmes.
			Pour plus de détails sur ces algorithmes, référez-vous à la section~\ref{section:3.3-DESCRIPTION-IMPLEMENTATION} pour avoir accès à leur description, à leurs paramètres et aux choix d'implémentation.
			
			% Description de l'évaluation.
			Pour évaluer l'équivalence entre la vérité terrain et notre segmentation des données obtenue au cours de la méthode, nous nous intéresserons à l'évolution de la \texttt{v-measure} entre ces deux jeu de données.
			Si le score du calcul de la \texttt{v-measure} est de $100$\%, cela signifierait que le clustering final et la vérité terrain propose une segmentation identique des données, donc que la vérité terrain a pu être retrouvée, et donc qu'il est possible d'obtenir une base d'apprentissage pour un assistant conversationnel à l'aide d'une méthodologie d'annotation basée sur le \textit{clustering} interactif.
			
			% Référence scripts.
			\begin{leftBarInformation}
				Les scripts de l'expérience (\textit{notebooks} Python) sont disponibles dans un dossier dédié de~\cite{schild:cognitivefactory-interactive-clustering-comparative-study:2021}.
			\end{leftBarInformation}

		%%% Résultats.
		\subsubsection{Résultats obtenus}
			
			% Graphe d'évolution de la v-measure moyenne, min et max.
			La figure~\ref{figure:4.1.1-ETUDE-CONVERGENCE-EVOLUTION} et le tableau~\ref{table:4.1.1-ETUDE-CONVERGENCE-EVOLUTION} représentent l'évolution moyenne de la \texttt{v-measure} du clustering en fonction du nombre d'itération de la méthode. Les tentatives les plus rapides et les plus lentes sont représentées sur la figure.
							
			% Tendance: Forte dispersion, Croissance générale.
			Malgré une forte dispersion des résultats (écart-type de \texttt{v-measure} pouvant être supérieur à $20$\%, forte différence entre les tentatives la plus rapide et la plus lente) et quelques sauts de performances (cf. à-coups de la tentative la plus lente sur la figure), une convergence générale vers la vérité terrain peut être constatée.
			
			% Tendance à courts termes: Croissance linéaire
			A l'itération $0$, une tentative commence avec une moyenne de $19.05$\% de \texttt{v-measure}  entre son \textit{clustering} initial (sans contraintes) et la vérité terrain.
			Cette \texttt{v-measure} moyenne croît presque linéairement (pente de $0.97$) jusqu'à l'itération $75$ où elle atteint la performance de $92.08$\% (cf. tableau~\ref{table:4.1.1-ETUDE-CONVERGENCE-EVOLUTION}).

			% Tendance à longs termes: Asymptote.
			Au delà de l'itération $75$, la courbe de la \texttt{v-measure} moyenne tend vers une asymptote de $100$\% (cf. figure~\ref{figure:4.1.1-ETUDE-CONVERGENCE-EVOLUTION}).
			Cette asymptote est atteinte par toute les $960$ tentatives ($192$ combinaisons de paramètres, $5$ tentatives pour chaque combinaison), la tentative l'ayant atteinte le plus tôt à l'itération $19$ et celle le plus tard à l'itération $326$.
			La courbe se prolonge jusqu'à l'itération $394$ pour que toutes les tentatives puisse annoter toutes les contraintes possibles sur le jeu de données.
			
			%
			\begin{figure}[!htb]
				\centering
				\includegraphics[width=0.8\textwidth]{figures/etude-efficacite-evolution-moyenne-0par-iteration}
				\caption{Évolution de la moyenne de la \texttt{v-measure} entre un résultat obtenu et la vérité terrain en fonction du nombre d'itération de la méthode de \textit{clustering} interactif, moyenne réalisée itération par itération sur l'ensemble des tentatives.
				Représentation des tentatives ayant été les plus rapides (\textit{un prétraitement \texttt{prep.simple}, une vectorisation \texttt{vect.tfidf}, un clustering \texttt{clust.hier.comp} ou \texttt{clust.hier.ward}, et un échantillonnage \texttt{samp.closest.diff}}) et les plus lentes (\textit{un prétraitement \texttt{prep.no}, une vectorisation \texttt{vect.tfidf}, un clustering \texttt{clust.spec}, et un échantillonnage de contraintes \texttt{samp.farthest.same}}) pour atteindre $100$\% de \texttt{v-measure}.}
				\label{figure:4.1.1-ETUDE-CONVERGENCE-EVOLUTION}
			\end{figure}
			%
			\begin{table}[!htb]
				\begin{center}
				\begin{tabular}{|c|r|r|r|r|r|}
					\hline
					% ENTETE DU TABLEAU
					\multicolumn{2}{|c|}{ \shortstack{ Annotations } }
						& \multicolumn{4}{c|}{ \shortstack{ Performances (\texttt{v-measure}) } }
						\tabularnewline
						\hline
					\multicolumn{1}{|c|}{ \shortstack{ Itérations } }
						& \multicolumn{1}{c|}{ \shortstack{ Contraintes } }
						& \multicolumn{1}{c|}{ \shortstack{ Moyenne } }
						& \multicolumn{1}{c|}{ \shortstack{ Écart-type } }
						& \multicolumn{1}{c|}{ \shortstack{ Minimum } }
						& \multicolumn{1}{c|}{ \shortstack{ Maximum } }
						\tabularnewline
						\hline
					%
					$0$		& $0$		& $19.05$\% \footnotesize $(\pm0.43)$ \par	& $13.38$\% & $03.42$\% & $47.75$\%
					\tabularnewline
					\hline
					%
					$25$	& $1~250$	& $49.09$\% \footnotesize $(\pm0.82)$ \par	& $25.43$\% & $09.09$\% & $100.00$\%
					\tabularnewline
					\hline
					%
					$50$	& $2~500$	& $73.66$\% \footnotesize $(\pm0.77)$ \par	& $23.98$\% & $16.78$\% & $100.00$\%
					\tabularnewline
					\hline
					%
					$75$	& $3~750$	& $92.08$\% \footnotesize $(\pm0.54)$ \par	& $16.70$\% & $21.74$\% & $100.00$\%
					\tabularnewline
					\hline
					%
					$100$	& $5~000$	& $95.19$\% \footnotesize $(\pm0.41)$ \par	& $12.67$\% & $26.93$\% & $100.00$\%
					\tabularnewline
					\hline
					%
					$125$	& $6~250$	& $97.43$\% \footnotesize $(\pm0.29)$ \par	& $09.09$\% & $34.99$\% & $100.00$\%
					\tabularnewline
					\hline
					%
					$150$	& $7~500$	& $98.73$\% \footnotesize $(\pm0.23)$ \par	& $07.22$\% & $38.14$\% & $100.00$\%
					\tabularnewline
					\hline
					
				\end{tabular}
				\end{center}
				\caption{Détails de l'évolution de la moyenne de la \texttt{v-measure} entre un résultat obtenu et la vérité terrain en fonction du nombre d'itération de la méthode de \textit{clustering} interactif, moyenne réalisée itération par itération sur l'ensemble des tentatives.}
				\label{table:4.1.1-ETUDE-CONVERGENCE-EVOLUTION}
			\end{table}

		%%% Discussion
		\subsubsection{Discussion}
			
			%%% Principale conclusion : il y a convergence !
			La première et principale conclusion de cette étude concerne la preuve que la méthode est efficace.
			En effet, les différentes simulations ont bien convergé vers la vérité terrain (atteinte de l'asymptote à $100$\% de \texttt{v-measure}), montrant qu'il est possible pour un expert métier de créer une base d'apprentissage à l'aide d'une méthodologie d'annotation basée sur le \textit{clustering} interactif. \\
			
			
			%%% Avantages.
			Cette découverte permet de confirmer plusieurs espoirs portés sur la méthode. 
			
			% Avantage 1 : Émergence d'une modélisation sur la base des contraintes
			Tout d'abord, la vérité terrain a été retrouvée sans formaliser concrètement la structure de données.
			Là où une annotation par label aurait requis au préalable une définition des catégories possibles pour les données à étiqueter, la méthodologie employant le \textit{clustering} interactif a permis de faire émerger naturellement cette structure de données.
			Cette émergence provient directement des contraintes annotées par l'expert métier, traduisant ainsi ses connaissances à l'aide d'instructions simples : \textit{les données sont-elles ou non similaires ?}
			
			% Avantage 2 : annotations plus simples et plus concrètes
			De plus, ces contraintes ont été l'objet d'une annotation guidée par les besoins de la machine afin de s'améliorer d'itération en itération (voir la croissance globale de la \texttt{v-measure} sur la figure~\ref{figure:4.1.1-ETUDE-CONVERGENCE-EVOLUTION}).
			Ainsi, l'expert métier corrige la base d'apprentissage à chaque itération : soit en affinant les clusters en cours de construction, améliorant ainsi la cohérence des clusters (cf. pentes croissantes) ; soit en remaniant les clusters mal formés pour repartir sur de bonnes bases, détériorant la cohérence des clusters le temps de la réorganisation (cf. oscillations ou pentes décroissantes). \\

			
			%%% Limites.
			Néanmoins, différentes pistes sont encore à explorer pour rendre le \textit{clustering} interactif utilisable en situation réelle.
			
			% Limite 1 : Nombre d'annotations ==> besoin d'optimisation.
			D'une part, nous échangeons le besoin de définir une structure de données contre la nécessité d'annoter un grand nombre de contraintes : pour $500$ points de données, et en considérant que l'asymptote à $100$\% est atteinte en moyenne autour de l'itération $200$, il faudrait $10~000$ annotations de contraintes pour être exhaustif, ce qui correspond à près de $20$ fois plus de contraintes que de données.
			Bien que l'annotation binaire demande a priori une charge mentale plus faible à un annotateur, un tel volume représente tout de même une grande quantité de travail.
			\todo{
				Commentaire Gautier 22/05/2023 :
				(A DÉTAILLER AILLEURS ?)
				Oui, complètement d'accord ici, mais en fait ça va plus loin que ça non ?
				Déjà, on a une quantité de ressources allouées à la tâche en effet plus fabile (car choisir entre "similaire" et "non similaire" est clairement plus simple que d'assigner un label parmi N).
				Mais on a aussi une diminution des ressources allouées au maintien d'une stratégie d'annotation : en effet, pas besoin de définir à l'avance de type system ou autre, tout est construit à la volée. Ce deuxième point est particulierement intéressant à discuter je pense, car on sait normalement que le maintien d'objectifs en mémoire de travail peut aider à maintenir un niveau d'engagement sur une tâche cognitive. Du coup, ça pose d'autant plus la question de l'expérience utilisateur : annoter avec un CI sera-t-il moins engageant qu'annoter avec une méthode classique ?
			}
			Cela peut décourager les experts métiers en début de projet, surtout pour des projets ayant des jeux de données de plus grandes tailles.
			Toutefois, les résultats obtenus montrent une forte dispersion du nombre d'itérations nécessaire, et certaines tentatives ont été bien plus efficientes dans l'utilisation de leurs contraintes. La tentative la plus rapide a convergé à l'itération $19$, soit $950$ contraintes, ce qui est un volume d'annotation bien plus abordable !
			On peut donc espérer trouver un paramétrage optimal de la méthode permettant de diminuer significativement le nombre moyen de contraintes nécessaires afin d'obtenir une base d'apprentissage exploitable avec un volume d'annotations acceptable.
			Cet aspect fait l'objet de l'étude décrite dans la section~\ref{section:4.2-HYPOTHESE-EFFICIENCE} (hypothèse d'efficience).
			
			% Limite 2 : Exhaustivité des annotations ==> evaluation de la rentabilité.
			D'autre part, le choix d'annoter toutes les contraintes possibles sur les données (\textbf{annotation exhaustive}) n'est pas forcément judicieux.
			En effet, si nous nous référons à la figure~\ref{figure:4.1.1-ETUDE-CONVERGENCE-EVOLUTION}), une moyenne de $90$\% de \texttt{v-measure} est déjà atteinte autour de l'itération $75$, alors que l'asymptote à $100$\% n'est atteinte qu'au delà de l'itération $200$. Afin d'être plus efficient, il faudrait envisager une \textbf{annotation partielle} permettant d'obtenir rapidement $90$\% de \texttt{v-measure} (quitte à affiner le résultat manuellement pour combler la "perte" moyenne de $10$\% de \texttt{v-measure}).
			Cet aspect sera ajouté à l'objectif de l'étude décrite dans la section~\ref{section:4.2-HYPOTHESE-EFFICIENCE} (hypothèse d'efficience).
			
			% Limite 3 : Expert métier parfait ==> simuler les erreurs.
			Pour finir, nous avons supposé dans cette étude que l'annotateur est un expert métier connaissant parfaitement le domaine traité.
			Cette hypothèse forte n'est a priori pas valable en situation réelle : En effet, des erreurs d'annotations peuvent intervenir (ambiguïtés sur les données, méconnaissance du domaine, erreurs d'inattention, différence d'opinions entre annotateurs, ...), ce qui peut entraîner des divergences ou des incohérences dans la construction de la base d'apprentissage.
			Il semble donc nécessaire d'étudier les impacts de ces incohérences, ainsi que de proposer une méthode pour les prévenir ou les corriger.
			Cet aspect sera traité à la fin de ce chapitre dans la section~\ref{section:4.6-HYPOTHESE-ROBUSTESSE} (hypothèse de robustesse).
	
	
	%%%%%--------------------------------------------------------------------
	%%%%% Section 4.2: Hypothèse d'efficience.
	%%%%%--------------------------------------------------------------------
	\newpage
	\section{Évaluation de l'hypothèse d'efficience}
\label{section:4.2-HYPOTHESE-EFFICIENCE}
% « \textit{quels sont les paramètres optimaux pour minimiser la charge de travail de l'annotateur ?} »

	%%% Formulation des hypothèses:
	Suite à la validation de l'hypothèse d'efficacité (convergence de la méthode, cf. section~\ref{section:4.1-HYPOTHESE-EFFICACITE}), nous aimerions vérifier l'hypothèse suivante :

	\begin{tcolorbox}[
		title=\faVial~\textbf{Hypothèse d'efficience}~\faVial,
		colback=colorTcolorboxHypothesis!15,
		colframe=colorTcolorboxHypothesis!75,
		width=\linewidth
	]

		% Hypothèse.
		« \textbf{
			La vitesse de convergence du \textit{clustering} interactif peut être optimisée en ajustant différents paramètres afin de minimiser la charge de travail de l'opérateur. Nous étudierons en particulier l'influence du prétraitement des données, de la vectorisation des données, de l'échantillonnage des contraintes à annoter et du \textit{clustering} sous contraintes.
		} » \\
		
		% Résumé de l'étude.
		Afin de vérifier cette hypothèse, nous mettrons en place une expérience de ré-annotation d'une base d'apprentissage (qui servira ici de vérité terrain) à l'aide de notre méthode, en simulant l'annotation d'un expert, et nous réaliserons l'analyse statistique de la taille d'effet de différents paramètres sur la vitesse de convergence du \textit{clustering} itératif.
		
		% Figure.
		La figure~\ref{figure:4.2-HYPOTHESE-EFFICIENCE} illustre cette hypothèse et l'espoir d'une convergence "optimale" d'une base d'apprentissage en cours de construction vers sa vérité terrain.
		%
		\begin{figure}[H]  % keep [H] to be in the tcolorbox.
			\centering
			\includegraphics[width=0.8\textwidth]{figures/hypotheses-02-efficience}
			\caption{Illustration des études réalisées sur le \textit{clustering} interactif (\textit{étape 2/6}) en schématisant l'évolution de la performance (\textit{accord avec la vérité terrain calculé en v-measure}) d'une base d'apprentissage en cours de construction en fonction du nombre d'itérations de la méthode (\textit{nombre d'annotations par un expert métier}).}
			\label{figure:4.2-HYPOTHESE-EFFICIENCE}
		\end{figure}

	\end{tcolorbox}
	
	%%%
	%%% Subsection 4.2.1: Étude d'optimisation des paramètres de convergence.
	%%%
	\subsection{Étude d'optimisation des paramètres de convergence}
	\label{subsection:4.2.1-ETUDE-OPTIMISATION}
			
		% Référence articles.
		\begin{leftBarInformation}
			Cette étude a été l'objet d'une présentation à la conférence \texttt{EGC (Extraction et Gestion des Connaissances)}~\citep{schild:conception-interactive-clustering:2021}, et d'une extension dans le journal \texttt{IJDWM (International Journal of Data Warehousing and Mining)}~\citep{schild:extension-interactive-clustering:2022}.
			\footnote{Les résultats et la discussion ont été mis à jour et réécrits pour mieux s'intégrer au discours ce manuscrit.}
		\end{leftBarInformation}

		%%% Protocole expérimental.
		\subsubsection{Protocole expérimental : analyser la taille d'effet des paramètres d’implémentation sur la vitesse de création d'une base d'apprentissage}

			% Objectif de l'expérience.
			Nous voulons étudier l'influence des paramètres de notre implémentation du \textit{clustering} interactif sur la vitesse de création d'une base d'apprentissage pour un assistant conversationnel.
			Nous allons donc compléter le protocole expérimental de l'étude de convergence en section~\ref{subsection:4.1.1-ETUDE-CONVERGENCE} visant à simuler la création d'une base d'apprentissage\todo{référence, lien vers ANNEXE}.
			
			% Pseudo-code.
			Pour résumer le protocole expérimental que nous décrivons c-dessous, vous pouvez vous référer au pseudo-code décrit dans Alg.~\ref{algorithm:4.2.1-ETUDE-OPTIMISATION-PROTOCOLE}.
			%
			\begin{algorithm}[!htb]
				\begin{algorithmic}[1]
					\Require jeu de données annoté (vérité terrain)
					\ForAll{arrangement d'algorithmes et de paramètres à tester}
						\State \textbf{initialisation}: récupérer les données de la vérité terrain sans leur label, créer une liste vide de contraintes
						\State \textbf{prétraitement}: supprimer le bruit dans les données
						\State \textbf{vectorisation}: transformer les données en vecteurs
						\State \textbf{clustering initial}: regrouper les données par similarité
						\State \textbf{évaluation}: estimer l'équivalence entre le clustering obtenu et la vérité terrain
						\Repeat
							\State \textbf{échantillonnage}: sélectionner de nouvelles contraintes à annoter
							\State \textbf{simulation d'annotation}: ajouter des contraintes grâce à la comparaison des labels de la vérité terrain
							\State \textbf{clustering}: regrouper les données par similarité avec les contraintes
							\State \textbf{évaluation}: estimer l'équivalence entre le clustering obtenu et la vérité terrain
						\Until{annotation de toutes les contraintes possibles}
					\EndFor
					\State \textbf{analyse}: déterminer les tailles d'effets des algorithmes et paramètres
					\Ensure meilleurs arrangements d'algorithmes et de paramètres
				\end{algorithmic}
				\caption{Description en pseudo-code du protocole expérimental de l'étude d'optimisation de la convergence du \textit{clustering} interactif vers une vérité terrain pré-établie.}
				\label{algorithm:4.2.1-ETUDE-OPTIMISATION-PROTOCOLE}
			\end{algorithm}
			
			% Détails de l'expérience.
			En s'appuyant sur les résultats précédemment obtenus, nous allons analyser l'influence des différentes tâches employées (\textbf{prétraitement}, \textbf{vectorisation}, \textbf{clustering sous contraintes}, \textbf{échantillonnage}) et de leurs paramètres sur la vitesse de convergence vers la vérité terrain.
			% Description implémentation de l'interactive clustering.
			Nous avons toujours $192$ combinaisons testées, et chaque tentative est répétée $5$ fois pour contrer les aléas statistiques de certains algorithmes.
			Pour plus de détails sur ces algorithmes, référez-vous à la section~\ref{section:3.3-DESCRIPTION-IMPLEMENTATION}.
			
			% Description de l'évaluation et Seuils d'évaluation.
			Comme lors de l'étude sur la convergence de la méthode, nous nous intéresserons à l'évolution de la \texttt{v-measure} entre la vérité terrain et notre segmentation des données obtenue, et nous affinerons notre évaluation en portant attention aux trois seuils d'annotations suivants :
			\begin{enumerate}
				\item le cas d'une \textbf{annotation partielle}, correspondant au nombre d'itérations nécessaires à la méthode pour avoir $90$\% de \texttt{v-measure} entre le résultat obtenu et la vérité terrain, c'est-à-dire un état de semi-parcours vers une convergence totale\footnote{Le seuil de $90$\% a été choisi au cours de l'étude de convergence (cf. hypothèse d'efficacité, section~\ref{section:4.1-HYPOTHESE-EFFICACITE}).} ;
				\item le cas d'une \textbf{annotation suffisante}, correspondant au nombre d'itérations nécessaires à la méthode pour avoir $100$\% de \texttt{v-measure} entre le résultat obtenu et la vérité terrain, c'est-à-dire avoir suffisamment de contraintes annotées par l'expert métier pour retrouver la vérité terrain ;
				\item le cas d'une \textbf{annotation exhaustive}, correspondant au nombre d'itérations nécessaires à la méthode pour parcourir toutes les contraintes possibles sur les données, et ainsi retranscrire exhaustivement la vision de l'expert métier.
			\end{enumerate}
			
			% Description de l'analyse ANOVA.
			Enfin, nous utiliserons une \texttt{ANOVA} à mesures répétées afin de déterminer l’effet des paramètres de notre implémentation sur le nombre d’annotations requis pour converger vers la vérité terrain. Ces analyses sont réalisées à l'aide du logiciel R\todo{citation}, et le test de \texttt{Tukey (HSD)} est utilisé pour les comparaisons post-hoc.
			
			% Référence scripts.
			\begin{leftBarInformation}
				Les scripts de l'expérience (\textit{notebooks} Python) sont disponibles dans un dossier dédié de~\cite{schild:cognitivefactory-interactive-clustering-comparative-study:2021}.
			\end{leftBarInformation}

		%%% Résultats
		\subsubsection{Résultats obtenus}
		
			%%% Analyse d'une annotation partielle.
			Pour obtenir une \textbf{annotation partielle} (\textit{atteindre une \texttt{v-measure} de $90$\%}), la moyenne des itérations est de $59.04$ (min: $11$, max: $315$, écart-type: $42.14$), soit une moyenne de $2~951.81$ annotations (min: $550$, max: $15~750$, écart-type: $2~106.72$).
			La figure~\ref{figure:4.2.1-ETUDE-OPTIMISATION-HISTOGRAMME-ANNOTATION-PARTIELLE} représente la répartition de ces itérations au cours des différentes tentatives.
			On peut noter les deux cas intéressants suivants :
			%
			\begin{itemize}
				\item[$\bullet$] Les tentatives les plus rapides furent celles avec un prétraitement des données \texttt{prep.no} ou \texttt{prep.simple} ou \texttt{prep.lemma}, une vectorisation des données \texttt{vect.tfidf}, un clustering sous contraintes \texttt{clust.hier.sing}, et un échantillonnage de contraintes \texttt{samp.closest.diff}. Ces tentatives ont requis $11$ itérations, soit $550$ annotations, dont $299$ (respectivement $304$ et $281$) contraintes \texttt{MUST-LINK}.
				\item[$\bullet$] Les tentatives les plus lentes furent celles avec un prétraitement des données \texttt{prep.no}, une vectorisation des données \texttt{vect.tfidf}, un clustering sous contraintes \texttt{clust.spec}, et un échantillonnage de contraintes \texttt{samp.farthest.same}. Ces tentatives ont requis $315$ itérations, soit $15~750$ annotations, dont $1~032$ contraintes \texttt{MUST-LINK}.
			\end{itemize}
			%
			\begin{figure}[!htb]
				\centering
				\includegraphics[width=0.7\textwidth]{figures/etude-efficience-histogramme-annotation-partielle}
				\caption{Répartition des tentatives en fonction de l'itération de la méthode à laquelle elles atteignent le seuil d'une annotation partielle, c'est-à-dire l'itération à laquelle elles parviennent à $90$\% de \texttt{v-measure} entre un résultat obtenu et la vérité terrain. L'histogramme est réduit à $60$ pics pour simplifier l'affichage.}
				\label{figure:4.2.1-ETUDE-OPTIMISATION-HISTOGRAMME-ANNOTATION-PARTIELLE}
			\end{figure}
			%
			Le tableau~\ref{table:4.2.1-ETUDE-OPTIMISATION-ANOVA-ANNOTATION-PARTIELLE} retranscrit l'influence de chacun des paramètres sur le nombre d'itérations nécessaires pour atteindre une \textbf{annotation partielle} (\textit{atteindre une \texttt{v-measure} de $90$\%}).
			Les analyses de variance mettent en relief l'effet significatif sur cette convergence du prétraitement (\texttt{eta-carré}: $0.320$, \texttt{p-valeur}: $< 10^{-3}$), de la vectorisation (\texttt{eta-carré}: $0.388$, \texttt{p-valeur}: $< 10^{-3}$), du clustering (\texttt{eta-carré}: $0.866$, \texttt{p-valeur}: $< 10^{-3}$) et de l'échantillonnage (\texttt{eta-carré}: $0.968$, \texttt{p-valeur}: $< 10^{-3}$).
			L'analyse post-hoc de ces effets indique que le meilleur paramétrage moyen pour atteindre une \textbf{annotation partielle} repose sur la prétraitement \texttt{prep.simple}, le vectorisation \texttt{vect.tfidf}, le clustering \texttt{clust.hier.avg}, et l'échantillonnage \texttt{samp.closest.diff}. La moyenne du nombre d'itération requis pour ce paramétrage est de $19.00$ (écart-type: $0.79$), soit $950$ annotations (écart-type: $39.34$).
			%
			\begin{table}[!htb]
				\begin{center}
				\begin{tabular}{|c|c|c|c|c|c|c|}
					\hline
					% ENTETE DU TABLEAU
					\multicolumn{2}{|c|}{ \shortstack{Description des \\ facteurs analysés } }
						& \multicolumn{3}{c|}{ \shortstack{ Description \\ statistique } }
						& \multicolumn{2}{c|}{ \shortstack{ Description des \\ tailles d'effets } }
						\tabularnewline
						\hline

					Facteur
						& Niveau 
						& Moyenne
						& Rang
						& SE
						& \texttt{ $ \eta^{2} $ }
						& \texttt{p-valeur}
						\tabularnewline
						\hline
					
					% PRETRAITEMENT
					\multirow{4}{*}{prétraitement}
						& \texttt{prep.simple}
						& $61.90$
						& (1)
						& \multirow{4}{*}{ $0.32$ }
						& \multirow{4}{*}{ $0.320$ }
						& \multirow{4}{*}{ \shortstack{ $< 10^{-3}$ \\ ($***$) } }
						\tabularnewline
						\cline{2-4}
						
						& \texttt{prep.lemma}
						& $63.08$
						& (2)
						&
						&
						&
						\tabularnewline
						\cline{2-4}
						
						& \texttt{prep.no}
						& $63.70$
						& (2)
						&
						& 
						&
						\tabularnewline
						\cline{2-4}
						
						& \texttt{prep.filter}
						& $71.90$
						& (4)
						&
						&
						&
						\tabularnewline
						\hline
					
					% VECTORISATION
					\multirow{2}{*}{vectorisation}
						& \texttt{vect.tfidf}
						& $60.61$
						& (1)
						& \multirow{2}{*}{ $0.29$ }
						& \multirow{2}{*}{ $0.388$ }
						& \multirow{2}{*}{ \shortstack{$< 10^{-3}$ \\ ($***$) } }
						\tabularnewline
						\cline{2-4}
						
						& \texttt{vect.frcorenewsmd}
						& $63.08$
						& (2)
						&
						&
						&
						\tabularnewline
						\hline
					
					% CLUSTERING
					\multirow{6}{*}{clustering}
						& \texttt{clust.hier.avg}
						& $50.64$
						& (1)
						& \multirow{6}{*}{ $0.35$ }
						& \multirow{6}{*}{ $0.866$ }
						& \multirow{6}{*}{ \shortstack{ $< 10^{-3}$ \\ ($***$) } }
						\tabularnewline
						\cline{2-4}
						
						& \texttt{clust.kmeans.cop}
						& $52.43$
						& (2)
						&
						&
						&
						\tabularnewline
						\cline{2-4}
						
						& \texttt{clust.hier.sing}
						& $54.08$
						& (3)
						&
						& 
						&
						\tabularnewline
						\cline{2-4}
						
						& \texttt{clust.hier.ward}
						& $72.41$
						& (4)
						&
						& 
						&
						\tabularnewline
						\cline{2-4}
						
						& \texttt{clust.hier.comp}
						& $73.48$
						& (5)
						&
						&
						&
						\tabularnewline
						\cline{2-4}
						
						& \texttt{clust.spec}
						& $87.84$
						& (6)
						&
						& 
						&
						\tabularnewline
						\hline
					
					% ECHANTILLONNAGE
					\multirow{4}{*}{échantillonnage}
						& \texttt{samp.closest.diff}
						& $33.66$
						& (1)
						& \multirow{4}{*}{ $0.32$ }
						& \multirow{4}{*}{ $0.968$ }
						& \multirow{4}{*}{ \shortstack{ $< 10^{-3}$ \\ ($***$) } }
						\tabularnewline
						\cline{2-4}
						
						& \texttt{samp.random.same}
						& $48.24$
						& (2)
						&
						&
						&
						\tabularnewline
						\cline{2-4}
						
						& \texttt{samp.random.full}
						& $65.83$
						& (3)
						&
						& 
						&
						\tabularnewline
						\cline{2-4}
						
						& \texttt{samp.farhtest.same}
						& $112.86$
						& (4)
						&
						&
						&
						\tabularnewline
						\hline
				\end{tabular}
				\end{center}
				\caption{ANOVA du nombre d'itérations nécessaires pour l'obtention de $90$\% de v-mesure. Les (\textit{$*$}) dénotent le niveau de significativité ($\alpha=0.05$). Pour les effets significatifs, les chiffres précisés entre parenthèses dans la colonne \texttt{Moyenne} indiquent le classement des niveaux selon les analyses post-hoc.}
				\label{table:4.2.1-ETUDE-OPTIMISATION-ANOVA-ANNOTATION-PARTIELLE}
			\end{table}
			

			%%% Analyse d'une annotation suffisante.
			Pour obtenir une \textbf{annotation suffisante} (\textit{atteindre une \texttt{v-measure} de $100$\%}), la moyenne des itérations est de $76.29$ (min: $19$, max: $328$, écart-type: $46.44$), soit une moyenne de $3~801.19$ annotations (min: $950$, max: $16~400$, écart-type: $2~314.91$).
			La figure~\ref{figure:4.2.1-ETUDE-OPTIMISATION-HISTOGRAMME-ANNOTATION-SUFFISANTE} représente la répartition de ces itérations au cours des différentes tentatives.
			On peut noter les deux cas intéressants suivants :
			%
			\begin{itemize}
				\item[$\bullet$] Les tentatives les plus rapides furent celles avec un prétraitement des données \texttt{prep.simple}, une vectorisation des données \texttt{vect.tfidf}, un clustering sous contraintes \texttt{clust.hier.comp} ou \texttt{clust.hier.ward}, et un échantillonnage de contraintes \texttt{samp.closest.diff}. Ces tentatives ont requis $19$ itérations, soit $950$ annotations, dont $638$ (respectivement $641$) contraintes \texttt{MUST-LINK}.
				\item[$\bullet$] Les tentatives les plus lentes furent celles avec un prétraitement des données \texttt{prep.no}, une vectorisation des données \texttt{vect.tfidf}, un clustering sous contraintes \texttt{clust.spec}, et un échantillonnage de contraintes \texttt{samp.farthest.same}. Ces tentatives ont requis $394$ itérations, soit $16~400$ annotations, dont $1~309$ contraintes \texttt{MUST-LINK}.
			\end{itemize}
			%
			\begin{figure}[!htb]
				\centering
				\includegraphics[width=0.7\textwidth]{figures/etude-efficience-histogramme-annotation-suffisante}
				\caption{Répartition des tentatives en fonction de l'itération de la méthode à laquelle elles atteignent le seuil d'une annotation suffisante, c'est-à-dire l'itération à laquelle elles parviennent à $100$\% de \texttt{v-measure} entre un résultat obtenu et la vérité terrain. L'histogramme est réduit à $60$ pics pour simplifier l'affichage.}
				\label{figure:4.2.1-ETUDE-OPTIMISATION-HISTOGRAMME-ANNOTATION-SUFFISANTE}
			\end{figure}
			%
			Le tableau~\ref{table:4.2.1-ETUDE-OPTIMISATION-ANOVA-ANNOTATION-SUFFISANTE} retranscrit l'influence de chacun des paramètres sur le nombre d'itérations nécessaires pour atteindre une \textbf{annotation suffisante}.
			Les analyses de variance mettent en relief l'effet significatif sur cette convergence du prétraitement (\texttt{eta-carré}: $0.987$, \texttt{p-valeur}: $< 10^{-3}$), de la vectorisation (\texttt{eta-carré}: $0.991$, \texttt{p-valeur}: $< 10^{-3}$), du clustering (\texttt{eta-carré}: $0.997$, \texttt{p-valeur}: $< 10^{-3}$) et de l'échantillonnage (\texttt{eta-carré}: $0.998$, \texttt{p-valeur}: $< 10^{-3}$).
			L'analyse post-hoc de ces effets indique que le meilleur paramétrage moyen pour atteindre une \textbf{annotation suffisante} repose sur la prétraitement \texttt{prep.lemma}, le vectorisation \texttt{vect.tfidf}, le clustering \texttt{clust.kmeans.cop}, et l'échantillonnage \texttt{samp.closest.diff}. La moyenne du nombre d'itération requis pour ce paramétrage est de $34.60$ (écart-type: $7.44$), soit $1~730$ annotations (écart-type: $372.00$).
			%
			\begin{table}[!htb]
				\begin{center}
				\begin{tabular}{|c|c|c|c|c|c|c|}
					\hline
					% ENTETE DU TABLEAU
					\multicolumn{2}{|c|}{ \shortstack{Description des \\ facteurs analysés } }
						& \multicolumn{3}{c|}{ \shortstack{ Description \\ statistique } }
						& \multicolumn{2}{c|}{ \shortstack{ Description des \\ tailles d'effets } }
						\tabularnewline
						\hline

					Facteur
						& Niveau 
						& Moyenne
						& Rang
						& SE
						& \texttt{ $\eta^{2}$ }
						& \texttt{p-valeur}
						\tabularnewline
						\hline
					
					% PRETRAITEMENT
					\multirow{4}{*}{prétraitement}
						& \texttt{prep.lemma}
						& $72.86$
						& (1)
						& \multirow{4}{*}{ $0.32$ }
						& \multirow{4}{*}{ $0.276$ }
						& \multirow{4}{*}{ \shortstack{ $< 10^{-3}$ \\ ($***$) } }
						\tabularnewline
						\cline{2-4}
						
						& \texttt{prep.simple}
						& $73.30$
						& (2)
						&
						&
						&
						\tabularnewline
						\cline{2-4}
						
						& \texttt{prep.no}
						& $75.24$
						& (2)
						&
						& 
						&
						\tabularnewline
						\cline{2-4}
						
						& \texttt{prep.filter}
						& $83.77$
						& (4)
						&
						&
						&
						\tabularnewline
						\hline
					
					% VECTORISATION
					\multirow{2}{*}{vectorisation}
						& \texttt{vect.tfidf}
						& $71.16$
						& (1)
						& \multirow{2}{*}{ $0.36$ }
						& \multirow{2}{*}{ $0.366$ }
						& \multirow{2}{*}{ \shortstack{$< 10^{-3}$ \\ ($***$)} }
						\tabularnewline
						\cline{2-4}
						
						& \texttt{vect.frcorenewsmd}
						& $81.43$
						& (2)
						&
						&
						&
						\tabularnewline
						\hline
					
					% CLUSTERING
					\multirow{6}{*}{clustering}
						& \texttt{clust.kmeans.cop}
						& $62.23$
						& (1)
						& \multirow{6}{*}{ $0.42$ }
						& \multirow{6}{*}{ $0.700$ }
						& \multirow{6}{*}{ \shortstack{$< 10^{-3}$ \\ ($***$)} }
						\tabularnewline
						\cline{2-4}
						
						& \texttt{clust.hier.avg}
						& $65.13$
						& (2)
						&
						&
						&
						\tabularnewline
						\cline{2-4}
						
						& \texttt{clust.hier.sing}
						& $75.44$
						& (3)
						&
						& 
						&
						\tabularnewline
						\cline{2-4}
						
						& \texttt{clust.hier.ward}
						& $80.44$
						& (4)
						&
						& 
						&
						\tabularnewline
						\cline{2-4}
						
						& \texttt{clust.hier.comp}
						& $81.46$
						& (5)
						&
						&
						&
						\tabularnewline
						\cline{2-4}
						
						& \texttt{clust.spec}
						& $93.06$
						& (6)
						&
						& 
						&
						\tabularnewline
						\hline
					
					% ECHANTILLONNAGE
					\multirow{4}{*}{échantillonnage}
						& \texttt{samp.closest.diff}
						& $50.29$
						& (1)
						& \multirow{4}{*}{ $0.39$ }
						& \multirow{4}{*}{ $0.950$ }
						& \multirow{4}{*}{ \shortstack{$< 10^{-3}$ \\ ($***$)} }
						\tabularnewline
						\cline{2-4}
						
						& \texttt{samp.random.same}
						& $56.38$
						& (2)
						&
						&
						&
						\tabularnewline
						\cline{2-4}
						
						& \texttt{samp.random.full}
						& $71.95$
						& (3)
						&
						& 
						&
						\tabularnewline
						\cline{2-4}
						
						& \texttt{samp.farhtest.same}
						& $126.55$
						& (4)
						&
						&
						&
						\tabularnewline
						\hline
				\end{tabular}
				\end{center}
				\caption{ANOVA du nombre d'itérations nécessaires pour l'obtention de $100$\% de v-mesure. Les (\textit{$*$}) dénotent le niveau de significativité ($\alpha=0.05$). Pour les effets significatifs, les chiffres précisés entre parenthèses dans la colonne \texttt{Moyenne} indiquent le classement des niveaux selon les analyses post-hoc.}
				\label{table:4.2.1-ETUDE-OPTIMISATION-ANOVA-ANNOTATION-SUFFISANTE}
			\end{table}
			
			%%% Analyse d'une annotation exhaustive.
			Enfin, pour avoir une \textbf{annotation exhaustive} (\textit{annoter toutes les contraintes possibles}), la moyenne des itérations est de $88.98$ (min: $20$, max: $394$, écart-type: $68.21$), soit une moyenne de $4~431.34$ annotations (min: $1~000$, max: $19~656$, écart-type: $3~405.16$).
			La figure~\ref{figure:4.2.1-ETUDE-OPTIMISATION-HISTOGRAMME-ANNOTATION-EXHAUSTIVE} représente la répartition de ces itérations au cours des différentes tentatives.
			On peut noter les deux cas intéressants suivant :
			%
			\begin{itemize}
				\item[$\bullet$] Les tentatives les plus rapides furent celles avec un prétraitement des données \texttt{prep.no} ou \texttt{prep.lemma}, une vectorisation des données \texttt{vect.tfidf}, un algorithme de clustering sous contraintes \texttt{clust.hier.comp} ou \texttt{clust.hier.wars}, et un échantillonnage de contraintes \texttt{samp.closest.diff}. Ces tentatives ont requis $20$ itérations, soit $1~000$ annotations, dont $653$ (respectivement $668$) contraintes \texttt{MUST-LINK}.
				\item[$\bullet$] Les tentatives les plus lentes furent celles avec un prétraitement des données \texttt{prep.simple}, une vectorisation des données \texttt{vect.frcorenewsmd}, un clustering sous contraintes \texttt{clust.hier.sing}, et un échantillonnage de contraintes \texttt{samp.closest.diff}. Ces tentatives ont requis $394$ itérations, soit $19~656$ annotations, dont $682$ contraintes \texttt{MUST-LINK}.
			\end{itemize}
			%
			\begin{figure}[!htb]
				\centering
				\includegraphics[width=0.7\textwidth]{figures/etude-efficience-histogramme-annotation-exhaustive}
				\caption{Répartition des tentatives en fonction de l'itération de la méthode à laquelle elles atteignent le seuil d'une annotation exhaustive, c'est-à-dire l'itération à laquelle toutes les contraintes possibles entre les données ont été annotées. L'histogramme est réduit à $60$ pics pour simplifier l'affichage.}
				\label{figure:4.2.1-ETUDE-OPTIMISATION-HISTOGRAMME-ANNOTATION-EXHAUSTIVE}
			\end{figure}
			%
			Le tableau~\ref{table:4.2.1-ETUDE-OPTIMISATION-ANOVA-ANNOTATION-EXHAUSTIVE} retranscrit l'influence de chacun des paramètres sur le nombre d'itérations nécessaires pour atteindre une \textbf{annotation exhaustive}.
			Les analyses de variance mettent en relief l'effet significatif sur cette convergence du prétraitement (\texttt{eta-carré}: $0.909$, \texttt{p-valeur}: $< 10^{-3}$), de la vectorisation (\texttt{eta-carré}: $0.985$, \texttt{p-valeur}: $< 10^{-3}$), du clustering (\texttt{eta-carré}: $0.999$, \texttt{p-valeur}: $< 10^{-3}$) et de l'échantillonnage (\texttt{eta-carré}: $0.997$, \texttt{p-valeur}: $< 10^{-3}$).
			L'analyse post-hoc de ces effets indique que le meilleur paramétrage moyen pour atteindre une \textbf{annotation exhaustive} repose sur la prétraitement \texttt{prep.lemma}, le vectorisation \texttt{vect.tfidf}, le clustering \texttt{clust.kmeans.cop}, et l'échantillonnage \texttt{samp.random.same}. La moyenne du nombre d'itération requis pour ce paramétrage est de $32.60$ (écart-type: $1.14$), soit $1~630$ annotations (écart-type: $57.00$).
			%
			\begin{table}[!htb]
				\begin{center}
				\begin{tabular}{|c|c|c|c|c|c|c|}
					\hline
					% ENTETE DU TABLEAU
					\multicolumn{2}{|c|}{ \shortstack{Description des \\ facteurs analysés } }
						& \multicolumn{3}{c|}{ \shortstack{ Description \\ statistique } }
						& \multicolumn{2}{c|}{ \shortstack{ Description des \\ tailles d'effets } }
						\tabularnewline
						\hline

					Facteur
						& Niveau 
						& Moyenne
						& Rang
						& SE
						& \texttt{ $\eta^{2}$ }
						& \texttt{p-valeur}
						\tabularnewline
						\hline
					
					% PRETRAITEMENT
					\multirow{4}{*}{prétraitement}
						& \texttt{prep.lemma}
						& $85.89$
						& (1)
						& \multirow{4}{*}{ $0.42$ }
						& \multirow{4}{*}{ $0.052$ }
						& \multirow{4}{*}{ \shortstack{$< 10^{-3}$ \\ ($***$)} }
						\tabularnewline
						\cline{2-4}
						
						& \texttt{prep.filter}
						& $89.55$
						& (2)
						&
						&
						&
						\tabularnewline
						\cline{2-4}
						
						& \texttt{prep.simple}
						& $89.64$
						& (2)
						&
						& 
						&
						\tabularnewline
						\cline{2-4}
						
						& \texttt{prep.no}
						& $90.81$
						& (4)
						&
						&
						&
						\tabularnewline
						\hline
					
					% VECTORISATION
					\multirow{2}{*}{vectorisation}
						& \texttt{vect.tfidf}
						& $85.50$
						& (1)
						& \multirow{2}{*}{ $0.39$ }
						& \multirow{2}{*}{ $0.165$ }
						& \multirow{2}{*}{ \shortstack{$< 10^{-3}$ \\ ($***$)} }
						\tabularnewline
						\cline{2-4}
						
						& \texttt{vect.frcorenewsmd}
						& $92.46$
						& (2)
						&
						&
						&
						\tabularnewline
						\hline
					
					% CLUSTERING
					\multirow{6}{*}{clustering}
						& \texttt{clust.kmeans.cop}
						& $64.99$
						& (1)
						& \multirow{6}{*}{ $0.39$ }
						& \multirow{6}{*}{ $0.894$ }
						& \multirow{6}{*}{ \shortstack{$< 10^{-3}$ \\ ($***$)} }
						\tabularnewline
						\cline{2-4}
						
						& \texttt{clust.hier.avg}
						& $78.54$
						& (2)
						&
						&
						&
						\tabularnewline
						\cline{2-4}
						
						& \texttt{clust.hier.ward}
						& $81.31$
						& (3)
						&
						& 
						&
						\tabularnewline
						\cline{2-4}
						
						& \texttt{clust.hier.comp}
						& $82.49$
						& (3)
						&
						& 
						&
						\tabularnewline
						\cline{2-4}
						
						& \texttt{clust.spec}
						& $93.78$
						& (5)
						&
						&
						&
						\tabularnewline
						\cline{2-4}
						
						& \texttt{clust.hier.comp}
						& $132.75$
						& (6)
						&
						& 
						&
						\tabularnewline
						\hline
					
					% ECHANTILLONNAGE
					\multirow{4}{*}{échantillonnage}
						& \texttt{samp.random.same}
						& $57.23$
						& (1)
						& \multirow{4}{*}{ $0.42$ }
						& \multirow{4}{*}{ $0.930$ }
						& \multirow{4}{*}{ \shortstack{$< 10^{-3}$ \\ ($***$)} }
						\tabularnewline
						\cline{2-4}
						
						& \texttt{samp.random.full}
						& $72.80$
						& (2)
						&
						&
						&
						\tabularnewline
						\cline{2-4}
						
						& \texttt{samp.closest.diff}
						& $98.38$
						& (3)
						&
						& 
						&
						\tabularnewline
						\cline{2-4}
						
						& \texttt{samp.farhtest.same}
						& $132.75$
						& (4)
						&
						&
						&
						\tabularnewline
						\hline
				\end{tabular}
				\end{center}
				\caption{ANOVA du nombre d'itérations nécessaires pour annoter toutes les contraintes possibles. Les (\textit{$*$}) dénotent le niveau de significativité ($\alpha=0.05$). Pour les effets significatifs, les chiffres précisés entre parenthèses dans la colonne \texttt{Moyenne} indiquent le classement des niveaux selon les analyses post-hoc.}
				\label{table:4.2.1-ETUDE-OPTIMISATION-ANOVA-ANNOTATION-EXHAUSTIVE}
			\end{table}
		
		
			% Graphe d'évolution de la v-measure moyenne, min et max.
			La figure~\ref{figure:4.2.1-ETUDE-OPTIMISATION-EVOLUTION-PAR-FACTEURS} représente les évolutions moyennes de la \texttt{v-measure} du clustering en fonction du nombre d'itération de la méthode pour les différentes valeurs des facteurs analysés (prétraitement en haut à gauche, vectorisation en haut à droite, clustering en bas à gauche, échantillonnage en bas à droite).
			La figure~\ref{figure:4.2.1-ETUDE-OPTIMISATION-EVOLUTION-MEILLEUR-PARAMETRAGE} représente cette même évolution pour les meilleurs paramétrages moyens destinés à atteindre les trois seuils d'annotation définis (partiel, suffisant, exhaustif).
			%
			\begin{figure}[!htb]
				\centering
				\includegraphics[width=\textwidth]{figures/etude-efficience-evolution-moyenne-par-vmeasure-par-facteur}
				\caption{Évolution des moyennes du nombre d'itérations nécessaire de la méthode de \textit{clustering} interactif pour obtenir un seuil défini de \texttt{v-measure} entre un résultat obtenu et la vérité terrain, moyennes réalisées sur les différentes valeurs que peuvent prendre les facteurs analysés et affichées par facteur : \textbf{(1)} prétraitement, \textbf{(2)} vectorisation, \textbf{(3)} clustering et \textbf{(4)} échantillonnage. Le seuil d'annotation exhaustive (annoter toutes les contraintes possibles) n'étant pas exprimé en terme de \texttt{v-measure}, ce seuil n'est pas affiché ici.}
				\label{figure:4.2.1-ETUDE-OPTIMISATION-EVOLUTION-PAR-FACTEURS}
			\end{figure}
			%
			\begin{figure}[!htb]
				\centering
				\includegraphics[width=0.8\textwidth]{figures/etude-efficience-evolution-moyenne-5best-par-vmeasure}
				\caption{Évolution des moyennes du nombre d'itérations nécessaire de la méthode de \textit{clustering} interactif pour obtenu un seuil défini de \texttt{v-measure} entre un résultat obtenu et la vérité terrain, moyennes réalisées sur les différentes seuils d'annotations étudiés : l'annotation partielle (\textit{atteindre une \texttt{v-measure} de $90$\%}), l'annotation suffisante (\textit{atteindre une \texttt{v-measure} de $100$\%}) et l'annotation exhaustive (\textit{annoter toutes les contraintes possibles}).}
				\label{figure:4.2.1-ETUDE-OPTIMISATION-EVOLUTION-MEILLEUR-PARAMETRAGE}
			\end{figure}

		%%% Discussion
		\subsubsection{Discussion}

			% Rappel de l'objectif : être efficient.
			L'objectif de l'étude est de trouver une implémentation "efficiente" du \textit{clustering} interactif permettant d'obtenir une base d'apprentissage correctement annotée en un minimum d'annotation.
			Pour trouver si une telle implémentation existe et quels en sont les paramètres optimaux, nous avons analysé l'impact de différentes paramétrages sur les tâches principales de la méthode (\textbf{prétraitement}, \textbf{vectorisation}, \textbf{clustering sous contraintes}, \textbf{échantillonnage}) en nous basant sur des simulations d'annotation d'un jeu de données.
			
			% Première remarque : Choix d'un seuil à 90\% de v-measure.
			Dans l'optique d'être efficient, nous excluons le désir d'annoter \textbf{exhaustivement} le jeu de données car la charge de travail estimée est trop importante.
			(cf. discussion de la section~\ref{section:4.1-HYPOTHESE-EFFICACITE} (hypothèse d'efficacité))
			Nous préférons donc nous concentrer sur deux seuils d'annotation plus réalistes : celui d'une \textbf{annotation partielle} (atteindre $90$\% de \texttt{v-measure} avec la vérité terrain) et celui d'une \textbf{annotation suffisante} (atteindre $100$\% de \texttt{v-measure} avec la vérité terrain en un minimum de contraintes). 
			
			% Meilleur paramétrage.
			L'étude réalisée met en avant l'impact significatif des quatre tâches principales (\textbf{prétraitement}\todo{remarque sur la valeur de eta2}, \textbf{vectorisation}\todo{remarque sur la valeur de eta2}, \textbf{clustering sous contraintes}, \textbf{échantillonnage}) sur la vitesse de convergence de la méthode pour atteindre les seuils définis de $90$\% et $100$\% de \texttt{v-measure}. Il existe donc bien un paramétrage permettant d'optimiser l'implémentation proposée et de réduire le nombre de contraintes nécessaires à annoter :
			\begin{enumerate}
				\item pour une \textbf{annotation partielle} ($90$\% de \texttt{v-measure}), le meilleur paramétrage moyen est constitué du prétraitement simple (\texttt{prep.simple}), de la vectorisation TF-IDF (\texttt{vect.tfidf}), du clustering hiérarchique à lien moyen (\texttt{clust.hier.avg}) et de l'échantillonnage des données les plus proches dans des clusters différents (\texttt{sampl.closest.diff}). Avec ce paramétrage, il faut en moyenne $950$ annotations de contraintes pour obtenir une \texttt{v-measure} de $90$\% ;
				\item pour une \textbf{annotation suffisante} ($100$\% de \texttt{v-measure}), le meilleur paramétrage moyen est constitué du prétraitement avec lemmatisation (\texttt{prep.lemma}), de la vectorisation TF-IDF (\texttt{vect.tfidf}), du clustering KMeans (\texttt{clust.kmeans.cop}) et de l'échantillonnage des données les plus proches dans des clusters différents (\texttt{sampl.closest.diff}). Avec ce paramétrage, il faut en moyenne $1~750$ annotations de contraintes pour obtenir une \texttt{v-measure} de $100$\% ;
				\item le cas d'une \textbf{annotation exhaustive} (annoter toutes les contraintes possibles sur les données) n'est pas explicité ici mais peut se déduire des résultats décrits plus haut.
			\end{enumerate}


			%%% Avantages.
			Ainsi, cette étude permet de répondre à certaines limites discutées dans la section~\ref{section:4.1-HYPOTHESE-EFFICACITE} (hypothèse d'efficacité). 
			
			% Avantage 1: Optimisation du nombre de contraintes.
			En effet, l'optimisation des paramètres de l'implémentation du \textit{clustering} interactif permet de réduire considérablement le nombre de contraintes nécessaires pour obtenir une base d'apprentissage exploitable.
			En nous basant sur le tableau~\ref{table:4.1.1-ETUDE-CONVERGENCE-EVOLUTION} de l'étude de convergence, et dans le cadre de l'annotation d'un jeu de $500$ données, nous sommes passé d'un paramétrage moyen nécessitant $3~750$ (respectivement $10~000$) contraintes à un paramétrage optimisé ne nécessitant que $950$ (respectivement $1~750$) contraintes pour atteindre un seuil de $90$\% (respectivement $100$\%) \texttt{v-measure}.
			L'ordre de grandeur de la charge de travail demandée aux annotateurs est donc située entre $2$ et $4$ fois la taille du jeu de données.
			
			% Avantage 2: La méthode devient réaliste !
			En considérant que les annotations sont binaires et demandent a priori une charge mental plus faible que les annotations par attribution de label ("\textit{les données sont-elles similaires ?}" vs "\textit{quel est l'étiquette de cette donnée ?}"), nous pouvons conclure que la charge totale nécessaire à l'annotation avec une méthodologie basée sur le \textit{clustering} interactif est comparable à celles des méthodes traditionnelles.
			De plus, cette méthode ne demande pas de formalisation concrète de la structure de données à annoter pour faire émerger une base d'apprentissage au cours des itérations, donc le \textit{clustering} interactif devient une méthode d'annotation adaptée à l'activité des annotateurs.
			
			%%% Limites.
			Néanmoins, quelques pistes sont encore à explorer pour compléter cette analyse d'efficience.
			
			% Limite 1 : Coût temporel.
			D'une part, une étude de coût est à réaliser pour trancher le choix de paramètre optimaux réalistes. En effet, il est intéressant d'étudier le coût machine (temps CPU utilisé) et le coût humain (temps d'annotation) afin d'affiner les choix techniques et de compléter les arguments sur l'utilisation en situation réelle d'une méthodologie d'annotation basée sur le \textit{clustering} interactif.
			Cet aspect sera traité dans la section~\ref{section:4.3-HYPOTHESE-COUTS} (hypothèse des coûts).
			
			% Limite 2 : Valeur métier de ce 90\% (pas de vérité terrain en pratique).
			D'autre part, l'étude réalisée se base sur des seuils de performance par rapport à une vérité terrain.
			Or en situation réelle, cette comparaison avec la vérité terrain n'est pas possible car elle est précisément en cours de conception (la base d'apprentissage finale devant être la vérité terrain).
			De plus, un tel score n'est pas le plus explicite pour pour un expert métier pour qui un score de \texttt{v-measure} n'est pas révélateur de la pertinence métier de la segmentation proposée des données.
			Il manque donc une stratégie d'évaluation de pertinence de la base d'apprentissage en cours de construction et de la suffisance des annotations réalisées pour faire refléter la vision de l'annotateur dans le résultat.
			Cet aspect sera traité dans la section~\ref{section:4.4-HYPOTHESE-PERTINENCE} (hypothèse de pertinence).
			
			% Limite 3 : Expert métier parfait ==> simuler les erreurs.
			Pour finir, comme pour l'étude de convergence réalisé en section~\ref{section:4.1-HYPOTHESE-EFFICACITE}, nous avons supposé dans cette étude que l'annotateur est un expert métier connaissant parfaitement le domaine traité.
			Cette hypothèse forte n'est a priori pas valable en situation réelle : En effet, des erreurs d'annotations peuvent intervenir (ambiguïtés sur les données, méconnaissance du domaine, erreurs d'inattention, différence d'opinions entre annotateurs, ...), ce qui peut entraîner des divergences ou des incohérences dans la construction de la base d'apprentissage.
			Il semble donc nécessaire d'étudier les impacts de ces incohérences, ainsi que de proposer une méthode pour les prévenir ou les corriger.
			Cet aspect sera traité à la fin de ce chapitre dans la section~\ref{section:4.6-HYPOTHESE-ROBUSTESSE} (hypothèse de robustesse).
	
	
	%%%%%--------------------------------------------------------------------
	%%%%% Section 4.3: Hypothèse sur les coûts.
	%%%%%--------------------------------------------------------------------
	\newpage
	\section{Évaluation de l'hypothèse sur les coûts}
\label{section:4.3-HYPOTHESE-COUTS}
% : « \textit{combien dois-je investir ?} »

	%%% Introduction / Transition.
	Dans les deux sections précédentes, nous avons estimé le paramétrage du \textit{clustering} interactif le plus efficient pour atteindre $90$\% de \texttt{v-measure} avec la vérité terrain, correspondant à ce que nous appelons une annotation partielle.
	Toutefois, pour compléter l'étude de faisabilité technique de notre méthode, nous devons nous intéresser aux coûts (matériel et humain) à investir pour atteindre notre objectif.
	Nous aimerions donc vérifier l'hypothèse suivante :
	
	%%% Formulation des hypothèses:
	\begin{tcolorbox}[
		title=\faVial~\textbf{Hypothèse sur les coûts}~\faVial,
		colback=colorTcolorboxHypothesis!15,
		colframe=colorTcolorboxHypothesis!75,
		width=\linewidth
	]
		% Hypothèse.
		«\textbf{
			Il est possible d'\textbf{estimer les coûts nécessaires} d'une méthodologie d'annotation basée sur le \textit{clustering} interactif pour obtenir une base d'apprentissage exploitable. Nous étudierons en particulier les coûts relatifs au temps d'annotation, au temps de calculs des algorithmes, ainsi que la durée totale de la méthode en fonction de la taille du jeu de données.
		} » \\

		% Résumé des études.
		Afin de vérifier cette hypothèse, nous organiserons plusieurs expériences pour simuler ou déterminer ces durées : une étude du temps d'annotation par un expert métier (cf. section~\ref{section:4.3.1-ETUDE-COUTS-TEMPS-ANNOTATION}), une étude du temps de calcul des algorithmes (cf. section~\ref{section:4.3.2-ETUDE-COUTS-TEMPS-CALCUL}) et une étude du nombre de contraintes nécessaires (cf. section~\ref{section:4.3.3-ETUDE-COUT-NOMBRE-CONTRAINTES}). Nous conclurons l'estimation du temps total d'un projet d'annotation en section~\ref{section:4.3.4-ETUDE-COUTS-TOTAL}.
		
		% Figure.
		La figure~\ref{figure:4.3-HYPOTHESE-COUTS} illustre cette hypothèse et l'espoir de pouvoir caractériser la qualité de la base d'apprentissage en cours de construction en fonction d'un coût temporel au lieu d'un nombre abstrait d'itérations de la méthode. 
		%
		
		\begin{figure}[H]  % keep [H] to be in the tcolorbox.
			\centering
			\includegraphics[width=0.8\textwidth]{figures/hypotheses-03-couts}
			\caption{Illustration des études réalisées sur le \textit{clustering} interactif (\textit{étape 3/6}) en schématisant l'évolution de la performance (\textit{accord avec la vérité terrain calculé en v-measure}) d'une base d'apprentissage en cours de construction en fonction du nombre d'itérations de la méthode (\textit{nombre d'annotations par un expert métier}).}
			\label{figure:4.3-HYPOTHESE-COUTS}
		\end{figure}

	\end{tcolorbox}
	
	
	%%%
	%%% Subsection 4.3.1: Étude du temps d'annotation nécessaire pour traiter un lot de contraintes en chronométrant des opérateurs en situation réelle
	%%%
	\subsection{Étude du temps d'annotation nécessaire pour traiter un lot de contraintes en chronométrant des opérateurs en situation réelle}
	\label{section:4.3.1-ETUDE-COUTS-TEMPS-ANNOTATION}
	
		%%% Protocole expérimental.
		\subsubsection{Protocole expérimental}
		
			% Objectif de l'expérience.
			Nous voulons estimer le temps nécessaire à un opérateur pour annoter un lot de contraintes.
			Pour cela, nous allons chronométrer plusieurs experts métiers en train d'annoter un même échantillon et modéliser le nombre de contraintes par minute, ainsi que son évolution au cours de plusieurs sessions d'annotation.
			
			% Axiome.
			\begin{leftBarWarning}
				Dans cette étude, nous supposons que les annotateurs de l'expérience connaissent parfaitement le domaine traité dans le jeu de données, et qu'ils sont capables de caractériser sans ambiguïté la similitude entre deux données issues de cet ensemble.
				Afin de pourvoir faire cette hypothèse forte, et ainsi limiter les bruits dans l'analyse des résultats, le jeu de données devra traiter d'un sujet de culture générale (ne nécessitant donc pas de connaissance particulière) et des réviseurs supprimeront en amont et d'un commun accord les données trop spécifiques ou trop ambiguës.
			\end{leftBarWarning}
			
			% Objectif supplémentaire.
			Pour aller plus loin, nous aimerions aussi confirmer que le temps d'annotation est caractéristique d'une tâche "courte".
			En utilisant des approximations communes en neuroscience (cf. CITATION)\todo{CITATION: Purves et al. (2008). Principles of Cognitive Neuroscience}, nous pouvons estimer le temps d'annotation d'une contraintes à environ $5$ secondes.
			En effet, il faut $1~200$ms pour lire et traiter deux phrases (on utilise la \texttt{P600} pour estimer la réaction à un stimulus pour une phrase), auquel on ajoute à nouveau $600$ ms pour traiter la concordance entre les deux phrases.
			En considérant une réponse motrice au alentours de $1$ seconde (clic de bouton) et un délais application de $1$ seconde (rechargement de la page), nous sommes autour des $5$ secondes ($4.6$ estimé).
			Bien entendu, cette estimation reste approximative, et dépend fortement de la taille des phrases ainsi que de leur proximité sémantique.
			Toutefois, cela donne un ordre de grandeur pour notre étude du temps d'annotation.
			
			% Pseudo-code.
			Pour résumer le protocole expérimental que nous décrivons ci-dessous, vous pouvez vous référer au pseudo-code décrit dans Alg.~\ref{algorithm:4.3.1-ETUDE-COUTS-TEMPS-ANNOTATION-PROTOCOLE}.
			%
			\begin{algorithm}[!htb]
				\begin{algorithmic}[1]
					\Require jeu de données annoté (vérité terrain)
					\Require plusieurs réviseurs, plusieurs annotateurs
					\State \textbf{initialisation} définir et revoir le jeu de données entre réviseurs
					\State \textbf{échantillonnage} sélectionner une base de contraintes avec \texttt{samp.rand.full}
					\ForAll{annotateur}
						\While{la base de contraintes n'a pas été entièrement annotée}
							\State \textbf{chronomètre: START}
							\State \textbf{annotation}: annoter une partie des contraintes
							\State \textbf{revue}: revue des contraintes en conflits d'annotation
							\State \textbf{chronomètre: STOP}
							\State \textbf{mesure}: estimer la différence de chronomètre pour cette session
						\EndWhile
					\EndFor
					\State \textbf{modélisation}: entraîner un modèle linéaire généralisé du temps d'annotation
					\State \textbf{simulation}: écrire l'équation du temps d'annotation d'un lot de contraintes
					\Ensure modélisation du temps d'annotation d'un lot de contraintes
				\end{algorithmic}
				\caption{Description en pseudo-code du protocole expérimental de l'étude du temps d'annotation d'un lot de contraintes par un expert métier.}
				\label{algorithm:4.3.1-ETUDE-COUTS-TEMPS-ANNOTATION-PROTOCOLE}
			\end{algorithm}
			
			% Détails de l'expérience : préparation du jeu de données.
			Pour cette étude, nous procéderons en plusieurs étapes.
			D'abord, il faut choisir un jeu de données approprié : pour valider notre hypothèse forte sur les compétence de nos annotateurs, nous cherchons un jeu de données traitant d'un sujet de culture général.
			Pour cette expérience, nous avons donc choisi \texttt{MLSUM} : une collecte d'articles de journaux, classés par catégorie de publication et décrits par leur titre et leur résumé.
			Nous nous intéressons ici à la tâche de classification d'un titre d'article en fonction de sa catégorie de publication.
			Comme certains titres peuvent porter à confusion (un titre d'article n'étant pas toujours explicite sur son contenu), deux réviseurs sont chargés de choisir les données les plus explicites sur un échantillon d'un millier de données représentatives des catégories les plus communes.
			L'échantillon résultant, noté \texttt{MLSUM FR TRAIN SUBSET (v1.0.0-schild)}, est composé de $744$ titres d'articles rédigés en français et répartis en $14$ classes (\textit{économie}, \textit{sport}, ...).
			Pour plus de détails, consultez l'annexe~\ref{annex:C.2-DATASET-MLSUM-SUBSET-SCHILD}.
			
			% Détails de l'expérience : sélection des contraintes à annoter. 
			A partir de ces données, nous sélectionnons un lot de $1~000$ contraintes à annoter. Comme nous nous intéressons exclusivement au temps d'annotation pour cette expérience (et que nous ne regardons pas le nombre d'itérations de la méthode), nous utilisons l'échantillonnage purement aléatoire (\texttt{samp.rand.full}).			
			
			% Détails de l'expérience : annotations et consignes.
			Ensuite, un groupe de $14$ annotateurs vont annoter la sélection de $1~000$ contraintes en plusieurs sessions.
			Les directives données aux opérateurs sont les suivantes:
			\begin{itemize}
				\item \textbf{Contexte de l'opérateur} :
				« \textit{Vous êtes des \textbf{experts de la presse et de l’actualité} ; Vous voulez classer des articles dans des catégories en fonction de leur titre ; Vous ne savez pas précisément quelles catégories vous allez utiliser pour classer vos articles ; Mais vous savez \textbf{caractériser la similitude} de deux articles} » ;
				\item \textbf{Contexte sur le jeu de données} :
				« \textit{Le thème sont les catégories d’articles de presse ; La vérité terrain contient entre $10$ et $20$ catégories parmi les plus communes de la presse ; La vérité terrain contient entre $30$ et $100$ articles par catégorie ; Vous \textbf{pouvez regarder le jeu de données non annoté} autant que vous le voulez (disponible dans l'onglet \texttt{TEXTS} de l'application)} » ;
				\item \textbf{Objectif de l'expérience} :
				« \textit{Je veux savoir le temps nécessaire pour annoter un certain nombre de contraintes ; Autrement dit : \textbf{Pour annoter 1000 contraintes, combien de temps me faut-il ?}} » ;
				\item \textbf{Consignes d'annotations} :
				« \textit{Faites des séries de \textbf{15 minutes minimum} pour avoir de la régularité ; Si possible, \textbf{isolez-vous} pour ne pas être dérangé et ne pas fausser les résultats ; Pour chaque série, \textbf{notez le temps et le nombre de contraintes annotés} ; Si vous ne savez pas quoi annoter (trop ambigu, vocabulaire inconnu, ...), \textbf{passez au suivant sans annoter} (vous êtes sensés être des experts de la presse !)} ».
			\end{itemize}
			%
			Pour réaliser l'annotation, les opérateurs auront accès à l'application web développée au cours de ce doctorat.
			Des captures d'écran sont disponibles en figures~\ref{figure:4.3.1-ETUDE-COUTS-TEMPS-ANNOTATION-APPLICATION-ANNOTATION}et~\ref{figure:4.3.1-ETUDE-COUTS-TEMPS-ANNOTATION-APPLICATION-LISTE-CONTRAINTES}.
			Une description plus détaillée de l'application et de ses fonctionnalités est disponible en section~\ref{section:3.3-DESCRIPTION-IMPLEMENTATION}\todo{description à faire}
			%
			\begin{figure}[!htb]
				\centering
				\includegraphics[width=\textwidth]{figures/etude-temps-annotation-0application-annotation}
				\caption{Capture d'écran de l'application web permettant utilisant notre méthodologie de \textit{clustering} interactif pour annoter des contraintes (page d'annotation). Les deux textes à annoter sont disposés à gauche et droite de l'écran. Chacun dispose d'un cache à cocher si le texte n'est pas pertinent à analyser (\textit{ambigu, hors périmètre, incompréhensible, ...}).\\
				Les boutons à disposition permettent respectivement d'annoter un \texttt{MUST-LINK} si les données sont similaires (\textit{bouton en vert}), un \texttt{CANNOT-LINK} si les données ne sont pas similaire (\textit{bouton en rouge}), d'ignorer la contrainte pour laisser la main à l'algorithme de \textit{clustering} (\textit{bouton en bleu}), et d'ajouter un commentaire pour revoir la contrainte plus tard (\textit{case à choser et champ de texte libre}). Deux éléments déroulant permettent d'avoir des informations supplémentaires (\textit{metadata de sélection et de \textit{clustering}, représentation graphique des liens entre contraintes annotées}). Les boutons de navigation (\textit{boutons flèches et liste}) sont disponibles en bas de page.}
				\label{figure:4.3.1-ETUDE-COUTS-TEMPS-ANNOTATION-APPLICATION-ANNOTATION}
			\end{figure}
			\begin{figure}[!htb]
				\centering
				\includegraphics[width=\textwidth]{figures/etude-temps-annotation-0application-liste-contraintes}
				\caption{Capture d'écran de l'application web permettant utilisant notre méthodologie de \textit{clustering} interactif pour annoter des contraintes (page d'inventaire des contraintes à annoter).\\
				La partie supérieure permet d'identifier le nombre de textes et de contraintes sur le projet, ainsi que les boutons destinés à calculer les transitivités entre les contraintes et à approuver le travail réalisé si aucune transitivité n'entre en conflit avec un contrainte annotée. La partie inférieure liste l'ensemble des contraintes du projet, avec les annotations réalisées, l'itération à laquelle la contraintes a été sélectionnée et annotée, si elle est à revoir ou si une incohérence la concernant est détectée.}
				\label{figure:4.3.1-ETUDE-COUTS-TEMPS-ANNOTATION-APPLICATION-LISTE-CONTRAINTES}
			\end{figure}
			
			
			% Détails de l'expérience : modélisation.
			Une fois les sessions d'annotations terminées, nous entraînons un modèle linéaire généralisé (\textit{GLM}) pour estimer le temps d'annotation moyen pour un lot de contraintes (dont la taille est notée $\texttt{batch\_size}$).
			Ce modèle sera caractérisé par le coefficient de détermination généralisé \texttt{R²} de \textit{Cox et Snel}\todo{CITATION}, la log-vraisemblance \texttt{llf}\todo{CITATION} et la log-vraisemblance \texttt{llf\_null} du modèle \textit{null}.
			Nous discuterons aussi de l'évolution de la vitesse d'un opérateur au cours des différentes sessions d'annotation.

			% Référence scripts.
			\begin{leftBarInformation}
				Ces analyses sont réalisées en Python à l'aide des librairies \texttt{datetime} et \texttt{statsmodels} (\cite{seabold:2010}).
				Le projet à importer dans l'outil d'annotation ainsi que les scripts de l'expérience (\textit{notebooks} Python) sont disponibles dans un dossier dédié de~\cite{schild:cognitivefactory-interactive-clustering-comparative-study:2021}.
			\end{leftBarInformation}
			\todo{citation}

		%%% Résultats.
		\subsubsection{Résultats obtenus}
		
			% Taux de participation.
			Durant cette expérience, $14$ annotateurs ont participé à l'annotation de $1~000$ contraintes aléatoires sur un jeu de données.
			Par manque de disponibilités, $4$ annotateurs n'ont que partiellement réalisé leur tâche : nous avons toutefois intégré leurs participations car elles contenaient toutes au moins $150$ annotations.
			
			% Statistiques descriptives.
			D'après les observations, un annotateur réalisait en moyenne $170.7$ contraintes par session d'annotation (min: $43$, max: $547$, médiane: $138$, écart-type: $106.4$) ce qui lui demandait en moyenne $23.1$ minutes (min: $3.0$, max: $92.0$, écart-type: $14.4$).
			De plus, la vitesse d'annotation moyenne était de $7.7$ contraintes par minute (min: $3.5$, max: $14.3$, écart-type: $2.9$).
			
			% Modélisation du temps d'annotation.
			Le modèle linéaire généralisé entraîné sur les mesures de temps d'annotations (\texttt{R²}: $0.910$, \texttt{llf}: $-499.15$, \texttt{llf\_null}: $-539.95$) nous permet de déduire l'équation suivante :
			%
			\begin{equation}
				\texttt{annotation\_time}~[s]~
				\propto~7.8 \cdot \texttt{batch\_size}
			\end{equation}
		
			% Affichage du temps d'annotation.
			La figure~\ref{figure:4.3.1-ETUDE-COUTS-TEMPS-ANNOTATION-SIMULATION} représente cette modélisation du temps d'annotation en comparaison avec les mesures réalisées lors de l'expérience.
			\begin{figure}[!htb]
				\centering
				\includegraphics[width=\textwidth]{figures/etude-temps-annotation-1-modelisation-temps}
				\caption{Estimation du temps nécessaire (en minutes) pour annoter un lot de contraintes.}
				\label{figure:4.3.1-ETUDE-COUTS-TEMPS-ANNOTATION-SIMULATION}
			\end{figure}
		
			% Etude de cas.
			En ce qui concerne l'évolution de la vitesse d'annotation au cours des sessions, aucune tendance significative n'a été identifiée. 
			La figure~\ref{figure:4.3.1-ETUDE-COUTS-TEMPS-ANNOTATION-EXEMPLE} représente cette évolution de vitesse d'annotation pour quatre opérateurs (les deux plus rapides et les deux plus lents).
			Ces données sont l'objet d'une étude de cas dans la discussion ci-dessous.
			\begin{figure}[!htb]
				\centering
				\includegraphics[width=\textwidth]{figures/etude-temps-annotation-3-etude-de-cas}
				\caption{Etude de cas d'évolution de la vitesse d'annotation de contraintes (en contraintes par minutes) en fonction des différentes sessions d'annotations}
				\label{figure:4.3.1-ETUDE-COUTS-TEMPS-ANNOTATION-EXEMPLE}
			\end{figure}

		%%% Discussion.
		\subsubsection{Discussion}
		
		% Généralités sur la modélisation du temps d'annotation sur une session.
		L'étude réalisée avec $14$ annotateurs sur des lots de $1~000$ contraintes a permis d'estimer à $7.8 \cdot \texttt{batch\_size}$ le temps nécessaire (en secondes) pour annoter un lot de contraintes (cf. figure~\ref{figure:4.3.1-ETUDE-COUTS-TEMPS-ANNOTATION-SIMULATION})
		
		% Compliqué de comparer ...
		\begin{leftBarAuthorOpinion}
			Avant poursuivre la discussion, il est nécessaire de préciser qu'il est compliqué de comparer ces résultats.
			% Forte disparité des mesures.
			D'une part, il y a une forte disparité des mesures, et il est idyllique de penser qu'une étude sur $14$ annotateurs peut représenter la diversité du comportement humain sur une tâche aussi complexe que l'annotation de données textuelles.
			% Peu de repères concrets.
			D'autre part, il y a un manque de repères concrets dans la littérature scientifique, entre autre à cause des nombreux facteurs intervenant dans une tâche d'annotation (\textit{objectifs à réaliser, données à manipuler, nombre de choix proposés à l'opérateur, complexité sémantique des données, des compétences de l'opérateur, fréquence d'exécution de la tâche, ...}), mais aussi en raison du manque d'intérêt du temps nécessaire au profit de l'analyse de la cohérence et de la qualité intra-ou-inter-annotateur.
			% Diversité des données.
			De plus, les résultats peuvent différer en fonction des contraintes à caractériser : on peut supposer que des couples de données très similaires ou très différentes sont simples à annoter, mais que des données plus ambiguës peuvent nécessiter davantage de temps pour être intégrées et étiquetées.
			
			% Angle d'attaque.
			Pour pallier ce problème, nous proposons de comparer nos estimations de temps d'annotations grâce aux pistes ci-dessous. Certes, ces repères sont approximatifs, mais ils nous permettront de discuter des ordres de grandeurs à manipuler.
		\end{leftBarAuthorOpinion}
		
		% 8 secondes, ce n'est pas "court".
		\todo[inline]{A REDIGER: 8 secondes c'est pas courts}
			%%%% (OLD) tache courte : caractérisation rapide d'une similitude ou d'une différence entre deux données \todo{citation: Kahneman (2011), Hancock (1988)}.
			%%%% comparé à 5 secondes, il y a quelques chose en plus
			%%%% hypothèse 1 : il y a un traitement cognitif en plus
			%%%% hypothèse 2 : il y a un problème applicatif
			
		% ... mais 8 secondes, c'est mieux que 17 secondes.
		\todo[inline]{A REDIGER: 8 secondes c'est mieux que 17}
			% Article Cheap and Fast — But is it Good? (Word Sense Disambiguation)
			%%%% Selon~\todo{citation: Snow et al. (2008) / Yuret (2007)}, qui délègue à \texttt{Amazon Mechanical Turk} l'annotation de phrases pour catégoriser leur contexte parmi trois possibilités préformatés, il faut $8.59$ heures pour étiqueter $1~770$ données, soit $17.8$ secondes par annotation ;
			%%%% ajouter du contexte du AMT (pourquoi c'est utilisable, sont-ils payés, ...)
			
		
		% Autres analyses sans conclusions.
		Sur d'autres aspects, nous avons analysé l'évolution de la vitesse d'annotation au cours des sessions d'annotation, en espérant observer une accélération des annotations au fur et à mesure que l'annotateur s'habitue avec la tâche, ainsi qu'un effet de fatigue pour des sessions d'annotations trop longues.
		Cependant, aucune de nos analyses n'a montré de résultats significatifs (on peut constater la forte dispersion des résultats grâce à la figure~\ref{figure:4.3.1-ETUDE-COUTS-TEMPS-ANNOTATION-EXEMPLE}).
		Nous ne pouvons donc pas conclure sur de telles tendances.
		\begin{leftBarAuthorOpinion}
			Nos intuitions initiales concernaient deux points :
			\begin{itemize}
				\item la diminution du \textbf{temps d'adaptation} au cours de sessions d'annotations : au fur et à mesure qu'il annote, l'opérateur pourrait entrer plus facilement dans sa tâche, lui permettant d'atteindre plus rapidement sa vitesse de croisière et ainsi gagner en efficacité sur plusieurs sessions. D'après (CITATION: Anderson (2013)), ce temps d'adaptation pourrait se définir en trois étapes : une phase déclarative (\textit{besoin d'instructions détaillées, exécution lente et avec erreurs}), une phase associative (\textit{quelques rappels clés suffisent pour retrouver les instructions, donc gain de vitesse}) et une phase autonome (\textit{les consignes sont acquises, donc exécution rapide et sans erreur}) ;
				\item l'intervention d'un \textbf{effet de fatigue} : si une session d'annotation dure trop longtemps, l'opérateur pourrait perdre en efficacité par manque de concentration et augmenter ses chances de faire des erreurs. D'après (CITATION: Jones et al. (2015)), la fatigue est considérée comme un inconfort qui s'installe après une tâche excessive, et (CITATION: Elkosantini et Gien (2009)) décrit cet état de fatigue par des capacités de travail réduites.
			\end{itemize}
			Ces différentes intuitions ont aussi été remontées par les annotateurs de notre expériences, mais aucun effet significatif n'a pu être observé.
		\end{leftBarAuthorOpinion}
		\todo{citation temps d'adaptation: Anderson (2013)}
		\todo{citation effet de fatigue: Jones et al. (2015), Elkosantini et Gien (2009)}
		
		% remarque sur le nombre médian de contraintes à annoter
		\todo[inline]{A REDIGER: taille batch}
		Par extension, nous ne pouvons pas non plus conclure sur la taille optimale d’échantillon de contraintes à sélection pour une session d'annotation.
		Toutefois, ...
			%%%% nombre médian de contraintes annotées durant les sessions d'annotation de cette expérience est de $138$ (moyenne à $170.70$), nous pouvons par exemple fixer la taille par défaut des lots à $100$ contraintes ($14.9$ minutes) ou $150$ contraintes ($20.7$ minutes).
			%%%% Attention toutefois à ne pas faire des lots trop conséquents (au delà de $200$), d'une part pour garder l'aspect itératif et interactif de la méthode, et d'autre part pour ne pas atteindre un pallier de \textbf{fatigue de} l'annotateur.\todo{citation: Jones et al. (2015), Elkosantini et Gien (2009)}
		
		% Autres remontées applicatives.
		\todo[inline]{A REDIGER: ergo appli}
		Enfin, diverses remontées des opérateurs de l'expérience concerne l'ergonomie de l'application.
		En effet, il est logique de penser que la conception du logiciel peut grandement impacté 
	

	%%%
	%%% Subsection 4.3.2: Étude du temps de calcul nécessaire aux algorithmes implémentés en chronométrant des exécutions dans différentes situations
	%%%
	\subsection{Étude du temps de calcul nécessaire aux algorithmes implémentés en chronométrant des exécutions dans différentes situations}
	\label{section:4.3.2-ETUDE-COUTS-TEMPS-CALCUL}
	
		%%% Protocole expérimental.
		\subsubsection{Protocole expérimental}
		
			% Transition.
			Maintenant que nous avons pu modéliser le temps nécessaire à un expert pour annoter un lot de contraintes, nous nous intéressons au temps nécessaire à la machine pour interpréter ces annotations et proposer une nouvelle segmentation des données.
			
			% Objectif de l'expérience.
			Pour cela, nous allons chronométrer plusieurs exécutions des algorithmes intervenant dans notre implémentation du \textit{clustering} interactif, et nous évaluerons l'importance de leurs différents arguments d'entrée (la taille du jeu de données, le nombre de clusters et le nombre de contraintes annotées, ...).
			Nous profiterons aussi de ces modélisations du temps de calcul pour confirmer le choix de paramétrage réalisé lors de l'étude d'efficience en section~\ref{section:4.2-HYPOTHESE-EFFICIENCE}, et ainsi faire un compromis entre l'algorithme le plus efficient et l'algorithme le plus rapide.
			
			% Remarques.
			\begin{leftBarWarning}
				Pour utiliser des jeux de données de tailles différentes tout en maîtrisant leur contenu, nous avons dupliqués aléatoirement des données issues de nos jeux de référence en générant des fautes de frappes.
				Pour cette étude, nous faisons l'hypothèse que cela n'a pas d'impact majeur sur le temps d'exécution des différents algorithmes.
			\end{leftBarWarning}
			
			% Pseudo-code.
			Pour résumer le protocole expérimental que nous décrivons ci-dessous, vous pouvez vous référer aux pseudo-code décrit dans Alg.~\ref{algorithm:4.3.2-ETUDE-COUTS-TEMPS-CALCUL-PROTOCOLE}.
			%
			\begin{algorithm}[!htb]
				\begin{algorithmic}[1]
					\Require jeux de données annotés (vérité terrain) de tailles différentes
					\ForAll{arrangement d'algorithmes et de paramètres à tester}
						\State \textbf{initialisation}: récupérer ou générer le jeu de données
						\If{estimation de la tâche de \textbf{prétraitement}}
							\State \textbf{chronomètre: START}
							\State \textbf{prétraitement (à étudier)}: supprimer le bruit dans les données
							\State \textbf{chronomètre: STOP}
						\ElsIf {estimation de la tâche de \textbf{vectorisation}}
							\State \textbf{prétraitement}: supprimer le bruit dans les données avec \texttt{prep.simple}
							\State \textbf{chronomètre: START}
							\State \textbf{vectorisation (à étudier)}: transformer les données en vecteurs
							\State \textbf{chronomètre: STOP}
						\ElsIf {estimation de la tâche de \textbf{clustering}}
							\State \textbf{prétraitement}: supprimer le bruit dans les données avec \texttt{prep.simple}
							\State \textbf{vectorisation}: transformer les données en vecteurs avec \texttt{vect.tfidf}
							\State \textbf{échantillonnage initial}: sélectionner une base de contraintes avec \texttt{samp.rand.full}
							\State \textbf{simulation d'annotation}: ajouter des contraintes en utilisant la vérité terrain
							\State \textbf{chronomètre: START}
							\State \textbf{clustering (à étudier)}: regrouper les données par similarité
							\State \textbf{chronomètre: STOP}
						\ElsIf {estimation de la tâche d'\textbf{échantillonnage}}
							\State \textbf{prétraitement}: supprimer le bruit dans les données avec \texttt{prep.simple}
							\State \textbf{vectorisation}: transformer les données en vecteurs avec \texttt{vect.tfidf}
							\State \textbf{échantillonnage initial}: sélectionner une base de contraintes avec \texttt{samp.rand.full}
							\State \textbf{simulation d'annotation}: ajouter des contraintes en utilisant la vérité terrain
							\State \textbf{clustering initial}: regrouper les données par similarité avec \texttt{clust.kmeans.cop}
							\State \textbf{chronomètre: START}
							\State \textbf{échantillonnage (à étudier)}: sélectionner de nouvelles contraintes à annoter
							\State \textbf{chronomètre: STOP}
						\EndIf
						\State \textbf{mesure}: estimer la différence de chronomètre pour cet algorithme
					\EndFor
					\ForAll{algorithme à modéliser}
						\State \textbf{cadrage}: définir les facteurs et les interactions intervenant dans la modélisation
						\State \textbf{simplification}: restreindre la modélisation aux facteurs les plus corrélés
						\State \textbf{modélisation}: entraîner un modèle linéaire généralisé avec les facteurs retenus
						\State \textbf{simulation}: écrire l'équation du temps d'exécution avec des paramètres obtenus
					\EndFor
					\Ensure modélisation du temps d'exécution des différents algorithmes
				\end{algorithmic}
				\caption{Description en pseudo-code du protocole expérimental de l'étude du temps d'exécution des algorithmes du \textit{clustering} interactif}
				\label{algorithm:4.3.2-ETUDE-COUTS-TEMPS-CALCUL-PROTOCOLE}
			\end{algorithm}
			
			% Description de la vérité terrain.
			Nous utiliserons deux vérités terrains comme références pour cette expérience :
			\begin{itemize}
				\item le jeu de données \texttt{bank cards (v2.0.0)} : ce dernier traite des demandes les plus fréquentes des clients en ce qui concerne la gestion de leur carte bancaire. Il est composé de $1~000$ questions rédigées en français et réparties en $10$ classes (\texttt{perte ou vol de carte}, \texttt{carte avalée}, \texttt{commande de carte}, ...). Pour plus de détails, consultez l'annexe~\ref{annex:C.1-DATASET-BANK-CARDS} ;
				\item le jeu de données \texttt{MLSUM FR TRAIN SUBSET (v1.0.0-schild)} : ce dernier concerne les titres d'articles de journaux issus des catégories de publication les plus communes. Il est composé de $744$  titres d'articles rédigés et répartis en $14$ classes (\textit{économie}, \textit{sport}, ...). Pour plus de détails, consultez l'annexe~\ref{annex:C.2-DATASET-MLSUM-SUBSET-SCHILD} ;
			\end{itemize}
			
			% Description des tâches, des algorithmes et des contextes.
			Pour cette étude, nous lançons plusieurs exécutions de chaque algorithme de notre implémentation du \textit{clustering} interactif (cf. section~\ref{section:3.3-DESCRIPTION-IMPLEMENTATION}) avec différentes variations de contexte d'utilisation. Cela comprend les tâches, algorithmes et contextes d'utilisation suivants :
			%
			\begin{enumerate}
				% Prétraitement.
				\item le \textbf{prétraitement} des données...
					\begin{itemize}
						\item avec les algorithmes suivants : \textbf{simple} (noté \texttt{prep.simple}), \textbf{avec lemmatisation} (noté \texttt{prep.lemma}) et \textbf{avec filtres} (noté \texttt{prep.filter}) ;
						\item avec les contextes d'utilisation suivants : \textbf{nombre de données} (variant de $1~000$ à $5~000$ par pas de $1~000$, noté $\texttt{dataset\_size}$) ;
					\end{itemize}
				% Vectorisation.
				\item la \textbf{vectorisation} des données...
					\begin{itemize}
						\item avec les algorithmes suivants : \textbf{TF-IDF} (noté \texttt{vect.tfidf}) et \textbf{SpaCy} (noté \texttt{vect.frcorenewsmd}) ;
						\item avec les contextes d'utilisation suivants : \textbf{nombre de données} (variant de $1~000$ à $5~000$ par pas de $1~000$, noté $\texttt{dataset\_size}$) ;
						\item précédé par un prétraitement \textbf{simple} ;
					\end{itemize}
				% Clustering.
				\item le \textbf{\textit{clustering} sous contraintes} des données...
					\begin{itemize}
						\item avec les algorithmes suivants : \textbf{KMeans} (modèle \textit{COP} noté \texttt{clust.kmeans.cop}), \textbf{Hiérarchique} (lien \textit{single} noté \texttt{clust.hier.sing} ; lien \textit{complete} noté \texttt{clust.hier.comp} ; lien \textit{average} noté \texttt{clust.hier.avg} ; lien \textit{ward} noté \texttt{clust.hier.ward}) et \textbf{Spectral} (modèle \textit{SPEC} noté \texttt{clust.spec}) ;
						\item avec les contextes d'utilisation suivants : \textbf{nombre de données} (variant de $1~000$ à $5~000$ par pas de $1~000$, noté $\texttt{dataset\_size}$), le \textbf{nombre de contraintes annotés} (variant de $0$ à $5~000$ par pas de $500$, noté $\texttt{previous\_nb\_constraints}$) et le \textbf{nombre de \textit{clusters} à trouver} (variant de $5$ à $50$ par pas de $5$, noté $\texttt{algorithm\_nb\_clusters}$) ;
						\item précédé par un prétraitement \textbf{simple} et une vectorisation \textbf{TF-IDF} et un échantillonnage initial \textbf{purement aléatoire} ;
					\end{itemize}
				% Sampling.
				\item l'\textbf{échantillonnage} des contraintes à annoter...
					\begin{itemize}
						\item avec les algorithmes suivants : \textbf{purement aléatoire} (noté \texttt{samp.random.full}), \textbf{pseudo-aléatoire} (noté \texttt{samp.random.same}), \textbf{même cluster et étant les plus éloignées} (noté \texttt{samp.farhtest.same}) et \textbf{clusters différents et étant les plus proches} (noté \texttt{samp.closest.diff}) ;
						\item avec les contextes d'utilisation suivants : \textbf{nombre de données} (variant de $1~000$ à $5~000$ par pas de $1~000$, noté $\texttt{dataset\_size}$), le \textbf{nombre de contraintes annotés} (variant de $0$ à $5~000$ par pas de $500$, noté $\texttt{previous\_nb\_constraints}$), le \textbf{nombre de \textit{clusters} existant} (variant de $10$ à $50$ par pas de $10$, noté $\texttt{previous\_nb\_clusters}$) et le \textbf{nombre de contraintes à sélectionner} (variant de $50$ à $250$ par pas de $50$, noté $\texttt{algorithm\_nb\_constraints}$) ;
						\item précédé par un prétraitement \textbf{simple}, une vectorisation \textbf{TF-IDF}, un \textit{clustering} initial \textbf{KMeans} (modèle \textit{COP}) et un échantillonnage initial \textbf{purement aléatoire} ;
					\end{itemize}
			\end{enumerate}
			
			Il y a donc $8~825$ combinaisons d'algorithmes (\texttt{15} pour le prétraitement, $10$ pour la vectorisation, $3~330$ pour le \textit{clustering}, $5~550$ pour l'échantillonnage), et chaque combinaison est répétée $5$ fois pour contrer les aléas statistiques des exécutions.
			De plus, chaque jeu de données est généré $5$ fois pour contrer les aléas statistiques de création, donc il y a $220~625$ exécutions d'algorithmes ($375$ pour le prétraitement, $250$ pour la vectorisation, $82~500$ pour le \textit{clustering}, $137~500$ pour l'échantillonnage).
			
			% Description de la modélisation.
			Sur la base de ces mesures, nous cherchons à modéliser le temps d'exécution de chaque algorithme en fonction de son contexte d'utilisation (dépendant de ses arguments d'entrée), et les interactions doubles entre paramètres sont envisagées.
			Afin de réduire la complexité des modélisations, nous ordonnons les interactions de facteurs possibles en fonction de leur corrélation avec le temps mesuré (la corrélation \texttt{r} de \textit{Pearson}\todo{CITATION} est utilisée) et nous nous limitons aux variables responsables d'un maximum de la variance des mesures (la méthode d'\textit{Elbow}\todo{CITATION} est utilisée pour choisir les facteurs pertinents).
			Sur cette base, nous entraînons un modèle linéaire généralisé (\textit{GLM}) pour représenter le temps d'exécution moyen de l'algorithme : ce modèle sera caractérisé par le coefficient de détermination généralisé \texttt{R²} de \textit{Cox et Snel}\todo{CITATION}, la log-vraisemblance \texttt{llf}\todo{CITATION} et la log-vraisemblance \texttt{llf\_null} du modèle \textit{null}.
			Pour finir, nous discuterons des valeurs des coefficients obtenus sur l'impact du temps d'exécution.
			
			% Référence scripts.
			\begin{leftBarInformation}
				Ces analyses sont réalisées en Python à l'aide des librairies \texttt{datetime} et \texttt{statsmodels} (\cite{seabold:2010}).
				Les scripts de l'expérience (\textit{notebooks} Python) sont disponibles dans un dossier dédié de~\cite{schild:cognitivefactory-interactive-clustering-comparative-study:2021}.
			\end{leftBarInformation}

		%%% Résultats.
		\subsubsection{Résultats obtenus}
				
			%%% Prétraitements
			
			% Première analyse.
			En ce qui concerne la tâche de \textbf{prétraitement}, une première analyse montre que les modélisations des trois implémentations sont similaires (\texttt{p-valeur}: $> 0.980$). Nous ferons donc une seule modélisation.
			
			% Modélisation du temps de calcul (prep.simple + prep.lemma + prep.filter).
			Pour les algorithmes de prétraitements (\texttt{prep.simple}, \texttt{prep.lemma} et \texttt{prep.filter}), l'analyse de la corrélation des facteurs avec les mesures de temps d'exécution indique qu'une modélisation minimale et suffisante peut être réalisée à partir du facteur $\texttt{dataset\_size}$ (\texttt{r}: $0.997$).
			Le modèle linéaire généralisé retenu (\texttt{R²}: $> 0.999$, \texttt{llf}: $-432.43$, \texttt{llf\_null}: $-1~353.98$) nous permet de déduire l'équation suivante\todo{ref complexité théorique algo en annexe} :
			%
			\begin{equation}
				\texttt{computation\_time}(\texttt{prep})~[s]~
				\propto~6.55 \cdot 10^{-3} \cdot \texttt{dataset\_size}
			\end{equation}
			
			% Affichage du temps de calcul.
			La figure~\ref{figure:4.3.2-ETUDE-COUTS-TEMPS-CALCUL-MODELISATION-PREPROCESSING} représente cette modélisation du temps de calcul des algorithmes de prétraitements en comparaison avec les mesures réalisées lors de l'expérience.
			\newline
			%		
			\begin{figure}[!htb]
				\centering
				\includegraphics[width=0.8\textwidth]{figures/etude-temps-calcul-modelisation-1prep}
				\caption{Estimation du temps nécessaire (en minutes) pour effectuer une tâche de \textbf{prétraitement} en fonction du nombre de données à traiter. Les paramétrages \texttt{prep.simple}, \texttt{prep.lemma} et \texttt{prep.filter} ayant des temps de calculs similaires, leurs modélisations n'ont pas été séparées.}
				\label{figure:4.3.2-ETUDE-COUTS-TEMPS-CALCUL-MODELISATION-PREPROCESSING}
			\end{figure}
			
			%%% Vectorization
			
			% Première analyse.
			En ce qui concerne la tâche de \textbf{vectorisation}, une première analyse montre que les modélisations des deux implémentations sont différentiables  (\texttt{p-valeur}: $< 10^{-3}$). Nous ferons donc une modélisation par algorithme.
		
			% Modélisation du temps de calcul (vect.tfidf).
			Pour les algorithmes de vectorisation \texttt{vect.tfidf}, l'analyse de la corrélation des facteurs avec les mesures de temps d'exécution indique qu'une modélisation minimale et suffisante peut être réalisée à partir du facteur $\texttt{dataset\_size}$ (\texttt{r}: $0.977$).
			Le modèle linéaire généralisé retenu (\texttt{R²}: $> 0.999$, \texttt{llf}: $259.89$, \texttt{llf\_null}: $70.04$) nous permet de déduire l'équation suivante\todo{ref complexité théorique algo en annexe} :
			%
			\begin{equation}
				\texttt{computation\_time}(\texttt{vect.tfidf})~[s]~
				\propto~-9.16 \cdot 10^{-5} \cdot \texttt{dataset\_size}
			\end{equation}
			
			% Modélisation du temps de calcul (vect.frcorenewsmd).
			Pour les algorithmes de vectorisation \texttt{vect.frcorenewsmd}, l'analyse de la corrélation des facteurs avec les mesures de temps d'exécution indique qu'une modélisation minimale et suffisante peut être réalisée à partir du facteur $\texttt{dataset\_size}$ (\texttt{r}: $0.983$).
			Le modèle linéaire généralisé retenu (\texttt{R²}: $> 0.999$, \texttt{llf}: $-214.44$, \texttt{llf\_null}: $-399.39$) nous permet de déduire l'équation suivante\todo{ref complexité théorique algo en annexe} :
			%
			\begin{equation}
				\texttt{computation\_time}(\texttt{vect.frcorenewsmd})~[s]~
				\propto~4.62 \cdot 10^{-3} \cdot \texttt{dataset\_size}
			\end{equation}
			
			% Affichage du temps de calcul.
			La figure~\ref{figure:4.3.2-ETUDE-COUTS-TEMPS-CALCUL-MODELISATION-VECTORIZATION} représente ces modélisations de temps de calcul des algorithmes de vectorisation en comparaison avec les mesures réalisées lors de l'expérience.
			\newline
			%
			\begin{figure}[!htb]
				\centering
				\includegraphics[width=0.8\textwidth]{figures/etude-temps-calcul-modelisation-2vect}
				\caption{Estimation du temps nécessaire (en minutes) pour effectuer une tâche de \textbf{vectorisation} en fonction du nombre de données à traiter.}
				\label{figure:4.3.2-ETUDE-COUTS-TEMPS-CALCUL-MODELISATION-VECTORIZATION}
			\end{figure}
			
			%%% Clustering
			
			% Première analyse.
			En ce qui concerne la tâche de \textbf{\textit{clustering} sous contraintes}, une première analyse montre que les modélisations des six implémentations sont différentiables  (\texttt{p-valeur}: $<$ \texttt{$10^{-3}$}). Nous ferons donc une modélisation par algorithme.
			
			% Remarques: hiérarchique trop long.
			\begin{leftBarWarning}
				Plusieurs exécutions des algorithmes de type \textit{hiérarchique} ont été annulées pour les jeux données de tailles supérieures à $4~000$ car la durée excédé plusieurs heures.
				Nous limitons dons l'analyse de \texttt{clust.hier.sing}, \texttt{clust.hier.comp}, \texttt{clust.hier.avg} et \texttt{clust.hier.ward} aux tailles de $1~000$ à $3~000$.
			\end{leftBarWarning}
			
			% Modélisation du temps de calcul (clust.kmeans.cop).
			Pour les algorithmes du \textit{clustering} sous contraintes \texttt{clust.kmeans.cop}, l'analyse de la corrélation des facteurs avec les mesures de temps d'exécution indique qu'une modélisation minimale et suffisante peut être réalisée à partir du facteur $\texttt{dataset\_size}$ (\texttt{r}: $0.837$).
			Le second facteur le plus corrélé (mais non retenu) est l'interaction $\texttt{dataset\_size}^{2} \cdot algorithm\_nb\_clusters$ (\texttt{r}: $0.545$).
			Le modèle linéaire généralisé retenu (\texttt{R²}: $0.802$, \texttt{llf}: $-9.37 \cdot 10^{4}$, \texttt{llf\_null}: $-1.00 \cdot 10^{5}$) nous permet de déduire l'équation suivante\todo{ref complexité théorique algo en annexe} :
			%
			\begin{equation}
				\texttt{computation\_time}(\texttt{clust.kmeans.cop})~[s]~
				\propto~1.45 \cdot 10^{-1} \cdot \texttt{dataset\_size}
			\end{equation}
			
			% Modélisation du temps de calcul (clust.hier.sing).
			Pour les algorithmes du \textit{clustering} sous contraintes \texttt{clust.hier.sing}, l'analyse de la corrélation des facteurs avec les mesures de temps d'exécution indique qu'une modélisation minimale et suffisante peut être réalisée à partir du facteur $\texttt{dataset\_size}^{2}$ (\texttt{r}: $0.940$).
			Le second facteur le plus corrélé (mais non retenu) est l'interaction $\texttt{dataset\_size}^{2} \cdot algorithm\_nb\_clusters$ (\texttt{r}: $0.729$).
			Le modèle linéaire généralisé retenu (\texttt{R²}: $0.987$, \texttt{llf}: $-5.54 \cdot 10^{4}$, \texttt{llf\_null}: $-6.10 \cdot 10^{4}$) nous permet de déduire l'équation suivante\todo{ref complexité théorique algo en annexe} :
			%
			\begin{equation}
				\texttt{computation\_time}(\texttt{clust.hier.sing})~[s]~
				\propto~-5.00 \cdot 10^{-4} \cdot \texttt{dataset\_size}^{2}
			\end{equation}
			
			% Modélisation du temps de calcul (clust.hier.comp).
			Pour les algorithmes du \textit{clustering} sous contraintes \texttt{clust.hier.comp}, l'analyse de la corrélation des facteurs avec les mesures de temps d'exécution indique qu'une modélisation minimale et suffisante peut être réalisée à partir du facteur $\texttt{dataset\_size}^{2}$ (\texttt{r}: $0.938$).
			Le second facteur le plus corrélé (mais non retenu) est l'interaction $\texttt{dataset\_size}^{2} \cdot algorithm\_nb\_clusters$ (\texttt{r}: $0.736$).
			Le modèle linéaire généralisé retenu (\texttt{R²}: $0.984$, \texttt{llf}: $-5.56 \cdot 10^{4}$, \texttt{llf\_null}: $-6.11 \cdot 10^{4}$) nous permet de déduire l'équation suivante\todo{ref complexité théorique algo en annexe} :
			%
			\begin{equation}
				\texttt{computation\_time}(\texttt{clust.hier.comp})~[s]~
				\propto~-4.99 \cdot 10^{-4} \cdot \texttt{dataset\_size}^{2}
			\end{equation}

			% Modélisation du temps de calcul (clust.hier.avg).
			Pour les algorithmes du \textit{clustering} sous contraintes \texttt{clust.hier.avg}, l'analyse de la corrélation des facteurs avec les mesures de temps d'exécution indique qu'une modélisation minimale et suffisante peut être réalisée à partir du facteur $\texttt{dataset\_size}^{2}$ (\texttt{r}: $0.915$).
			Le second facteur le plus corrélé (mais non retenu) est l'interaction $\texttt{dataset\_size}^{2} \cdot algorithm\_nb\_clusters$ (\texttt{r}: $0.713$).
			Le modèle linéaire généralisé retenu (\texttt{R²}: $0.981$, \texttt{llf}: $-5.90 \cdot 10^{4}$, \texttt{llf\_null}: $-6.45 \cdot 10^{4}$) nous permet de déduire l'équation suivante\todo{ref complexité théorique algo en annexe} :
			%
			\begin{equation}
				\texttt{computation\_time}(\texttt{clust.hier.avg})~[s]~
				\propto~-8.51 \cdot 10^{-4} \cdot \texttt{dataset\_size}^{2}
			\end{equation}

			% Modélisation du temps de calcul (clust.hier.ward).
			Pour les algorithmes du \textit{clustering} sous contraintes \texttt{clust.hier.ward}, l'analyse de la corrélation des facteurs avec les mesures de temps d'exécution indique qu'une modélisation minimale et suffisante peut être réalisée à partir du facteur $\texttt{dataset\_size}^{2}$ (\texttt{r}: $0.945$).
			Le second facteur le plus corrélé (mais non retenu) est l'interaction $\texttt{dataset\_size}^{2} \cdot algorithm\_nb\_clusters$ (\texttt{r}: $0.734$).
			Le modèle linéaire généralisé retenu (\texttt{R²}: $0.989$, \texttt{llf}: $-5.57 \cdot 10^{4}$, \texttt{llf\_null}: $-6.14 \cdot 10^{4}$) nous permet de déduire l'équation suivante\todo{ref complexité théorique algo en annexe} :
			%
			\begin{equation}
				\texttt{computation\_time}(\texttt{clust.hier.ward})~[s]~
				\propto~-5.30 \cdot 10^{-4} \cdot \texttt{dataset\_size}^{2}
			\end{equation}
			
			% Modélisation du temps de calcul (clust.spec).
			Pour les algorithmes du \textit{clustering} sous contraintes \texttt{clust.spec}, l'analyse de la corrélation des facteurs avec les mesures de temps d'exécution indique qu'une modélisation minimale et suffisante peut être réalisée à partir du facteur $\texttt{dataset\_size}^{2}$ (\texttt{r}: $0.658$).
			Le second facteur le plus corrélé (mais non retenu) est l'interaction $\texttt{dataset\_size}^{2} \cdot algorithm\_nb\_clusters$ (\texttt{r}: $0.595$).
			Le modèle linéaire généralisé retenu (\texttt{R²}: $0.527$, \texttt{llf}: $-7.89 \cdot 10^{5}$, \texttt{llf\_null}: $-8.27 \cdot 10^{5}$) nous permet de déduire l'équation suivante\todo{ref complexité théorique algo en annexe} :
			%
			\begin{equation}
				\texttt{computation\_time}(\texttt{clust.spec})~[s]~
				\propto~8.18 \cdot 10^{-6} \cdot \texttt{dataset\_size}^{2}
			\end{equation}
			
			% Affichage du temps de calcul.
			La figure~\ref{figure:4.3.2-ETUDE-COUTS-TEMPS-CALCUL-MODELISATION-CLUSTERING} représente ces modélisations de temps de calcul des algorithmes de \textit{clustering} en comparaison avec les mesures réalisées lors de l'expérience.
			\newline
			%
			\begin{figure}[!htb]
				\centering
				\includegraphics[width=0.8\textwidth]{figures/etude-temps-calcul-modelisation-3clust}
				\caption{Estimation du temps nécessaire (en minutes) pour effectuer une tâche de \textbf{clustering} en fonction du nombre de données à traiter.}
				\label{figure:4.3.2-ETUDE-COUTS-TEMPS-CALCUL-MODELISATION-CLUSTERING}
			\end{figure}
			
			%%% Sampling
			
			% Première analyse.
			En ce qui concerne la tâche d'\textbf{échantillonnage de contraintes}, une première analyse montre que les modélisations des quatre implémentations sont différentiables  (\texttt{p-valeur}: $<$ \texttt{$10^{-3}$}). Nous ferons donc une modélisation par algorithme.
			
			% Modélisation du temps de calcul (samp.rand.full).
			Pour les algorithmes de l'échantillonnage de contraintes \texttt{samp.rand.full}, l'analyse de la corrélation des facteurs avec les mesures de temps d'exécution indique qu'une modélisation minimale et suffisante peut être réalisée à partir du facteur $\texttt{dataset\_size}^{2}$ (\texttt{r}: $0.993$).
			Le second facteur le plus corrélé (mais non retenu) est l'interaction $\texttt{dataset\_size}^{2} \cdot previous\_nb\_clusters$ (\texttt{r}: $0.791$).
			Le modèle linéaire généralisé retenu (\texttt{R²}: $> 0.999$, \texttt{llf}: $-4.52 \cdot 10^{4}$, \texttt{llf\_null}: $-1.17 \cdot 10^{5}$) nous permet de déduire l'équation suivante\todo{ref complexité théorique algo en annexe} :
			%
			\begin{equation}
				\texttt{computation\_time}(\texttt{samp.rand.full})~[s]~
				\propto~-8.20 \cdot 10^{-7} \cdot \texttt{dataset\_size}^{2}
			\end{equation}
			
			% Modélisation du temps de calcul (samp.rand.same).
			Pour les algorithmes de l'échantillonnage de contraintes \texttt{samp.rand.same}, l'analyse de la corrélation des facteurs avec les mesures de temps d'exécution indique qu'une modélisation minimale et suffisante peut être réalisée à partir du facteur $\texttt{dataset\_size}^{2}$ (\texttt{r}: $0.939$).
			Le second facteur le plus corrélé (mais non retenu) est l'interaction $\texttt{dataset\_size}^{2} \cdot algorithm\_nb\_constraints$ (\texttt{r}: $0.611$).
			Le modèle linéaire généralisé retenu (\texttt{R²}: $> 0.999$, \texttt{llf}: $-3.20 \cdot 10^{4}$, \texttt{llf\_null}: $-6.84 \cdot 10^{4}$) nous permet de déduire l'équation suivante\todo{ref complexité théorique algo en annexe} :
			%
			\begin{equation}
				\texttt{computation\_time}(\texttt{samp.rand.same})~[s]~
				\propto~1.85 \cdot 10^{-7} \cdot \texttt{dataset\_size}^{2}
			\end{equation}
			
			% Modélisation du temps de calcul (samp.farhtest.same).
			Pour les algorithmes de l'échantillonnage de contraintes \texttt{samp.farhtest.same}, l'analyse de la corrélation des facteurs avec les mesures de temps d'exécution indique qu'une modélisation minimale et suffisante peut être réalisée à partir du facteur $\texttt{dataset\_size}^{2}$ (\texttt{r}: $0.981$).
			Le second facteur le plus corrélé (mais non retenu) est l'interaction $\texttt{dataset\_size}^{2} \cdot previous\_nb\_clusters$ (\texttt{r}: $0.700$).
			Le modèle linéaire généralisé retenu (\texttt{R²}: $> 0.999$, \texttt{llf}: $-4.56 \cdot 10^{4}$, \texttt{llf\_null}: $-1.02 \cdot 10^{5}$) nous permet de déduire l'équation suivante\todo{ref complexité théorique algo en annexe} :
			%
			\begin{equation}
				\texttt{computation\_time}(\texttt{samp.farhtest.same})~[s]~
				\propto~5.19 \cdot 10^{-7} \cdot \texttt{dataset\_size}^{2}
			\end{equation}
			
			% Modélisation du temps de calcul (samp.closest.diff).
			Pour les algorithmes de l'échantillonnage de contraintes \texttt{samp.closest.diff}, l'analyse de la corrélation des facteurs avec les mesures de temps d'exécution indique qu'une modélisation minimale et suffisante peut être réalisée à partir du facteur $\texttt{dataset\_size}^{2}$ (\texttt{r}: $0.995$).
			Le second facteur le plus corrélé (mais non retenu) est l'interaction $\texttt{dataset\_size}^{2} \cdot previous\_nb\_clusters$ (\texttt{r}: $0.815$).
			Le modèle linéaire généralisé retenu (\texttt{R²}: $> 0.999$, \texttt{llf}: $-5.96 \cdot 10^{4}$, \texttt{llf\_null}: $-1.36 \cdot 10^{5}$) nous permet de déduire l'équation suivante\todo{ref complexité théorique algo en annexe} :
			%
			\begin{equation}
				\texttt{computation\_time}(\texttt{samp.closest.diff})~[s]~
				\propto~1.43 \cdot 10^{-6} \cdot \texttt{dataset\_size}^{2}
			\end{equation}
			
			% Affichage du temps de calcul.
			La figure~\ref{figure:4.3.2-ETUDE-COUTS-TEMPS-CALCUL-MODELISATION-SAMPLING} représente ces modélisations de temps de calcul des algorithmes d'échantillonnage en comparaison avec les mesures réalisées lors de l'expérience.
			\newline
			%
			\begin{figure}[!htb]
				\centering
				\includegraphics[width=0.8\textwidth]{figures/etude-temps-calcul-modelisation-4samp}
				\caption{Estimation du temps nécessaire (en minutes) pour effectuer une tâche d'\textbf{échantillonnage de contraintes} en fonction du nombre de données à traiter.}
				\label{figure:4.3.2-ETUDE-COUTS-TEMPS-CALCUL-MODELISATION-SAMPLING}
			\end{figure}

		%%% Discussion.
		\subsubsection{Discussion}
		
			% Rappel de l'objectif : estimer le temps d'exécution.
			Dans cette étude, nous avons estimé le temps de calcul des différents algorithmes implémentés afin de confirmer le choix de paramétrage pour une convergence optimal (cf. hypothèse d'efficience en section~\ref{section:4.2-HYPOTHESE-EFFICIENCE}).
			Ces estimations ont été réalisées sur la base de plusieurs exécutions et fonction de divers contextes d'utilisation : nombre de données, nombre de contraintes annotées, nombre de contraintes à sélectionner, nombre de \textit{clusters} existant, nombre de \textit{clusters} à trouver.
			
			% Remarque générale : Dépend principalement du nombre de données.
			En premier lieu, on peut constater que les différentes modélisations dépendent majoritairement de la taille du jeu de données manipulé ($\texttt{dataset\_size}$ ou $\texttt{dataset\_size}^{2}$) avec un score de corrélation \texttt{r} avec le temps mesuré généralement supérieur à $0.9$ et des modèles \textit{GLM} avec des coefficients de détermination généralisé \texttt{R²} généralement proches de $0.999$.
			Bien que d'autres facteurs peuvent intervenir dans ces estimations (notamment les interactions doubles entre la taille du jeu de données et le nombre de \textit{clusters} ou le nombre de contraintes), ces derniers semblent avoir un impact négligeable sur le temps d'exécution.
			
			% Note: remarque sur le nombre de contraintes.
			\begin{leftBarAuthorOpinion}
				Certains paramétrages de la méthode du \textit{clustering} interactif semblent cependant avoir un temps de calcul décroissant au cours des itérations, mais nous n'avons cependant pas pu montrer de tendances globales significatives.
				Il est probable que l'ajout de contraintes judicieusement placées permettent à certains algorithmes de \textit{clustering} de s'exécuter plus rapidement, notamment lorsque ceux-ci exploitent les composants connexes du graphe de contraintes (cf. section~\ref{section:3.3.2-GESTION-DES-CONTRAINTES}). En effet, :
				\begin{itemize}
					\item les \textit{clustering} hiérarchiques s'initialisent autant de \textit{clusters} que de groupes de données liées entre elles par des contraintes \texttt{MUST-LINK} : or s'il y a plus de contraintes, alors les composants connexes sont davantage développés, donc il y a moins de \textit{clusters} à initialiser et donc moins d'époques de l'algorithme ;
					\item le \textit{clustering} KMeans (modèle COP) attire auprès d'un barycentre l'ensemble des données liées par un \texttt{MUST-LINK} : or s'il y a plus de contraintes, alors il y a des données attirées, donc les noyaux de \textit{clusters} peuvent se stabiliser plus rapidement.  
				\end{itemize}
				Toutefois, ces suppositions n'ont pas pu être démontrées, et certains contre-exemples tendent à conclure que ces comportements sont très dépendants du jeu de données manipulé et de l'ordre d'ajout des contraintes. Par exemple :
				\begin{itemize}
					\item l'ajout d'un trop grand nombre de contraintes \texttt{CANNOT-LINK} peut engendrer un surplus de vérification pour estimer quelles formations de \textit{clusters} sont autorisées sans violer de contraintes ;
					\item l'algorithme KMeans (modèle COP) peut osciller autour de plusieurs noyaux de \textit{clusters} instables si les contraintes violent trop la similarité intrinsèque des données.
				\end{itemize}
			\end{leftBarAuthorOpinion}
			
			% Cas du clustering.
			En ce qui concerne la tâche de \textit{clustering}, on note des différences significatives dans les temps d'exécution des divers algorithmes implémentés.
			En effet, l'algorithme KMeans (modèle COP) est nettement plus rapide (complexité en $ \mathcal{O}(\texttt{dataset\_size}) $, nécessitant quelques dizaines de minutes pour $5~000$ données) que les implémentations du \textit{clustering} hiérarchique (complexité en $ \mathcal{O}(\texttt{dataset\_size}^{2}) $, nécessitant plusieurs heures dès $3~000$ données).
			Cette différence, visible en figure~\ref{figure:4.3.2-ETUDE-COUTS-TEMPS-CALCUL-MODELISATION-CLUSTERING}, a un réel impact sur l'expérience utilisateur de l'opérateur.
			En effet, bien qu'il soit théoriquement plus efficient pour atteindre une annotation suffisante (cf. hypothèse d'efficience en section~\ref{section:4.2-HYPOTHESE-EFFICIENCE}), l'usage d'un \textit{clustering} hiérarchique imposerait de longs temps d'attente à l'opérateur, interdisant des interactions rapides avec la machines.
			Or l'intérêt principal de notre méthodologie d'annotation à l'aide du \textit{clustering} interactif repose sur ces interactions homme-machine via l'ajout régulier de contraintes pertinentes (cf. hypothèse d'efficacité en section~\ref{section:4.1-HYPOTHESE-EFFICACITE}).
			Nous décidons donc d'exclure l'usage des algorithmes de \textit{clustering} hiérarchique au profit du \textit{clustering} KMeans (modèle COP).
			
			% Note: Cas du projet étudiant avec TPS.
			\begin{leftBarInformation}
				Dans le cadre du projet étudiant avec l'école Télécom Physique Strasbourg visant à implémenter d'autres algorithmes de \textit{clustering} sous contraintes, un résonnement similaire a été utilisé pour filtrer les algorithmes. Ainsi, l'implémentation de KMeans (modèle MPC) a été exclu (complexité en $ \mathcal{O}(\texttt{dataset\_size}^{3}) $) et l'implémentation de la propagation par affinité écarte la gestion des contraintes \texttt{CANNOT-LINK} pour avoir un temps d'exécution comparable au \textit{clustering} KMeans (modèle COP). L'algorithme DBScan (modèle C-DBScan) est quand à lui un rival possible avec une complexité théorique en $ \mathcal{O}(\texttt{dataset\_size}) $.
			\end{leftBarInformation}
			
			% Cas du prétraitement + vectorisation + échantillonnage.
			En ce qui concerne les tâches de prétraitements (figure~\ref{figure:4.3.2-ETUDE-COUTS-TEMPS-CALCUL-MODELISATION-PREPROCESSING}), de vectorisation (figure~\ref{figure:4.3.2-ETUDE-COUTS-TEMPS-CALCUL-MODELISATION-VECTORIZATION}), et d'échantillonnage de contraintes (cf. figure~\ref{figure:4.3.2-ETUDE-COUTS-TEMPS-CALCUL-MODELISATION-SAMPLING}) ont des complexités presque négligeables au regard des temps d'exécution du \textit{clustering} (pour $5~000$ données : moins de $2$ minute contre près de $12.1$ minutes pour \texttt{clust.kmeans.cop} et près de $3.5$ heures pour \texttt{clust.hier.sing}).
			Nous maintenons donc les paramétrages obtenus pour ces tâches en section~\ref{section:4.2-HYPOTHESE-EFFICIENCE} sans analyses complémentaires.
			
			% Conclusion.
			Pour conclure, dans l'optique d'atteindre de manière efficiente $90$\% de \texttt{v-measure}
			\footnote{$90$\% de \texttt{v-measure}: cas d'une annotation dite partielle, dont le paramétrage le plus efficient est constitué du prétraitement simple (\texttt{prep.simple}), de la vectorisation TF-IDF (\texttt{vect.tfidf}), du \textit{clustering} hiérarchique à lien moyen (\texttt{clust.hier.avg}) et de l'échantillonnage des données les plus proches dans des clusters différents (\texttt{sampl.closest.diff}}
			avec un coût global minimal, nous retenons l'usage du \textbf{paramétrage favori} constitué du prétraitement simple (\texttt{prep.simple}), de la vectorisation TF-IDF (\texttt{vect.tfidf}), du \textit{clustering} KMeans avec modèle COP (\texttt{clust.kmeans.cop}) et de l'échantillonnage des données les plus proches dans des clusters différents (\texttt{sampl.closest.diff}).
			On estime le temps d'exécution de ce paramétrage avec l'équation suivante\footnote{Temps du paramétrage favori : environ $2.6$ minutes pour $1~000$ données ; environ $13.3$ minutes pour $5~000$ données.} :
			%
			\begin{equation}
				\label{equation:4.3.2-ETUDE-COUTS-TEMPS-CALCUL-PARAMETRAGE-FAVORI}
				\texttt{computation\_time}(\texttt{settings.favorite})~[s]~
				\propto~1.52 \cdot 10^{-1} \cdot \texttt{dataset\_size} + 1.43 \cdot 10^{-6} \cdot \texttt{dataset\_size}^{2}
			\end{equation}
	
	%%%
	%%% Subsection 4.3.3: Étude du nombre de contraintes nécessaires à la convergence vers une vérité terrain pré-établie en fonction de la taille du jeu de données 
	%%%
	\subsection{Étude du nombre de contraintes nécessaires à la convergence vers une vérité terrain pré-établie en fonction de la taille du jeu de données}
	\label{section:4.3.3-ETUDE-COUT-NOMBRE-CONTRAINTES}
	
		%%% Protocole expérimental.
		\subsubsection{Protocole expérimental}
			
			% Transition.
			Avec les deux précédentes études, nous sommes capable d'estimer le temps nécessaire à un expert pour annoter des contraintes et le temps nécessaire à la machine pour proposer un nouveau \textit{clustering} adapté aux suggestions de l'expert.
			Pour poursuivre nos études et pouvoir estimer le coût total d'un projet d'annotation, il nous reste à estimer le nombre total de contraintes à devoir renseigner en fonction de la taille du jeu de données.
			
			% Objectif de l'expérience.
			Pour cela, nous allons simuler la création de cette base d'apprentissage en adaptant le protocole utilisé lors de notre étude d'efficacité (cf. section~\ref{section:4.1.1-ETUDE-CONVERGENCE}) :
			nous emploierons notre méthode de \textit{clustering} interactif avec notre \textbf{paramétrage favori}
			\footnote{Paramétrage favori (atteindre $90$\% de \texttt{v-measure} avec un coût minimal): prétraitement simple (\texttt{prep.simple}), vectorisation TF-IDF (\texttt{vect.tfidf}), \textit{clustering} KMeans avec modèle COP (\texttt{clust.kmeans.cop}) et échantillonnage des données les plus proches dans des clusters différents (\texttt{sampl.closest.diff})}
			sur des jeux de données de différentes tailles et mesurerons le nombre de contraintes nécessaires pour converger vers la vérité terrain.
			
			% Axiome et Remarque.
			\begin{leftBarWarning}
				Dans le cadre de cette étude, nous supposons que l'expert métier connaît parfaitement le domaine traité dans ce jeu de données, et qu'il est capable de caractériser sans ambiguïté la similitude entre deux données issues de cet ensemble.
				De plus, pour utiliser des jeux de données de tailles différentes tout en maîtrisant leur contenu, nous avons dupliqués aléatoirement des données issues de deux jeux de référence en générant des fautes de frappes.
				Pour cette étude, nous faisons l'hypothèse que cela n'a pas d'impact majeur sur le nombre de contraintes nécessaires pour converger vers la vérité terrain.
			\end{leftBarWarning}
			
			% Pseudo-code.
			Pour résumer le protocole expérimental que nous décrivons ci-dessous, vous pouvez vous référer au pseudo-code décrit dans Alg.~\ref{algorithm:4.3.3-ETUDE-COUT-NOMBRE-CONTRAINTES-PROTOCOLE}.
			%
			\begin{algorithm}[!htb]
				\begin{algorithmic}[1]
					\Require jeux de données annotés (vérité terrain) de tailles différentes
					\ForAll{jeux de données à tester}
						\State \textbf{initialisation (données)}: récupérer ou générer les données et la vérité terrain
						\State \textbf{initialisation (contraintes)}: créer une liste vide de contraintes
						\State \textbf{prétraitement}: supprimer le bruit dans les données avec \texttt{prep.simple}
						\State \textbf{vectorisation}: transformer les données en vecteurs avec \texttt{vect.tfidf}
						\State \textbf{clustering initial}: regrouper les données par similarité avec \texttt{clust.kmeans.cop}
						\State \textbf{évaluation}: estimer l'équivalence entre le \textit{clustering} obtenu et la vérité terrain
						\Repeat
							\State \textbf{échantillonnage}: sélectionner de nouvelles contraintes à annoter
							\State \textbf{simulation d'annotation}: ajouter des contraintes avec \texttt{samp.closest.diff}
							\State \textbf{clustering}: regrouper les données par similarité avec \texttt{clust.kmeans.cop}
							\State \textbf{évaluation}: estimer l'équivalence entre le \textit{clustering} obtenu et la vérité terrain
						\Until{annotation de toutes les contraintes possibles}
					\EndFor						
					\State \textbf{analyse}: entraîner un modèle linéaire généralisé du nombre de contraintes nécessaires
					\Ensure modélisation du nombre de contraintes nécessaires pour un jeu de données
				\end{algorithmic}
				\caption{Description en pseudo-code du protocole expérimental de l'étude du nombre de contraintes nécessaires pour converger vers une vérité terrain pré-établie avec notre paramétrage favori du \textit{clustering} interactif.}
				\label{algorithm:4.3.3-ETUDE-COUT-NOMBRE-CONTRAINTES-PROTOCOLE}
			\end{algorithm}
			
			% Description des jeux de données.
			Pour cette étude, nous utilisons comme références les jeux de données \textbf{TODO:JDD}\todo{TODO: reference dataset bank card} et \textbf{TODO:JDD}\todo{TODO: reference dataset mlsum}.
			La taille des jeux de données générée, noté $\texttt{dataset\_size}$, varie entre $1~000$ à $5~000$ par pas de $250$, et chaque taille de jeu est générée $3$ fois pour contrer les aléas statistiques de création.
			Il y a donc $51$ variations de chaque jeu de références, soit $102$ jeux utilisés de tailles différentes.
			
			% Description des tentatives de la méthode.
			Sur chacun de ces jeux générés, une tentative complète
			\footnote{Tentative complète : itérations d'échantillonnage, d'annotation et de \textit{clustering} jusqu'à annotation de toutes les contraintes possibles.}
			de la méthode du \textit{clustering} interactif utilisant notre paramétrage favori est exécuté, et chaque tentative est répétée $5$ fois pour contrer les aléas statistiques des exécutions.
			Il y a donc $510$ tentatives de \textit{clustering} interactif réalisées.
			
			% Description de l'évaluation.
			Pour chacune de ces tentatives, nous nous intéresserons au nombre de contraintes nécessaires pour atteindre le seuil d'annotation partielle (caractérisé par $90$\% de \texttt{v-measure} entre la vérité terrain et la segmentation des données obtenue), et nous entraînerons un modèle linéaire généralisé (\textit{GLM}) pour modéliser le nombre de contraintes requis en fonction de la taille du jeu de données (noté $\texttt{dataset\_size}$).
			Ce modèle sera caractérisé par le coefficient de détermination généralisé \texttt{R²} de \textit{Cox et Snel}\todo{CITATION}, la log-vraisemblance \texttt{llf}\todo{CITATION} et la log-vraisemblance \texttt{llf\_null} du modèle \textit{null}.
			Pour finir, nous discuterons des valeurs des coefficients obtenus sur l'impact du nombre d'itérations de la méthode à prévoir.

			% Référence scripts.
			\begin{leftBarInformation}
				Ces analyses sont réalisées en Python à l'aide de la librairie \texttt{statsmodels} (\cite{seabold:2010}).
				Les scripts de l'expérience (\textit{notebooks} Python) sont disponibles dans un dossier dédié de~\cite{schild:cognitivefactory-interactive-clustering-comparative-study:2021}.
			\end{leftBarInformation}

		%%% Résultats.
		\subsubsection{Résultats obtenus}
		
			% Modélisation du nombre de contraintes.
			Le modèle linéaire généralisé entraîné sur les mesures du nombre de contraintes requis pour atteindre $90$\% de \texttt{v-measure} (\texttt{R²}: $> 0.999$, \texttt{llf}: $-4~327.6$, \texttt{llf\_null}: $-4~942.9$) nous permet de déduire l'équation suivante :
			%
			\begin{equation}
				\label{equation:4.3.3-ETUDE-COUT-NOMBRE-CONTRAINTES}
				\texttt{constraints\_needed}(\texttt{settings.favorite})~[\#]~
				\propto~3.15 \cdot \texttt{dataset\_size}
			\end{equation}
			%
			La figure~\ref{figure:4.3.3-ETUDE-COUT-NOMBRE-CONTRAINTES} représente cette modélisation.
			\newline
			%
			\begin{figure}[!htb]
				\centering
				\includegraphics[width=0.8\textwidth]{figures/etude-nombre-contraintes-1-modelisation-nombre}
				\caption{Estimation du nombre moyen de contraintes nécessaire à notre \textbf{paramétrage favori} du \textit{clustering} interactif afin d'obtenir une annotation partielle (\textit{atteindre une \texttt{v-measure} de $90$\%}) en fonction de la taille du jeu de données à modéliser.}
				\label{figure:4.3.3-ETUDE-COUT-NOMBRE-CONTRAINTES}
			\end{figure}
		
			% Note de l'auteur.
			\begin{leftBarAuthorOpinion}
				% Estimation de points de références.
				On peut considérer les points de références suivants :
				%
				\begin{itemize}
					% Estimation grossière.
					\item le nombre de contraintes possibles (avec doublons) est de $\texttt{dataset\_size}^{2}$
					(\textit{caractériser chaque couple de données présent dans la matrice d'adjacence}) ;
					% Estimation sans doublon.
					\item le nombre de contraintes possibles (sans doublons) est de $\frac{1}{2} \cdot (\texttt{dataset\_size}^{2} - \texttt{dataset\_size})$
					(\textit{considérer la symétrie des contraintes, donc seul le triangle supérieur de la matrice d'adjacence a besoin d'être renseigné}) ;
					% Estimation minimale.
					\item le nombre minimal de contraintes à annoter sur une partition en $k$ \textit{clusters} $\{K_{1}, K_{2}, ..., K_{k}\} $ du jeu de données et en étant exhaustif est estimé à ${\displaystyle \sum\limits_{0 \leq i < k}{\|K_{i}\|-1} + \sum\limits_{0 \leq i < k}{k-1-i}} $ (
					\textit{considérer d'abord le chemin minimal pour créer des composants connexes avec des contraintes \texttt{MUST-LINK}, puis ajouter le nombre minimal des contraintes \texttt{CANNOT-LINK} pour distinguer les composants connexes en \textit{cluster}}).
				\end{itemize}
				%
				% Annonce de la figure.
				La figure~\ref{figure:4.3.3-ETUDE-COUT-NOMBRE-CONTRAINTES-EXEMPLES} illustre ces propos sur un jeu d'exemple comportant $10$ points de données réparties en $3$ classes, et met en avant l'explosion du nombre de contraintes possibles même sur un petit jeu de données (cf.~\ref{figure:4.3.3-ETUDE-COUT-NOMBRE-CONTRAINTES-EXEMPLES}~\textbf{(2)}).
				
				% Application de ces points de référence.
				Avec ces références, le nombre de contraintes est borné approximativement
				entre $1~035$ et $499~500$ pour un jeu de $1~000$ données équilibré en $10$ classes,
				et entre $6~175$ et $12~497~500$ pour un jeu de $5~000$ données équilibré en $50$ classes.
				%
				\begin{figure}[H]
					\centering
					\includegraphics[width=0.7\textwidth]{figures/etude-nombre-contraintes-2-bornes-limites}
					\caption{Exemple de caractérisation exhaustive d'un jeu de données ($10$ données, $3$ classes) en ajoutant un nombre minimal de contraintes (cf. \textbf{(1)}) ou en ajoutant toutes les contraintes possibles (cf. \textbf{(2)}).}
					\label{figure:4.3.3-ETUDE-COUT-NOMBRE-CONTRAINTES-EXEMPLES}
				\end{figure}
			\end{leftBarAuthorOpinion}

		%%% Discussion.
		\subsubsection{Discussion}
		
			% Rappel de l'objectif.
			L'objectif de cette étude était de déterminer le nombre moyen de contraintes à devoir annoter pour modéliser un jeu de données avec un accord $90$\% de \texttt{v-measure} avec la vérité terrain utilisée.
			Cette estimation, dépendant de la taille du jeu de données manipulé, est représentée par l'équation~\ref{equation:4.3.3-ETUDE-COUT-NOMBRE-CONTRAINTES}.
			
			% Discussion générale sur la pente.
			On peut constater que la relation entre la taille du jeu de données et le nombre de contraintes à annoter est linéaire (pente de $3.15$) : doubler la taille d'un jeu de données doublera donc la charge de travail incombant à l'expert métier.
			\todo[inline]{A REDIGER: Ca aurait pu être pire (cf. notes de l'auteur)}
			\todo[inline]{A REDIGER: Toutefois, ça parait beaucoup et ça pourrait gêner... cf. 4.3.4}
			
			% Influence du jeu de données.
			Bien évidemment, une telle estimation est sensible au jeu de données utilisé comme référence (cf. figure~\ref{figure:4.3.3-ETUDE-COUT-NOMBRE-CONTRAINTES}).
			Ici, la différence de pente mesurée est de $0.25$ (\texttt{p-valeur}: $> 0.999$), soit un écart moyen d'environ $8$\% par rapport à la modélisation moyenne.
			Toutefois, comme l'impact semble limité, nous maintenons la modélisation moyenne représentée par l'équation~\ref{equation:4.3.3-ETUDE-COUT-NOMBRE-CONTRAINTES} pour la suite de nos estimations de coûts.
		
			% Note de l'auteur.
			\begin{leftBarAuthorOpinion}
				Il n'y a pas davantage de matière à discussion pour cette étude, car le principal résultat (l'équation~\ref{equation:4.3.3-ETUDE-COUT-NOMBRE-CONTRAINTES}) est un résultat temporaire nécessaire à l'estimation du coût global d'un projet utilisant une méthodologie de \textit{clustering} interactif.
			\end{leftBarAuthorOpinion}
	
	%%%
	%%% Subsection 4.3.4: Estimation du temps total d'un projet d'annotation en combinant les précédentes études de coûts
	%%%
	\subsection{Estimation du temps total d'un projet d'annotation en combinant les précédentes études de coûts}
	\label{section:4.3.4-ETUDE-COUTS-TOTAL}
	
		% Equation finale.
		Résumons l'ensemble des modélisations réalisées lors des précédentes études (cf. sections \ref{section:4.3.1-ETUDE-COUTS-TEMPS-ANNOTATION}, \ref{section:4.3.2-ETUDE-COUTS-TEMPS-CALCUL} et \ref{section:4.3.3-ETUDE-COUT-NOMBRE-CONTRAINTES}) afin d'estimer le coût total d'un projet d'annotation employant une méthodologie basée sur le \textit{clustering} interactif et utilisant notre \textbf{paramétrage favori}
		\footnote{Paramétrage favori (atteindre $90$\% de \texttt{v-measure} avec un coût minimal).}.
		Dans les notations, $\texttt{dataset\_size}$ représente la taille du jeu de données à modéliser, et $\texttt{batch\_size}$ représente le nombre de contraintes que l'expert annote à chaque itération.

		%%% Résultats.
		\subsubsection{Synthèse des résultats}
			
			% Temps pour une itération.
			Tout d'abord, nous pouvons estimer le \textbf{temps moyen d'une itération de la méthode}, comprenant les temps de d'exécution des algorithmes (\textit{prétraitement}, \textit{vectorisation}, \textit{clustering}, \textit{échantillonnage}) et le temps d'annotation d'un lot de contraintes, grâce aux équations suivantes :
			\begin{equation}
				\label{equation:4.3.4-ETUDE-COUT-UNE-ITERATION}
				\begin{cases}
					% Computation time.
					\texttt{computation\_time}~[s]&
						~\propto~1.52 \cdot 10^{-1} \cdot \texttt{dataset\_size} + 1.43 \cdot 10^{-6} \cdot \texttt{dataset\_size}^{2} \\
					% Annotatim time.
					\texttt{annotation\_time}~[s]&
						~\propto~7.77 \cdot \texttt{batch\_size} \\
					% One iteration time.
					\texttt{one\_iteration\_time}~[s]&
						~\propto~\texttt{computation\_time} + \texttt{annotation\_time} \\
				\end{cases}
			\end{equation}
			
			% Nombre d'itérations.
			Ensuite, nous sommes en mesure d'anticiper le \textbf{nombre moyen de contraintes à annoter} pour modéliser le jeu de données avec un seuil de $90$\% de \texttt{v-measure}, et donc de déduire le nombre d'itérations nécessaire de la méthode, grâce aux équations suivantes :
			\begin{equation}
				\label{equation:4.3.4-ETUDE-COUT-NOMBRE-ITERATIONS}
				\begin{cases}
					% Constraints number.
					\texttt{constraints\_number\_needed}~[\#] &
						~\propto~3.15 \cdot \texttt{dataset\_size} \\
					% Iterations number.
					\texttt{iterations\_number\_needed}~[\#] &
						~\propto~\texttt{constraints\_number\_needed} / \texttt{batch\_size} \\
				\end{cases}
			\end{equation}
			
			% Temps total pour un projet.
			Enfin, il suffit de combiner les deux équations~\ref{equation:4.3.4-ETUDE-COUT-UNE-ITERATION} et~\ref{equation:4.3.4-ETUDE-COUT-NOMBRE-ITERATIONS} pour estimer le temps total nécessaire à un projet d'annotation utilisant le \textit{clustering} interactif pour converger vers $90$\% de \texttt{v-measure} :
			\begin{equation}
				\label{equation:4.3.4-ETUDE-COUT-TOTAL}
				\begin{cases}
					% Constraints number.
					\texttt{total\_time}~[s] &
						~\propto~\texttt{one\_iteration\_time} \cdot \texttt{iterations\_number\_needed}
				\end{cases}
			\end{equation}
			
			% Figure.
			La figure~\ref{figure:4.3.4-ETUDE-COUT-TOTAL} représente cette estimation globale en fonction de plusieurs taille de jeu de données et plusieurs tailles de lots d'annotation.
			%		
			\begin{figure}[!htb]
				\centering
				\includegraphics[width=0.8\textwidth]{figures/etude-temps-total-1-modelisation}
				\caption{Estimation du temps total nécessaire (en heures) pour modéliser un jeu de données avec notre \textbf{paramétrage favori} du \textit{clustering} interactif afin d'obtenir une annotation partielle (\textit{atteindre une \texttt{v-measure} de $90$\%}), en fonction de la taille du jeu de données et de la taille des lots d'annotations.}
				\label{figure:4.3.4-ETUDE-COUT-TOTAL}
			\end{figure}

		%%% Discussion finale.
		\subsubsection{Discussion finale}
		
		
			\todo[inline]{A REDIGER: (1) Rappel sur l'objectif des précédentes études}
			\todo[inline]{A REDIGER: (2) Annonce des résultats : c'est long !}
			\todo[inline]{A REDIGER: (3) Comparer avec une annotation classique}
			\todo[inline]{A REDIGER: (4) Augmenter la taille des lots d'annotation + Augmenter le nombre d'annotateurs + paralléliser annotation et clustering}
			\todo[inline]{A REDIGER: (5) Estimer la rentabilité d'une nouvelle itération pour savoir quand s'arrêter}
		
		
		\todo[inline]{A REDIGER : Calcul total pour le dataset de 500 points de données. Rejetter l'annotation suffisante car trop longue.}
		
		%%	A l'aide de la modélisation temporelle réalisée ci-dessus, nous pouvons estimer qu'un lot de $50$ contraintes (comme employé dans nos précédentes études) nécessite en moyenne $548$ secondes, soit $9.13$ minutes.
		%%	De plus, en considérant les résultats de l'étude d'efficience (cf. section~\ref{section:4.2-HYPOTHESE-EFFICIENCE}) sur un jeu de $500$ points de données, nous pouvons déduire que :
		%%	\begin{itemize}
		%%		\item l'obtention d'une annotation partielle (atteindre $90$\% de \texttt{v-measure}) peut se faire avec $950$ contraintes, soit $19$ itérations de $50$ contraintes, ce qui équivaut à une moyenne de $2.89$ heures d'annotations ;
		%%		\item l'obtention d'une annotation suffisante (atteindre $100$\% de \texttt{v-measure}) peut se faire en $1~730$ contraintes, soit $34.6$ itérations de $50$ contraintes, ce qui équivaut à une moyenne de $5.27$ heures d'annotations.
		%%	\end{itemize}
		
		\todo[inline]{A REDIGER : comparaison du total avec 5 points : le temps nécessaire à créer banking cards, le temps nécessaire à adapter MLSUM, et trois études de Snow et al. (2008)}
		
			% Commentaire sur la difficulté de comparer les durées totales.
		%%	En ce qui concerne les durées totales pour annoter un jeu de $500$ points de données, estimées respectivement à $2.89$ et à $5.27$ heures suivant les seuils d'annotation, il est difficile de discuter de leur compétitivité à cause du manque de repères concrets.
		%%	En effet, les références dépendent de la tâche d'annotation, des données manipulées, du nombre de choix proposés à l'opérateur, de la complexité sémantique des données, des compétences des annotateurs, ...
		%%	\footnote{De plus, la plupart des études des tâches d'annotation en \textit{machine learning} se concentrent sur la cohérence inter-annotateurs plutôt que sur le temps nécessaire.}
		%%	Pour pallier ce problème, nous proposons de comparer nos estimations de temps d'annotations grâce aux pistes ci-dessous.
		%%	Ces repères sont approximatifs et assez disparates, mais ils nous permettront de discuter des ordres de grandeurs à manipuler.
			
			% Cas de taches de complexité similaire.
		%%	D'une part, comparons nos estimations à trois exemples de tâches ayant une \textbf{complexité similaire} à la vérité terrain que nous manipulons lors de cette expérience. Ces points de repères font appel à la catégorisation du thème traité par des textes courts.
			%
		%%	\begin{enumerate}
		%%		% Conception du JDD bank cards.
		%%		\item Pour concevoir le jeu de données~\todo{citation: bank cards} ($10$ classes équilibrées, $500$ points de données) servant de vérité terrain aux études d'efficacité et d'efficience, nous avons requis approximativement $5$ heures, dont $2.5$ d'annotation
		%%		\footnote{La conception de $bank cards$ a demandé $\sim 2$ heure de définition du périmètre (1 personne), $\sim 2$ heures de collecte et d'annotation des étiquettes des données (1 personne), et $\sim 1$ heure de revue de la cohérence des annotations (2 personne)} ;
		%%		% Conception du JDD MLSUM subset.
		%%		\item Pour adapter le jeu de données~\todo{citation: mlsum} en~\todo{citation: mlsum subset SCHILD} ($14$ classes retenues, $744$ points de données sélectionnées sur un échantillon de $1~050$) servant de base à notre étude du coût de l'annotation, nous avons requis approximativement $5$ heures, dont $2$ d'annotation binaire
		%%		\footnote{L'adaptation de $TODO:mlsum$ a demandé $\sim 1$ heure de collecte des catégories les plus communes (1 personne), $\sim 2$ heures d'annotation binaire de la pertinence des données (2 personnes) et $\sim 2$ heures de revue de la cohérence des annotations (2 personne)} ;
		%%		% Article Cheap and Fast — But is it Good? (Word Sense Disambiguation)
		%%		\item Selon~\todo{citation: Snow et al. (2008) / Yuret (2007)}, qui délègue à \texttt{Amazon Mechanical Turk} l'annotation de phrases pour catégoriser leur contexte parmi trois possibilités préformatés, il faut $8.59$ heures pour étiqueter $1~770$ données. En pondérant cette approximation pour un jeu de $500$ points de données, on peut estimer le temps d'annotation à $2.43$ heures environ ;
		%%	\end{enumerate}
			%
		%%	À l'aide de ces trois exemples, on remarque qu'une annotation partielle ($950$ contraintes pour atteindre $90$\% de \texttt{v-measure}) nécessite une durée comparable à une annotation par label ($2.89$ heures vs. entre $2$ et $3$ heures), mais une annotation suffisante ($1~730$ contraintes pour atteindre $100$\% de \texttt{v-measure}) nécessite 2 à 3 fois plus de temps ($5.27$ heures).
		%%	Il est donc préférable de nous concentrer sur une \textbf{annotation partielle} afin de garder une méthode efficiente. \todo{lien vers la section 4.4 a faire dans la discussion globale de cette section}
				% Ce choix implique de pouvoir compléter la différence qualité en remaniant manuellement certains \textit{clusters} obtenus lors d'une phase de revue.
				% Cet aspect sera complété à partir de l'étude décrite dans la section~\ref{section:4.4-HYPOTHESE-PERTINENCE} (hypothèse de pertinence).
			
			% cas de taches de complexité différentes
		%%	D'autre part, comparons nos estimations à deux exemples de tâches ayant des \textbf{complexité différentes} (une premier presque triviale, une seconde plutôt complexe) :
			%
		%%	\begin{enumerate}
		%%		\setcounter{enumi}{3}
		%%		% Article Cheap and Fast — But is it Good? (Word Similarity)
		%%		\item Selon~\todo{citation: Snow et al. (2008) / Rubenstein et Goodenough (1965)}, qui délègue à \texttt{Amazon Mechanical Turk} l'annotation de couples de mots pour identifier par similarité des synonymes, il faut $0.17$ heures pour ordonner $300$ paires de données de la plus à la moins similair
		%%		\footnote{La tâche d'annotation de synonymes par ordonnancement peut s'apparenter à de l'annotation de contraintes en se demandant : « \textit{laquelle de ces deux paires est la plus adéquate ? »}. Cette tâche est relativement simple (similarité intrinsèque triviale, peu de vocabulaire).}. En pondérant cette approximation pour un lot de $950$ contraintes, on peut estimer le temps d'annotation à environ $0.53$ heures ;
		%%		% Article Cheap and Fast — But is it Good? (Recognizing Textual Entailment)
		%%		\item Encore selon~\todo{citation: Snow et al. (2008) / Dagan et Magnini (2005)}, qui délègue à \texttt{Amazon Mechanical Turk} l'annotation binaire de la véracité d'une implication, il faut $89.3$ heures pour étiqueter $8~000$ données
		%%		\footnote{La tâche d'annotation de la véracité d'une implication peut s'apparenter à de l'annotation de contraintes en se demandant : « \textit{est-ce que l'implication est vraie ? »}. Cette tâche est relativement complexe (implication logique non trivial). }. En pondérant cette approximation pour un lot de $950$ contraintes, on peut estimer le temps d'annotation à environ $10.60$ heures ;
		%%	\end{enumerate}
			%
		%%	Avec ces deux exemples, on constate que les variations peuvent être très grandes (allant du cinquième au quadruple !).
		%%	Il est donc important de nuancer nos précédentes conclusions en fonction de la complexité de la thématique traitée.
		%%	Toutefois, 
		%%	\todo[inline]{données intrinsèquement différentes : peuvent être caractérisées très rapidement}
		%%	\todo[inline]{données ambiguës : peuvent demander plus de temps, plus de réflexion}
		
		\todo[inline]{A REDIGER : TRANSISTION VERS PERTINENCE car la notion de v-measure est impossible à calculer en situation réelle}



	
	
	%%%%%--------------------------------------------------------------------
	%%%%% Section 4.4: Hypothèse de pertinence.
	%%%%%--------------------------------------------------------------------
	\newpage
	\section{Hypothèse de pertinence}
\label{section:4.4-HYPOTHESE-PERTINENCE}
% « \textit{est-ce le résultat est exploitable ?} »

	%%% Formulation des hypothèses:
	Nous aimerions vérifier l'hypothèse suivante :
	\todo{à compléter}

	\begin{tcolorbox}[
		title=\faVial~\textbf{Hypothèse de pertinence}~\faVial,
		colback=colorTcolorboxHypothesis!15,  % gray!20
		colframe=colorTcolorboxHypothesis!75,  % gray!50!black!75,
		width=\linewidth
	]
		« La vitesse de convergence du \textit{clustering} interactif \textbf{peut être optimisée} en réglant différents paramètres. Nous étudierons l'influence du prétraitement des données, de la vectorisation des données, de l'échantillonnage des contraintes à annoter et du \textit{clustering} sous contraintes (cf. figure~\ref{figure:4.4-HYPOTHESE-PERTINENCE}. »
		
		
		\begin{figure}[H]  % keep [H] to be in the tcolorbox.
			\centering
			\includegraphics[width=0.95\textwidth]{figures/hypotheses-04-pertinence}
			\caption{Illustration des études réalisées sur le \textit{clustering} interactif (\textit{étape 4/6}) en schématisant l'évolution de la performance (\textit{accord avec la vérité terrain calculé en v-measure}) d'une base d'apprentissage en cours de construction en fonction du nombre d'itérations de la méthode (\textit{nombre d'annotations par un expert métier}).}
			\label{figure:4.4-HYPOTHESE-PERTINENCE}
		\end{figure}

	\end{tcolorbox}
	
	%%%
	%%% Subsection 4.4.1: Étude de la cohérence statistique de la base d'apprentissage en cours de construction
	%%%
	\subsection{Étude de la cohérence statistique de la base d'apprentissage en cours de construction}
	
		%%% Protocole expérimental.
		\subsubsection{Protocole expérimental}
			\todo[inline]{Description succincte du protocole expérimental dans l'encadré d'hypothèse ?}

		%%% Résultats
		\subsubsection{Résultats obtenus}
			%
			\begin{figure}[!htb]
				\centering
				\includegraphics[width=0.95\textwidth]{figures/etude-pertinence-consistence}
				\caption{Évolution du score de cohérence moyen des tentatives en fonction de leur paramétrage : \textbf{(1)} meilleur paramétrage moyen une annotation partielle (\texttt{90}\% de \texttt{v-measure}), \textbf{(2)} meilleur paramétrage moyen une annotation suffisante (\texttt{100}\% de \texttt{v-measure}), \textbf{(3)} meilleur paramétrage moyen une annotation exhaustive (annoter toutes les contraintes possibles), et \textbf{(4)} paramétrage favori (\texttt{90}\% de \texttt{v-measure} avec un coût minimal). \\
				Note : \textit{Le score de cohérence de la vérité terrain peut varier en fonction des méthodes de prétraitements et de vectorisation utilisées.}}
				\label{figure:4.3.1-ETUDE-PERTINENCE-CONHERENCE-ANNOTATION}
			\end{figure}

		%%% Discussion
		\subsubsection{Discussion}
	
	%%%
	%%% Subsection 4.4.2: Étude de la pertinence sémentique de la base d'apprentissage en cours de construction
	%%%
	\subsection{Étude de la pertinence sémantique de la base d'apprentissage en cours de construction}
	
		%%% Protocole expérimental.
		\subsubsection{Protocole expérimental}
			\todo[inline]{Description succincte du protocole expérimental dans l'encadré d'hypothèse ?}

		%%% Résultats
		\subsubsection{Résultats obtenus}

		%%% Discussion
		\subsubsection{Discussion}
	
	
	%%%%%--------------------------------------------------------------------
	%%%%% Section 4.5: Hypothèse de rentabilité.
	%%%%%--------------------------------------------------------------------
	\newpage
	\section{Évaluation de l'hypothèse de rentabilité}
\label{section:4.5-HYPOTHESE-RENTABILITE}

	%%% Introduction / Transition.
	Dans les études précédentes, le cas d'arrêt de notre méthodologie d'annotation basée sur le \textit{clustering} interactif était conditionné à la vérité terrain.
	En effet, nous utilisions un seuil de $90$\% de \texttt{v-measure}, caractérisant une annotation dite "partielle" de la base d'apprentissage.
	Cependant, une telle référence n'est pas accessible en situation réelle car l'objectif de notre méthode est précisément de la construire cette vérité terrain.
	Nous devons donc nous intéresser à d'autres moyens pour estimer la rentabilité d'une itération supplémentaire et pouvoir ainsi définir de nouveaux cas d'arrêt pour le \textit{clustering} interactif.
	Pour cela, nous aimerions vérifier l'hypothèse suivante :
	
	%%% Formulation des hypothèses:
	\begin{tcolorbox}[
		title=\faVial~\textbf{Hypothèse de rentabilité}~\faVial,
		colback=colorTcolorboxHypothesis!15,
		colframe=colorTcolorboxHypothesis!75,
		width=\linewidth
	]
		« \textbf{
			Au cours d'une méthodologie d'annotation basée sur le \textit{clustering} interactif, il est possible d'estimer la rentabilité d'une itération supplémentaire de la méthode, et ainsi d'établir des cas d'arrêt indépendant d'une vérité terrain pour obtenir une base d’apprentissage satisfaisante.
		} » \\
		
		% Figure.
		La \textsc{Figure~\ref{figure:4.5-HYPOTHESE-RENTABILITE}} illustre cette hypothèse et l'espoir de pouvoir estimer le rapport entre le gain de pertinence obtenu et le coût nécessaire pour l'obtenir.
		%
		\begin{figure}[H]  % keep [H] to be in the tcolorbox.
			\centering
			\includegraphics[width=0.95\textwidth]{figures/hypotheses-05-rentabilite}
			\caption{
				Illustration des études réalisées sur le \textit{clustering} interactif (\textit{étape 5/6}) en schématisant l'évolution de la pertinence (\textit{valeur métier évaluée par l'expert et exprimé en nombre de clusters}) d'une base d'apprentissage en cours de construction en fonction du coût temporel de la méthode (\textit{temps nécessaire à l'expert métier et à la machine}), ainsi que la rentabilité de chaque itération de la méthode (\textit{rapport entre le gain potentiel de pertinence et le coût à investir}).
			}
			\label{figure:4.5-HYPOTHESE-RENTABILITE}
		\end{figure}
	\end{tcolorbox}
		
	% Résumé de l'étude.
	Afin de vérifier cette hypothèse, nous explorons deux approches :
	\begin{itemize}
		\item l'évolution de l'\textbf{accord entre l'annotation de l'expert et le \textit{clustering}} sur lequel est basé l'échantillon d'annotation, permettant d'estimer si la machine doit encore être corrigée par l'annotateur  (cf. \textsc{Section~\ref{section:4.5.1-ETUDE-RENTABILITE-ACCORD-ANNOTATION-CLUSTERING}}) ;
		\item et l'évolution de la \textbf{différence entre deux \textit{clusterings} successifs}, permettant de mesurer s'il y a eu des changements visibles dans le partitionnement des données après l'ajout des dernières contraintes (cf. \textsc{Section~\ref{section:4.5.2-ETUDE-RENTABILITE-SIMILARITE-CLUSTERING}}).
	\end{itemize}
	
	
	%%%
	%%% Subsection 4.5.1: Étude de l'évolution d'accord entre l'annotation et le \textit{clustering}.
	%%%
	\subsection{Étude de l'évolution d'accord entre l'annotation et le \textit{clustering}}
	\label{section:4.5.1-ETUDE-RENTABILITE-ACCORD-ANNOTATION-CLUSTERING}
		
		% Objectif de l'expérience.
		Nous cherchons à trouver un cas d'arrêt du \textit{clustering} interactif ne nécessitant pas de comparaison avec une vérité terrain, et notre première intuition concerne l'étude des annotations réalisées.
		En effet, à chaque itération, l'expert annote un échantillon de contraintes dans le but de confirmer ou de corriger le \textit{clustering} de l'itération précédente.
		Or, après un nombre suffisant d'itérations, le \textit{clustering} commence à se stabiliser : il devrait donc y avoir davantage d’annotations qui confirment le \textit{clustering} que d'annotations qui le corrigent, puis n'avoir que des accords entre les annotations et le \textit{clustering}.
		Ainsi, nous allons étudier l'évolution du nombre de contraintes annotées qui approuvent le partitionnement des données obtenu et essayer d'adapter cette analyse en cas d'arrêt pour notre méthode d'annotation.
	
		%%% Protocole expérimental.
		\subsubsection{Protocole expérimental}
			
			% Axiome.
			\begin{leftBarWarning}
				Dans le cadre de cette étude, nous supposons que l'expert métier connaît parfaitement le domaine traité dans ce jeu de données, et qu'il est capable de caractériser sans ambiguïté la similitude entre deux données issues de cet ensemble.
			\end{leftBarWarning}
			
			% Pseudo-code.
			Pour résumer le protocole expérimental que nous décrivons ci-dessous, vous pouvez vous référer au pseudo-code décrit dans \textsc{Algorithme~\ref{algorithm:4.5.1-ETUDE-RENTABILITE-ACCORD-ANNOTATION-CLUSTERING-PROTOCOLE}}.
			
			\begin{algorithm}
				\KwData{jeu de données annotés (vérité terrain)}
				%
				\ForEach{jeux de données à tester}{
					\textbf{initialisation (données)}: récupérer les données et la vérité terrain \;
					\textbf{initialisation (contraintes)}: créer une liste vide de contraintes \;
					\textbf{prétraitement}: supprimer le bruit dans les données avec \texttt{prep.simple} \;
					\textbf{vectorisation}: transformer les données en vecteurs avec \texttt{vect.tfidf} \;
					\textbf{clustering initial}: regrouper les données par similarité avec \texttt{clust.kmeans.cop} \;
					\Repeat{annotation de toutes les contraintes possibles}{
						\textbf{échantillonnage}: sélectionner des contraintes avec \texttt{samp.closest.diff} \;
						\textbf{simulation d'annotation}: caractériser les contraintes grâce à la vérité terrain \;
						\textbf{intégration}: ajouter les nouvelles contraintes au gestionnaire de contraintes \;
						\textbf{rentabilité}: calculer l'accord entre l'annotation et le \textit{clustering} précédent \;
						\textbf{clustering}: regrouper les données par similarité avec \texttt{clust.kmeans.cop} \;
					}
				}
				\textbf{analyse 1}: afficher l'évolution de l'accord entre annotation et \textit{clustering} \;
				\textbf{analyse 2}: calculer la corrélation entre le score d'accord et le score de performance \;
				%
				\KwResult{discussion sur la rentabilité d'après l'accord entre annotation et \textit{clustering}}
				%
				\caption{\textit{
					Description en pseudo-code du protocole expérimental de l'étude de l'évolution d'accord entre l'annotation et le \textit{clustering}.
				}}
				\label{algorithm:4.5.1-ETUDE-RENTABILITE-ACCORD-ANNOTATION-CLUSTERING-PROTOCOLE}
			\end{algorithm}
			
			% Description de la vérité terrain.
			Nous utilisons comme vérité terrain le jeu de données \texttt{Bank Cards (v1.0.0)} : ce dernier traite des demandes les plus fréquentes des clients en ce qui concerne la gestion de leur carte bancaire.
			Il est composé de $500$ questions rédigées en français et réparties en $10$ classes (\texttt{perte ou vol de carte}, \texttt{carte avalée}, \texttt{commande de carte}, ...).
			Pour plus de détails, consultez l'annexe~\ref{annex:C.1-DATASET-BANK-CARDS}.
			
			% Description des tentatives de la méthode et du calcul de rentabilité.
			Sur ce jeu de données, nous exécutons une tentative complète
			\footnote{Tentative complète : itérations d'échantillonnage, d'annotation et de \textit{clustering} jusqu'à annotation de toutes les contraintes possibles.}
			\todo{Utiliser 'footmisc' et 'footref' pour faire des notes de bas de pages communes ? ou lien vers des conclusions ?}
			de la méthode du \textit{clustering} interactif en utilisant notre paramétrage favori, et cette tentative est répétée $5$ fois pour contrer les aléas statistiques des exécutions.
			À chaque itération, un lot de $50$ contraintes est sélectionné puis annotés en simulant l'action d'un expert métier, et nous évaluons l'accord entre ces nouvelles annotations et la proposition de partitionnement des données réalisé par le \textit{clustering} à l'itération précédente :
			\begin{itemize}
				\item il y a \textbf{accord} lorsqu'une contrainte de deux données issues d'un même \textit{cluster} est annotée \texttt{MUST-LINK}, ou lorsqu'une contrainte de deux données issues de deux \textit{clusters} différents est annotée \texttt{CANNOT-LINK} (cf. \textsc{Figure~\ref{figure:4.5.1-ETUDE-RENTABILITE-ACCORD-ANNOTATION-CLUSTERING-EXEMPLE} (1)}) ;
				\item il y a \textbf{désaccord} lorsqu'une contrainte de deux données issues d'un même \textit{cluster} est annotée \texttt{CANNOT-LINK}, ou lorsqu'une contrainte de deux données issues de deux \textit{clusters} différents est annotée \texttt{MUST-LINK} (cf. \textsc{Figure~\ref{figure:4.5.1-ETUDE-RENTABILITE-ACCORD-ANNOTATION-CLUSTERING-EXEMPLE} (2)}).
			\end{itemize}
			Nous pouvons ainsi calculer un score d'accord défini par la ratio entre le nombre d'accords et le nombre de contraintes annotées.
			Pour nous permettre de discuter de l'utilité de ce score pour prédire la stabilisation du \textit{clustering} et ainsi définir un cas d'arrêt de notre méthodologie d'annotation, nous calculons aussi le score de corrélation entre cet accord et la performance obtenu à l'aide d'une vérité terrain (la corrélation \texttt{r} de \textit{Pearson} (\cite{kirch:2008:pearson-correlation-coefficient}) est utilisée).

			\begin{figure}[!htb]
				\centering
				\includegraphics[width=0.7\textwidth]{figures/example-accord-annotation-clustering}
				\caption{
					Exemples d'accords et de désaccord entre les annotations d'une itération et le résultat du \textit{clustering} de l'itération précédente.
					Des contraintes \texttt{MUST-LINK} (flèches vertes) et \texttt{CANNOT-LINK} (flèches rouges) sont représentées dans deux situations : \textbf{(1)} montre des cas d'accords (\texttt{MUST-LINK} dans un même \textit{cluster}, \texttt{CANNOT-LINK} entre deux \textit{clusters} différents), et \textbf{(2)} montre des cas de désaccords (\texttt{MUST-LINK} entre deux \textit{clusters} différents, \texttt{CANNOT-LINK} dans un même \textit{cluster}).
				}
				\label{figure:4.5.1-ETUDE-RENTABILITE-ACCORD-ANNOTATION-CLUSTERING-EXEMPLE}
			\end{figure}
			
			\begin{leftBarIdea}
				Nous concentrons l'étude sur notre paramétrage favori (voir \textsc{Section~\ref{section:4.4.3-ETUDE-PERTINENCE-RESUME-AUTOMATIQUE}}).
				Cependant, afin de compléter notre discussion avec d'autres points de comparaison, nous analysons aussi les autres paramétrages implémentés, notamment les meilleurs paramétrages moyens identifiés lors de l'hypothèse d'efficience (voir \textsc{Section~\ref{section:4.2-HYPOTHESE-EFFICIENCE}}).
			\end{leftBarIdea}
			
			% Référence scripts.
			\begin{leftBarInformation}
				Les scripts de l'expérience, réalisés avec des \textit{notebooks} Python (\cite{van-rossum-drake:2009:python-reference-manual}), sont disponibles dans un dossier dédié de~\cite{schild:2021:cognitivefactory-interactiveclusteringcomparativestudy}.
			\end{leftBarInformation}

		%%% Résultats
		\subsubsection{Résultats obtenus}
			
			% Figure : croissance générale.
			La \textsc{Figure~\ref{figure:4.5.1-ETUDE-RENTABILITE-ACCORD-ANNOTATION-CLUSTERING}} représente l'évolution moyenne du score d'accord entre annotation et \textit{clustering} pour les quatre paramétrages mis en avant lors de nos études.
			Nous pouvons constater une tendance générale à la croissance de ce score d'accord : pour le paramétrage favori \textbf{(4)}, l'accord est plutôt faible au début de la méthode (inférieur à $45$\% avant l'itération $15$), puis devient de plus en plus fort (dépassant les $60$\%) pour finalement atteider les $100$\% vers l'itération $45$.
			\begin{figure}[!htb]
				\centering
				\includegraphics[width=0.95\textwidth]{figures/etude-rentabilite-accord-annotation}
				\caption{
					Évolution au cours des itérations de l'accord entre l'annotation de contraintes d'un expert et le résultat de \textit{clustering} sur lequel est basé l'échantillonnage de contraintes.
					Ces accords sont exprimés grâce à des lots de $50$ contraintes annotées.
					Les évolutions moyennes de différents paramétrages de la méthode sont exposées :
					\textbf{(1)} meilleur paramétrage moyen pour atteindre une annotation partielle ;
					\textbf{(2)} meilleur paramétrage moyen pour atteindre une annotation suffisante ;
					\textbf{(3)} meilleur paramétrage moyen pour atteindre une annotation exhaustive ;
					et \textbf{(4)} paramétrage favori.
					À titre d'information, les courbes en noir représentent l'évolution de la \texttt{v-measure} entre le \textit{clustering} et la vérité terrain.
				} 
				\label{figure:4.5.1-ETUDE-RENTABILITE-ACCORD-ANNOTATION-CLUSTERING}
			\end{figure}
			
			% Tableau : corrélation modérée.
			La \textsc{Table~\ref{table:4.5.1-ETUDE-RENTABILITE-CORRELATION-ACCORD-PERFORMANCE}} contient le score de corrélation entre cet accord et la performance théoriques obtenue grâce à la vérité terrain.
			Cette corrélation est modérée : $0.49$ sur l'ensemble des tentatives, $0.69$ sur les tentatives utilisant notre paramétrage favori.
			\begin{table}[!htb]
				\begin{center}
				\begin{tabular}{|c|r|}
				
					\hline
					% ENTETE DU TABLEAU
					\multicolumn{1}{|c|}{\shortstack[c]{
						Paramétrage
					}}
						& \multicolumn{1}{c|}{\shortstack[c]{
							Corrélation \texttt{r}
						}}
						\tabularnewline
						\hline
					
					% Annotation partielle.
					Meilleur paramétrage moyen pour une annotation partielle \textbf{(1)}
						& $0.92$
						\tabularnewline
						\hline
					
					% Annotation suffisante.
					Meilleur paramétrage moyen pour une annotation suffisante \textbf{(2)}
						& $0.74$
						\tabularnewline
						\hline
					
					% Annotation exhaustive.
					Meilleur paramétrage moyen pour une annotation exhaustive \textbf{(3)}
						& $0.57$
						\tabularnewline
						\hline
					
					% Paramétrage favori
					Paramétrage favori \textbf{(4)}
						& $0.69$
						\tabularnewline
						\hline
					
					% Moyenne des $960$ tentatives.
					Moyenne des $960$ tentatives
						& $0.49$
						\tabularnewline
						\hline
					
				\end{tabular}
				\end{center}
				\caption{
					Score de corrélation \texttt{r} de \textit{Pearson} entre la performance du \textit{clustering} obtenu à l'aide d'une vérité terrain (\texttt{v-measure}) et le score d'accord entre annotation et \textit{clustering}.
				}
				\label{table:4.5.1-ETUDE-RENTABILITE-CORRELATION-ACCORD-PERFORMANCE}
			\end{table}
			
			% Description de la figure : croissance instable.
			Cependant, la tendance constatée est aussi saccadée par de nombreux pics pouvant faire perdre ou gagner jusqu'à $40$\% d'accord entre deux itérations.
			Des chutes d'accord peuvent intervenir à des itérations où la similarité du \textit{clustering} avec la vérité terrain est pourtant forte, comme c'est le cas autour des itérations $29$ et $36$ où l'accord chute de plus de $25$\% alors que la \texttt{v-measure} avec la vérité terrain est constamment au dessus de de $95$\%.
			
			% Description de la figure : autres paramétrages
			Les autres paramétrages représentés dans \textbf{(1)}, \textbf{(2)} et \textbf{(3)} comportent des tendances similaires (corrélation forte mais variations soudaines d'accord, chute d'accords malgré des \textit{clustering} aux performances élevées, ...).

		%%% Discussion
		\subsubsection{Discussion}
		
			% Rappel de l'objectif : trouver un cas d'arrêt en regardant l'accord entre l'annotation et le clustering.
			Dans cette étude, nous avons analysé l'évolution de l'accord entre les annotations et le partitionnement de données proposé par un \textit{clustering} dans l'espoir de définir un cas d'arrêt de notre méthodologie d'annotation qui soit indépendant d'une vérité terrain pré-établie.
			Cependant, en considérant les résultats obtenus, ce score d'accord ne semble pas répondre à cette objectif.
			
			% Trop instable pour définir un cas d'arrêt.
			Tout d'abord, malgré une corrélation acceptable avec la performance théorique du \textit{clustering} (moyenne à $0.49$, voir \textsc{Table~\ref{table:4.5.1-ETUDE-RENTABILITE-CORRELATION-ACCORD-PERFORMANCE}}), l'évolution du score d'accord reste instable.
			En effet, les nombreuses variations et saccades rendent toute analyse de rentabilité difficile voire impossible, ce qui ne permet pas de définir un cas d'arrêt pour notre méthode d'annotation.
			
			\begin{leftBarExamples}
				Concernant l'évolution du paramétrage favori (\textsc{Figure~\ref{figure:4.5.1-ETUDE-RENTABILITE-ACCORD-ANNOTATION-CLUSTERING}} \textbf{(4)}), nous ne pouvons pas précisément définir à partir de quelle itération les résultats semblent intéressant car le score d'accord oscille longuement entre $50$\% et $100$\% avec des pics de plus de $25$\% entre deux itérations. 
			\end{leftBarExamples}
			
			\begin{leftBarAuthorOpinion}
				% Rappel: l'objectif de notre méthode d'annotation est de corriger le plus rapidement un clustering.
				Après réflexion, ce score d'accord est probablement infructueux à cause du fonctionnement même de notre méthode, dont l'objectif est de corriger le partitionnement des données en utilisant un minimum de contraintes.
				En effet, dans le cadre de l'optimisation des paramètres réalisée en \textsc{Section~\ref{section:4.2-HYPOTHESE-EFFICIENCE}}, nous avons retenu dans notre paramétrage favori la sélection des contraintes les plus proches entre deux \textit{clusters} différents (\texttt{samp.closest.diff}) : cette sélection permet ainsi de décrire efficacement l'emplacement des frontières de \textit{clusters}.
				
				% Cet échantillonnage est non supervisé : il y a de nombreuses saccades.
				Or, cet échantillonnage reste une méthode non-supervisée : aux premières itérations, les contraintes sélectionnées ont de bonnes chances de mettre en avant une frontière mal positionnée, mais au fur et à mesure que des contraintes s'ajoutent, les nouvelles contraintes ont moins de chances de trouver des bordures de \textit{clusters} qui ne soient pas encore caractérisées.
				De ce fait, il se peut que les dernières sélections n'identifient aucune nouvelle frontière, qu'elles se concentrent sur des frontières déjà bien positionnées ou déjà décrites par d'autres contraintes, ou qu'elles nécessitent plusieurs itérations pour caractériser des frontières complexes (le comportement des autres méthodes de sélections représentées en \textsc{Figure~\ref{figure:4.5.1-ETUDE-RENTABILITE-ACCORD-ANNOTATION-CLUSTERING}} peut être illustré par des raisonnements similaires).
				L'ensemble de ces cas de figures peut ainsi expliquer les nombreuses saccades dans l'évolution du score d'accord : tantôt la sélection semble pertinente, tantôt la sélection semble inutile.
			\end{leftBarAuthorOpinion}
			
			% Trop instable pour caratériser une itération.
			Pour aller plus loin, nous pouvons aussi critiquer le score de corrélation qui ne semble pas montrer de lien fort entre les performances théoriques et les accords calculés, tant sur l'ensemble des tentatives que pour le paramétrage favori.
			Il est même rare d'observer des chutes importantes d'accords qui soient accompagnées d'une variation significative de \texttt{v-measure} avec la vérité terrain.
			Au final, ce score d'accord n'est donc pas vraiment représentatif de la rentabilité d'une itération ou de l'évolution de la pertinence du \textit{clustering}.
			\begin{leftBarAuthorOpinion}
				Pour expliquer cette absence de corrélation, il est possible que l'analyse des annotations réalisées ait été une idée infructueuse : les $50$ contraintes annotées peuvent peut-être exprimer un désaccord avec le précédent \textit{clustering}, mais ce n'est pas pour autant que l'ajout de ces nouvelles contraintes impacte significativement la pertinence globale du partitionnement des données.
			\end{leftBarAuthorOpinion}
			
			% Conclusions et suggestion.
			En conclusion, \textbf{le score d'accord entre l'annotation courante et le \textit{clustering} précédent n'est pas adéquat pour estimer un cas d'arrêt de notre méthode d'annotation}, principalement car il est trop instable et qu'il ne représente pas bien les bénéfices obtenus à chaque itération.
			Ainsi, si une analyse de l'annotation réalisée n'est pas fructueuse, nous nous tournerons vers l'analyse plus abstraite des différences entre deux résultats de \textit{clustering}
	
	%%%
	%%% Subsection 4.5.2: Étude de l'évolution de la différence entre deux \textit{clusterings} consécutifs.
	%%%
	\subsection{Étude de l'évolution de la différence entre deux \textit{clusterings} consécutifs}
	\label{section:4.5.2-ETUDE-RENTABILITE-SIMILARITE-CLUSTERING}
		
		% Objectif de l'expérience.
		Nous venons de conclure que l'analyse de l'accord entre l'annotation et le partitionnement des données ne permet pas d'estimer la rentabilité d'une itération de notre méthode d'annotation.
		Parmi les explications possibles, nous avons mis en cause l'analyse du lot de contraintes annotées : en effet, ce n'est pas parce que l'annotation de contraintes est en désaccord avec le précédent partitionnement des données que les correctifs associés auront un impact significatif sur le prochain partitionnement.
		Ainsi, nous voulons analyser l'évolution de la différence entre deux \textit{clusterings} successifs : en effet, si une itération apporte des correctifs ayant un impact, alors il devrait y avoir des différences visibles entre les deux itérations de \textit{clustering}.
	
		%%% Protocole expérimental.
		\subsubsection{Protocole expérimental}
			
			% Axiome.
			\begin{leftBarWarning}
				Dans le cadre de cette étude, nous supposons que l'expert métier connaît parfaitement le domaine traité dans ce jeu de données, et qu'il est capable de caractériser sans ambiguïté la similitude entre deux données issues de cet ensemble.
			\end{leftBarWarning}
			
			% Pseudo-code.
			Pour résumer le protocole expérimental que nous décrivons ci-dessous, vous pouvez vous référer au pseudo-code décrit dans \textsc{Algorithme~\ref{algorithm:4.5.1-ETUDE-RENTABILITE-SIMILARITE-CLUSTERING-PROTOCOLE}}.
			
			\begin{algorithm}
				\KwData{jeu de données annotés (vérité terrain)}
				%
				\ForEach{jeux de données à tester}{
					\textbf{initialisation (données)}: récupérer les données et la vérité terrain \;
					\textbf{initialisation (contraintes)}: créer une liste vide de contraintes \;
					\textbf{prétraitement}: supprimer le bruit dans les données avec \texttt{prep.simple} \;
					\textbf{vectorisation}: transformer les données en vecteurs avec \texttt{vect.tfidf} \;
					\textbf{clustering initial}: regrouper les données par similarité avec \texttt{clust.kmeans.cop} \;
					\Repeat{annotation de toutes les contraintes possibles}{
						\textbf{échantillonnage}: sélectionner des contraintes avec \texttt{samp.closest.diff} \;
						\textbf{simulation d'annotation}: caractériser les contraintes grâce à la vérité terrain \;
						\textbf{intégration}: ajouter les nouvelles contraintes au gestionnaire de contraintes \;
						\textbf{clustering}: regrouper les données par similarité avec \texttt{clust.kmeans.cop} \;
						\textbf{rentabilité}: calculer la différence entre les deux précédents \textit{clusterings} \;
					}
				}
				\textbf{analyse 1}: afficher l'évolution de la différence entre deux \textit{clustering} consécutifs \;
				\textbf{analyse 2}: calculer la corrélation entre le score de différence et le score de performance \;
				%
				\KwResult{discussion sur la rentabilité d'après la différence entre \textit{clusterings}}
				%
				\caption{\textit{
					Description en pseudo-code du protocole expérimental de l'étude de l'évolution de la différence entre deux \textit{clustering} consécutifs.
				}}
				\label{algorithm:4.5.1-ETUDE-RENTABILITE-SIMILARITE-CLUSTERING-PROTOCOLE}
			\end{algorithm}

			% Détails de l'expérience.
			Nous nous appuyons sur le même protocole que l'expérience précédente (cf. \textsc{Section~\ref{section:4.5.1-ETUDE-RENTABILITE-ACCORD-ANNOTATION-CLUSTERING}}) : nous utilisons comme vérité terrain le jeu de données \texttt{Bank Cards (v1.0.0)}, nous réalisons $5$ tentatives complètes de la méthode du \textit{clustering} interactif en utilisant notre paramétrage favori, et nous simulons l'annotation par un expert d'un lot de $50$ contraintes à chaque itération.
			
			% Ajout de la comparaison entre clustering.
			Cependant, au lieu de calculer un score d'accord entre annotation et \textit{clustering}, nous estimons la différence entre le \textit{clustering} précédent et le \textit{clustering} obtenu grâce aux dernières annotations.
			Cette différence entre deux \textit{clustering} $X$ et $Y$ est obtenue par la formule $1-\texttt{v-measure}(X,Y)$ où la \texttt{v-measure} caractérise la ressemblance entre deux partitionnements des données (\cite{rosenberg-hirschberg:2007:vmeasure-conditional-entropybased}).
			Pour nous permettre de discuter de l'utilité de ce score pour prédire la stabilisation du \textit{clustering} et ainsi définir un cas d'arrêt de notre méthodologie d'annotation, nous calculons aussi le score de corrélation entre cette différence et la performance obtenue à l'aide d'une vérité terrain (la corrélation \texttt{r} de \textit{Pearson} (\cite{kirch:2008:pearson-correlation-coefficient}) est utilisée).
			
			\begin{leftBarIdea}
				Comme précédemment, nous concentrons l'étude sur notre paramétrage favori (voir \textsc{Section~\ref{section:4.4.3-ETUDE-PERTINENCE-RESUME-AUTOMATIQUE}}).
				Cependant, afin de compléter notre discussion avec d'autres points de comparaison, nous analysons aussi les autres paramétrages implémentés, notamment les meilleurs paramétrages moyens identifiés lors de l'hypothèse d'efficience (voir \textsc{Section~\ref{section:4.2-HYPOTHESE-EFFICIENCE}}).
			\end{leftBarIdea}
			
			% Référence scripts.
			\begin{leftBarInformation}
				Les scripts de l'expérience, réalisés avec des \textit{notebooks} Python (\cite{van-rossum-drake:2009:python-reference-manual}), sont disponibles dans un dossier dédié de~\cite{schild:2021:cognitivefactory-interactiveclusteringcomparativestudy}.
			\end{leftBarInformation}

		%%% Résultats
		\subsubsection{Résultats obtenus}
			
			% Figure : décroissance générale.
			La \textsc{Figure~\ref{figure:4.5.2-ETUDE-RENTABILITE-SIMILARITE-CLUSTERING}} représente l'évolution moyenne du score de différence entre deux \textit{clusterings} pour les quatre paramétrages mis en avant lors de nos études.
			Nous pouvons constater une tendance générale à la décroissance vers $0$\% de ce score de différence : pour le paramétrage favori, la différence moyenne entre deux \textit{clustering} est initialement comprise entre $25$\% et $35$\% jusqu'à l'itération $10$, elle chute ensuite pour être inférieure à $5$\% après l'itération $20$, et elle termine enfin en oscillant très légèrement ($\pm1$\%) autour de $0$\% jusqu'à la fin des annotations.
			\begin{figure}[!htb]
				\centering
				\includegraphics[width=0.95\textwidth]{figures/etude-rentabilite-similarite-clustering}
				\caption{
					Évolution de la différence de résultats entre deux itérations de \textit{clustering}.
					Les évolutions moyennes de différents paramétrages de la méthode sont exposées :
					\textbf{(1)} meilleur paramétrage moyen pour atteindre une annotation partielle ;
					\textbf{(2)} meilleur paramétrage moyen pour atteindre une annotation suffisante ;
					\textbf{(3)} meilleur paramétrage moyen pour atteindre une annotation exhaustive ;
					et \textbf{(4)} paramétrage favori.
					À titre d'information, les courbes en noir représentent l'évolution de la \texttt{v-measure} entre le \textit{clustering} et la vérité terrain.
				}
				\label{figure:4.5.2-ETUDE-RENTABILITE-SIMILARITE-CLUSTERING}
			\end{figure}
			
			% Tableau : corrélation forte.
			La \textsc{Table~\ref{table:4.5.2-ETUDE-RENTABILITE-CORRELATION-SIMILARITE-PERFORMANCE}} contient le score de corrélation entre cette différence et la performance théoriques obtenue grâce à la vérité terrain.
			Cette corrélation est forte : $0.75$ sur l'ensemble des tentatives, $0.93$ sur les tentatives utilisant notre paramétrage favori.
			La \textsc{Figure~\ref{figure:4.5.2-ETUDE-RENTABILITE-SIMILARITE-CLUSTERING}} confirme cette corrélation :
			\begin{itemize}
				\item un score de \texttt{v-measure} avec la vérité terrain proche de $100$\% est accompagné d'un score de différence proche de $0$\% (après l'itération $20$ pour \textbf{(1)}, après l'itération $20$ pour \textbf{(2)}, après l'itération $30$ pour \textbf{(3)} et après l'itération $22$ pour \textbf{(4)}) ;
				\item une croissance de performance est généralement accompagnée d'un score non nul de différence (voir \textbf{(2)} et \textbf{(4)} entre les itérations $0$ et $20$), et plusieurs pics de performance sont accompagnés de scores forts de différence (particulièrement visible sur \textbf{(1)} vers l'itération $5$ et entre les itérations $10$ et $15$) ;
				\item il est toutefois à noter que l'inverse n'est pas vrai : un score non nul de différence n'accompagne pas forcément une croissance de performance, mais peut simplement caractériser un changement de partitionnement, comme c'est le cas dans \textbf{(3)} entre les itérations $0$ et $10$ où des modifications ont lieu (score de différence non nul) mais où la performance par rapport à la vérité terrain stagne.
			\end{itemize}
			\begin{table}[!htb]
				\begin{center}
				\begin{tabular}{|c|r|}
				
					\hline
					% ENTETE DU TABLEAU
					\multicolumn{1}{|c|}{\shortstack[c]{
						Paramétrage
					}}
						& \multicolumn{1}{c|}{\shortstack[c]{
							Corrélation
						}}
						\tabularnewline
						\hline
					
					% Annotation partielle.
					Meilleur paramétrage moyen pour une annotation partielle \textbf{(1)}
						& $0.96$
						\tabularnewline
						\hline
					
					% Annotation suffisante.
					Meilleur paramétrage moyen pour une annotation suffisante \textbf{(2)}
						& $0.92$
						\tabularnewline
						\hline
					
					% Annotation exhaustive.
					Meilleur paramétrage moyen pour une annotation exhaustive \textbf{(3)}
						& $0.85$
						\tabularnewline
						\hline
					
					% Paramétrage favori
					Paramétrage favori \textbf{(4)}
						& $0.93$
						\tabularnewline
						\hline
					
					% Moyenne des $960$ tentatives.
					Moyenne des $960$ tentatives
						& $0.75$
						\tabularnewline
						\hline
					
				\end{tabular}
				\end{center}
				\caption{
					Score de corrélation \texttt{r} de \textit{Pearson} entre la performance du \textit{clustering} obtenu à l'aide d'une vérité terrain (\texttt{v-measure}) et le score de différence entre deux \textit{clusterings} consécutifs.
				}
				\label{table:4.5.2-ETUDE-RENTABILITE-CORRELATION-SIMILARITE-PERFORMANCE}
			\end{table}
			
			% Description de la figure : autres paramétrages
			Les autres paramétrages représentés dans \textbf{(1)}, \textbf{(2)} et \textbf{(3)} comportent des tendances similaires (décroissance générale, forte corrélation avec la performance théorique) à quelques détails (\textbf{(1)} commence avec des scores de différence très forts avant décroître avec de nombreux pics ; \textbf{(3)} croît légèrement avant d'entamer sa décroissance, ...).
			
		%%% Discussion
		\subsubsection{Discussion}
		
			% Rappel de l'objectif : trouver un cas d'arrêt en regardant l'évolution de la différence entre deux clusterings.
			Dans cette étude, nous avons analysé l'évolution du score de différence entre deux itérations de \textit{clustering} dans l'espoir de définir un cas d'arrêt de notre méthodologie d'annotation qui soit indépendant d'une vérité terrain pré-établie.
			
			% Avantage 1 : Caractérise la rentabilité.
			Tout d'abord, nous pouvons affirmer qu'il y une forte corrélation entre l'évolution de ce score de différence et l'évolution du score de performance (voir \textsc{Table~\ref{table:4.5.2-ETUDE-RENTABILITE-CORRELATION-SIMILARITE-PERFORMANCE}} : \texttt{r} moyen de $0.75$ ; \texttt{r} supérieur à $0.85$ pour les paramétrages mis en avant).
			Cette corrélation est confirmée visuellement grâce à la \textsc{Figure~\ref{figure:4.5.2-ETUDE-RENTABILITE-SIMILARITE-CLUSTERING}} : plus les différences entre \textit{clusterings} sont faibles, plus les performances des \textit{clusterings} sont fortes.
			
			% Attention : Peut ne caractériser qu'un gros changement sans pour autant une amélioration.
			Un point d'attention est toutefois à retenir : une modification du partitionnement des données n'entraîne pas forcément un gain de performance (voir \textbf{(3)} entre les itérations $0$ et $10$ et \textbf{(4)} entre les itérations $0$ et $8$).
			Nous ne pouvons donc pas conclure que l'analyse de la différence entre eux itération de \textit{clustering} permet de caractériser totalement la rentabilité d'une itération.
			
			% Avantage 2 : Permet de définir un cas d'arrêts.
			Cependant, nous pouvons tout de même nous servir de ce score pour définir un cas d'arrêt pour notre méthodologie d'annotation lorsque la différence entre deux \textit{clusterings} est faible.
			Pour cela, il nous suffit de fixer un seuil bas du score de différence en dessous duquel il n'est plus rentable de faire de nouvelles itérations de la méthode car les performances devraient être suffisantes.
			Une analyse manuelle ou semi-manuelle (voir hypothèse de pertinence en \textsc{Section~\ref{section:4.4-HYPOTHESE-PERTINENCE}}) reste nécessaire pour confirmer la valeur métier du résultat obtenu.
			
			\begin{leftBarIdea}
				Si nous restons sur notre seuil théorique de $90$\% de \texttt{v-measure} (voir \textsc{Section~\ref{section:4.2-HYPOTHESE-EFFICIENCE}}) et que nous nous basons sur la \textsc{Figure~\ref{figure:4.5.2-ETUDE-RENTABILITE-SIMILARITE-CLUSTERING}} \textbf{(4)}, nous pouvons visuellement fixer ce seuil autour de $5$\% de différences.
				Le réglage fin de ce seuil pourra être le sujet de futures analyses complémentaires.
			\end{leftBarIdea}
			
			% Conclusions et suggestion.
			En conclusion, \textbf{le score de différences entre deux résultats de \textit{clustering} semble être un bon indicateur pour estimer un cas d'arrêt de notre méthodologie d'annotation}, et nous proposer d'utiliser un seuil de $5$\% pour implémenter ce cas d'arrêt.
			
	%%%
	%%% Subsection 4.5.3: Mise en commun des stratégies d'évaluation de la rentabilité d'une itération de la méthode et définition d'un cas d'arrêt indépendant d'une vérité terrain.
	%%%
	\subsection{Mise en commun des stratégies d'évaluation de la rentabilité d'une itération de la méthode et définition d'un cas d'arrêt indépendant d'une vérité terrain.}
	\label{section:4.5.3-ETUDE-RENTABILITE-MISE-EN-COMMUN}
			
		% Conclusion.
		\begin{leftBarSummary}
			Au cours de cette étude de rentabilité, nous avons pu voir que :
			\begin{itemize}
				\item[\itemko] l'analyse du score d'accord entre l'annotation courante et le \textit{clustering} précédent ne permet pas d'estimer la rentabilité d'une itération, ni de définir un cas d'arrêt de notre méthodologie d'annotation (cf. \textsc{Section~\ref{section:4.5.1-ETUDE-RENTABILITE-ACCORD-ANNOTATION-CLUSTERING}}) ;
				\item[\itemok] l'analyse des différences entre deux itérations de \textit{clusterings} est une approche prometteuse pour estimer la rentabilité d'une itération, bien qu'une modification significative entre deux résultats de \textit{clustering} n'implique pas forcément un gain de performance (les deux \textit{clustering} peuvent avoir une \texttt{v-measure} équivalente avec la vérité terrain) ;
				\item[\itemok] l'usage de différences entre deux itérations de \textit{clusterings} permet de définir un cas d'arrêt de de notre méthodologie d'annotation : si les différences sont faibles (par exemple : inférieures à $5$\%), alors les performances stagnent ou plafonnent, donc il peut être intéressant d'interrompre le \textit{clustering} interactif après avoir vérifier la manuellement pertinence des résultats obtenus (cf. (cf. \textsc{Section~\ref{section:4.5.2-ETUDE-RENTABILITE-SIMILARITE-CLUSTERING}}) et \textsc{Section~\ref{section:4.4.4-ETUDE-PERTINENCE-MISE-EN-COMMUN}}).
			\end{itemize}
		\end{leftBarSummary}
		
		% Transition: Vers Simulation d'erreurs.
		Pour terminer nos différentes analyses, il convient maintenant d'anticiper la présence d'erreurs d’annotation.
		En effet, nous avons fait jusqu'à présent l'hypothèse que l'annotateur ne se trompe jamais, mais cette hypothèse forte n'est pas toujours vérifiée en pratique.
		Pour estimer l'impact de ces erreurs ou incohérences d'annotation, nous devons donc réaliser une analyse de robustesse de notre méthode d'annotation : celle-ci sera réalisé en \textsc{Section~\ref{section:4.6-HYPOTHESE-ROBUSTESSE}}.
	
	
	%%%%%--------------------------------------------------------------------
	%%%%% Section 4.6: Hypothèse de robustesse.
	%%%%%--------------------------------------------------------------------
	\newpage
	\section{Évaluation de l'hypothèse de robustesse}
\label{section:4.6-HYPOTHESE-ROBUSTESSE}

	%%% Introduction / Transition.
	Dans les précédentes études, nous avons presque toujours analysé le \textit{clustering} interactif en supposant que l'annotateur connaît parfaitement le domaine traité par le jeu de données et qu'il est capable de caractériser sans ambiguïté la similitude entre deux données issues de cet ensemble.
	Bien entendu, cette hypothèse forte n'est pas toujours vérifiée en situation réelle : l'interprétation du langage peut contenir certaines ambiguïtés, l'opérateur peut faire des erreurs d’inattention, et deux annotateurs peuvent avoir des avis contraire sur un même sujet.
	Or, comme notre méthode d'annotation est itérative, elle est a priori sensible aux dérives fonctionnement liées à ce type de contradictions.
	Dans cette section, nous nous intéresserons donc à la robustesse du \textit{clustering} interactif en présence d'incohérences dans les contraintes et aux moyens de les contrer.
	Pour cela, nous aimerions donc vérifier l'hypothèse suivante :
	
	%%% Formulation des hypothèses:
	\begin{tcolorbox}[
		title=\faVial~\textbf{Hypothèse de robustesse}~\faVial,
		colback=colorTcolorboxHypothesis!15,
		colframe=colorTcolorboxHypothesis!75,
		width=\linewidth
	]
		% Hypothèse.
		« \textbf{
			Au cours d'une méthodologie d'annotation basée sur le \textit{clustering} interactif, il est possible d'estimer le taux d'incohérences dans les contraintes ainsi que leur impact sur les performances de la méthode.
		} » \\
		
		% Figure.
		La \textsc{Figure~\ref{figure:4.6-HYPOTHESE-ROBUSTESSE}} illustre cette hypothèse et l'espoir de estimer l'impact différences ou d'erreurs d'annotations sur le nombre d'itérations de la méthode.
		%
		\begin{figure}[H]  % keep [H] to be in the tcolorbox.
			\centering
			\includegraphics[width=0.95\textwidth]{figures/hypotheses-06-robustesse}
			\caption{
				Illustration des études réalisées sur le \textit{clustering} interactif (\textit{étape 6/6}) en schématisant l'évolution de la pertinence (\textit{valeur métier évaluée par l'expert et exprimé en nombre de clusters}) d'une base d'apprentissage en cours de construction en fonction du coût temporel de la méthode (\textit{temps nécessaire à l'expert métier et à la machine}), ainsi que les marges d'erreurs représentant l'impact de différences d'annotation sur le nombre d'itérations nécessaire à la méthode.
			}
			\label{figure:4.6-HYPOTHESE-ROBUSTESSE}
		\end{figure}
	\end{tcolorbox}
		
	% Résumé de l'étude.
	Afin de vérifier cette hypothèse, nous organisons trois expériences :
	\begin{itemize}
		\item une étude de cas sur la \textbf{correction des incohérences d'annotation} (cf. \textsc{Section~\ref{section:4.6.1-ETUDE-ROBUSTESSE-INTERETS-CORRECTION-ERREURS}}) ;
		\item une simulation de l'\textbf{impact des incohérences d'annotation} sur les performances de la méthode (cf. \textsc{Section~\ref{section:4.6.2-ETUDE-ROBUSTESSE-SIMULATION-IMPACT-ERREURS}}) ;
		\item une étude de cas sur le \textbf{score inter-annotateurs} obtenu lors d'une annotation de contraintes en situation réelle avec plusieurs opérateurs (cf. \textsc{Section~\ref{section:4.6.3-ETUDE-ROBUSTESSE-SCORE-INTER-ANNOTATEURS}}).
	\end{itemize}
	
	
	%%%
	%%% Subsection 4.6.1: Étude de l'intérêt de la correction des incohérences d'annotation.
	%%%
	\subsection{Étude de l'intérêt de la correction des incohérences d'annotation}
	\label{section:4.6.1-ETUDE-ROBUSTESSE-INTERETS-CORRECTION-ERREURS}
		
		% Objectif de l'expérience.
		Nous cherchons à estimer la robustesse du \textit{clustering} interactif face aux incohérences d'annotation.
		Dans cette étude, nous nous intéressons plus particulièrement à l'intérêt de la détection et de la correction des conflits présents dans les contraintes.
		En effet, deux approches de travail s'opposent :
		\begin{itemize}
			\item une approche naïve ignorant simplement les conflits : celle-ci n'engendre pas de coûts supplémentaires, mais elle s'expose aux risques de dérives de fonctionnement ;
			\item une seconde approche avec correction des conflits : celle-ci nécessite de revoir ou de ré-annoter des contraintes, impliquant un coût supplémentaire, mais permet de limiter l'impact de dérives potentielles.
		\end{itemize}
		Pour trancher entre ces deux options, nous simulons ces deux approches afin d'estimer si l'absence de correction induit une régression significative des performances de notre méthode d'annotation.
	
		%%% Protocole expérimental.
		\subsubsection{Protocole expérimental}
			
			% Pseudo-code.
			Pour résumer le protocole expérimental que nous décrivons ci-dessous, vous pouvez vous référer au pseudo-code décrit dans \textsc{Algorithme~\ref{algorithm:4.6.1-ETUDE-ROBUSTESSE-INTERETS-CORRECTION-ERREURS-PROTOCOLE}}.
			\todo{FORMAT: ajuster l’enchaînement pour ne pas avoir de page "blanche" avec l'algorithme.}
			
			\begin{algorithm}
				\KwData{jeu de données annoté (vérité terrain)}
				\KwIn{taux d'erreurs à tester}
				%
				\ForEach{arrangement de taux d'erreur à tester}{
					\textbf{initialisation (données)}: récupérer les données et la vérité terrain \;
					\textbf{initialisation (contraintes)}: créer une liste vide de contraintes \;
					\textbf{prétraitement}: supprimer le bruit dans les données avec \texttt{prep.simple} \;
					\textbf{vectorisation}: transformer les données en vecteurs avec \texttt{vect.tfidf} \;
					\textbf{clustering initial}: regrouper les données par similarité avec \texttt{clust.kmeans.cop} \;
					\textbf{évaluation}: estimer l'équivalence entre le \textit{clustering} obtenu et la vérité terrain \;
					\Repeat{annotation de toutes les contraintes possibles}{
						\textbf{échantillonnage}: sélectionner des contraintes avec \texttt{samp.closest.diff} \;
						\textbf{échantillonnage d'erreurs}: définir les contraintes qui seront erronées \;
						\textbf{simulation d'annotation}: caractériser les contraintes grâce à la vérité terrain \;
						\If{absence de correction des conflits d'annotation}{
							\textbf{intégration naïve}: ignorer les conflits avec le gestionnaire de contraintes \;
						}
						\ElseIf{détection et correction des conflits d'annotation}{
							\textbf{intégration corrective}: changer les annotations erronées en conflit \;
						}
						\textbf{clustering}: regrouper les données par similarité avec \texttt{clust.kmeans.cop} \;
						\textbf{évaluation}: estimer l'équivalence entre le \textit{clustering} obtenu et la vérité terrain \;
					}
				}
				\textbf{analyse}: déterminer l'impact par itération et par taux d'erreurs de la correction \;
				%
				\KwResult{discussion sur l'impact de la correction des incohérences}
				%
				\caption{\textit{
					Description en pseudo-code du protocole expérimental de l'étude d'intérêt de la correction des incohérences d'annotation.
				}}
				\label{algorithm:4.6.1-ETUDE-ROBUSTESSE-INTERETS-CORRECTION-ERREURS-PROTOCOLE}
			\end{algorithm}
			
			% Description de la vérité terrain.
			Nous utilisons comme vérité terrain le jeu de données \texttt{Bank Cards (v1.0.0)} : ce dernier traite des demandes les plus fréquentes des clients en ce qui concerne la gestion de leur carte bancaire.
			Il est composé de $500$ questions rédigées en français et réparties en $10$ classes (\texttt{perte ou vol de carte}, \texttt{carte avalée}, \texttt{commande de carte}, ...).
			Pour plus de détails, consultez l'annexe~\ref{annex:C.1-DATASET-BANK-CARDS}.
			
			% Description des tentatives de la méthode avec simulation d'erreurs.
			Sur ce jeu de données, nous exécutons une tentative complète
			\footnote{Tentative complète : itérations d'échantillonnage, d'annotation et de \textit{clustering} jusqu'à annotation de toutes les contraintes possibles.}
			de la méthode du \textit{clustering} interactif en utilisant notre paramétrage favori (voir \textsc{Section~\ref{section:4.4.3-ETUDE-PERTINENCE-RESUME-AUTOMATIQUE}}).
			Toutefois, contrairement aux précédents expériences, nous allons ajouter un pourcentage de contraintes erronées à chaque itération :
			\begin{itemize}
				\item Le taux d'erreurs insérées, variant de $0$\% à $50$\% par pas de $5$\%, reste fixe tout au long d'une même tentative de notre méthode : nous pouvons ainsi analyser l'impact d'un taux d'erreur fixe sur les performances au courant des itérations ;
				\item Les contraintes erronées à insérer sont tirées aléatoirement parmi le lot de contraintes qui aurait été échantillonné au cours d'une tentative sans erreur : ainsi, nous pouvons comparer itération par itération toutes ces simulations car elles partagent la même base de contraintes (aux valeurs de \texttt{MUST-LINK} et \texttt{CANNOT-LINK} près) ;
			\end{itemize}
			
			% Description des tentatives de la méthode avec gestion des conflits.
			Puisque nous introduisons des erreurs d'annotations, des conflits vont apparaître dans le gestionnaire de contraintes.
			Pour rappel, un conflit est détecté dans le cas où l'ajout d'une nouvelle contrainte annotée contredit ce qui a été précédemment déduit grâce aux propriétés de transitivité des contraintes de types \texttt{MUST-LINK}et \texttt{CANNOT-LINK} (voir \textsc{Figure~\ref{figure:3.3-CONTRAINTES-TRANSITIVITE}} en \textsc{Section~\ref{section:3.3.2-GESTION-DES-CONTRAINTES}}).
			Pour les traiter, nous testons deux approches :
			\begin{itemize}
				\item l'approche naïve ignorant simplement les conflits : si la prochaine contrainte à ajouter est incompatible avec la base de contraintes déjà intégrées au gestionnaire, alors nous ignorons simplement son existence sans remettre en question les précédentes annotations ;
				\item l'approche avec correction des conflits : pour simuler la correction d'un expert, nous recréons à chaque itération le gestionnaire de contraintes en intégrant d'abord les contraintes correctes puis les contraintes erronées ; ainsi, les conflits ne peuvent arrivent qu'à l'ajout d'une contrainte erronée, et il suffit d'ajouter sa version exacte pour simuler la correction de l'expert.
			\end{itemize}
			
			% Description des tentatives de la méthode avec les répétitions.
			Ainsi, il y a donc $11$ taux d'erreurs différents à simuler, chacun suivant $2$ approches de gestion de conflits différentes, et chacune de ces simulations d'erreurs seront répétées $10$ fois sur chaque tentative complète de la méthode pour limiter contrer les aléas statistiques des tirages de contraintes erronées, ce qui représente $220$ simulation par tentatives.
			Enfin, chaque tentative complète de \textit{clustering} interactif est répétée $5$ fois pour contrer les aléas statistiques des exécutions, ce qui représente un total de $1100$ tentatives complètes à simuler.

			% Description de l'analyse.
			Enfin, nous afficherons l'évolution de la performance moyenne du \textit{clustering} obtenu en fonction des divers taux d'erreurs simulés, et discuterons de l'impact au cours des itérations de la présence ou de l'absence de corrections des conflits d'annotations détectés.
			
			% Référence scripts.
			\begin{leftBarInformation}
				Les scripts de l'expérience, réalisés avec des \textit{notebooks} Python (\cite{van-rossum-drake:2009:python-reference-manual}), sont disponibles dans un dossier dédié de~\cite{schild:2021:cognitivefactory-interactiveclusteringcomparativestudy}.
			\end{leftBarInformation}

		%%% Résultats
		\subsubsection{Résultats obtenus}
		
			% Description statistiques.
			La \textsc{Figure~\ref{figure:4.6.1-ETUDE-ROBUSTESSE-INTERETS-CORRECTION-ERREURS}} représente l'évolution moyenne de la \texttt{v-measure} du \textit{clustering} en fonction du nombre d'itération de la méthode, déclinée avec les $11$ taux d'erreurs simulés et les $2$ approches de gestion des conflits.
			Les contraintes utilisées sont basées sur les échantillonnages réalisées au cours des tentatives sans erreurs : comme les mêmes contraintes sont donc utilisées (aux valeurs d'annotations près), toutes les courbes sont comparables point par point.
			
			\begin{leftBarWarning}
				Toutefois, il est important de noter que les tentatives sans contraintes ont besoin de maximum $3~000$ contraintes pour annoter toutes les contraintes possibles et leurs transitivités (moyenne: $2~488$, écart-type: $327$).
				Ainsi, toutes les courbes simulant les différents taux d'erreurs sont tronquées à $3~000$ contraintes, que les tentatives aient convergé ou non.
				Nous serons sensible à cette information pour ne pas faire de mauvaises interprétations, car le dernier point des différentes courbes ne représente pas forcément le point de convergence des tentatives associées.
			\end{leftBarWarning}
			
			\begin{figure}[!htb]
				\centering
				\includegraphics[width=0.95\textwidth]{figures/etude-erreur-simulation-impact-closest}
				\caption{
					Évolutions de la moyenne de la \texttt{v-measure} entre un résultat obtenu et la vérité terrain en fonction du nombre de contraintes annotées au cours des itérations du \textit{clustering} interactif.
					Les courbes en dégradé de couleur représentent les déclinaisons de cette évolution en intégrant un pourcentage d'annotations erronées (allant de $0$\% et $50$\%).
					\textbf{(1)} et \textbf{(2)} représente respectivement l'approche ignorant les conflits dans les contraintes et l'approche corrigeant les conflits détectés par le gestionnaire de contraintes.
					Toutes les courbes sont tronquées à $3~000$ contraintes.
				}
				\label{figure:4.6.1-ETUDE-ROBUSTESSE-INTERETS-CORRECTION-ERREURS}
			\end{figure}
			
			% Description de la figure.
			\todo[inline]{A REDIGER: description de la figure}

		%%% Discussion
		\subsubsection{Discussion}
		
			% Rappel de l'objectif : ...
			\todo[inline]{A REDIGER: rappel de l'objectif}
		
			% Remaques expérience utilisateur.
			\todo[inline]{A REDIGER: Super important de corriger !}
			\todo[inline]{A REDIGER: Besoin de mécanisme pour prédire et mettre de la redondance}
			
			% Conclusions et suggestion.
			\todo[inline]{A REDIGER: ouverture sur l'impact des incohérences}
	
	
	%%%
	%%% Subsection 4.6.2: Étude de l'impact des incohérences d'annotation sur les performances.
	%%%
	\subsection{Étude de l'impact des incohérences d'annotation sur les performances}
	\label{section:4.6.2-ETUDE-ROBUSTESSE-SIMULATION-IMPACT-ERREURS}
		
		% Objectif de l'expérience.
		\todo[inline]{A REDIGER: objectif de l'expérience}
	
		%%% Protocole expérimental.
		\subsubsection{Protocole expérimental}
			
			% Axiome.
			% Pseudo-code.
			% Détails de l'expérience.
			\todo[inline]{A REDIGER}
			
			% Référence scripts.
			\begin{leftBarInformation}
				Les scripts de l'expérience, réalisés avec des \textit{notebooks} Python (\cite{van-rossum-drake:2009:python-reference-manual}), sont disponibles dans un dossier dédié de~\cite{schild:2021:cognitivefactory-interactiveclusteringcomparativestudy}.
			\end{leftBarInformation}

		%%% Résultats
		\subsubsection{Résultats obtenus}
			\todo[inline]{A REDIGER}
		
			% Description statistiques.
			\todo[inline]{A REDIGER:}
			La \textsc{Table~\ref{table:4.6.2-ETUDE-ROBUSTESSE-SIMULATION-IMPACT-ERREURS}}
			
			\begin{table}[!htb]
				\begin{center}
				\begin{tabular}{|c|r|r|r|r|r|r|r|r|r|}
					% ENTETE DU TABLEAU
					% Erreur 0.00%
					% Erreur 0.05%
					% Erreur 0.10%
					% Erreur 0.15%
					% Erreur 0.20%
					% Erreur 0.25%
					% Erreur 0.30%
					% Erreur 0.35%
					% Erreur 0.40%
					% Erreur 0.45%
					% Erreur 0.50%
				\end{tabular}
				\end{center}
				\caption{
					Simulation ...
				}
				\label{table:4.6.2-ETUDE-ROBUSTESSE-SIMULATION-IMPACT-ERREURS}
			\end{table}

		%%% Discussion
		\subsubsection{Discussion}
		
			% Rappel de l'objectif : ...
			\todo[inline]{A REDIGER: rappel de l'objectif}
		
			% Remaques expérience utilisateur.
			\todo[inline]{A REDIGER: prédiction de retard}
			
			% Conclusions et suggestion.
			\todo[inline]{FIN}
			
			
	%%%
	%%% Subsection 4.6.3: Étude du score inter-annotateurs obtenu avec des opérateurs en situation réelle.
	%%%
	\subsection{Étude du score inter-annotateurs obtenu avec des opérateurs en situation réelle}
	\label{section:4.6.3-ETUDE-ROBUSTESSE-SCORE-INTER-ANNOTATEURS}
		
		% Objectif de l'expérience.
		Nous voulons étudier le score d'accord inter-annotateurs calculé lors d'une annotation de contraintes par plusieurs expert métiers en situation réelle.
		Pour cela, nous reprenons l'expérience de la \textsc{Section~\ref{section:4.3.1-ETUDE-COUTS-TEMPS-ANNOTATION}} visant à estimer le temps moyen d'annotation d'un lot de contraintes, et nous adaptons son protocole expérimental pour estimer l'accord inter-annotateurs.
		
	
		%%% Protocole expérimental.
		\subsubsection{Protocole expérimental}
			
			% Axiome.
			\begin{leftBarWarning}
				Dans cette étude, nous supposons que les annotateurs de l'expérience connaissent parfaitement le domaine traité dans le jeu de données, et qu'ils sont capables de caractériser sans ambiguïté la similitude entre deux données issues de cet ensemble.
				Afin de pourvoir faire cette hypothèse forte, et ainsi limiter les bruits dans l'analyse des résultats, le jeu de données devra traiter d'un sujet de culture générale (ne nécessitant donc pas de connaissance particulière) et des réviseurs supprimeront en amont et d'un commun accord les données trop spécifiques ou trop ambiguës.
			\end{leftBarWarning}
			
			% Pseudo-code.
			Pour résumer le protocole expérimental que nous décrivons ci-dessous, vous pouvez vous référer au pseudo-code décrit dans \textsc{Algorithme~\ref{algorithm:4.6.3-ETUDE-ROBUSTESSE-SCORE-INTER-ANNOTATEURS-PROTOCOLE}}.

			\begin{algorithm}
				\KwData{jeu de données annoté (vérité terrain)}
				\KwIn{plusieurs réviseurs, plusieurs annotateurs}
				%
				\textbf{initialisation}: définir et revoir le jeu de données entre réviseurs \;
				\textbf{échantillonnage}: sélectionner une base de contraintes équilibrée \;
				\ForEach{annotateur}{
					 \While{la base de contraintes n'a pas été entièrement annotée}{
						\textbf{annotation}: annoter une partie des contraintes \;
						\textbf{revue}: revue des contraintes en conflits d'annotation \;
					}
				}
				%
				\KwResult{modélisation du score inter-annotateurs sur le lot de contraintes}
				%
				\caption{\textit{
					Description en pseudo-code du protocole expérimental de l'étude du score inter-annotateurs d'annotation d'un lot de contraintes par plusieurs experts métiers en situation réelle.
				}}
				\label{algorithm:4.3.3-ETUDE-COUTS-TEMPS-ANNOTATION-PROTOCOLE}
			\end{algorithm}
			
			% Détails de l'expérience : préparation du jeu de données.
			Nous allons procéder en plusieurs étapes.
			D'abord, il faut choisir un jeu de données approprié : pour valider notre hypothèse forte sur les compétences de nos annotateurs, nous cherchons un jeu de données traitant d'un sujet de culture général.
			Pour cette expérience, nous avons donc choisi \texttt{MLSUM} : une collecte d'articles de journaux, classés par catégorie de publication et décrits par leur titre et leur résumé.
			Nous nous intéressons ici à la tâche de classification d'un titre d'article en fonction de sa catégorie de publication.
			Comme certains titres peuvent porter à confusion (un titre d'article n'étant pas toujours explicite sur son contenu), deux réviseurs sont chargés de choisir les données les plus explicites sur un échantillon d'un millier de données représentatives des catégories les plus communes.
			L'échantillon résultant, noté \texttt{MLSUM FR Train Subset (v1.0.0-schild)}, est composé de $744$ titres d'articles rédigés en français et répartis en $14$ classes (\textit{économie}, \textit{sport}, ...).
			Pour plus de détails, consultez l'annexe~\ref{annex:C.2-DATASET-MLSUM-SUBSET-SCHILD}.
		
			% Détails de l'expérience : sélection des contraintes à annoter.
			A partir de ces données, nous sélectionnons un lot de $400$ contraintes à annoter.
			Pour faciliter l'analyse, l'échantillonnage sera un aléatoire équilibré d'après la vérité terrain en $200$ \texttt{MUST-LINK} et en $200$ \texttt{CANNOT-LINK}.
			
			% Détails de l'expérience : annotations et consignes.
			Ensuite, un groupe de $3$ annotateurs vont annoter la sélection de $400$ contraintes en plusieurs sessions.
			Les directives données aux opérateurs sont les suivantes:
			\begin{itemize}
				\item \textbf{Contexte de l'opérateur} :
				« \textit{Vous êtes des \textbf{experts de la presse et de l’actualité} ; Vous voulez classer des articles dans des catégories en fonction de leur titre ; Vous ne savez pas précisément quelles catégories vous allez utiliser pour classer vos articles ; Mais vous savez \textbf{caractériser la similitude} de deux articles} » ;
				\item \textbf{Contexte sur le jeu de données} :
				« \textit{Le thème sont les catégories d’articles de presse ; La vérité terrain contient entre $10$ et $20$ catégories parmi les plus communes de la presse ; La vérité terrain contient entre $30$ et $100$ articles par catégorie ; Vous \textbf{pouvez regarder le jeu de données non annoté} autant que vous le voulez (disponible dans l'onglet \texttt{TEXTS} de l'application)} » ;
				\item \textbf{Consignes d'annotations} :
				« \textit{Faites des séries de \textbf{15 minutes minimum} pour avoir de la régularité ; Si possible, \textbf{isolez-vous} pour ne pas être dérangé et ne pas fausser les résultats ; Pour chaque série, \textbf{notez le temps et le nombre de contraintes annotés} ; Si vous ne savez pas quoi annoter (trop ambigu, vocabulaire inconnu, ...), \textbf{passez au suivant sans annoter} (vous êtes sensés être des experts de la presse !)} ».
			\end{itemize}
			%
			Pour réaliser l'annotation, les opérateurs auront accès à l'application web développée au cours de ce doctorat.
			Des captures d'écran sont disponibles en \textsc{Figure~\ref{figure:4.3.1-ETUDE-COUTS-TEMPS-ANNOTATION-APPLICATION-ANNOTATION}} et \textsc{Figure~\ref{figure:4.3.1-ETUDE-COUTS-TEMPS-ANNOTATION-APPLICATION-LISTE-CONTRAINTES}}.
			Une description plus détaillée de l'application et de ses fonctionnalités est disponible en \textsc{Section~\ref{section:3.3-DESCRIPTION-IMPLEMENTATION}}\todo{description à faire}.
			
			% Détails de l'expérience.
			\todo[inline]{A REDIGER / A COMPLETER}
			
			% Référence scripts.
			\begin{leftBarInformation}
				Les scripts de l'expérience, réalisés avec des \textit{notebooks} Python (\cite{van-rossum-drake:2009:python-reference-manual}), sont disponibles dans un dossier dédié de~\cite{schild:2021:cognitivefactory-interactiveclusteringcomparativestudy}.
			\end{leftBarInformation}
			
		%%% Résultats
		\subsubsection{Résultats obtenus}
		
			% Description statistiques.
			La \textsc{Table~\ref{table:4.6.3-ETUDE-ROBUSTESSE-SCORE-INTER-ANNOTATEURS}} expose les scores inter-annotateurs sur les $3$ opérateurs et le réviseur de cette expérience, ainsi que l'accord avec la vérité terrain.
			Le score d'accord moyen avec la vérité terrain est de $0.86$ (écart-type: $0.01$) ; ce score est de $0.81$ (écart-type: $0.05$) pour les contraintes de types \texttt{MUST-LINK} et est de $0.92$ (écart-type: $0.03$) pour les contraintes de types \texttt{CANNOT-LINK}.
			Le score inter-annotateurs moyen (sans le réviseur) est de $0.84$ (écart-type: $0.03$).
			
			\begin{table}[!htb]
				\begin{center}
				\begin{tabular}{|c|r|r|r|r|}
				
					\hline
					% ENTETE DU TABLEAU
					
						& \multicolumn{1}{c|}{\shortstack[c]{
							1 (Relecteur)
						}}
						& \multicolumn{1}{c|}{\shortstack[c]{
							7 (Annotateur)
						}}
						& \multicolumn{1}{c|}{\shortstack[c]{
							9 (Annotateur)
						}}
						& \multicolumn{1}{c|}{\shortstack[c]{
							12 (Annotateur)
						}}
						\tabularnewline
						\hline

					% Vérité terrain
					\multicolumn{1}{|c|}{\shortstack[c]{
						Vérité terrain
					}}
						& $0.95$
						& $0.86$
						& $0.84$
						& $0.87$
						\tabularnewline
						\hline

					% Vérité terrain
					\multicolumn{1}{|c|}{\shortstack[c]{
						1 (Relecteur)
					}}
						&
						& $0.91$
						& $0.86$
						& $0.89$
						\tabularnewline
						\hline

					% Vérité terrain
					\multicolumn{1}{|c|}{\shortstack[c]{
						7 (Annotateur)
					}}
						&
						&
						& $0.86$
						& $0.85$
						\tabularnewline
						\hline

					% Vérité terrain
					\multicolumn{1}{|c|}{\shortstack[c]{
						9 (Annotateur)
					}}
						&
						&
						&
						& $0.80$
						\tabularnewline
						\hline
					
				\end{tabular}
				\end{center}
				\caption{
					Score d'accord inter-annotateurs obtenu avec $1$ réviseur et $3$ annotateurs sur un lot commun de $400$ contraintes ($200$ \texttt{MUST-LINK}, $200$ \texttt{CANNOT-LINK}).
				}
				\label{table:4.6.3-ETUDE-ROBUSTESSE-SCORE-INTER-ANNOTATEURS}
			\end{table}
			
			% Note.
			\begin{leftBarInformation}
				Dans une autre expérience, où $14$ opérateurs devaient annoter une base de $1~000$ contraintes aléatoires, nous obtenons un accord moyen avec la vérité terrain de $0.93$ (écart-type: $0.02$) et un score inter-annotateurs moyen de $0.91$ (écart-type: $0.03$).
				Toutefois, nous ne mettons pas en avant ces résultats car le lot de contraintes à annoté est déséquilibré à cause de l'utilisation de l'échantillonnage aléatoire ($932$ \texttt{CANNOT-LINK}, $68$ \texttt{MUST-LINK}).
			\end{leftBarInformation}
	
		%%% Discussion
		\subsubsection{Discussion}
		
			% Rappel de l'objectif : ...
			\todo[inline]{A REDIGER: rappel de l'objectif}
		
			% Avantage 1 : peu d'erreurs
			\todo[inline]{A REDIGER: Peu d'erreurs (environ 16\% d'erreurs sans concertations)}
			
			% Remarque 1 : MUST-LINK < CANNOT-LINK
			\todo[inline]{A REDIGER: CANNOT-LINK légèrement plus facile que les MUST-LINK}
			
			% Conclusions et suggestion.
			\todo[inline]{A REDIGER: ouverture sur l'impact des non-corrections}
			
	%%%
	%%% Subsection 4.6.3: Bilan concernant la robustesse du \textit{clustering} interactif
	%%%
	\subsection{Bilan concernant la robustesse du \textit{clustering} interactif}
	\label{section:4.6.3-ETUDE-ROBUSTESSE-MISE-EN-COMMUN}
	
		\todo[inline]{A REDIGER}
	
		% Conclusion.
		\begin{leftBarSummary}
			Au cours de cette étude de pertinence, nous avons pu voir que :
			\begin{itemize}
				\item[\itemok] ...
				\item[\itemok] ...
				\item[\itemok] ...
			\end{itemize}
		\end{leftBarSummary}
	
	
	%%%%%--------------------------------------------------------------------
	%%%%% Section 4.7: Hypothèses non vérifiées.
	%%%%%--------------------------------------------------------------------
	\newpage
	\section{Autres hypothèses non vérifiées}
\label{section:4.7-HYPOTHESES-NON-VERIFIEES}

	%%%
	%%% Introduction / Transition.
	%%%
	Lors des études précédentes, nous avons vérifié un certain nombre d'hypothèses et avons exploré plusieurs détails pratiques pour mettre en oeuvre une méthodologie d'annotation basée sur le \texttt{Clustering Interactif}.
	Toutefois, certains points n'ont pas pu être étudiés en profondeur lors de ce doctorat, par manque de temps ou de moyens.
	Nous exposons ici un ensemble de pistes intéressantes pouvant nourrir de futurs travaux afin d'améliorer notre méthode.
	
	
	%%%
	%%% Subsection 4.7.1: Étude du nombre de \textit{clusters} optimal.
	%%%
	\subsection{Étude du nombre de \textit{clusters} optimal}
	\label{section:4.7.1-HYPOTHESES-NON-VERIFIEES-NOMBRE-CLUSTERS}
	
		% Problème ouvert de la recherche: Estimer le nombre optimal de \textit{clusters}.
		Un problème ouvert de la recherche lors de l'utilisation d'algorithmes de \textit{clustering} concerne le choix du nombre de \textit{clusters} à trouver.
		En effet, à part une connaissance \textit{a priori} du nombre de thématiques présentes dans le jeu de données, il est difficile d'estimer le nombre optimal de \textit{clusters}, d'autant plus que celui-ci peut changer en fonction de la granularité de modélisation requise pour répondre au cas d'usage.
		
		% Pistes déjà explorées.
		Nous avons déjà exploré partiellement deux pistes :
		\begin{itemize}
			\item l'\textbf{exploration du graphe de contraintes} : en effet, il est possible d'estimer le nombre maximal de \textit{clusters} grâce aux composants connexes de contraintes \texttt{MUST-LINK}, et d'estimer le nombre minimal de \textit{clusters} grâce à la coloration du graphe de contraintes \texttt{CANNOT-LINK} ;
			\item les \textbf{études de pertinence} avec l'analyse des patterns linguistiques et le résumé thématique des \textit{clusters} (cf. \textsc{Section~\ref{section:4.4-HYPOTHESE-PERTINENCE}}) : ces deux approches permettent de rapidement évaluer si les thématiques obtenues sont trop générales (\textit{i.e. s'il n'y a pas assez de clusters}) ou si elles semblent trop spécifiques (\textit{i.e. s'il y en a trop}).
		\end{itemize}
		
		% Piste potentielles à explorer.
		Toutefois, pour aller plus loin, deux pistes potentielles pourraient être explorées :
		\begin{itemize}
			\item l'exploration brute du nombre de \textit{clusters} par la \textbf{méthode du coude} : bien que ces approches sont plus coûteuses en temps de calcul, elles permettent d'estimer le nombre de \textit{clusters} pour lequel la stabilité du \textit{clustering} est la plus élevée ;
			\item l'utilisation d'algorithmes n'ayant pas de nombre de \textit{clusters} en paramètres, comme des versions contraintes de \texttt{DBScan} (par exemple dans sa version \texttt{C-DBScan}, \cite{ruiz-etal:2010:densitybased-semisupervised-clustering}) ou de la \textbf{propagation par affinité} (\cite{givoni-frey:2009:semisupervised-affinity-propagation}) : ces alternatives semblent prometteuses car elles retirent la complexité due à ce paramétrage abstrait.
		\end{itemize}
		
		\setcounter{localCounterOfFootnoteValue}{\value{footnote}}
		\begin{leftBarInformation}
			L'étude de \texttt{C-DBScan} a été en partie réalisée dans le cadre d'un projet étudiants avec l'École d'Ingénieurs Télécom Physique Strasbourg (au cours de l'année 2022).
			Les résultats montraient que le temps de calcul était similaire à celui du \texttt{KMeans} (dans sa version \texttt{COP}).
			La difficulté d'utilisation résidait plutôt dans la définition du rayon de voisinage \texttt{eps} à parcourir pour établir des liens entre données.
			Celui-ci peut être estimé en analysant la densité vectorielle du jeu de données.
			Le code informatique est disponible dans \cite{schild:2022:cognitivefactory-interactiveclustering} \footnotemark.
		\end{leftBarInformation}
		% Rattraper les footnote.
			\stepcounter{localCounterOfFootnoteValue}
			\footnotetext[\value{localCounterOfFootnoteValue}]{
				Implémentation de \texttt{C-DBScan} : \textit{Pull Request} en attente pour une version \texttt{0.6.0} après ajout de documentation et de tests unitaires.
			}
	
	
	%%%
	%%% Subsection 4.7.2: Étude d'autres méthodes de vectorisation.
	%%%
	\subsection{Étude d'autres méthodes de vectorisation}
	\label{section:4.7.2-HYPOTHESES-NON-VERIFIEES-VECTORISATION}
	
		% Introduction.
		Au début de ce doctorat, nous avons conclu que les algorithmes de vectorisation n'avaient pas d'impact réel sur l'efficience de notre méthodologie d'annotation.
		Toutefois, les modèles de langues se sont largement développés, et il est fort probable que l'utilisation d'un \textbf{modèle pré-entraîné} permette désormais d'avoir un gain de performance.
		
		% Piste potentielles à explorer.
		Nous pourrions par exemple tester les \textbf{architectures à base de \textit{Transformers}} (\cite{uszkoreit:2017:transformer-novel-neural}) comme \texttt{BERT} (\cite{devlin-etal:2019:bert-pretraining-deep}) et essayer différents modèles pré-entraînés sur des données françaises pour compléter nos études réalisées dans \cite{schild:2021:cognitivefactory-interactiveclusteringcomparativestudy}
	
	
	%%%
	%%% Subsection 4.7.3: Étude d'autres méthodes d'échantillonnage.
	%%%
	\subsection{Étude d'autres méthodes d'échantillonnage}
	\label{section:4.7.3-HYPOTHESES-NON-VERIFIEES-ECHANTILLONNAGE}
	
		% Introduction.
		Comme nous avons pu le voir dans \textsc{Section~\ref{section:4.6-HYPOTHESE-ROBUSTESSE}}, il peut-être intéressant d'introduire un mécanisme de création de redondance dans le graphe de contraintes annotées pour identifier les erreurs d'annotation.
		Un tel mécanisme n'a pas encore été implémenté mais pourrait facilement être intégré aux implémentations \texttt{Python} déjà existantes (\cite{schild:2022:cognitivefactory-interactiveclustering}).
		
		% Piste potentielles à explorer.
		Pour ce faire, le parcours de graphe et la création de cycle permettraient de vérifier la présence de conflits et ainsi de \textbf{provoquer des phases de revues de contraintes} si cela est nécessaire.
		Une telle page de revue pourrait aussi contenir des analyses complémentaires, comme l'estimation du taux de contraintes n'ayant pas de redondance et représentant ainsi des erreurs cachées potentielles.
	
	
	%%%
	%%% Subsection 4.7.4: Étude de techniques de transfert d'apprentissage.
	%%%
	\subsection{Étude de techniques de transfert d'apprentissage}
	\label{section:4.7.4-HYPOTHESES-NON-VERIFIEES-TRANSFERT-APPRENTISSAGE}
	
		% Introduction.
		Dans la \textsc{Section~\ref{section:2.3-DEFIS-ANNOTATION}}, nous avions déjà évoqué le fait que la modélisation d'un phénomène peut être assisté par des techniques telles que la pré-annotation (\cite{dandapat-etal:2009:complex-linguistic-annotation}) ou le transfert d'apprentissage (\cite{zhuang-etal:2021:comprehensive-survey-transfer}).
		Nous pourrions nous inspirer davantage de ces approches pour démarrer plus efficacement les premières itérations d'un \texttt{Clustering Interactif}.
	
		% Piste potentielles à explorer.
		Voici quelques idées inspirées de ces méthodes :
		\begin{itemize}
			\item \textbf{pré-annoter} certaines contraintes simples à l'aide de règles (\textit{basées par exemple sur la présence de mots de vocabulaire en commun}) ou grâce à l'utilisation d'un modèle déjà disponible ; 
			\item \textbf{introduire des données synthétiques ou empruntées} à d'autres bases d'apprentissage pour initialiser le \textit{clustering}, et permettre ainsi ajouter d'emblée des connaissances générales dans la modélisation.
		\end{itemize}
	
	
	%%%
	%%% Subsection 4.7.5: Étude ergonomique de l'interface d'annotation.
	%%%
	\subsection{Étude ergonomique de l'interface d'annotation}
	\label{section:4.7.5-HYPOTHESES-NON-VERIFIEES-ERGONOMIQUE}
	
		% Introduction.
		L'application web développée au cours de ce doctorat (\cite{schild-etal:2022:cognitivefactory-interactiveclusteringgui}) permet d'essayer rapidement notre méthodologie d'annotation.
		Cependant, cette dernière n'a pu faire l'objet d'études poussées pour estimer la meilleure disposition des composants ou l'intérêt de certaines fonctionnalités d'annotation.
		
		% Piste potentielles à explorer.
		Parmi les pistes potentielles à explorer, nous avons évoqué la possibilité d'\textbf{annoter plusieurs contraintes} dans une même interface (\textit{par exemple : annoter visuellement un mini-graphe de $4$ données plutôt que d'annoter simplement une paire de données}) et le besoin de \textbf{réaliser des analyses rapides} sur les \textit{clusters} ou sur le graphe de contraintes (voir \textsc{Section~\ref{section:4.4-HYPOTHESE-PERTINENCE}} et \textsc{Section~\ref{section:4.5-HYPOTHESE-RENTABILITE}}).
		Pour aller plus loin, \cite{bae-etal:2021:interactive-clustering-comprehensive} proposent d'autres listes d'interactions qui sont possibles d'avoir avec un algorithme de \textit{clustering}, notamment sur la manipulation de son résultat (\textit{fusion, suppression, verrouillage, ...}) et de ses hyperparamètres (\textit{nombre de \textit{clusters}, adaptation du vocabulaire autorisé, ...})
		
		% A essayer !
		Toutes ces idées pourraient être l'objet de développements et d'études dédiées avec des groupes d'annotateurs différents pour voir l'impact sur les performances et les biais de conception de modèles.
	
	
	%%%%%--------------------------------------------------------------------
	%%%%% Section 4.8: Bilan des études réalisées
	%%%%%--------------------------------------------------------------------
	%\newpage
	%\section{Bilan des études réalisées}
\label{section:4.8-BILAN-HYPOTHESES}
		\todo[inline]{SECTION À RÉDIGER}

%%%%%--------------------------------------------------------------------
%%%%% Chapitre: Guide d'utilisation
%%%%%--------------------------------------------------------------------
\chapter{Bilan et Guide d'utilisation du \textit{Clustering Interactif}}
\label{chapter:5-GUIDE}
	
	% Introduction.
	Dans nos études, nous nous sommes intéressé aux assistants conversationnels orientés par tâches et aux méthodes de conception des bases d'apprentissage nécessaires à leur entraînement.
	Dans ce chapitre, nous dressons une synthèse des découvertes et des conseils d'utilisation de notre méthodologie d'annotation basée sur un \texttt{Clustering Interactif} ayant pour but d'assister les experts métiers dans la phase de modélisation des textes en intentions de dialogue.
	
	\begin{leftBarReminder}
		% Rappel: Complexité et cycle MATTER
		Nous partons du constat selon lequel la conception d'une base d'apprentissage de textes annotés en intentions est connue pour être complexe, subjective et sensible aux erreurs (voir \textsc{Section~\ref{section:2.3-DEFIS-ANNOTATION}}).
		Pour limiter ces problèmes, un projet de labellisation s'organise généralement autour du cycle \texttt{MATTER} (\cite{pustejovsky-stubbs:2012:natural-language-annotation}) durant lequel une modélisation abstraite des intentions est définie pour annoter les données ; cette modélisation est ensuite affinée ou remise en cause plusieurs fois au cours du cycle pour mieux s'adapter au projet (voir \textsc{Section~\ref{section:2.2-ORGANISATION-ANNOTATION}}).
		
		% Besoin d'experts métiers.
		Pour modéliser et annoter les données, les experts métiers ont besoin de leurs connaissances métiers, mais aussi de compétences analytiques et techniques afin d'assurer de la qualité de la base d'apprentissage en cours de construction.
		Par conséquent, un projet d'annotation devient rapidement onéreux, notamment à cause :
		\begin{itemize}
			\item des formations analytiques nécessaires aux annotateurs pour intervenir dans le projet ;
			\item des nombreux ateliers de modélisation en mode essai-erreur nécessaires pour trouver une base d'apprentissage stable et pertinente ; et
			\item de la complexité engendrée par la manipulation de concepts abstraits (\textit{intentions, entités, ...}) dans le but de représenter les connaissances des experts métiers.
		\end{itemize}
		
		% Remise en cause.
		Au cours de ce doctorat, et sur la base des constatations décrites ci-dessus, nous avons décidé de reconsidérer cette approche de la tâche de modélisation.
		Nous avons alors proposé une nouvelle méthodologie d'annotation dans le but d'impliquer les experts métiers pour leurs vraies compétences tout en leur demandant un minimum de bagages analytiques et techniques.
	\end{leftBarReminder}
	
	
	%%%%%--------------------------------------------------------------------
	%%%%% Section 5.1: Présentation rapide du \textit{Clustering Interactif}.
	%%%%%--------------------------------------------------------------------
	\newpage
	\section{Présentation rapide du \textit{Clustering Interactif}}
		\label{section:5.1-GUIDE-PRESENTATION-RAPIDE}
		
		% Intuitions : annotation par différences de cas d'usages.
		Sur la base d'intuitions issues de la littérature (\textsc{Section~\ref{section:3.1-INTUITIONS-ORIGINES}}), nous avons décidé de centrer notre méthodologie d'annotation sur l'\textbf{annotation des similarités et des différences entre les données}.
		En effet, une telle stratégie semble moins complexe à appliquer, car elle ne dépend pas d'une modélisation abstraite des connaissances de l'expert, mais elle se base directement sur ses connaissances pour décrire si deux données ont ou n'ont pas un cas d'usage équivalent.
		
		% Proposition de \texttt{Clustering Interactif}.
		Nous mettons en oeuvre cette stratégie d'annotation par similarité au sein d'une méthodologie basée sur un \texttt{Clustering Interactif} (voir \textsc{Section~\ref{section:3.2-DESCRIPTION-THEORIQUE}}).
		Cette méthode repose sur les avantages des interactions Homme/Machine, en déléguant la conception de la base d'apprentissage à la machine (\textit{à l'aide d'un algorithme de regroupement automatique de texte (clustering)}) et en faisant intervenir l'expert pour affiner itérativement la base d'apprentissage proposée (\textit{en annotant des contraintes entre les données pour corriger le regroupement des textes}).
		\textbf{Trois étapes principales} se répètent ainsi au cours de ce \textbf{processus itératif} :
		\begin{itemize}
			\item une \textbf{sélection de données à annoter} :
			la machine propose un ensemble de données dont la similarité serait à confirmer (ou à infirmer) afin de corriger efficacement le regroupement automatique des données opéré à l'itération suivante ;
			\item une \textbf{annotation des contraintes} :
			l'expert caractérise chaque couple de données en répondant à la question \textguillemets{\textit{est-ce que les deux données ont un cas d'usage similaire ?}}, en ajoutant une contrainte \texttt{MUST-LINK} si oui (\textit{similaires}) et en ajoutant une contrainte \texttt{CANNOT-LINK} sinon (\textit{non similaires}) ;
			\item une \textbf{segmentation automatique des données} : la machine regroupe les données en fonction de leurs similarités intrinsèques et des contraintes annotées par l'expert.
		\end{itemize}
		
		
		% Figure schéma.
		\begin{figure}[H]
			\centering
			\includegraphics[width=0.85\textwidth]{figures/interactive-clustering-architecture-sequentielle}
			\caption{
				Schéma illustrant l'architecture du \texttt{Clustering Interactif}.
			}
			\label{figure:5.1-GUIDE-PRESENTATION-RAPIDE-CLUSTERING-INTERACTIF}
		\end{figure}
		
		% Figure exemple
		\begin{figure}[H]
			\centering
			\includegraphics[width=0.85\textwidth]{figures/example-iteration-clustering-interatif}
			\caption{
				Exemple d'une itération de \texttt{Clustering Interactif}.
			}
			\label{figure:5.1-GUIDE-PRESENTATION-RAPIDE-EXEMPLE}
		\end{figure}
		
		% Références
		\begin{leftBarReminder}
			Consulter la \textsc{Section~\ref{section:3.2-DESCRIPTION-THEORIQUE}} pour plus de détails.
		\end{leftBarReminder}
	
	
	%%%%%--------------------------------------------------------------------
	%%%%% Section 5.2: Avantages et limites de la méthode.
	%%%%%--------------------------------------------------------------------
	\newpage
	\section{Avantages et limites de la méthode}
	\label{section:5.2-GUIDE-AVANTAGES-ET-LIMITES}
	
		%%% Avantages.
		Au sujet des avantages de la méthode :
		\begin{itemize}
			% Apprentissage semi-supervisé.
			\item[\textcolor{colorDarkPastelGreen}{\textcolor{colorDarkPastelGreen}{\faThumbsUp}}]
				Par conception, la méthode tire parti des avantages de l'apprentissage actif et des méthodes de regroupement semi-supervisé : \textbf{l'expert n'est plus responsable de l'ensemble du travail de labellisation} ; il intervient simplement là où il offre une valeur ajoutée (\textit{annotation des contraintes pour améliorer la pertinence de la base d'apprentissage en cours de construction}), et il délègue le reste à la machine (\textit{regroupement automatique, sélection des contraintes intéressantes à annoter, identification des incohérences}).
			% Plus de modélisation.
			\item[\textcolor{colorDarkPastelGreen}{\textcolor{colorDarkPastelGreen}{\faThumbsUp}}]
				\textbf{Il n'est plus nécessaire de définir une modélisation abstraite de la connaissance d'un expert métier pour labelliser un jeu de données} : cette modélisation est construite au cours des itérations de la méthode, à l'aide des regroupements automatiques réalisés par la machine.
			% Annotation binaire
			\item[\textcolor{colorDarkPastelGreen}{\textcolor{colorDarkPastelGreen}{\faThumbsUp}}]
				Par extension, il n'est plus nécessaire de manipuler cette modélisation abstraite pouvant contenir des dizaines d'intentions de dialogue :
				\textbf{l'expert se contente de décrire la similarité entre deux données au cours d'une annotation binaire (\texttt{MUST-LINK} ou \texttt{CANNOT-LINK})} pour corriger le résultat proposé par la machine.
			% Connaissance métier.
			\item[\textcolor{colorDarkPastelGreen}{\textcolor{colorDarkPastelGreen}{\faThumbsUp}}]
				\textbf{Cette annotation binaire se base directement sur la ressemblance entre cas d'usage métiers} : les experts intervenant dans le projet peuvent traiter les données comme ils le feraient professionnellement au quotidien, sans avoir à manipuler ou à interpréter des concepts abstraits et potentiellement non adaptés à la situation (\textit{intentions, ...}).
			% Optimisation technique et coûts compétitifs.
			\item[\textcolor{colorDarkPastelGreen}{\textcolor{colorDarkPastelGreen}{\faThumbsUp}}]
				L'implémentation de notre méthodologie d'annotation a pu être optimisée afin de converger vers une base d'apprentissage stable en un minimum de contraintes : \textbf{les coûts à engager pour cette approche semblent raisonnables et compétitifs avec l'approche d'annotation traditionnelle} (voir \textsc{Section~\ref{section:4.3.4-ETUDE-COUTS-TOTAL}}).
		\end{itemize}
	
		%%% Limites.
		Néanmoins, il faut aussi considérer quelques limites et pistes non explorées :
		\begin{itemize}
			% Sensible aux erreurs.
			\item[\textcolor{colorDarkPastelRed}{\textcolor{colorDarkPastelRed}{\faThumbsDown}}]
				Le \texttt{Clustering Interactif} possède les défauts des approches incrémentales : ainsi, \textbf{une erreur d'annotation peut rapidement se propager et pénaliser le processus}.
				Par conséquent, il est important de vérifier ses annotations (\textit{par exemple lors de session de revue avec d'autres opérateurs}), et il peut être intéressant d'introduire des contraintes redondantes afin de mieux détecter les incohérences d'annotation (voir \textsc{Section~\ref{section:4.6.2-ETUDE-ROBUSTESSE-ERREURS-ANNOTATION-ET-CORRECTION}}).
			% Sensible à la subjectivité.
			\item[\textcolor{colorDarkPastelRed}{\textcolor{colorDarkPastelRed}{\faThumbsDown}}]
				\textbf{Les différentes d'opinion entre plusieurs annotateurs concernant la caractérisation de certaines contraintes peuvent entraîner des divergences de résultat si elles ne sont pas identifiées et traitées}.
				De ce fait, il est primordial de confronter les points de vue des experts en leur faisant annoter les mêmes contraintes et en organisant des ateliers de revue pour débattre des différences d'annotation (voir \textsc{Section~\ref{section:4.6.3-ETUDE-ROBUSTESSE-SUBJECTIVITE-ANNOTATION-ET-DIVERGENCE}}).
			% Perte de visibilité dû à l'absence de modélisation.
			\item[\textcolor{colorDarkPastelRed}{\textcolor{colorDarkPastelRed}{\faThumbsDown}}]
				\textbf{La méthode peut souffrir d'un manque de visibilité et d'une sensation de perte de contrôle sur la modélisation en cours} : en effet, le processus peut être perçu comme répétitif (\textit{environ $145$ itérations à réaliser}), et le cas d'arrêt de la méthode n'est pas clairement identifiable (\textit{il est toujours possible de réaliser un itération supplémentaire pour améliorer légèrement le résultat}).
				Pour ces raisons, il est important de fournir des outils d'analyse pour tenir l'opérateur informé de son avancement et de la rentabilité de chacune de ses actions (voir \textsc{Sections}~\textsc{\ref{section:4.4-HYPOTHESE-PERTINENCE}} et~\textsc{\ref{section:4.5-HYPOTHESE-RENTABILITE}}).
		\end{itemize}
	
	
	%%%%%--------------------------------------------------------------------
	%%%%% Section 5.3: Démarche d'annotation et d'analyse de la méthode.
	%%%%%--------------------------------------------------------------------
	\newpage
	\section{Démarche d'annotation et d'analyse de la méthode}
	\label{section:5.3-GUIDE-UTILISER}
	
		%%% Annotation collaborative.
		\paragraph{\textcolor{colorSilverLakeBlue}{\faCheckSquare} Annotation collaborative.}
			
			% Constat.
			L'annotation est un acte relevant du domaine de l'interprétation et de la subjectivité.
			Par conséquent, \textbf{deux annotateurs peuvent avoir des divergences d'opinions lors de la caractérisation de la similarité entre certaines données}, ce qui peut introduire des incohérences dans le fonctionnement de notre méthode.
			Afin de limiter ces incohérences, il est nécessaire de bien définir l'objectif de l'annotation (\textit{quel est la finalité du modèle à entraîner ?}), mais aussi d'harmoniser les points de vue des différents annotateurs.
			
			% Solution.
			Au cours de l'\texttt{étude de robustesse} (voir \textsc{Section~\ref{section:4.6-HYPOTHESE-ROBUSTESSE}}), \textbf{nous conseillons d'employer au moins $3$ experts et de régulièrement (voire toujours) leur soumettre les mêmes contraintes à annoter}.
			Un tel choix semble contre-intuitif à première vue (\textit{nous triplons le coût d'embauche alors que nous pourrions tripler le nombre de contraintes annotées}), mais ce parti pris a l'avantage de contraindre les annotateurs à discuter de leurs divergences d'opinion (\textit{s'ils n'annotent pas de façon analogue, c'est qu'ils ne sont pas d'accord sur la manière de traiter les données}).
			Ainsi, même si ce choix engendre des coûts supplémentaires, il s'avère être un investissement à long terme pour garantir la qualité et la cohérence de la base d'apprentissage en cours de construction.
			
			% Notes.
			Les experts pourront annoter des contraintes différentes après plusieurs itérations de la méthode lorsque leur score d'accord inter-annotateurs sera considéré comme (très) \texttt{fort}.
		
		
		%%% Rentabilité d'une itération.
		\paragraph{\textcolor{colorSilverLakeBlue}{\faCheckSquare} Rentabilité d'une itération.}
			
			% Constat.
			Comme notre méthodologie est itérative, il faut être capable de définir un cas d'arrêt (\textit{annoter toutes les contraintes possibles étant trop ambitieux}).
			
			% Solution.
			Au cours de l'\texttt{étude de rentabilité} (voir \textsc{Section~\ref{section:4.5-HYPOTHESE-RENTABILITE}}), nous avons mis en évidence l'intérêt d'\textbf{analyser l'évolution de la différence de résultats de \textit{clustering} entre deux itérations consécutives afin de quantifier l'intérêt d'une itération supplémentaire}.
			En effet, si le nouveau \textit{clustering} reste fortement similaire au \textit{clustering} précédent, malgré l'ajout de contraintes supplémentaires, nous pouvons en déduire que la méthode d'annotation stagne.
			Il peut alors être intéressant d'envisager de stopper les itérations de la méthode, d'évaluer la pertinence métier du \textit{clustering} actuel, et de corriger manuellement les reliquats.
			
			% Mise en oeuvre.
			Pour estimer cette différence entre deux \textit{clustering} consécutifs, nous utilisons le score de \texttt{v-measure} (voir \textsc{Annexe~\ref{annex:D-ANNEXE-EVALUATION-CLUSTERING}}).
		
		
		%%% Pertinence d'un résultat.
		\paragraph{\textcolor{colorSilverLakeBlue}{\faCheckSquare} Pertinence d'un résultat.}
			
			% Constat.
			L'analyse d'un résultat de \textit{clustering} n'est pas une tâche aisée, et elle peut vite devenir fastidieuse si elle doit être faite à chaque itération de notre méthode.
			
			% Solution.
			Au cours de l'\texttt{étude de pertinence} (voir \textsc{Section~\ref{section:4.4-HYPOTHESE-PERTINENCE}}), nous avons proposé deux solutions prometteuses pour assister l'expert dans cette évaluation :
			\begin{itemize}
				% FMC.
				\item Une \textbf{analyse des patterns linguistiques caractéristiques de chaque \textit{cluster} en utilisant la \texttt{FMC}} (voir \textsc{Annexe~\ref{annex:C.3-DESCRIPTION-IMPLEMENTATION-FEATURES-MAXIMIZATION-METRIC}}) :
				un \textit{cluster} peut ainsi être identifié comme pertinent s'il possède un vocabulaire caractéristique cohérent du point de vue de l'expert.
				Il est aussi possible de se servir de cette analyse pour identifier dans chaque texte les mots caractéristiques d'un ou plusieurs \textit{clusters}.
				% LLM.
				\item Un \textbf{résumé thématique des \textit{clusters} par un large modèle de langage (\texttt{LLM})} :
				cette approche permet d'estimer efficacement la cohérence d'un \textit{cluster} à l'aide d'une courte description en langage naturel.
				Il faut toutefois être vigilant aux hallucinations du \texttt{LLM}.
			\end{itemize}
			
			% Constat.
			Note : \textbf{Une correction manuelle est parfois nécessaire} pour valider un \textit{cluster}.
	
	
	%%%%%--------------------------------------------------------------------
	%%%%% Section 5.4: Implémentation et paramétrages de la méthode.
	%%%%%--------------------------------------------------------------------
	\newpage
	\section{Implémentation et paramétrages de la méthode}
		\label{section:5.4-GUIDE-PARAMETRAGES}
		
		%%% Références aux implémentations.
		\paragraph{\textcolor{colorSilverLakeBlue}{\faCheckSquare} Développements logiciels.}
		
			% Références.
			Pendant ce doctorat, nous avons implémenté plusieurs algorithmes de \textit{clustering} et d'échantillonnage (\textit{disponibles dans la librairie \texttt{cognitivefactory-interactive-clustering}}) et nous avons intégré notre méthodologie d'annotation dans une application web (\textit{accessible dans la librairie \texttt{cognitivefactory-interactive-clustering-gui}}).
			Ces développements logiciels sont décrits en \textsc{Annexe~\ref{annex:C-ANNEXE-IMPLEMENTATIONS}}.
		
		%%% Paramétrage optimal.
		\paragraph{\textcolor{colorSilverLakeBlue}{\faCheckSquare} Choix des paramétrages.}
		
			% Description.
			Au cours de nos études (voir \textsc{Chapitre~\ref{chapter:4-ETUDES}}), nous avons mis en avant un \textbf{paramétrage favori} de notre méthode.
			Ce paramétrage, basé sur un compromis entre un maximum d'efficience et un minimum de coûts, est composé des algorithmes suivants :
			
			% Liste paramètres.
			\begin{itemize}
				\item \texttt{prep.simple}: les \textbf{prétraitements} \textbf{simples} supprimant les minuscules, les ponctuations, les accents et les espaces blancs ;
				\item \texttt{vect.tfidf}: la \textbf{vectorisation} utilisant une représentation statistique du vocabulaire à l'aide d'un \texttt{TF-IDF} (\cite{ramos:2003:using-tfidf-determine}) ;
				\item \texttt{clust.kmeans.cop}: le \textbf{\textit{clustering} sous contraintes} utilisant l'algorithme \textbf{\texttt{COP-KMeans}} (\cite{wagstaff-etal:2001:constrained-kmeans-clustering}) ;
				\item \texttt{samp.closest.diff}: l'\textbf{échantillonnage de contraintes} concernant des données issues de \textbf{\textit{clusters} différents} et étant \textbf{les plus proches} les unes des autres.
			\end{itemize}
		
			% Temps de calcul.
			Une approximation du temps de calcul d'une itération de \texttt{Clustering Interactif} utilisant ce paramétrage favori est estimée avec l'\textsc{Équation~\ref{equation:5.4-GUIDE-PARAMETRAGES-TEMPS-CALCUL}}.
			
			% Equation.
			\begin{equation}
				\label{equation:5.4-GUIDE-PARAMETRAGES-TEMPS-CALCUL}
				\texttt{computation\_time}~[s]~\propto~0.17 \cdot \texttt{dataset\_size}
			\end{equation}
		
		
		%%% Parallélisation.
		\paragraph{\textcolor{colorSilverLakeBlue}{\faCheckSquare} Choix d'architecture}
		
			% Description.
			Au lieu de réaliser les étapes du \texttt{Clustering Interactif} de manière séquentielle, il peut être intéressant d'\textbf{implémenter une architecture parallèle} de notre méthode (voir \textsc{Figure~\ref{figure:5.4-GUIDE-PARAMETRAGES-ARCHITECTURE-PARALLELE}}).
			Une telle approche consiste à faire coïncider les temps de calcul et d'annotation dans le but de diminuer les temps d'attente : une approximation du nombre de contraintes déterminant l'équivalence de ces durées est estimée avec l'\textsc{Équation~\ref{equation:5.4-GUIDE-PARAMETRAGES-ANNOTATION-BATCH}}.
			
			% Equation.
			\begin{equation}
				\label{equation:5.4-GUIDE-PARAMETRAGES-ANNOTATION-BATCH}
				\texttt{annotation\_batch\_size}~[\#]~\propto~0.0218 \cdot \texttt{dataset\_size}
			\end{equation}
			
			% Figure.
			\begin{figure}[H]
				\centering
				\includegraphics[width=0.85\textwidth]{figures/interactive-clustering-architecture-parallele}
				\caption{
					Schéma illustrant l'architecture du \texttt{Clustering Interactif} en mode parallèle.
				}
				\label{figure:5.4-GUIDE-PARAMETRAGES-ARCHITECTURE-PARALLELE}
			\end{figure}
	
	
	%%%%%--------------------------------------------------------------------
	%%%%% Section 5.5: Estimation des coûts de la méthode.
	%%%%%--------------------------------------------------------------------
	\newpage
	\section{Estimation des coûts de la méthode}
	\label{section:5.5-GUIDE-COUTS}
	
		% Introduction.
		Dans nos études, et principalement au cours de la \textsc{Section~\ref{section:4.3-HYPOTHESE-COUTS}}, nous avons pu estimer un ensemble de coûts théoriques (temporels et humains).
		
		
		%%% Vitesse d'annotation.
		\paragraph{\textcolor{colorSilverLakeBlue}{\faCheckSquare} Vitesse d'annotation}
			
			% Description.
			À l'aide d'une expérience en situation réelle, nous avons estimé qu'\textbf{il faut environ $7.8$ secondes pour caractériser une contrainte}.
			Nous avons montré que ce temps est inférieur à celui d'une annotation d'étiquettes faisant intervenir une modélisation ;
			l'annotation de contraintes est donc \textit{a priori} moins complexe et plus rapide.
		
		
		%%% Nombre de contraintes.
		\paragraph{\textcolor{colorSilverLakeBlue}{\faCheckSquare} Nombre de contraintes nécessaires}
		
			% Description.
			À l'aide d'une simulation, nous avons estimé qu'\textbf{il faut environ $3.15$ contraintes par donnée pour obtenir une base d'apprentissage stable} (\textit{bien entendu, cette estimation doit varier grandement en fonction la complexité des données et la finalité du modèle à entraîner}).
			Nous rappelons qu'un expert métier devra confirmer la pertinence du \textit{clustering} avant d'arrêter la méthode.
			Il est toutefois intéressant de noter que cette estimation du nombre de contraintes nécessaires est linéaire alors que le nombre de contraintes possibles augmente de manière quadratique avec le nombre de données.
		
		
		%%% Temps total.
		\paragraph{\textcolor{colorSilverLakeBlue}{\faCheckSquare} Temps total théorique}
		
			% Description.
			En utilisant une architecture parallèle, nous estimons qu'\textbf{il faut environ $24.6$ secondes par donnée pour obtenir une base d'apprentissage stable}.
			Nous pouvons aussi noter qu'une telle approche propose un nombre constant de $144.5$ itérations pour annoter le nombre de contraintes requis afin d'obtenir une base d'apprentissage stable.
			
			% Comparaison avec une approche classique.
			Une telle estimation semble compétitive avec une annotation d'étiquettes faisant intervenir une modélisation : en effet, le temps d'annotation est légèrement plus long (\textit{à cause du nombre de contraintes à traiter}), mais cela économise le temps nécessaire à la formation des experts ainsi que le temps alloué aux nombreuses révisions de la modélisation abstraite des intentions.
		
		
		%%% Robustesse et nombre d'annotateurs.
		\paragraph{\textcolor{colorSilverLakeBlue}{\faCheckSquare} Coûts d'analyse et de robustesse}
			
			% Attention.
			Il est à noter que \textbf{de nombreux coûts n'ont pas pu être estimés} (\textit{ces derniers étaient impossibles à simuler ou à reproduire}) :
			\begin{itemize}
				% Temps d'analyse.
				\item Le \textbf{temps nécessaire à l'analyse de la pertinence et de la rentabilité}
				(\textit{les gains de temps liés à nos approches pour assister ces étapes n'ont pas pu être estimés}).
				% Trois annotateurs.
				\item Nous avons conseillé d'\textbf{employer plusieurs experts annotant les mêmes données} : ce choix augmente le coût d'embauche, mais cet investissement garantit la cohérence des annotations et évite la conception d'une modélisation abstraite des données.
				% Temps de revue et de débats.
				\item Le \textbf{temps nécessaire aux revues d'annotations et aux débats d'opinion entre annotateurs} : de tels coûts existent aussi pour les approches traditionnelles, cependant les échanges sont ici davantage concrets car ils se basent sur les similarités de cas d'usage.
			\end{itemize}
			
			% Conseils.
			Par conséquent, \textbf{nous conseillons d'utiliser les estimations ci-dessous avec une marge d'erreur proportionnelle à la quantité et à la complexité des données à annoter}.
			
		% Equation.
		\begin{equation}
			\label{equation:5.5-GUIDE-COUTS}
			\begin{cases}
				\texttt{annotation\_time}~[s] &
					~\propto~7.8 \cdot \texttt{batch\_size} \\
				\texttt{constraints\_needed}~[\#] &
					~\propto~3.15 \cdot \texttt{dataset\_size} \\
				% Total time.
				\texttt{total\_time}~[s] &
					~\propto~24.6 \cdot \texttt{dataset\_size} \\
				% Iterations number.
				\texttt{iterations\_needed}~[\#] &
					~\propto~144.5
			\end{cases}
		\end{equation}
	
	
	%%%%%--------------------------------------------------------------------
	%%%%% Section 5.6: Conseils pour rédiger le guide d'annotation.
	%%%%%--------------------------------------------------------------------
	\newpage
	\section{Conseils pour rédiger le guide d'annotation}
		\label{section:5.6-GUIDE-REDIGER}
		
		% Introduction.
		Nous terminons ce chapitre de bilan en listant quelques conseils pour bien cadrer un projet de conception de base d'apprentissage utilisant notre méthode d'annotation.
		\begin{leftBarInformation}
			Nous nous référons aux $7$ maximes proposées par \cite{leech:1993:corpus-annotation-schemes} (et détaillées par \cite{fort:2022:manual-annotation-what}) que nous allons adapter pour le cas d'annotation de contraintes.
		\end{leftBarInformation}
		
		% Conseils basés sur les 7 maximes de Leech.
		\begin{enumerate}
			%  1. It should always be possible to come back to initial data.
			\item \textbf{La donnée initiale doit toujours pouvoir être accessible} \footnote{
				Maxime $1$ : \textguillemets{\textit{It should always be possible to come back to initial data.}}
			} :
			en effet, il faut faire attention aux étapes de prétraitements et de nettoyage qui peuvent effacer des informations potentiellement importantes pour l'analyse.
			% 2. Annotations should be extractable from the text.
			\item \textbf{L'annotation d'une contrainte doit se baser sur des similarités ou des différences observables} \footnote{
				Maxime $2$ : \textguillemets{\textit{Annotations should be extractable from the text.}}
			} :
			en effet, si aucun indice ne permet de faire un choix objectif, il est préférable de s'abstenir (\textit{voire de supprimer la donnée si elle est trop ambiguë}).
			% 3. The annotation procedure should be documented.
			\item \textbf{Il faut documenter l'objectif de l'annotation de contraintes et expliciter clairement sur quels critères elle se base} \footnote{
				Maxime $3$ : \textguillemets{\textit{The annotation procedure should be documented.}}
			} :
			d'une part, il est important de décrire sur quoi caractériser la similarité entre les données (\textit{sur l'action ? sur l'objet de l'action ? sur le sentiment associé ? ...}) ;
			d'autre part, cette documentation doit être complétée au fur et à mesure des revues d'annotation et des résolutions d'incohérences dans la base de contraintes (\textit{dans telle situation, la similarité sera caractérisée de telle manière}).
			% 4. Mention should be made of the annotator(s) and the way annotation was made.
			\item \textbf{Il est important de  décrire les opérateurs réalisant les annotations} \footnote{
				Maxime $4$ : \textguillemets{\textit{Mention should be made of the annotator(s) and the way annotation was made.}}
			} :
			leur nombre, leur expertise du sujet traité, leur formation à la tâche d'annotation, les outils d'assistance à leur disposition, les biais potentiels au cours de l'annotation, \textit{etc}.
			% 5. Annotation is an act of interpretation (cannot be infallible).
			\item \textbf{L'annotation est toujours une action d'interprétation} \footnote{
				Maxime $5$ : \textguillemets{\textit{Annotation is an act of interpretation (cannot be infallible).}}
			} : elle est donc forcément subjective et entraînera des différences d'annotations qui devront être discutées lors de revues entre opérateurs.
			% 6. Annotation schemas should be as independent as possible on formalisms.
			\item \textbf{Il est nécessaire d'être rigoureux dans l'annotation de contraintes pour ne pas introduire des incohérences ou produire des regroupements imprévus} \footnote{
				Maxime $6$ : \textguillemets{\textit{Annotation schemas should be as independent as possible on formalisms.}}
			} :
			dans le doute, il est préférable de s'abstenir de caractériser une contrainte et de laisser la machine décider du regroupement le plus adéquat au regard du reste du jeu de données.
			% 7. No annotation schema should consider itself a standard (it possibly becomes one).
			\item \textbf{À cause de la subjectivité de la tâche, plusieurs visions peuvent être envisagées pour annoter la similarité} \footnote{
				Maxime $7$ : \textguillemets{\textit{No annotation schema should consider itself a standard (it possibly becomes one).}}
			} :
			l'important est de pouvoir en discuter (\textit{au moins à trois personnes}), de choisir le point de vue qui semble le plus adapté au regard de la finalité du modèle à entraîner, et de s'y tenir pour limiter les incohérences.
		\end{enumerate}

%%%%%--------------------------------------------------------------------
%%%%% Conclusion
%%%%%--------------------------------------------------------------------
\chapter{Conclusion}
\label{chapter:6-CONCLUSION}
	
	%%% Rappel objectif et bilan.
	Au cours de ce manuscrit, nous avons abordé la difficulté de la tâche d'annotation 
	
	
	%%%%%--------------------------------------------------------------------
	%%%%% Section 6.1:
	%%%%%--------------------------------------------------------------------
	\section*{?}
	\addcontentsline{toc}{section}{
		\protect\numberline{}
		S6.1
	}  % A RENOMMER
		
		\todo[inline]{SECTION 6.1: À RÉDIGER}
	
	
	%%%%%--------------------------------------------------------------------
	%%%%% Section 6.2:
	%%%%%--------------------------------------------------------------------
	\section*{?}
	\addcontentsline{toc}{section}{
		\protect\numberline{}
		S6.2
	}  % A RENOMMER
		
		\todo[inline]{SECTION 6.2: À RÉDIGER}
	
	
	%%%%%--------------------------------------------------------------------
	%%%%% Section 6.3:
	%%%%%--------------------------------------------------------------------
	\section*{?}
	\addcontentsline{toc}{section}{
		\protect\numberline{}
		S6.3
	}  % A RENOMMER
		
		\todo[inline]{SECTION 6.3: À RÉDIGER}
		% RLHF \cite{griffith-etal:2013:policy-shaping-integrating}, \cite{arumugam-etal:2019:deep-reinforcement-learning}
	
	
	%%%%%--------------------------------------------------------------------
	%%%%% Section 6.4:
	%%%%%--------------------------------------------------------------------
	\section*{?}
	\addcontentsline{toc}{section}{
		\protect\numberline{}
		S6.4
	}  % A RENOMMER
		
		\todo[inline]{SECTION 6.4: À RÉDIGER}


%%%%%%%%%%%%%%%%%%%%%%%%%%%%%%%%%%%%%%%%%%%%%%%%%%%%%%%%%%%%%%%%%%%%%%%
%%%%% ANNEXES
%%%%%%%%%%%%%%%%%%%%%%%%%%%%%%%%%%%%%%%%%%%%%%%%%%%%%%%%%%%%%%%%%%%%%%%

\PutLineInToc
\Annexes

%%%%%--------------------------------------------------------------------
%%%%% Annexe A: JEUX DE DONNÉES
%%%%%--------------------------------------------------------------------
%\newpage
\DontFrameThisInToc
\Annex{Jeux de données utilisés dans nos études}
\label{annex:A-ANNEXE-DATASET}

	% INTRODUCTION DE L'ANNEXE.
	Pour les différentes études réalisées au cours de ce doctorat (cf. \textsc{Chapitre~\ref{chapter:4-ETUDES}}), nous avons utilisé les deux jeux de données suivants.

	% TABLE DES MATIÈRES DE L'ANNEXE.
	\minitoc


	%%%%%--------------------------------------------------------------------
	%%%%% Annexe A.1: \texttt{Bank Cards}: Jeu d'entraînement en français d'assistants conversationnels traitant des demandes courantes sur les cartes bancaires
	%%%%%--------------------------------------------------------------------
	\newpage
	\section[
		Jeu de données \texttt{Bank Cards}
	]{
		\texttt{Bank Cards}: Jeu d'entraînement en français d'assistants conversationnels traitant des demandes courantes sur les cartes bancaires
	}
	\label{annex:A.1-DATASET-BANK-CARDS}
		
		
		% Description.
		\paragraph{Description :}
		Cet ensemble de données représente des exemples de demandes usuelles des clients concernant la gestion des cartes bancaires.
		Il peut être utilisé comme jeu d'entraînement pour un petit assistant conversationnel destiné à traiter ces demandes courantes.
		
		% Contenu.
		\paragraph{Contenu :}
		Les questions sont formulées en français.
		L'ensemble de données est divisé en $10$ intentions (classes) dont un aperçu est disponible dans la \textsc{Table~\ref{table:A.1-DATASET-BANK-CARDS}}.
		Ces intentions sont construites de telle manière que toutes les questions issues d'une même intention ont la même réponse ou action.
		La version \texttt{1.0.0} du jeu de données contient $50$ questions par intention, soit un total de $500$ questions ;
		La version \texttt{2.0.0} du jeu de données contient $100$ questions par intention, soit un total de $1~000$ questions.
		
		\begin{table}[!htb]
			\begin{center}
			\begin{scriptsize}
			\begin{tabular}{|c|c|c|c|}
			
				\hline
				% ENTETE DU TABLEAU
				\rowcolor{colorTableHeader!15}
				\textbf{Intention}
					& \textbf{Définition}
					& \textbf{Exemple}
					\tabularnewline
					\hline \hline
				% alerte\_perte\_vol\_carte
				\multirow{2}{*}{\texttt{alerte\_perte\_vol\_carte}}
					& Affichage de la procédure de blocage
					& \textit{Comment signaler une perte}
					\tabularnewline
					& d'une carte perdue ou volée
					& \textit{de carte de paiement ?}
					\tabularnewline
					\hline
				% carte\_avalee
				\multirow{2}{*}{\texttt{carte\_avalee}}
					& Affichage de la procédure de
					& \textit{Comment récupérer}
					\tabularnewline
					& récupération d'une carte avalée
					& \textit{une carte avalée ?}
					\tabularnewline
					\hline
				% commande\_carte
				\multirow{2}{*}{\texttt{commande\_carte}}
					& Affichage des cartes disponibles,
					& \textit{Je souhaite changer}
					\tabularnewline
					& de la procédure de commande,
					& \textit{de carte bancaire.}
					\tabularnewline
					\hline
				% consultation\_solde
				\multirow{2}{*}{\texttt{consultation\_solde}}
					& Affichage d'une synthèse des
					& \textit{Où retrouver le solde}
					\tabularnewline
					& soldes bancaires du client.
					& \textit{ de mon compte ?}
					\tabularnewline
					\hline
				% couverture\_assurrance
				\multirow{2}{*}{\texttt{couverture\_assurrance}}
					& Affichage d'une synthèse des garanties
					& \textit{Que couvre ma carte bancaire}
					\tabularnewline
					& d'assurances de la carte bancaire du client
					& \textit{en cas d'hospitalisation ?}
					\tabularnewline
					\hline
				% deblocage\_carte
				\multirow{2}{*}{\texttt{deblocage\_carte}}
					& Affichage de gestion du statut
					& \textit{ma carte de paiement est}
					\tabularnewline
					& des cartes du client.
					& \textit{bloquée, que faire ?}
					\tabularnewline
					\hline
				% gestion\_carte\_virtuelle
				\multirow{2}{*}{\texttt{gestion\_carte\_virtuelle}}
					& Affichage de gestion des cartes
					& \textit{Comment faire pour créer une}
					\tabularnewline
					& virtuelles du client.
					& \textit{carte de paiements virtuelle ?}
					\tabularnewline
					\hline
				% gestion\_decouvert
				\multirow{2}{*}{\texttt{gestion\_decouvert}}
					& Affichage d'une synthèse des autorisations
					& \textit{Est-ce que j'ai un}
					\tabularnewline
					& de découverts de leur procédure de gestion
					& \textit{découvert autorisé ?}
					\tabularnewline
					\hline
				% gestion\_plafond
				\multirow{2}{*}{\texttt{gestion\_plafond}}
					& Affichage de gestion des plafonds
					& \textit{Le plafond de ma carte est}
					\tabularnewline
					& des cartes du client.
					& \textit{trop bas, que faire ?}
					\tabularnewline
					\hline
				% gestion\_sans\_contact
				\multirow{2}{*}{\texttt{gestion\_sans\_contact}}
					& Affichage de gestion des
					& \textit{Je veux désactiver le}
					\tabularnewline
					& fonctionnalités des cartes du client.
					& \textit{sans contact sur ma carte.}
					\tabularnewline
					\hline
			\end{tabular}
			\end{scriptsize}
			\end{center}
			\caption{
				Présentation du jeu de données \texttt{Bank Cards} avec quelques exemples.
				La version \texttt{2.0.0} contient $100$ questions par intention.
			}
			\label{table:A.1-DATASET-BANK-CARDS}
		\end{table}
		
		% Origine.
		\paragraph{Origine :}
		Le périmètre des intentions est inspiré d'un chatbot actuellement en production.
		Les données ont été sélectionnées aléatoirement et reformulées manuellement pour garantir la confidentialité des utilisateurs : aucune donnée personnelle ne subsiste dans ce jeu de données.
		Enfin, deux réviseurs extérieurs à l'équipe de recherche (\textit{des Data Analyst}), ayant un profil d'analystes métiers du domaine bancaire, ont validé le périmètre et le contenu de ces intentions.
		
		% Disponibilité.
		\paragraph{Disponibilité :}
		Le jeu de données est archivé sur la plateforme \texttt{Zenodo} et est accessible ici: \cite{schild:2022:french-trainset-chatbots}.


	%%%%%--------------------------------------------------------------------
	%%%%% Annexe A.2: \texttt{MLSUM} (The Multilingual Summarization Corpus): Échantillon de titres d'articles de journaux en français associés à leur classification thématique
	%%%%%--------------------------------------------------------------------
	\newpage
	\section[
		Jeu de données \texttt{MLSUM}
	]{
		\texttt{MLSUM} (The Multilingual Summarization Corpus): Échantillon de titres d'articles de journaux en français associés à leur classification thématique
	}
	\label{annex:A.2-DATASET-MLSUM-SUBSET-SCHILD}
		
		
		% Description.
		\paragraph{Description :}
		C'est un ensemble d'articles de journaux avec leur titre, leur résumé et leur classification thématique.
		Nous l'utilisons (1) pour estimer le temps nécessaire pour annoter la similarité des titres avec des contraintes (\texttt{MUST-LINK}, \texttt{CANNOT-LINK}) et (2) pour tester la méthodologie de \texttt{Clustering Interactif} (annotation de contraintes et \textit{clustering} sous contraintes).
		
		% Contenu.
		\paragraph{Contenu :}
		Les titres de journaux sont formulés en français.
		L'ensemble de données est divisé en $14$ thèmes (classes) dont un aperçu est disponible dans la \textsc{Table~\ref{table:A.2-DATASET-MLSUM-SUBSET-SCHILD}}.
		La version \texttt{1.0.0 [subset: fr+train+filtered]} contient $744$ articles.
		
		\begin{table}[!htb]
			\begin{center}
			\begin{scriptsize}
			\begin{tabular}{|c|c|c|c|}
			
				\hline
				% ENTETE DU TABLEAU
				\rowcolor{colorTableHeader!15}
				\textbf{Thème}
					& \textbf{Définition}
					& \textbf{Exemple}
					& \textbf{Taille}
					\tabularnewline
					\hline \hline
				% arts
				\multirow{2}{*}{\texttt{arts}}
					& Actualités artistiques (spectacles,
					& \textit{La rencontre de l'art et de la }
					& \multirow{2}{*}{$50$}
					\tabularnewline
					& oeuvres, événements, expositions)
					& \textit{gastronomie au château du Feÿ}
					&
					\tabularnewline
					\hline
				% disparitions
				\multirow{2}{*}{\texttt{disparitions}}
					& Actualités nécrologiques 
					& \textit{Le traducteur Jean-Pierre}
					& \multirow{2}{*}{$48$}
					\tabularnewline
					& (décès ou disparition)
					& \textit{Carasso est mort à 73 ans}
					&
					\tabularnewline
					\hline
				% ecologie
				\multirow{2}{*}{\texttt{ecologie}}
					& Actualités sur la pollution
					& \textit{Comment Lyon a banni les}
					& \multirow{2}{*}{$34$}
					\tabularnewline
					& et la transition écologique
					& \textit{pesticides de ses parcs et jardins}
					&
					\tabularnewline
					\hline
				% economie
				\multirow{2}{*}{\texttt{economie}}
					& Actualités économiques,
					& \textit{La guerre des prix s'intensifie}
					& \multirow{2}{*}{$41$}
					\tabularnewline
					& financières et boursières
					& \textit{sur le marché du mobile en Israël}
					&
					\tabularnewline
					\hline
				% education
				\multirow{2}{*}{\texttt{education}}
					& Actualités liées à l'éducation
					& \textit{Plainte de parents d'élève sur des notes}
					& \multirow{2}{*}{$62$}
					\tabularnewline
					& et à la filière enseignante
					& \textit{jugées trop basses au bac}
					&
					\tabularnewline
					\hline
				% emploi
				\multirow{2}{*}{\texttt{emploi}}
					& Actualités liées au marché du
					& \textit{Plus d'un tiers des CDI prennent}
					& \multirow{2}{*}{$54$}
					\tabularnewline
					& travail et aux actions syndicales
					& \textit{ fin avant la première année}
					&
					\tabularnewline
					\hline
				% immobilier
				\multirow{2}{*}{\texttt{immobilier}}
					& Actualités liées au marché de
					& \textit{Depuis la fin des années 2000, l'accession}
					& \multirow{2}{*}{$65$}
					\tabularnewline
					& l'immobilier et logements locatifs
					& \textit{à la propriété se complique en France}
					&
					\tabularnewline
					\hline
				% meteo
				\multirow{2}{*}{\texttt{meteo}}
					& Actualités météorologiques
					& \textit{L'Eure et l'est de la France}
					& \multirow{2}{*}{$35$}
					\tabularnewline
					& (bulletins, catastrophes, canicule)
					& \textit{balayés par les intempéries}
					&
					\tabularnewline
					\hline
				% musiques
				\multirow{2}{*}{\texttt{musiques}}
					& Actualités liées aux chanteurs,
					& \textit{Opéra : Elsa Dreisig,}
					& \multirow{2}{*}{$55$}
					\tabularnewline
					& concerts et sorties d'albums
					& \textit{une soprano à voix nue}
					&
					\tabularnewline
					\hline
				% police-justice
				\multirow{2}{*}{\texttt{police-justice}}
					&Actualités liées aux affaires
					& \textit{Bygmalion : Nicolas Sarkozy}
					& \multirow{2}{*}{$67$}
					\tabularnewline
					& policières et aux tribunaux
					& \textit{directement visé}
					&
					\tabularnewline
					\hline
				% politique
				\multirow{2}{*}{\texttt{politique}}
					& Actualités de la scène
					& \textit{Le Sénat donne son aval à la}
					& \multirow{2}{*}{$52$}
					\tabularnewline
					& politique et législative
					& \textit{prolongation de l'état d'urgence}
					&
					\tabularnewline
					\hline
				% sante
				\multirow{2}{*}{\texttt{sante}}
					& \multirow{2}{*}{Actualités sanitaires}
					& \textit{Chine : un nouveau}
					& \multirow{2}{*}{$70$}
					\tabularnewline
					&
					& \textit{cas de grippe aviaire H7N9}
					&
					\tabularnewline
					\hline
				% sciences
				\multirow{2}{*}{\texttt{sciences}}
					& Actualités scientifiques
					& \textit{L'ordinateur quantique}
					& \multirow{2}{*}{$47$}
					\tabularnewline
					& et vulgarisation
					& \textit{au banc d'essai}
					&
					\tabularnewline
					\hline
				% sport
				\multirow{2}{*}{\texttt{sport}}
					& \multirow{2}{*}{Actualités sportives}
					& \textit{F1 : Webber partira}
					& \multirow{2}{*}{$64$}
					\tabularnewline
					&
					& \textit{en tête à Monaco}
					&
					\tabularnewline
					\hline
			\end{tabular}
			\end{scriptsize}
			\end{center}
			\caption{
				Présentation du jeu de données échantillonné à partir de \texttt{MLSUM} avec quelques exemples.
			}
			\label{table:A.2-DATASET-MLSUM-SUBSET-SCHILD}
		\end{table}
		
		% Origine.
		\paragraph{Origine :}
		L'ensemble de données \texttt{MLSUM} a été proposé par \cite{scialom-etal:2020:mlsum-multilingual-summarization}.
		Notre ensemble de données en est un échantillon (\textit{sélection au hasard de $75$ articles dans les $14$ sujets les plus utilisés}) filtré (\textit{conservation des articles qui ont un sujet évident par rapport à leur titre, sans leur corps}).
		Deux réviseurs (\textit{une Data Scientist et moi-même}) ont travaillé sur cette tâche afin de limiter la subjectivité du filtrage : l'échantillon final contient $744$ articles.
		
		% Disponibilité.
		\paragraph{Disponibilité :}
		Le jeu de donnée original est archivé sur \texttt{arXiv} et est accessible ici : \cite{scialom-etal:2020:mlsum-multilingual-summarization}.
		L'échantillon réalisé par nos soins est archivé sur la plateforme \texttt{Zenodo} et est accessible ici: \cite{schild-adler:2023:subset-mlsum-multilingual}.


%%%%%--------------------------------------------------------------------
%%%%% Annexe B: ANNEXE CHATBOT
%%%%%--------------------------------------------------------------------
%\newpage
\Annex{Annexe des architectures d'assistants conversationnels (\textit{chatbot})}
\label{annex:B-ANNEXE-CHATBOT}
	
	% INTRODUCTION DE L'ANNEXE.
	
	% Engoument pour les chatbots en 2019.
	Au début de ce doctorat (\texttt{octobre 2019}), nous pouvions noter que :
	\begin{itemize}
		\item selon \cite{costello-lodolce:2019:gartner-top-technologies}, \textguillemets{seuls $4$\% des clients de \texttt{Gartner} [déclaraient] utiliser des \textit{chatbot} sur leur lieu de travail, mais $40$\% [avaient] l'intention de les mettre en oeuvre à court terme} ;
		\item et selon \cite{goasduff:2019:chatbots-will-appeal}, \textguillemets{d'ici 2022, 70\% des employés [interagiraient] quotidiennement avec les plateformes conversationnelles}.
	\end{itemize}
	
	% Engoument pour les chatbots en 2023.
	Aujourd'hui (\texttt{octobre 2023}), le mot \textit{chatbot} est sur toutes les lèvres, surtout depuis la révolution des \texttt{IA} génératives lancées par \texttt{ChatGPT} (\cite{openai:2023:chatgpt}) :
	\begin{itemize}
		\item selon \cite{costello-lodolce:2022:gartner-predicts-chatbots}, une entreprise sur deux aurait actuellement recours à une forme de \textit{chatbot} pour gérer sa relation client, et \textguillemets{d'ici 2027, les \textit{chatbot} deviendront le principal canal de service client pour environ un quart des organisations}.
	\end{itemize}

	% Annonce du plan.
	Dans cette annexe, nous allons détailler brièvement les philosophies de conceptions d'assistants conversationnels.
	
	% Citation.
	\begin{leftBarInformation}
		Cette présentation s'inspire des articles de \cite{chen-etal:2017:survey-dialogue-systems} et de \cite{adamopoulou-moussiades:2020:overview-chatbot-technology}.
	\end{leftBarInformation}
	
	% TABLE DES MATIÈRES DE L'ANNEXE.
	\minitoc
	
	
	% Exemple de projet task-oriented: \cite{yan-etal:2017:building-taskoriented-dialogue}
	
	% \cite{brabra-etal:2022:dialogue-management-conversational}: plusieurs système de gestion du dialogue : soit implémenté manuellement, soit entraîné, soit hybride.
	
	% Note auteur: une confusion fréquente est "task-based" => "not deep-learning"
	
	% % Introduction: classification des \textit{chatbots} en .
	\todo[inline]{A REDIGER}
	D'après \cite{chen-etal:2017:survey-dialogue-systems} et \cite{adamopoulou-moussiades:2020:overview-chatbot-technology}, nous pouvons distinguer grossièrement les \textit{chatbots} en deux approches en fonction de leur objectif principal :
	\begin{itemize}
		\item d'une part, il y a les \textit{chatbots} \textguillemets{\textit{task-oriented}}, axés sur l'accomplissement d'une tâche précise ;
		\item d'autre part, il y a les \textit{chatbots} \textguillemets{\textit{chat-oriented}}, dont l'utilité première est basée sur leur capacité à entretenir une conversation avec l'utilisateur.
	\end{itemize}
	
	La \textsc{Figure~\ref{figure:B.2-CHATBOT-ARCHITECTURES}} représente ces deux approches par leurs architectures les plus communes.
	\begin{figure}[!htb]
		\centering
		\includegraphics[width=0.95\textwidth]{figures/annexe-chatbots-architectures}
		\caption{
			Schéma illustrant les deux architecture usuelles pour implémenter un assistant conversationnel :
			\textbf{(1)} représente les \textbf{approches symboliques} avec un schéma d'architecture de gestion d'états de dialogue,
			et \textbf{(2)} représente les \textbf{approches génératives} avec un schéma d'architecture à base de \textit{transformers}, composée d'un encodeur et d'un décodeur.
		}
		\label{figure:B.2-CHATBOT-ARCHITECTURES}
	\end{figure}
	
	
	%%%%%--------------------------------------------------------------------
	%%%%% Annexe B.1: Classification des \textit{chatbots} suivant leurs objectifs.
	%%%%%--------------------------------------------------------------------

		% \subsection{Les \textit{chatbots} \textguillemets{\textit{task-oriented}}}
		% \label{annex:B.1.1-CHATBOT-CLASSIFICATION-TASK-ORIENTED}
		
			% % Définition.
				% Un \textit{chatbot} \textguillemets{\textit{task-oriented}} est conçu pour \textbf{accomplir une tâche spécifique}.
				% Le parcours de dialogue proposé est généralement prédéfini et comporte peu de digressions : l'assistant détermine l'action à réaliser grâce à l'énoncé de l'utilisateur, demande éventuellement des compléments d'informations si la requête n'est pas assez précise, puis effectue sa mission.
				% Le périmètre de fonctionnalités est généralement restreint pour contrôler les actions de l'assistant et s'assurer de son comportement.
			
			% % Exemples de fonctionnalités.
				% Gérer la relation client (\textit{suivi de commande, formulaire de satisfaction, ...}) ;
				% Accéder à des informations documentaires ou personnelles (\textit{accès à une documentation technique, accès au solde d'un compte bancaire, ...}) ;
				% Pré-remplir d'un formulaire (\textit{réserver un billet de voyage, payer ses contraventions, faire opposition à sa carte de crédit, ...}) ;
				 % Gérer la domotique (\textit{allume la lumière, joue de la musique, active l'alarme, ...}).
			
			% % Exemples d'assistants conversationnels connus.
				% \texttt{Assistant Virtuel SNCF} (\cite{sncf:2018:agent-virtuel-sncf}) pour la réservation de billets de train ;
				% \texttt{Google Assistant} (\cite{google:2016:google-assistant-your}) et \texttt{Alexa} (\cite{alexa-internet:2018:keyword-research-competitor}) pour la gestion de la domotique.
		
		
		% \subsection{Les \textit{chatbots} \textguillemets{\textit{chat-oriented}}}
		% \label{annex:B.1.2-CHATBOT-CLASSIFICATION-CHAT-ORIENTED}
		
			% % Définition.
				% Un \textit{chatbot} \textguillemets{\textit{chat-oriented}} se concentre davantage sur les interactions avec l'utilisateur dans le but d'\textbf{engager la conversation}.
				% Son objectif principale est de rendre le dialogue agréable.
				% Son périmètre de connaissance n'est en général pas restreint pour pouvoir facilement engager la conversation sur n'importe quel sujet.
		
			% % Exemples de fonctionnalités.
				% Offrir du divertissement (\textit{raconter une histoire ou une blague, organiser un jeu narratif, ...}) ;
				% Proposer une assistance générale (\textit{donner une définition, demander une explication, ...}) ;
				% Simplement discuter (\textit{mimer les interactions sociales}).
			
			% % Exemples d'assistants conversationnels connus.
				% \texttt{ELIZA} (\cite{weizenbaum:1966:eliza-computer-program}) pour simuler un entretien clinique en psychothérapie ;
				% \texttt{AI Dungeon} (\cite{latitude-inc.-oasis-tech-inc.:2019:ai-dungeon}) pour participer à la narration d'histoires interactives ;
				% \texttt{ChatGPT} (\cite{openai:2023:chatgpt}) et \texttt{BARD} (\cite{google:2023:bard-chat-based}) pour discuter avec des larges modèles de langues (\texttt{LLM}).
	
	
	%%%%--------------------------------------------------------------------
	%%%% Annexe B.2: Architectures usuelles des chatbots.
	%%%%--------------------------------------------------------------------
	% \section{Architectures usuelles des chatbots}
	% \label{annex:B.2-CHATBOT-ARCHITECTURES}

		% % Introduction: classification des \textit{chatbots}.
		% Toujours en s'inspirant de \cite{chen-etal:2017:survey-dialogue-systems} et de \cite{adamopoulou-moussiades:2020:overview-chatbot-technology}, nous pouvons distinguer deux types d'approches principales pour implémenter un \textit{chatbot} :
		% \begin{itemize}
			% \item les \textit{approches symboliques}, traditionnellement utilisées pour traiter le langage en essayant d'en réaliser un modélisation abstraite ;
			% \item les \textit{approches génératives}, reproduisant les capacités du langage en employant les capacités actuelles des réseaux de neurones sur des immenses quantités de données.
		% \end{itemize}
		
		% Ces deux types d'approches sont représentées dans la \textsc{Figure~\ref{figure:B.2-CHATBOT-ARCHITECTURES}}.
		
		% \begin{figure}[!htb]
			% \centering
			% \includegraphics[width=0.95\textwidth]{figures/annexe-chatbots-architectures}
			% \caption{
				% Schéma illustrant les deux architecture usuelles pour implémenter un assistant conversationnel :
				% \textbf{(1)} représente les \textbf{approches symboliques} avec un schéma d'architecture de gestion d'états de dialogue,
				% et \textbf{(2)} représente les \textbf{approches génératives} avec un schéma d'architecture à base de \textit{transformers}, composée d'un encodeur et d'un décodeur.
			% }
			% \label{figure:B.2-CHATBOT-ARCHITECTURES}
		% \end{figure}
		
		
		%%
		%% Subsection B.2.1. Les approches symboliques.
		%%
		% \subsection{Les approches symboliques}
		% \label{annex:B.2.1-CHATBOT-ARCHITECTURES-SYMBOLIQUE}
		% \todo[inline]{A REDIGER}
		
			% % Définition.
			% \paragraph{Définition :}
			
				% \cite{schuurmans-frasincar:2020:intent-classification-dialogue}  \\ % definition intention
			
			% % Exemples d'assistants conversationnels connus.
			% \paragraph{Exemples de moteurs connus ayant une approche symbolique :}
			
				% \texttt{RASA} \cite{bocklisch-etal:2017:rasa-open-source}
				% ou \texttt{WATSON} (\cite{hoyt-etal:2016:ibm-watson-analytics}).
		
		
		%%
		%% Subsection B.2.2. Les approches génératives.
		%%
		% \subsection{Les approches génératives}
		% \label{annex:B.2.2-CHATBOT-ARCHITECTURES-GENERATIVE}
		% \todo[inline]{A REDIGER}
		
			% % Définition.
			% \paragraph{Définition :}
			
				% \cite{uszkoreit:2017:transformer-novel-neural}  \\ % architecture transformers
				% \cite{ni-etal:2022:recent-advances-deep}  \\ % avancer en deep learning
				% \cite{openai:2023:chatgpt}  \\ % exemple
				% \cite{touvron-etal:2023:llama-open-foundation}  \\ % exemple
				% \cite{kaddour-etal:2023:challenges-applications-large} % plusieurs challenges
			
			% % Exemples d'assistants conversationnels connus.
			% \paragraph{Exemples de moteurs connus ayant une approche générative :}
				% \texttt{GPT} (\cite{openai:2023:chatgpt})
				% ou \texttt{LLAMA2} (\cite{touvron-etal:2023:llama-open-foundation}).
	
	
	%%%%--------------------------------------------------------------------
	%%%% Annexe B.3: Discussions sur le niveau d'automatisation.
	%%%%--------------------------------------------------------------------
	% \section{Discussions sur le niveau d'automatisation}
	% \label{annex:B.2-CHATBOT-DISCUSSION-AUTOMATISATION}
		
		% \todo[inline]{A REDIGER}
			
		% \begin{leftBarAuthorOpinion}
			% Il est fréquent d'amalgamer approche générative et chatbot \textit{chat-oriented}.
		% \end{leftBarAuthorOpinion}
		
		% \cite{sheridan-verplank:1978:human-computer-control} repris par \cite{parasuraman-etal:2000:model-types-levels} \\ % 10 niveaux de contrôles


%%%%%--------------------------------------------------------------------
%%%%% Annexe C: ANNEXE IMPLENTATIONS
%%%%%--------------------------------------------------------------------
%\newpage
\DontFrameThisInToc
\Annex{Annexe sur nos implémentations de notre \texttt{Clustering Interactif}}
\label{annex:C-ANNEXE-IMPLEMENTATIONS}

	% INTRODUCTION : annoncer 3 librairies.
	Au cours de ce doctorat, nous avons réalisé un ensemble d'implémentations en \texttt{Python} afin de mettre en oeuvre notre méthodologie de \texttt{Clustering Interactif}.
	Celle-ci est répartie en trois librairies :
	\begin{enumerate}
		% cognitivefactory-interactive-clustering
		\item \texttt{cognitivefactory-interactive-clustering} \footnote{
			\url{https://pypi.org/project/cognitivefactory-interactive-clustering/}
		} (\cite{schild:2022:cognitivefactory-interactiveclustering}), regroupant les gestions de données et des contraintes, les algorithmes de \textit{clustering} et d'échantillonnage ;
		% cognitivefactory-interactive-clustering-gui
		\item \texttt{cognitivefactory-interactive-clustering-gui} \footnote{
			\url{https://pypi.org/project/cognitivefactory-interactive-clustering-gui/}
		} (\cite{schild-etal:2022:cognitivefactory-interactiveclusteringgui}), intégrant la logique de la méthodologie dans une application web ;
		% cognitivefactory-features-maximization-metric
		\item \texttt{cognitivefactory-features-maximization-metric} \footnote{
			\url{https://pypi.org/project/cognitivefactory-features-maximization-metric/}
		} (\cite{schild:2023:cognitivefactory-featuresmaximizationmetric}), disposant d'une méthode de sélection des patterns linguistiques caractéristiques d'un jeu de données labellisées, permettant ainsi d'analyser la pertinence d'un résultat de \textit{clustering}.
	\end{enumerate}
	
	%Information : Développements Python.
	\setcounter{localCounterOfFootnoteValue}{\value{footnote}}
	\begin{leftBarInformation}
		Ces implémentations sont disponibles sur le GitHub \url{https://github.com/cognitivefactory}.
		Les \textit{pipeline} d'intégration continue contiennent les étapes
		% Vérification formatage.
		de formatage du code (\textit{grâce aux librairies \texttt{isort} \footnotemark et \texttt{black} \footnotemark}),
		% Vérification qualité et typage.
		d vérification de la qualité et du typage du code (\textit{grâce aux librairies \texttt{flake8} \footnotemark et \texttt{mypy} \footnotemark}),
		% Vérification vulnérabilités.
		de la vérification des vulnérabilités et des failles de sécurité (\textit{grâce à la librairie \texttt{safety} \footnotemark}),
		% Vérification tests unitaires.
		l'exécution de tests unitaires et la vérification de la couverture du code testé (\textit{grâce aux librairies \texttt{pytest} \footnotemark et \texttt{coverage} \footnotemark}),
		% Vérification documentation.
		et la génération de la documentation technique (\textit{grâce à la librairie \texttt{mkdocs} \footnotemark}).
	\end{leftBarInformation}
	% Rattraper les footnote.
		\stepcounter{localCounterOfFootnoteValue}
		\footnotetext[\value{localCounterOfFootnoteValue}]{
			\url{https://pypi.org/project/isort/}
		}
		\stepcounter{localCounterOfFootnoteValue}
		\footnotetext[\value{localCounterOfFootnoteValue}]{
			\url{https://pypi.org/project/black/}
		}
		\stepcounter{localCounterOfFootnoteValue}
		\footnotetext[\value{localCounterOfFootnoteValue}]{
			\url{https://pypi.org/project/flake8/}
		}
		\stepcounter{localCounterOfFootnoteValue}
		\footnotetext[\value{localCounterOfFootnoteValue}]{
			\url{https://pypi.org/project/mypy/}
		}
		\stepcounter{localCounterOfFootnoteValue}
		\footnotetext[\value{localCounterOfFootnoteValue}]{
			\url{https://pypi.org/project/safety/}
		}
		\stepcounter{localCounterOfFootnoteValue}
		\footnotetext[\value{localCounterOfFootnoteValue}]{
			\url{https://pypi.org/project/pytest/}
		}
		\stepcounter{localCounterOfFootnoteValue}
		\footnotetext[\value{localCounterOfFootnoteValue}]{
			\url{https://pypi.org/project/coverage/}
		}
		\stepcounter{localCounterOfFootnoteValue}
		\footnotetext[\value{localCounterOfFootnoteValue}]{
			\url{https://pypi.org/project/mkdocs/}
		}
	
	Dans cette annexe, nous allons détailler ces implémentations, leurs fonctionnalités et certains des choix de mises en oeuvres.
	
	
	% TABLE DES MATIÈRES DE L'ANNEXE.
	\minitoc
	
	
	%%%%%--------------------------------------------------------------------
	%%%%% Section C.1: Implémentation de la librairie \texttt{cognitivefactory-interactive-clustering}
	%%%%%--------------------------------------------------------------------
	\newpage
	\section[
		\texttt{cognitivefactory-interactive-clustering}
	]{
		Implémentation de la librairie \\ \texttt{cognitivefactory-interactive-clustering}
	}
\label{annex:C.1-DESCRIPTION-IMPLEMENTATION-INTERACTIVE-CLUSTERING}
	
	% Généralités.
	La librairie \texttt{cognitivefactory-interactive-clustering} \footnote{
		\url{https://pypi.org/project/cognitivefactory-interactive-clustering/}
	} (\cite{schild:2022:cognitivefactory-interactiveclustering}) a été implémentée au cours de ce doctorat dans le but de mettre à disposition un ensemble d'algorithmes nécessaires à l'utilisation de notre méthodologie de \texttt{Clustering Interactif}.
	Cette librairie comporte plusieurs fonctionnalités :
	\begin{itemize}
		\item la gestion des données avec leurs prétraitements et leur vectorisation (cf. \textsc{Section~\ref{annex:C.1.1-DESCRIPTION-IMPLEMENTATION-INTERACTIVE-CLUSTERING-GESTION-DES-DONNEES}}) ;
		\item la gestion des contraintes avec le calcul des propriétés de transitivité et la détection des conflits (cf. \textsc{Section~\ref{annex:C.1.2-DESCRIPTION-IMPLEMENTATION-INTERACTIVE-CLUSTERING-GESTION-DES-CONTRAINTES}}) ;
		\item l'exécution d'algorithmes de \textit{clustering} sous contraintes pour proposer une segmentation des données (cf. \textsc{Section~\ref{annex:C.1.3-DESCRIPTION-IMPLEMENTATION-INTERACTIVE-CLUSTERING-ALGORITHMES-CLUSTERING-SOUS-CONTRAINTES}}) ;
		\item l'exécution d'algorithmes d'échantillonnage pour sélectionner les prochaines contraintes à annoter (cf. \textsc{Section~\ref{annex:C.1.4-DESCRIPTION-IMPLEMENTATION-INTERACTIVE-CLUSTERING-ALGORITHMES-ECHANTILLONNAGE-DE-CONTRAINTES}}).
	\end{itemize}
	
	Nous présentons succinctement cette librairie avec certains choix d'implémentation.
	
	% Information : comme y accéder.
	\begin{leftBarInformation}
		La documentation technique de cette librairie est accessible au lien suivant : \url{https://cognitivefactory.github.io/interactive-clustering/}.
	\end{leftBarInformation}
	
	% Exemple.
	Pour les sections suivantes, nous suivrons l'exemple suivant (cf. \textsc{Code~\ref{code:C.1-DESCRIPTION-IMPLEMENTATION-INTERACTIVE-CLUSTERING-DATA}}) pour présenter nos implémentations.
	
	\begin{lstlisting}[
		language=Python,
		caption={Jeu exemple pour présenter notre implémentation du \texttt{Clustering Interactif}.},
		label={code:C.1-DESCRIPTION-IMPLEMENTATION-INTERACTIVE-CLUSTERING-DATA},
	]
# Définir les données.
dict_of_texts = {
"0": "Comment signaler un vol de carte bancaire ?",
"1": "J'ai égaré ma carte bancaire, que faire ?",
"2": "J'ai perdu ma carte de paiement",
"3": "Le distributeur a avalé ma carte !",
"4": "En retirant de l'argent, le GAB a gardé ma carte...",
"5": "Le distributeur ne m'a pas rendu ma carte bleue.",
# ...
"N": "Pourquoi le sans contact ne fonctionne pas ?",
}
	\end{lstlisting}
	
	
	%%%
	%%% Subsection C.1.1: Gestion des données.
	%%%
	\subsection{Gestion des données}
	\label{annex:C.1.1-DESCRIPTION-IMPLEMENTATION-INTERACTIVE-CLUSTERING-GESTION-DES-DONNEES}
	
	% cognitivefactory.interactive-clustering.utils
	Tout d'abord, en ce qui concerne la \textbf{manipulation de données}, nous utilisons le module \texttt{utils} de la librairie \texttt{cognitivefactory-interactive-clustering}.
	Les données sont stockées dans un dictionnaire \texttt{Python} afin de tracer les manipulations à l'aide d'une clé servant d'identifiant de la donnée.
	
	% cognitivefactory.interactive-clustering.utils.preprocessing : Implémentation.
	Nous avons d'une part la partie \texttt{utils.preprocessing} \footnote{
		\url{https://cognitivefactory.github.io/interactive-clustering/reference/cognitivefactory/interactive_clustering/utils/preprocessing/}
	} qui permet de normaliser les données.
	Par défaut :
	\begin{itemize}
		\item[\(\bullet\)] le texte est passé en \textit{minuscule} (de \textguillemets{\texttt{Bonjour}} à \textguillemets{\texttt{bonjour}}),
		\item[\(\bullet\)] la \textit{ponctuation} est supprimée \textguillemets{\texttt{c'est-à-dire ?!}} à \textguillemets{\texttt{c est a dire}}), %(\texttt{.}, \texttt{,}, \texttt{;}, \texttt{:}, \texttt{!}, \texttt{¡}, \texttt{?}, \texttt{¿}, \texttt{…}, \texttt{•}, \texttt{(}, \texttt{)}, \texttt{\{}, \texttt{\}}, \texttt{[}, \texttt{]}, \texttt{\textguillemets{}, \texttt{}}, \texttt{^}, \texttt{\`}, \texttt{'}, \texttt{"}, \texttt{\\}, \texttt{/}, \texttt{|}, \texttt{-}, \texttt{\_}, \texttt{#}, \texttt{\&}, \texttt{\~}, \texttt{\@}),
		\item[\(\bullet\)] les \textit{accents} sont enlevés (de \textguillemets{\texttt{crédit}} à \textguillemets{\texttt{credit}}),
		\item[\(\bullet\)] et les multiples \textit{espaces blancs} sont convertis en un unique espace simple (de \textguillemets{\texttt{au~~~~revoir}} à \textguillemets{\texttt{au revoir}}).
	\end{itemize}
	
	Si besoin, trois options "avancées" sont disponibles pour réaliser des prétraitements plus destructif :
	\begin{itemize}
		\item[\(\bullet\)] la suppression des mots vides (\textit{stopwords}, \cite{nothman-etal:2018:stop-word-lists}),
		\item[\(\bullet\)] la conversion des mots vers leur forme racine (\textit{lemmatisation}, \cite{manning-schutze:2000:foundations-statistical-natural}),
		\item[\(\bullet\)] et la suppression des mots en fonction de leur profondeur dans l'arbre de dépendances syntaxiques (\cite{nivre:2006:inductive-dependency-parsing}).
	\end{itemize}
	
	% cognitivefactory.interactive-clustering.utils.preprocessing : Dépendances.
	Ces traitements sont réalisés en bénéficiant des fonctionnalités mises à disposition d'un modèle de langue de type SpaCy (\cite{honnibal-montani:2017:spacy-natural-language}), avec par défaut l'utilisation du modèle \texttt{fr-core-news-md}.
	
	% cognitivefactory.interactive-clustering.utils.preprocessing : Par défaut.
	Pour nos études, nous définissons quatre niveaux de prétraitements facilement identifiables :
	\begin{enumerate}
		\item L'\textbf{absence de prétraitements}, soit la conservation de la donnée brute, noté \texttt{prep.no} ;
		\item Les \textbf{prétraitements simples}, correspondant au traitement traitement de base (minuscules, ponctuations, accents, espaces blancs), notés \texttt{prep.simple} ; 
		\item Les \textbf{prétraitements avec lemmatisation}, correspondant au traitement de base auquel s'ajoute la conversion des mots vers leur forme racine, notés \texttt{prep.lemma} ;
		\item les \textbf{prétraitements avec filtres}, correspondant au traitement de base avec l'élagage de l'arbre de dépendance syntaxique de la phrase, notés \texttt{prep.filter}.
	\end{enumerate}
	
	
	% cognitivefactory.interactive-clustering.utils.vectorization
	D'autre part, la partie \texttt{utils.vectorization} \footnote{
		\url{https://cognitivefactory.github.io/interactive-clustering/reference/cognitivefactory/interactive_clustering/utils/vectorization/}
	} permet de transformer les données en une représentation exploitable pour la machine.
	Deux modes de vectorisation sont mis à disposition :
	\begin{enumerate}
		\item \textbf{TF-IDF} (\cite{ramos:2003:using-tfidf-determine}), utilisant la fréquence d'occurrence des mots pour représenter une phrase, et noté \texttt{vect.tfidf} pour nos études ;
		\item \textbf{SpaCy} (\cite{honnibal-montani:2017:spacy-natural-language}), utilisant le modèle de langue \texttt{fr-core-news-md}, et noté \texttt{vect.frcorenewsmd}.
	\end{enumerate}
	
	% cognitivefactory.interactive-clustering.utils : Exemple.
	Vous avez un exemple d'utilisation des modules de prétraitements et de vectorisation dans \textsc{Code~\ref{code:C.1.1-DESCRIPTION-IMPLEMENTATION-INTERACTIVE-CLUSTERING-GESTION-DONNEES}}.
	
	\begin{lstlisting}[
		language=Python,
		caption={Démonstration de notre implémentation des prétraitements et de la vectorisation sur le jeu d'exemple.},
		label={code:C.1.1-DESCRIPTION-IMPLEMENTATION-INTERACTIVE-CLUSTERING-GESTION-DONNEES},
	]
# Import des dépendances.
from cognitivefactory.interactive_clustering.utils.preprocessing import preprocess
from cognitivefactory.interactive_clustering.utils.vectorization import vectorize

# Prétraitement des données.
dict_of_preprocess_texts = preprocess(
dict_of_texts=dict_of_texts,
apply_stopwords_deletion=False,
apply_parsing_filter=False,
apply_lemmatization=False,
spacy_language_model="fr_core_news_md",
)
"""
{"0": "comment signaler un vol de carte bancaire",
 "1": "j ai egare ma carte bancaire, que faire",
 "2": "j ai perdu ma carte de paiement",
 "3": "le distributeur a avale ma carte",
 "4": "en retirant de l argent le gab a garde ma carte",
 "5": "le distributeur ne m a pas rendu ma carte bleue",
 # ...
 "N": "pourquoi le sans contact ne fonctionne pas"}
"""

# Vectorisation des données.
dict_of_vectors = vectorize(
dict_of_texts=dict_of_preprocess_texts,
vectorizer_type="tfidf",
)
	\end{lstlisting}
	
	%%%
	%%% Subsection C.1.2: Gestion des contraintes
	%%%
	\subsection{Gestion des contraintes}
	\label{annex:C.1.2-DESCRIPTION-IMPLEMENTATION-INTERACTIVE-CLUSTERING-GESTION-DES-CONTRAINTES}
	
	% cognitivefactory.interactive-clustering.constraints
	En ce qui concerne la \textbf{manipulation de contraintes}, nous utilisons le module \texttt{contraints} \footnote{
		\url{https://cognitivefactory.github.io/interactive-clustering/reference/cognitivefactory/interactive_clustering/constraints/}
	} de la librairie \texttt{cognitivefactory-interactive-clustering}.
	
	% cognitivefactory.interactive-clustering.constraints: Types
	Deux types de contraintes sont prises en charge (cf. \cite{wagstaff-cardie:2000:clustering-instancelevel-constraints}) :
	\begin{itemize}
		\item[\(\bullet\)] les contraintes \texttt{MUST-LINK} permettant de réunir deux données,
		\item[\(\bullet\)] et les contraintes \texttt{CANNOT-LINK} permettant à l'inverse de les séparer.
	\end{itemize}

	% cognitivefactory.interactive-clustering.constraints: Transitivité.
	Ces types de contraintes respectent les propriétés de transitivité décrites dans l'\textsc{Equation~\ref{equation:C.1.2-DESCRIPTION-IMPLEMENTATION-INTERACTIVE-CLUSTERING-CONTRAINTES-TRANSITIVITE}}) et sont illustrées dans la \textsc{Figure~\ref{figure:C.1.2-DESCRIPTION-IMPLEMENTATION-INTERACTIVE-CLUSTERING-CONTRAINTES-TRANSITIVITE}} (\textbf{(1)} et \textbf{(2)}).
	Nous notons ainsi qu'il est possible de déduire la troisième contrainte d'un triangle de trois points si nous connaissons déjà les deux premières.
	
	\begin{equation}
		\label{equation:C.1.2-DESCRIPTION-IMPLEMENTATION-INTERACTIVE-CLUSTERING-CONTRAINTES-TRANSITIVITE}
		(\forall A,B,C)~
		\begin{cases}
			% ML + ML => ML
			~\textcolor{colorDarkPastelGreen}{\texttt{MUST\_LINK}}(A,B)
			~\wedge~\textcolor{colorDarkPastelGreen}{\texttt{MUST\_LINK}}(B,C)
			~\Rightarrow~\textcolor{colorDarkPastelGreen}{\texttt{MUST\_LINK}}(A,C)  \\
			% ML + CL => CL
			~\textcolor{colorDarkPastelGreen}{\texttt{MUST\_LINK}}(A,B)
			~\wedge~\textcolor{colorDarkPastelRed}{\texttt{CANNOT\_LINK}}(B,C)
			~\Rightarrow~\textcolor{colorDarkPastelRed}{\texttt{CANNOT\_LINK}}(A,C)
		\end{cases}
	\end{equation}
	
	Pour respecter ces propriétés, le gestionnaire de contraintes doit ainsi calculer les transitivités à chaque ajout ou suppression de contraintes.
	Nous distinguerons donc une contrainte ajoutée (\texttt{added}) d'une contrainte déduite par transitivité (\texttt{inferred}).
	
	% cognitivefactory.interactive-clustering.constraints: Conflits.
	Il se peut que la contrainte en cours d'ajout contredise les contraintes précédemment déduites : nous parlons alors d'incohérence ou de conflit (cf. \textsc{Figure~\ref{figure:C.1.2-DESCRIPTION-IMPLEMENTATION-INTERACTIVE-CLUSTERING-CONTRAINTES-TRANSITIVITE}} et \textsc{Equation~\ref{equation:C.1.2-DESCRIPTION-IMPLEMENTATION-INTERACTIVE-CLUSTERING-CONTRAINTES-CONFLITS}}).
	Dans ce cas, l'ajout de la dernière contrainte n'est pas prise en compte et le gestionnaire renvoie une erreur permettant d'identifier ce conflit.
	Ce conflit peut venir simplement venir d'une erreur d'inattention, mais peut aussi venir d'une déduction basée sur des ajouts antérieurs erronés.
	Sémantiquement, un conflit indique une contradiction dans la gestion des données, car les données concernées doivent à la fois être réunies et séparées...
	
	\begin{equation}
		\label{equation:C.1.2-DESCRIPTION-IMPLEMENTATION-INTERACTIVE-CLUSTERING-CONTRAINTES-CONFLITS}
		(\exists A,B,C)~
		~\textcolor{colorDarkPastelGreen}{\texttt{MUST\_LINK}}(A,B)
		~\wedge~\textcolor{colorDarkPastelGreen}{\texttt{MUST\_LINK}}(B,C)
		~\wedge~\textcolor{colorDarkPastelRed}{\texttt{CANNOT\_LINK}}(A,C)
	\end{equation}
	
	% cognitivefactory.interactive-clustering.constraints: Composants connexe.
	À partir d'une donnée \(D\), et par application de la propriété de transitivité des \texttt{MUST-LINK}, nous appelons \textbf{composant connexe} de \(D\) l'ensemble des données \(D_i\) liées par une succession de contraintes \texttt{MUST-LINK} à \(D\) (cf. \textsc{Figure~\ref{figure:C.1.2-DESCRIPTION-IMPLEMENTATION-INTERACTIVE-CLUSTERING-CONTRAINTES-TRANSITIVITE}}).
	Ce composant peut être vu comme un noyau de \textit{cluster}.
	Il pourra être associé à d'autres noyaux par similarité pour former un \textit{cluster} plus conséquent, ou être distingué d'autres noyaux pour former plusieurs \textit{cluster}.

	\begin{figure}[!htb]
		\centering
		\includegraphics[width=0.70\textwidth]{figures/example-constraints-transitivity}
		\caption{
			Exemples des propriétés de transitivité des contraintes \texttt{MUST-LINK} (flèches vertes) et \texttt{CANNOT-LINK} (flèches rouges). \textbf{(1)} et \textbf{(2)} représente les possibilités de déduction d'une contrainte (\texttt{(c)}) en fonction des deux autres (\texttt{(a)} et \texttt{(b)}). \textbf{(3)} représente deux composants connexes définis par la transitivité des contraintes \texttt{MUST-LINK}. Enfin, \textbf{(4)} représente un cas de conflit où une contrainte (\texttt{(c)}) ne correspond pas à sa déduction faite à partir des autres contraintes (\texttt{(a)} et \texttt{(b)}).
		}
		\label{figure:C.1.2-DESCRIPTION-IMPLEMENTATION-INTERACTIVE-CLUSTERING-CONTRAINTES-TRANSITIVITE}
	\end{figure}
	
	% cognitivefactory.interactive-clustering.constraints : Exemple.
	Un exemple d'utilisation du module de gestion de contraintes est consultable dans \textsc{Code~\ref{code:C.1.2-DESCRIPTION-IMPLEMENTATION-INTERACTIVE-CLUSTERING-GESTION-CONTRAINTES}}.
	
	\begin{lstlisting}[
		language=Python,
		caption={Démonstration de notre implémentation de gestion des contraintes sur le jeu d'exemple.},
		label={code:C.1.2-DESCRIPTION-IMPLEMENTATION-INTERACTIVE-CLUSTERING-GESTION-CONTRAINTES},
	]
# Import des dépendances.
from cognitivefactory.interactive_clustering.constraints.factory import managing_factory

# Création du gestionnaire de contraintes.
constraints_manager = managing_factory(
manager="binary",
list_of_data_IDs = list(dict_of_texts.keys()),  # ["0", "1", "2", "3", "4", "5", ..., "N"]
)

# Ajout de contraintes.
constraints_manager.add_constraint(
data_ID1="0",  # "Comment signaler un vol de carte bancaire ?"
data_ID2="1",  # "J'ai égaré ma carte bancaire, que faire ?"
constraint_type="MUST_LINK",
)
constraints_manager.add_constraint(
data_ID1="3",  # "Le distributeur a avalé ma carte !"
data_ID2="4",  # "En retirant de l'argent, le GAB a gardé ma carte..."
constraint_type="MUST_LINK",
)
constraints_manager.add_constraint(
data_ID1="0",  # "Comment signaler un vol de carte bancaire ?"
data_ID2="N",  # "Pourquoi le sans contact ne fonctionne pas ?"
constraint_type="CANNOT_LINK",
)
# NB: ajouter une contrainte "MUST_LINK" entre "1" et "N" lèverait une erreur.

# Récupération des composants connexes.
connected_components = constraints_manager.get_connected_components()
"""
[['0', '1'],
 ['2'],
 ['3', '4'],
 ['5'],
 ['N']]
"""
	\end{lstlisting}
	
	
	%%%
	%%% Subsection C.1.3: Algorithme de \textit{clustering} sous contraintes.
	%%%
	\subsection{Algorithme de \textit{clustering} sous contraintes}
	\label{annex:C.1.3-DESCRIPTION-IMPLEMENTATION-INTERACTIVE-CLUSTERING-ALGORITHMES-CLUSTERING-SOUS-CONTRAINTES}
	
	% cognitivefactory.interactive-clustering.clustering
	En ce qui concerne le \textbf{regroupement automatique} des données par similarité, nous utilisons le module \texttt{clustering} \footnote{
		\url{https://cognitivefactory.github.io/interactive-clustering/reference/cognitivefactory/interactive_clustering/clustering/}
	} de la librairie \texttt{cognitivefactory-interactive-clustering}.
	
	% cognitivefactory.interactive-clustering.utils.clustering : Implémentation.
	Ce module met à disposition six algorithmes de \textit{clustering} sous contraintes :
	\begin{enumerate}
		\item \textbf{\texttt{KMeans}}, dans sa version \texttt{COP-KMeans} (\cite{wagstaff-etal:2001:constrained-kmeans-clustering}), noté \texttt{clust.kmeans.cop}, et sa version \texttt{MPC-KMeans} (\cite{khan-etal:2012:multiple-parameter-based}), noté \texttt{clust.kmeans.mpc}) ;
		\item \textbf{\texttt{DBscan}}, dans sa version \texttt{C-DBScan} (\cite{ruiz-etal:2010:densitybased-semisupervised-clustering}), noté \texttt{clust.cdbscan} ;
		\item \textbf{Hiérarchique} (\cite{davidson-ravi:2005:agglomerative-hierarchical-clustering}), avec quatre métriques de distances : \texttt{single} (noté \texttt{clust.hier.sing}), \texttt{complete} (noté \texttt{clust.hier.comp}), \texttt{average} (noté \texttt{clust.hier.avg}) et \texttt{ward} (noté \texttt{clust.hier.ward}) ;
		\item \textbf{Spectral}, dans sa version \texttt{SPEC} (\cite{kamvar-etal:2003:spectral-learning}), noté \texttt{clust.spec} ;
		\item \textbf{Propagation par affinité} (\cite{givoni-frey:2009:semisupervised-affinity-propagation}), noté \texttt{clust.affprop}.
	\end{enumerate}
	
	Une classe abstraite définit les prérequis des algorithmes implémentés (avoir une méthode \texttt{cluster}) et une \textit{factory} est disponible pour instancier rapidement un objet de \textit{clustering}.
	% cognitivefactory.interactive-clustering.clustering : Exemple.
	Enfin, un exemple d'utilisation ce module est consultable dans \textsc{Code~\ref{code:C.1.3-DESCRIPTION-IMPLEMENTATION-INTERACTIVE-CLUSTERING-CLUSTERING}}.
	
	
	\begin{lstlisting}[
		language=Python,
		caption={Démonstration de notre implémentation du \textit{clustering} sous contraintes sur le jeu d'exemple.},
		label={code:C.1.3-DESCRIPTION-IMPLEMENTATION-INTERACTIVE-CLUSTERING-CLUSTERING},
	]
# Import des dépendances.
from cognitivefactory.interactive_clustering.clustering.factory import clustering_factory

# Initialiser un objet de clustering.
clustering_model = clustering_factory(
algorithm="kmeans",
model="COP",
random_seed=42,
)

# Lancer le clustering.
clustering_result = clustering_model.cluster(
constraints_manager=constraints_manager,  # contient les contraintes
nb_clusters=2,
vectors=dict_of_vectors,
)
"""
{"0": 0,  # "Comment signaler un vol de carte bancaire ?"
 "1": 0,  # "J'ai égaré ma carte bancaire, que faire ?"
 "2": 0,  # "J'ai perdu ma carte de paiement"
 "3": 1,  # "Le distributeur a avalé ma carte !"
 "4": 1,  # "En retirant de l'argent, le GAB a gardé ma carte..."
 "5": 1,  # "Le distributeur ne m'a pas rendu ma carte bleue."
 # ...
 "N": 1}  # "Pourquoi le sans contact ne fonctionne pas ?"
"""
	\end{lstlisting}
	
	% cognitivefactory.interactive-clustering.utils.clustering : Historique.
	\begin{leftBarInformation}
		Dans le cadre d'un projet étudiant avec l'école d'ingénieur Télécom Physique Strasbourg (au cours de l'année 2022), les implémentations des algorithmes \texttt{MPC-KMeans}, \texttt{C-DBScan} et propagation par affinité ont été ajoutées. Les élèves ont conclu ce projet d'extension en suggérant de se concentrer sur l'étude du C-DBScan car les deux autres algorithmes étaient soit trop instables, soit trop gourmand en temps de calcul.
		Les autres algorithmes (\texttt{COP-KMeans}, hiérarchique et spectral) ont été implémentés au début de ce doctorat.
	\end{leftBarInformation}
	
	
	%%%
	%%% Subsection C.1.4: Algorithme d'échantillonnage de contraintes.
	%%%
	\subsection{Algorithme d'échantillonnage de contraintes}
	\label{annex:C.1.4-DESCRIPTION-IMPLEMENTATION-INTERACTIVE-CLUSTERING-ALGORITHMES-ECHANTILLONNAGE-DE-CONTRAINTES}
	
	% cognitivefactory.interactive-clustering.sampling
	En ce qui concerne l'\textbf{échantillonnage} de contraintes à annoter, nous utilisons le module \texttt{sampling} \footnote{
		\url{https://cognitivefactory.github.io/interactive-clustering/reference/cognitivefactory/interactive_clustering/sampling/}
	} de la librairie \texttt{cognitivefactory-interactive-clustering}.
	
	% cognitivefactory.interactive-clustering.utils.sampling : Implémentation.
	Cet échantillonnage correspond à la sélection de couple de données.
	Par défaut, l'échantillonnage est purement aléatoire.
	Cependant, plusieurs options sont disponibles :
	
	\begin{itemize}
		\item[\(\bullet\)] une restriction sur la \textit{distance} pouvant imposer aux données d'être les plus proches ou les plus éloignées du corpus ;
		\item[\(\bullet\)] une restriction sur le \textit{résultat du clustering} pouvant imposer aux données d'être issues d'un même \textit{cluster} ou de \textit{cluster} différents,
		\item[\(\bullet\)] une restriction pour exclure les contraintes \textit{déjà annotées},
		\item[\(\bullet\)] et enfin une restriction pour exclure les contraintes \textit{déjà déduites} par transitivité.
	\end{itemize}
	
	% cognitivefactory.interactive-clustering.utils.sampling : Par défaut.
	Sur cette base, nous définissons quatre niveaux d'échantillonnage facilement identifiables pour nos études :
	\begin{enumerate}
		\item Un échantillonnage \textbf{purement aléatoire} en excluant toutes les contraintes déjà annotées ou déduites, noté \texttt{samp.random.full} ;
		\item Un échantillonnage \textbf{pseudo-aléatoire} de données issues d'un \textbf{même \textit{cluster}}, en excluant toutes les contraintes déjà annotées ou déduites, noté \texttt{samp.random.same} ;
		\item Un échantillonnage des données issues d'un \textbf{même \textit{cluster}} et étant \textbf{les plus éloignées} les unes des autres, noté \texttt{samp.farhtest.same} (cf. \textsc{Figure~\ref{figure:C.1.4-DESCRIPTION-IMPLEMENTATION-INTERACTIVE-CLUSTERING-CONTRAINTES-SAMPLING}}) ;
		\item Un échantillonnage des données issues de \textbf{\textit{cluster} différents} et étant \textbf{les plus proches} les unes des autres, noté \texttt{samp.closest.diff} (cf. \textsc{Figure~\ref{figure:C.1.4-DESCRIPTION-IMPLEMENTATION-INTERACTIVE-CLUSTERING-CONTRAINTES-SAMPLING}}).
	\end{enumerate}
	
	\begin{figure}[!htb]
		\centering
		\includegraphics[width=0.35\textwidth]{figures/example-sampling}
		\caption{
			Exemples d'échantillonnages, sur la base de trois clusters, de données issues de mêmes \textit{cluster} et étant les plus éloignées les unes des autres (\texttt{samp.farhtest.same}), et de données issues de clusters différents et étant les plus proches les unes des autres (\texttt{samp.closest.diff}).
		}
		\label{figure:C.1.4-DESCRIPTION-IMPLEMENTATION-INTERACTIVE-CLUSTERING-CONTRAINTES-SAMPLING}
	\end{figure}

	Une classe abstraite définit les prérequis des algorithmes implémentés (avoir une méthode \texttt{sample}) et une \textit{factory} est disponible pour instancier rapidement un objet d'échantillonnage.
	% cognitivefactory.interactive-clustering.sampling : Exemple.
	Un exemple d'utilisation ce module est consultable dans \textsc{Code~\ref{code:C.1.4-DESCRIPTION-IMPLEMENTATION-INTERACTIVE-CLUSTERING-SAMPLING}}.
	
	\begin{lstlisting}[
		language=Python,
		caption={Démonstration de notre implémentation de l'échantillonnage sur le jeu d'exemple.},
		label={code:C.1.4-DESCRIPTION-IMPLEMENTATION-INTERACTIVE-CLUSTERING-SAMPLING},
	]
# Import des dépendances.
from cognitivefactory.interactive_clustering.sampling.factory import sampling_factory

# Initialiser un objet d'échantillonnage.
sampler = sampling_factory(
algorithm="random",
random_seed=42,
)

# Run sampling.
selection = sampler.sample(
constraints_manager=constraints_manager,
nb_to_select=2,
clustering_result=clustering_result,  # optionnel pour "random"
vectors=dict_of_vectors,  # optionnel pour "random"
)
"""
[("0", '5"),  # "Comment signaler un vol de carte bancaire ?" vs "Le distributeur ne m'a pas rendu ma carte bleue."
 ("0", '2"),  # "Comment signaler un vol de carte bancaire ?" vs "J'ai perdu ma carte de paiement"
 ("2", 'N")]  # "J'ai perdu ma carte de paiement" vs "Pourquoi le sans contact ne fonctionne pas ?"
"""
	\end{lstlisting}
	
	
	%%%%%--------------------------------------------------------------------
	%%%%% Section C.2: Implémentation de la librairie \texttt{cognitivefactory-interactive-clustering}
	%%%%%--------------------------------------------------------------------
	\newpage
	\section[
		\texttt{cognitivefactory-interactive-clustering-gui}
	]{
		Implémentation de l'application web \\ \texttt{cognitivefactory-interactive-clustering-gui}
	}
\label{annex:C.2-DESCRIPTION-IMPLEMENTATION-INTERACTIVE-CLUSTERING-GUI}

	% INTRODUCTION DE L'ANNEXE.
	La librairie \texttt{cognitivefactory-interactive-clustering-gui} \footnote{
		\url{https://pypi.org/project/cognitivefactory-interactive-clustering-gui/}
	} (\cite{schild-etal:2022:cognitivefactory-interactiveclusteringgui}) a été implémentée au cours de ce doctorat dans le but d'intégrer notre méthodologie de \texttt{Clustering Interactif} au sein d'une application web.
	Cette application dispose de plusieurs fonctionnalités telles que :
	\begin{itemize}
		\item la gestion du projet, de ses paramétrages et de ses données (cf. \textsc{Figures}~\textsc{\ref{figure:C-WEB-APPLICATION-LISTE-PROJETS}},~\textsc{\ref{figure:C-WEB-APPLICATION-ACCUEIL-PROJET}},~\textsc{\ref{figure:C-WEB-APPLICATION-PARAMETRAGE}} et~\textsc{\ref{figure:C-WEB-APPLICATION-INVENTAIRE-TEXTES}}) ;
		\item la gestion et l'annotation de contraintes, ainsi que la vérification des propriétés de transitivité (cf. \textsc{Figures}~\textsc{\ref{figure:C-WEB-APPLICATION-INVENTAIRE-CONTRAINTES}},~\textsc{\ref{figure:C-WEB-APPLICATION-ANNOTATION}} et~\textsc{\ref{figure:C-WEB-APPLICATION-CONFLIT}}) ;
		\item la gestion des étapes d'une itération et de l'exécution asynchrone des divers algorithmes (cf. \textsc{Figures}~\textsc{\ref{figure:C-WEB-APPLICATION-ACCUEIL-PROJET}} et~\textsc{\ref{figure:C-WEB-APPLICATION-DIAGRAMME-ETATS}}) ;
		\item quelques scripts d'analyses.
	\end{itemize}
	
	Nous présentons succinctement cette application ci-dessous à l'aide de captures d'écrans.

	
	% Information : comme y accéder.
	\begin{leftBarInformation}
		La documentation technique de cette librairie est accessible au lien suivant : \url{https://cognitivefactory.github.io/interactive-clustering-gui/}.
	\end{leftBarInformation}
	
	% Information : projet ingénieur TPS
	\setcounter{localCounterOfFootnoteValue}{\value{footnote}}
	\begin{leftBarAuthorOpinion}
		L'étude d'une interface graphique et de ses fonctionnalités a été l'objet d'un premier projet étudiant avec l'École d'Ingénieurs Télécom Physique Strasbourg (au cours de l'année 2021).
		Lors de nos échanges, une idée consistait à s'inspirer des fonctionnalités de l'application \texttt{TINDER} \footnotemark pour \textit{swipe left} (respectivement \textit{swipe right}) l'annotation d'une contrainte \texttt{MUST-LINK} (respectivement d'une contrainte \texttt{CANNOT-LINK}).
		Bien qu'aucune version mobile de cette application n'a été développée, une telle fonctionnalité pourrait être envisagée afin d'améliorer le confort de l'utilisateur.
		Nous pouvons toutefois noter qu'un reliquat de cette discussion à mener au choix du logo de l'application, proche du logo de celui de l'application \texttt{TINDER}, ainsi qu'au design de la page d'annotation (cf. \textsc{Figure~\ref{figure:C-WEB-APPLICATION-ANNOTATION}}).
	\end{leftBarAuthorOpinion}
		% Rattraper les footnote.
			\stepcounter{localCounterOfFootnoteValue}
			\footnotetext[\value{localCounterOfFootnoteValue}]{
				\url{https://tinder.com/fr}
			}
	
	% Note de l'auteur : en cours de maintenance.
	\begin{leftBarAuthorOpinion}
		Suite aux diverses études menées au cours de ce doctorat, certaines pages sont en cours de développement, notamment :
		\begin{itemize}
			\item les pages d'analyses dont le but d'intégrer les conclusions du \textsc{Chapitre~\ref{chapter:4-ETUDES}} ;
			\item les pages de documentation pour intégrer les discussions du \textsc{Chapitre~\ref{chapter:5-GUIDE}}.
		\end{itemize}
	\end{leftBarAuthorOpinion}
	
	
	%%%
	%%% Subsection C.2.1: Accueil et Gestion de projets.
	%%%
	\newpage
	\subsection{Accueil et Gestion de projets}
	\label{annex:C.2.1-DESCRIPTION-IMPLEMENTATION-INTERACTIVE-CLUSTERING-GUI-ACCUEIL}
	
		%%% Page d'accueil de l'application
		%\newpage
		\paragraph{Page d'accueil de l'application (\textsc{Figure~\ref{figure:C-WEB-APPLICATION-ACCUEIL}}) :}
			
			% Capture d'écran: Page d'accueil de l'application.
			\begin{figure}[H]
				\centering
				\includegraphics[width=0.95\textwidth]{figures/interactive-clustering-application-accueil-application}
				\caption{
					Capture d'écran de l'application web implémentant notre méthodologie de \texttt{Clustering Interactif} : \textbf{page d'accueil de l'application}.
				}
				\label{figure:C-WEB-APPLICATION-ACCUEIL}
			\end{figure}
			
			% Description générale.
			C'est la page de bienvenu de l'application.
			Nous y trouvons une description rapide de la méthode ainsi qu'une liste des questions fréquentes à son sujet.
			A terme, la documentation de la méthodologie d'annotation y sera intégrée (cf. discussions du \textsc{Chapitre~\ref{chapter:5-GUIDE}}).
			
			% Boutons accessibles.
			Concernant les boutons accessibles :
			\begin{itemize}
				\item Le bouton d'accueil en haut à gauche redirigera toujours sur cette page ;
				\item Le bouton de contact en haut à droite permet de contacter l'équipe de recherche ;
				\item Le bouton \textguillemets{\texttt{LET'S GO}} permet d'accéder à la page de listant les projets d'annotation (cf. \textsc{Figure~\ref{figure:C-WEB-APPLICATION-LISTE-PROJETS}}).
			\end{itemize}
			
			% Information : comme y accéder.
			\begin{leftBarInformation}
				Dans toutes les pages suivantes, il est à noter que :
				\begin{itemize}
					\item Tous les boutons peuvent être survolés pour afficher une courte description de leur action ou de leur état, ainsi que les raccourcis clavier qui permettent de les activer ;
					\item Si besoin, tous les encadrés sont repliables pour gagner en visibilité.
				\end{itemize}
			\end{leftBarInformation}
		
		
		%%% Page de gestion des projets
		%\newpage
		\paragraph{Page de gestion des projets (\textsc{Figure~\ref{figure:C-WEB-APPLICATION-LISTE-PROJETS}}) :}
			
			% Capture d'écran: liste des projets.
			\begin{figure}[H]
				\centering
				\includegraphics[width=0.95\textwidth]{figures/interactive-clustering-application-liste-projets}
				\caption{
					Capture d'écran de l'application web implémentant notre méthodologie de \texttt{Clustering Interactif} : \textbf{page de gestion des projets}.
				}
				\label{figure:C-WEB-APPLICATION-LISTE-PROJETS}
			\end{figure}
			
			% Description générale.
			Cette page liste les projets existants sous la forme de tuiles contenant les informations importantes : nom, date de création, nombre d'itérations de la méthode, et l'état du projet (cf. \textsc{Figure~\ref{figure:C-WEB-APPLICATION-DIAGRAMME-ETATS}}).
			
			% Boutons accessibles.
			Concernant les boutons d'action de cette page :
			\begin{itemize}
				\item Les boutons d'accueil en haut à gauche permettent de naviguer entre cette page et la page d'accueil de l'application (cf. \textsc{Figure~\ref{figure:C-WEB-APPLICATION-ACCUEIL}}) ;
				\item Il est possible de télécharger un projet au format \texttt{.zip} ou de le supprimer grâce aux boutons \textguillemets{\faDownload} et \textguillemets{\faTrash} en haut à droite de chaque tuile ;
				\item Pour créer un projet, le bouton \textguillemets{\texttt{ADD NEW}} ouvre un formulaire demandant le nom du projet et la liste des textes à annoter (fichier au format \texttt{.csv} avec séparateur '\texttt{;}') ;
				\item Il est aussi possible d'importer un projet contenu dans une archive \texttt{.zip} grâce au bouton \textguillemets{\texttt{IMPORT}} ;
				\item Enfin, le bouton \textguillemets{\texttt{LOAD}} mène à la page d'accueil du projet sélectionné (cf. \textsc{Figure~\ref{figure:C-WEB-APPLICATION-ACCUEIL-PROJET}}).
			\end{itemize}
	
	
	%%%
	%%% Subsection C.2.2: Projet, Diagramme d'états et Paramétrages.
	%%%
	\newpage
	\subsection{Projet, Diagramme d'états et Paramétrages}
	\label{annex:C.2.2-DESCRIPTION-IMPLEMENTATION-INTERACTIVE-CLUSTERING-GUI-PROJET}
	
		%%% Page d'accueil du projet en cours
		%\newpage
		\paragraph{Page d'accueil du projet en cours (\textsc{Figure~\ref{figure:C-WEB-APPLICATION-ACCUEIL-PROJET}}) :}
			
			% Capture d'écran: accueil projet.
			\begin{figure}[H]
				\centering
				\includegraphics[width=0.95\textwidth]{figures/interactive-clustering-application-accueil-projet}
				\caption{
					Capture d'écran de l'application web implémentant notre méthodologie de \texttt{Clustering Interactif} : \textbf{page d'accueil du projet en cours}.
				}
				\label{figure:C-WEB-APPLICATION-ACCUEIL-PROJET}
			\end{figure}
			
			% Description générale.
			C'est la page principale de l'application.
			Elle contient en partie supérieure les informations du projet d'annotation en cours (\textit{date de création, numéro d'itération, gestion des textes et des contraintes}), et en partie inférieure les étapes d'une itération de \texttt{Clustering Interactif} (\textit{descriptions, boutons d'actions et de paramétrages}).
			
			% Boutons accessibles: gestion de projet.
			Concernant la gestion du projet (partie supérieure) :
			\begin{itemize}
				\item Les boutons d'accueil en haut à gauche permettent de naviguer entre cette page, la page de gestion des projets (cf. \textsc{Figure~\ref{figure:C-WEB-APPLICATION-LISTE-PROJETS}} et la page d'accueil de l'application (cf. \textsc{Figure~\ref{figure:C-WEB-APPLICATION-ACCUEIL}}) ;
				\item Au centre, il est possible de télécharger le projet au format \texttt{.zip} ou de le supprimer grâce aux boutons \textguillemets{\faDownload} et \textguillemets{\faTrash} ;
				\item Le bouton \textguillemets{\texttt{TEXTS}} mène vers la page d'inventaire et de gestion des textes du projet (cf. \textsc{Figure~\ref{figure:C-WEB-APPLICATION-INVENTAIRE-TEXTES}}) ;
				\item Le bouton \textguillemets{\texttt{CONSTRAINTS}} mène vers la page d'inventaire et de gestion des contraintes annotées ou en cours d'annotation (cf. \textsc{Figure~\ref{figure:C-WEB-APPLICATION-INVENTAIRE-CONTRAINTES}}).
			\end{itemize}
			
			% Boutons accessibles: gestion d'une itération.
			Concernant la gestion d'une itération de \texttt{Clustering Interactif} (partie inférieure), les différentes étapes sont représentées de bas en haut à l'aide d'éléments descriptifs repliables et de boutons d'actions.
			Nous retrouvons quatre étapes :
			\begin{enumerate}
				\item l'échantillonnage de contraintes, exécuté en tâche de fond grâce au bouton \textguillemets{\texttt{SAMPLING}}, et dont les paramètres sont accessibles via le bouton \textguillemets{\faCog} ;
				\item l'annotation de contraintes, avec le bouton \textguillemets{\texttt{ANNOTATE}} qui redirige vers la prochaine contrainte à annoter.
				Cette étape contient aussi une gestion de la modélisation, c'est-à-dire une vérification des prétraitements et de la vectorisation des textes, ainsi qu'une vérification de la cohérence des contraintes par l'absence de conflits d'annotation : le bouton \textguillemets{\texttt{UPDATE}} permet de recalculer cette modélisation en tâche de fond et le bouton \textguillemets{\texttt{APPROVE}} permet de la fixer jusqu'à la fin de l'itération en cours ;
				\item l'\textit{clustering} sous contraintes, exécuté en tâche de fond grâce au bouton \textguillemets{\texttt{CLUSTERING}}, et dont les paramètres sont accessibles via le bouton \textguillemets{\faCog} ;
				\item la confirmation du passage à la nouvelle itération, exécutée grâce au bouton \textguillemets{\texttt{NEXT ITERATION}}.
			\end{enumerate}
			
			% Notes:
			Il est à noter que :
			\begin{itemize}
				\item Les éléments de gauche sont repliables : au chargement de la page, seul l'élément de l'étape en cours est déplié ;
				\item La gestion de l'itération se fait à l'aide d'un diagramme d'état (cf. \textsc{Figure~\ref{figure:C-WEB-APPLICATION-DIAGRAMME-ETATS}}) : celui-ci se manifeste par un code couleur et l'activation/désactivation des boutons.
			\end{itemize}
		
		
		%%% Diagramme d'états de l'application et gestion des exécutions asynchrones
		%\newpage
		\paragraph{Diagramme d'états de l'application et gestion des exécutions asynchrones (\textsc{Figure~\ref{figure:C-WEB-APPLICATION-DIAGRAMME-ETATS}}) :}
			
			% Capture d'écran: accueil projet.
			\begin{figure}[H]
				\centering
				\includegraphics[width=0.95\textwidth]{figures/interactive-clustering-application-diagramme-etats}
				\caption{
					\textbf{Diagramme d'états} simplifié de l'application web implémentant notre méthodologie de \texttt{Clustering Interactif}.
				}
				\label{figure:C-WEB-APPLICATION-DIAGRAMME-ETATS}
			\end{figure}
			
			% Description générale.
			Comme décrit en \textsc{Section~\ref{section:3.2-DESCRIPTION-THEORIQUE}} et dans la \textsc{Figure~\ref{figure:3.2.1-DESCRIPTION-THEORIQUE-GENERALE}}, une itération de \textit{Clustering Interactif} contient trois étapes majeures : \textbf{(1)} l'échantillonnage de contraintes, \textbf{(2)} l'annotation de contraintes, et \textbf{(3)} le \textit{clustering} sous contraintes.
			Ces étapes sont représentées par le diagramme d'état en \textsc{Figure~\ref{figure:C-WEB-APPLICATION-DIAGRAMME-ETATS}} : ce dernier définit l'activation ou la désactivation des boutons d'action de l'application (cf. \textsc{Figure~\ref{figure:C-WEB-APPLICATION-ACCUEIL-PROJET}}).
			
			% Gestion de couleur.
			Afin de représenter l'état en cours et les actions possibles de manière pragmatique dans l'interface, un code couleur implicite est utilisé en plus de l'activation/désactivation des boutons :
			\begin{itemize}
				\item objet \textguillemets{\textcolor{colorApplicationNOTAVAILABLE}{\textbf{grisé}}} et généralement désactivé : action inaccessible pour le moment ;
				\item objet en \textguillemets{\textcolor{colorApplicationAVAILABLE}{\textbf{vert}}} et activé : prochaine action à réaliser ;
				\item objet en \textguillemets{\textcolor{colorApplicationWORKING}{\textbf{cyan}}} : action en cours de traitement ;
				\item objet en \textguillemets{\textcolor{colorApplicationERROR}{\textbf{rouge}}} et activé : action en erreur ou à recommencer ;
				\item objet en \textguillemets{\textcolor{colorApplicationDONE}{\textbf{vert grisé}}} et généralement désactivé : action réalisée avec succès.
			\end{itemize}
			
			% Gestion des actions asynchrones.
			D'autre part, comme certains algorithmes peuvent être lents, ces derniers sont exécutés en tâche de fond.
			La gestion d'état est alors affinée en quatre sous-états :
			\begin{itemize}
				\item \textguillemets{\textcolor{colorApplicationAVAILABLE}{\textbf{\texttt{TODO}}}} : l'action est à faire, la machine attend l'ordre de l'utilisateur ;
				\item \textguillemets{\textcolor{colorApplicationWORKING}{\textbf{\texttt{PENDING}}}} : l'action a été ordonnée par l'utilisateur, mais elle n'a pas encore été prise en charge par la machine ;
				\item \textguillemets{\textcolor{colorApplicationWORKING}{\textbf{\texttt{WORKING}}}} : l'action est en cours d'exécution en tâche de fond.
				Une barre d'avancement apparaît pour maintenir l'utilisateur informé de l'évolution de cet état ;
				\item \textit{Note} : l'état \textguillemets{\textcolor{colorApplicationDONE}{\textbf{\texttt{DONE}}}} (action faite) n'existe pas réellement, elle est représentée par le fait que la prochaine étape ait un état \textguillemets{\texttt{TODO}}.
			\end{itemize}
			
			% Warning: Architecture asynchrone en production.
			\begin{leftBarWarning}
				Pour une simplicité d'usage et afin d'offrir une démonstration rapide de notre méthodologie, nous avons décidé d'exécuter simplement les algorithmes en tâche de fond.
				Toutefois, pour favoriser les performances de l'application ainsi que sa sûreté pour une utilisation en production, nous vous conseillons de ré-implémenter cette gestion des exécutions en privilégiant une architecture asynchrone utilisant des \textit{workers} dédiés.
			\end{leftBarWarning}
		
		
		%%% Page de gestion des paramètres
		%\newpage
		\paragraph{Page de gestion des paramètres (\textsc{Figure~\ref{figure:C-WEB-APPLICATION-PARAMETRAGE}}) :}
		
			% Capture d'écran: gestion des paramètrages.
			\begin{figure}[H]
				\centering
				\includegraphics[width=0.95\textwidth]{figures/interactive-clustering-application-parametres}
				\caption{
					Capture d'écran de l'application web implémentant notre méthodologie de \texttt{Clustering Interactif} : \textbf{page de gestion des paramètres}.
				}
				\label{figure:C-WEB-APPLICATION-PARAMETRAGE}
			\end{figure}
			
			% Description générale.
			Accessible depuis les différents boutons \textguillemets{\faCog}, cette page liste les divers paramètres des algorithmes pour chaque itération.
			\begin{itemize}
				\item Chaque tuile représente une tâche (prétraitements, vectorisation, échantillonnage et \textit{clustering}) : divers algorithmes et hyperparamètres son disponibles ;
				\item Les boutons \textguillemets{\texttt{UPDATE}} permettent de valider les changements, les boutons \textguillemets{\texttt{RESET}} rétablissent les paramètres par défaut ;
				\item Ces différents formulaires sont modifiables tant que l'étape n'est pas en cours d'exécution, sinon ils sont juste consultables ;
				\item En bas de page, il est possible de changer d'itération pour consulter les paramètres des itérations précédentes (boutons \textguillemets{\faAngleLeft} et \textguillemets{\faAngleRight}), et de revenir vers la page d'accueil du projet (bouton \textguillemets{\faHome}, cf. \textsc{Figure~\ref{figure:C-WEB-APPLICATION-ACCUEIL-PROJET}}).
			\end{itemize}
	
	
	%%%
	%%% Subsection C.2.3: Textes et Contraintes.
	%%%
	\newpage
	\subsection{Textes et Contraintes}
	\label{annex:C.2.3-DESCRIPTION-IMPLEMENTATION-INTERACTIVE-CLUSTERING-GUI-DONNEES}
	
		%%% Page d'inventaire des textes
		%\newpage
		\paragraph{Page d'inventaire des textes (\textsc{Figure~\ref{figure:C-WEB-APPLICATION-INVENTAIRE-TEXTES}}) :}
		
			% Capture d'écran: inventaire des textes.
			\begin{figure}[H]
				\centering
				\includegraphics[width=0.95\textwidth]{figures/interactive-clustering-application-textes}
				\caption{
					Capture d'écran de l'application web implémentant notre méthodologie de \texttt{Clustering Interactif} : \textbf{page d'inventaire des textes}.
				}
				\label{figure:C-WEB-APPLICATION-INVENTAIRE-TEXTES}
			\end{figure}
			
			% Description générale.
			Cette page permet de lister les textes du projet à annoter.
			La page est divisée en deux : la partie supérieure donne des informations générales (\textit{nombre de textes, nombre de contraintes à annoter, rappel de la modélisation en cours}), et la partie inférieure liste les textes dans un tableau.
			
			% Boutons accessibles: gestion des textes.
			Concernant les informations générales (partie supérieure) :
			\begin{itemize}
				\item Le bouton \textguillemets{\texttt{UPDATE}} permet de mettre à jour la modélisation si des contraintes ont été ajoutées, des paramètres de prétraitements ou de vectorisation ont été mis à jour, ou si des textes ont été modifiés : cette action est exécutée en tâche de fond.
				La couleur de se bouton est définie par le diagramme d'état (cf. \textsc{Figure~\ref{figure:C-WEB-APPLICATION-DIAGRAMME-ETATS}}) ;
				\item Le bouton \textguillemets{\texttt{CONSTRAINTS}} mène vers la page d'inventaire et de gestion des contraintes annotées ou en cours d'annotation (cf. \textsc{Figure~\ref{figure:C-WEB-APPLICATION-INVENTAIRE-CONTRAINTES}}).
			\end{itemize}
			
			% Boutons accessibles: liste des textes.
			Concernant le tableau listant les textes (partie inférieure) :
			\begin{itemize}
				\item Le texte brut et sa version prétraitée sont affichés ;
				\item Grâce au bouton \textguillemets{\faPen}, il est possible de corriger un texte s'il contient une faute de frappe ;
				\item Grâce au bouton \textguillemets{\textcolor{colorApplicationDELETE}{\faTrash}}, il est possible de supprimer (\textit{ne plus prendre en compte}) un texte s'il n'est pas pertinent pour le projet : celui-ci est alors rayé en \textcolor{colorApplicationDELETE}{orange} ;
				\item En haut du tableau, il est possible de trier les textes suivant différents critères (\textit{ordre alphabétique, supprimé ou non, ...}) ;
				\item \textbf{Attention} : Toute action de modification (renommage, suppression) nécessite de mettre à jour la modélisation par la suite.
				De plus, ces actions sont désactivées si le projet n'est pas à l'étape d'annotation (cf. diagramme d'états en \textsc{Figure~\ref{figure:C-WEB-APPLICATION-DIAGRAMME-ETATS}}) ;
			\end{itemize}
		
		
		%%% Page d'inventaire des contraintes
		%\newpage
		\paragraph{Page d'inventaire des contraintes (\textsc{Figure~\ref{figure:C-WEB-APPLICATION-INVENTAIRE-CONTRAINTES}}) :}
		
			% Capture d'écran: inventaire des contraintes.
			\begin{figure}[H]
				\centering
				\includegraphics[width=0.95\textwidth]{figures/interactive-clustering-application-contraintes}
				\caption{
					Capture d'écran de l'application web implémentant notre méthodologie de \texttt{Clustering Interactif} : \textbf{page d'inventaire des contraintes}.
				}
				\label{figure:C-WEB-APPLICATION-INVENTAIRE-CONTRAINTES}
			\end{figure}
			
			% Description générale.
			Cette page permet de lister les contraintes du projet à annoter.
			La page est divisée en deux : la partie supérieure donne des informations générales (\textit{nombre de textes, nombre des contraintes à annoter, rappel de la modélisation en cours}), et la partie inférieure liste les contraintes dans un tableau.
			
			% Boutons accessibles: gestion des contraintes.
			Concernant les informations générales (partie supérieure) :
			\begin{itemize}
				\item Le bouton \textguillemets{\texttt{ANNOTATE}} redirige vers la prochaine contrainte à annoter (s'il en reste) ;
				\item Le bouton \textguillemets{\texttt{UPDATE}} permet de mettre à jour la modélisation si des contraintes ont été ajoutées, des paramètres de prétraitements ou de vectorisation ont été mis à jour, ou si des textes ont été modifiés : cette action est exécutée en tâche de fond.
				La couleur de se bouton est définie par le diagramme d'état (cf. \textsc{Figure~\ref{figure:C-WEB-APPLICATION-DIAGRAMME-ETATS}}) ;
				\item Le bouton \textguillemets{\texttt{TEXTS}} mène vers la page d'inventaire et de gestion des données du projet (cf. \textsc{Figure~\ref{figure:C-WEB-APPLICATION-INVENTAIRE-TEXTES}}).
			\end{itemize}
			
			% Boutons accessibles: liste des contraintes.
			Concernant le tableau listant les contraintes (partie inférieure) :
			\begin{itemize}
				\item Les deux textes d'une même contrainte sont affichés de part et d'autre de la valeur annotée : \textcolor{colorApplicationMUSTLINK}{\texttt{MUST-LINK}}, \textcolor{colorApplicationCANNOTLINK}{\texttt{CANNOT-LINK}} ou \texttt{SKIP} (\textit{pour une contrainte non-annotée ou temporairement ignorée}) ;
				\item Il est possible de marquer une contrainte pour la revoir plus tard grâce à la coche \textguillemets{\textcolor{colorApplicationREVIEW}{\faCheckSquare}} ;
				\item Le bouton \textguillemets{\faAngleRight} à droite permet d'accéder à la page d'annotation de cette contrainte (cf. \textsc{Figure~\ref{figure:C-WEB-APPLICATION-ANNOTATION}}) ;
				\item Diverses informations sont disponibles à la droite du tableau : l'itération à laquelle la contrainte a été échantillonnée, sa dernière date de modification, son besoin d'annotation (\textit{\textguillemets{\faQuestion} pour une contrainte encore jamais été annotée}), et la présence ou non de conflits (\textit{\textguillemets{\textcolor{colorApplicationMUSTLINK}{\faCheck}} ou \textguillemets{\textcolor{colorApplicationERROR}{\faExclamation}}}) ;
				\item En haut du tableau, il est possible de trier les contraintes suivant différents critères (\textit{ordre alphabétique, valeur d'annotation, date d'échantillonnage ou de modification, présence de conflit, ...}) ;
				\item \textbf{Attention} : Toute action de modification de la valeur d'annotation nécessite de mettre à jour la modélisation par la suite ;
				De plus, cette action est désactivée si le projet n'est pas à l'étape d'annotation (cf. diagramme d'états en \textsc{Figure~\ref{figure:C-WEB-APPLICATION-DIAGRAMME-ETATS}}) ;
				\item \textit{Note} : Si une contrainte concerne au moins un texte qui a été supprimé (cf. \textsc{Figure~\ref{figure:C-WEB-APPLICATION-INVENTAIRE-TEXTES}}), la contrainte n'apparaît pas dans ce tableau mais existe toujours dans l'application (\textit{elle n'est plus prise en compte}).
			\end{itemize}
	
	
	%%%
	%%% Subsection C.2.4: Annotation et Conflits.
	%%%
	\newpage
	\subsection{Annotation et Conflits}
	\label{annex:C.2.4-DESCRIPTION-IMPLEMENTATION-INTERACTIVE-CLUSTERING-GUI-ANNOTATION}
	
	
		%%% Page d'annotation d'une contrainte
		%\newpage
		\paragraph{Page d'annotation d'une contrainte (\textsc{Figure~\ref{figure:C-WEB-APPLICATION-ANNOTATION}} et \textsc{Figure~\ref{figure:C-WEB-APPLICATION-CONFLIT}}) :}
		
			% Capture d'écran: annotation.
			\begin{figure}[H]
				\centering
				\includegraphics[width=0.95\textwidth]{figures/interactive-clustering-application-annotation-0full}
				\caption{
					Capture d'écran de l'application web implémentant notre méthodologie de \texttt{Clustering Interactif} : \textbf{page d'annotation d'une contrainte}.
				}
				\label{figure:C-WEB-APPLICATION-ANNOTATION}
			\end{figure}
			
			% Description générale.
			Cette page est le coeur de cette application d'annotation : la partie supérieure permet d'annoter une contraintes entre deux textes, les parties inférieures sont des détails (éléments repliés par défaut).
			
			% Boutons accessibles: annotation.
			Concernant l'annotation de la contrainte (partie supérieure) :
			\begin{itemize}
				\item Les deux textes de la contrainte sont affichés en haut du bloc d'annotation ;
				\item Les deux boutons principaux d'annotation sont le \textguillemets{\textcolor{colorApplicationMUSTLINK}{\faEquals}} pour un \texttt{MUST-LINK} et le \textguillemets{\textcolor{colorApplicationCANNOTLINK}{\faNotEqual}} pour un \texttt{CANNOT-LINK}.
				Les raccourcis claviers de ses boutons sont respectivement \textguillemets{\texttt{A}} (\textit{accept}) et \textguillemets{\texttt{R}} (\textit{reject}).
				Il est aussi possible d'ignorer la contrainte avec le bouton \textguillemets{\textcolor{colorApplicationSKIP}{\faQuestion}} (raccourcis avec la barre espace) ;
				\item Si c'est la première fois que cette contrainte est annotée, la prochaine contrainte est automatiquement chargée lors d'un choix d'annotation (\textguillemets{\textcolor{colorApplicationMUSTLINK}{\faEquals}}, \textguillemets{\textcolor{colorApplicationCANNOTLINK}{\faNotEqual}} ou \textguillemets{\textcolor{colorApplicationSKIP}{\faQuestion}}).
				Sinon, une confirmation est demandée pour valider le changement de la valeur de la contrainte ;
				\item Une gestion de revue d'annotation est possible grâce à un champ de commentaire, et \textguillemets{\textcolor{colorApplicationREVIEW}{\faCheckSquare}} permet de marquer la contrainte pour la pour revoir plus tard ;
				\item Grâce à l'icône info-bulle, il est possible d'afficher la version non-prétraitée de chaque texte
				De plus, grâce à la coche \textguillemets{\textcolor{colorApplicationDELETE}{\faTrash}}, il est possible de supprimer (\textit{ne plus prendre en compte}) un texte non pertinent pour le projet : la contrainte sera alors masquée de la liste des contraintes ;
				\item \textbf{Attention} : Toute action de modification de la valeur d'annotation nécessite de mettre à jour la modélisation par la suite.
				De plus, cette action est désactivée si le projet n'est pas à l'étape d'annotation (cf. diagramme d'états en \textsc{Figure~\ref{figure:C-WEB-APPLICATION-DIAGRAMME-ETATS}}) ;
			\end{itemize}
			
			% Boutons accessibles: détails.
			Concernant les détails sur la contraintes (partie inférieure) :
			\begin{itemize}
				\item L'encadré du milieu donne quelques informations sur l'annotation : ce qu'a effectué la machine lors de la précédente étape de \textit{clustering} (même \textit{cluster} ou différentes \textit{cluster} ?), l'historique de ce que l'annotateur a renseigné au préalable, et la déduction faite par le gestionnaire de contraintes grâce aux propriétés de transitivité ;
				\item L'encadré du bas propose une visualisation du graphe de contraintes à l'aide de la librairie \texttt{d3js} \footnote {
					\url{https://d3js.org/}
				} ;
				\item Les boutons en bas de page permettent de naviguer entre les pages d'annotation (boutons \textguillemets{\faAngleLeft} et \textguillemets{\faAngleRight}) ou à revenir vers la page d'inventaire des contraintes (bouton \textguillemets{\faList}, cf. \textsc{Figure~\ref{figure:C-WEB-APPLICATION-INVENTAIRE-CONTRAINTES}}).
			\end{itemize}
			
			% Capture d'écran: conflit d'annotation.
			\begin{figure}[H]
				\centering
				\includegraphics[width=0.95\textwidth]{figures/interactive-clustering-application-annotation-4conflit}
				\caption{
					Capture d'écran de l'application web implémentant notre méthodologie de \texttt{Clustering Interactif} : \textbf{graphe de contraintes présentant un conflit d'annotation}.
				}
				\label{figure:C-WEB-APPLICATION-CONFLIT}
			\end{figure}
			
			% Graphe de contraintes.
			Concernant le graphe local de contraintes :
			\begin{itemize}
				\item Les cercles représentent les données : chaque numéro correspond à un identifiant, et leurs textes respectifs sont visibles par un survol de souris.
				Ces cercles peuvent être déplacés pour une meilleur visibilité ;
				\item Les liens entre les cercles représentent les contraintes : en \textcolor{colorApplicationMUSTLINK}{\textbf{vert}} pour les \texttt{MUST-LINK} et en \textcolor{colorApplicationCANNOTLINK}{\textbf{rouge}} pour les \texttt{CANNOT-LINK}, en \textbf{gras} pour les contraintes annotées et en \textit{fin} pour les contraintes déduites par transitivité, et en \texttt{pointillés} pour les conflits détectés.
				Un clic de souris sur un lien redirige vers la page d'annotation de la contrainte associée (si elle existe) ;
				\item Comme il peut y avoir un nombre important de contraintes dans un projet, ce graphe ne représente que la partie des contraintes impliquées dans l'annotation des deux textes de cette page : nous retrouvons ainsi les deux textes en cours d'annotation et leurs composants connexes respectifs (voir \textsc{Section~\ref{annex:C.1.2-DESCRIPTION-IMPLEMENTATION-INTERACTIVE-CLUSTERING-GESTION-DES-CONTRAINTES}}) ;
				\item Dans l'exemple en \textsc{Figure~\ref{figure:C-WEB-APPLICATION-CONFLIT}}, nous pouvons voir le conflit suivant : \textbf{(1)} $118$ et $270$ doivent être séparés car $118$ est différent de $271$ qui est similaire à $270$, mais \textbf{(2)} $118$ et $270$ doivent être rapprochés car $118$ est similaire à $220$ qui est similaire à $270$...
				\item \textbf{Attention} : En cas de conflit, plusieurs boutons deviennent \textcolor{colorApplicationERROR}{\textbf{rouge}}, et l'approbation de la modélisation est désactivée tant que le conflit n'est pas résolu.
			\end{itemize}
	
	
	%%%%%--------------------------------------------------------------------
	%%%%% Section C.3: Implémentation de la librairie \texttt{cognitivefactory-interactive-clustering}
	%%%%%--------------------------------------------------------------------
	\newpage
	\section[
		\texttt{cognitivefactory-features-maximization-metric}
	]{
		Implémentation de l'application web \\ \texttt{cognitivefactory-features-maximization-metric}
	}
\label{annex:C.3-DESCRIPTION-IMPLEMENTATION-FEATURES-MAXIMIZATION-METRIC}
	
	% Généralités.
	La librairie \texttt{cognitivefactory-features-maximization-metric} \footnote{
		\url{https://pypi.org/project/cognitivefactory-features-maximization-metric/}
	} (\cite{schild:2023:cognitivefactory-featuresmaximizationmetric}) a été implémentée au cours de ce doctorat pour pouvoir utiliser la \texttt{Maximisation des traits} (\textit{Features Maximization} notée \texttt{FMC}).
	Cette technique, proposée par \cite{lamirel-etal:2017:novel-approach-feature}, permet de sélectionner les composantes vectorielles pertinentes (\textit{features}) d'une base d'apprentissage.
	Nous allons détailler cette librairie comme suit :
	\begin{itemize}
		\item le calcul du score de \textit{Features F-Measure} associé à cette méthode (cf. \textsc{Section~\ref{annex:C.3.1-DESCRIPTION-IMPLEMENTATION-FEATURES-MAXIMIZATION-METRIC-CALCUL-FMEASURE}}) ;
		\item la sélection des composantes vectorielles pertinentes à l'aide du score de \textit{Features F-Measure} (cf. \textsc{Section~\ref{annex:C.3.2-DESCRIPTION-IMPLEMENTATION-FEATURES-MAXIMIZATION-METRIC-SELECTION-FEATURES}}) ;
		\item l'activation des composantes vectorielles pertinentes pour chaque classe de la base d'apprentissage (cf. \textsc{Section~\ref{annex:C.3.3-DESCRIPTION-IMPLEMENTATION-FEATURES-MAXIMIZATION-METRIC-ACTIVATION-FEATURES}}) ;
		\item l'application de cette méthode à l'analyse des patterns linguistiques pertinent d'une base d'apprentissage utilisée pour de la classification de textes intention (cf. \textsc{Section~\ref{annex:C.3.4-DESCRIPTION-IMPLEMENTATION-FEATURES-MAXIMIZATION-METRIC-APPLICATION-TEXTES}}).
	\end{itemize}
	
	% Information : comme y accéder.
	\begin{leftBarInformation}
		La documentation technique de cette librairie est accessible au lien suivant : \url{https://cognitivefactory.github.io/features-maximization-metric/}.
	\end{leftBarInformation}
	
	% Notations.
	Pour la suite de l'exposé, nous allons utiliser les notations suivantes :
	\begin{itemize}
		% Data.
		\item $\mathcal{D}$ représente l'ensemble des données $d$ de la base d'apprentissage,
		et $\texttt{vecteur}[d]$ représente la description vectorielle d'une donnée $d \in \mathcal{D}$ ;
		% Features.
		\item $\mathcal{F}$ représente l'ensemble des composantes vectorielles $f$ (\textit{features}),
		et $\texttt{vecteur}[d][f]$ représente le poids de la composante $f \in \mathcal{F}$ du vecteur décrivant la donnée $d \in \mathcal{D}$ ;
		% Classes.
		\item $\mathcal{C}$ représente l'ensemble des classes $c$ possibles de la base d'apprentissage,
		et $\texttt{classe}[d]$ représente la classe d'une donnée $d \in \mathcal{D}$ ;
	\end{itemize}
	
	
	%%%
	%%% Subsection C.3.1: Calcul du score de \texttt{F-Measure}.
	%%%
	\subsection{Calcul du score de \textit{Features F-Measure}}
	\label{annex:C.3.1-DESCRIPTION-IMPLEMENTATION-FEATURES-MAXIMIZATION-METRIC-CALCUL-FMEASURE}
	
		% Introduction.
		D'abord, il faut calculer le score de \textit{Features F-Measure} (\texttt{FM}) à partir de deux termes : la \textit{Features Recall} (\texttt{FR}) et la \textit{Features Predominance} (\texttt{FP}).
		\newline
		
		
		%%% FEATURES RECALL.
		
		% Description du calcul.
		La \textbf{\textit{Features Recall}} d'une composante vectorielle $f \in \mathcal{F}$ et d'une classe $c \in \mathcal{C}$, notée $\texttt{FR}[f][c]$, est un score compris entre $0$ et $1$ qui est obtenu par le ratio entre :
		\begin{itemize}
			\item la somme des poids des vecteurs pour la composante $f$ et pour les données de la classe $c$, et
			\item la somme des poids des vecteurs pour la composante $f$ et pour toutes les données.
		\end{itemize}
		
		% Equation.
		\begin{equation}
			\label{equation:C.3.1-DESCRIPTION-IMPLEMENTATION-FEATURES-MAXIMIZATION-METRIC-FEATURES-RECALL}
			\texttt{FR}[f][c]~=~\frac{
				\sum\limits_{
					\substack{
						d \in \mathcal{D} \\
						\texttt{classe}[d]=c
					}
				} \texttt{vector}[d][f]
			}{
				\sum\limits_{
					c' \in \mathcal{C}
				}
				\sum\limits_{
					\substack{
						d' \in \mathcal{D} \\
						\texttt{classe}[d']=c'
					}
				} \texttt{vector}[d'][f]
			}
		\end{equation}
		
		% Interprétation.
		\begin{leftBarAuthorOpinion}
			Ce score répond à la question :
			\textguillemets{\textit{est-ce que la \textit{feature} $f$ permet de distinguer la classe $c$ des autres classes $c'$ ?}}
		\end{leftBarAuthorOpinion}
		
		
		%%% FEATURES PREDOMINANCE.
		
		% Description du calcul.
		La \textbf{\textit{Features Predominance}} d'une composante vectorielle $f \in \mathcal{F}$ et d'une classe $c \in \mathcal{C}$, notée $\texttt{FP}[f][c]$, est un score compris entre $0$ et $1$ qui est obtenu par le ratio entre :
		\begin{itemize}
			\item la somme des poids des vecteurs pour la composante $f$ et pour les données de la classe $c$, et
			\item la somme des poids des vecteurs pour toutes les composantes et pour les données de la classe $c$.
		\end{itemize}
		
		% Equation.
		\begin{equation}
			\label{equation:C.3.1-DESCRIPTION-IMPLEMENTATION-FEATURES-MAXIMIZATION-METRIC-FEATURES-PREDOMINANCE}
			\texttt{FP}[f][c]~=~\frac{
				\sum\limits_{
					\substack{
						d \in \mathcal{D} \\
						\texttt{classe}[d]=c
					}
				} \texttt{vector}[d][f]
			}{
				\sum\limits_{
					f' \in \mathcal{F}
				}
				\sum\limits_{
					\substack{
						d \in \mathcal{D} \\
						\texttt{classe}[d]=c
					}
				} \texttt{vector}[d][f']
			}
		\end{equation}
		
		% Interprétation.
		\begin{leftBarAuthorOpinion}
			Ce score répond à la question :
			\textguillemets{\textit{est-ce que la composante $f$ identifie mieux la classe $c$ que les autres composantes $f'$ ?}}
		\end{leftBarAuthorOpinion}
		
		%%% FEATURES F-MEASURE.
		
		% Description du calcul.
		La \textbf{\textit{Features F-Measure}} d'une composante vectorielle $f \in \mathcal{F}$ et d'une classe $c \in \mathcal{C}$, notée $\texttt{FM}[f][c]$, est un score entre $0$ et $1$ qui est obtenu par la moyenne harmonique entre la \textit{Features Recall} et la \textit{Features Predominance}.
		
		% Equation.
		\begin{equation}
			\label{equation:C.3.1-DESCRIPTION-IMPLEMENTATION-FEATURES-MAXIMIZATION-METRIC-FEATURES-FMEASURE}
			\texttt{FM}[f][c]~=~2 \cdot \frac{
				\texttt{FR}[f][c] \cdot \texttt{FP}[f][c]
			}{
				\texttt{FR}[f][c] + \texttt{FP}[f][c]
			}
		\end{equation}
		
		% Interprétation.
		\begin{leftBarAuthorOpinion}
			Ce score répond à la question :
			\textguillemets{\textit{combien d'information contient la \textit{feature} $f$ au sujet de la classe $c$ ?}}
		\end{leftBarAuthorOpinion}
	
	
	%%%
	%%% Subsection C.3.2: Sélection de \textit{features} à l'aide de la \texttt{F-Measure}.
	%%%
	\subsection{Sélection de \textit{features} à l'aide de la \texttt{F-Measure}}
	\label{annex:C.3.2-DESCRIPTION-IMPLEMENTATION-FEATURES-MAXIMIZATION-METRIC-SELECTION-FEATURES}
	
		% Introduction.
		L'objectif de cette seconde étape est de supprimer les composantes vectorielles qui n'apportent pas d'information de manière générale.
		Pour cela, un seuil de sélection est défini grâce à la moyenne globale des scores de \textit{Features F-Measure}.
		\newline
		
		%%% FEATURES FMEASURE OVERALL AVERAGE.
		
		% Description du calcul.
		La \textbf{\textit{Features Overall Average}}, notée $\overline{\overline{\texttt{FM}}}$, est un score entre $0$ et $1$ qui est obtenu par la moyenne de la \texttt{Features F-Measure} pour toutes les composantes et pour toutes les classes.
		
		% Equation.
		\begin{equation}
			\label{equation:C.3.1-DESCRIPTION-IMPLEMENTATION-FEATURES-MAXIMIZATION-METRIC-FEATURES-OVERALL-AVERAGE}
			\overline{\overline{\texttt{FM}}}~=~\frac{
				\sum\limits_{
					f \in \mathcal{F}
				}
				\sum\limits_{
					c \in \mathcal{C}
				} \texttt{FM}[f][c]
			}{
				|\mathcal{F}| \cdot |\mathcal{C}|
			}
		\end{equation}
		
		% Interprétation.
		\begin{leftBarAuthorOpinion}
			Ce seuil répond à la question :
			\textguillemets{\textit{quelle est la moyenne d'information contenue dans cette représentation vectorielle ?}}
		\end{leftBarAuthorOpinion}
		
		%%% FEATURES SELECTION.
		
		% Description du calcul.
		La \textbf{sélection de \textit{features}} se fait en comparant les valeurs de \texttt{Features F-Measures} à cette moyenne globale : si une composante $f \in \mathcal{F}$ a un score \texttt{FM}[f][c'] supérieur à la moyenne $\overline{\overline{\texttt{FM}}}$ pour au moins une classe $c' \in \mathcal{C}$, alors la composante $f$ est sélectionnée ; sinon, elle est supprimée.
		
		% Equation.
		\begin{equation}
			\label{equation:C.3.1-DESCRIPTION-IMPLEMENTATION-FEATURES-MAXIMIZATION-METRIC-FEATURES-SELECTION}
			\begin{cases}
				\mathcal{F}_{\texttt{SELECTED}}
					& :=~\bigl\{~
						f \in \mathcal{F}
						~\big|~
						(\exists c' \in \mathcal{C})~\texttt{FM}[f][c'] \geq \overline{\overline{\texttt{FM}}}
					~\bigr\} \\
				\mathcal{F}_{\texttt{DELETED}}
					& :=~\bigl\{~
						f \in \mathcal{F}
						~\big|~
						(\forall c' \in \mathcal{C})~\texttt{FM}[f][c'] < \overline{\overline{\texttt{FM}}}
					~\bigr\}
			\end{cases}
		\end{equation}
		
		% Interprétation.
		\begin{leftBarAuthorOpinion}
			Cette sélection répond à la question :
			\textguillemets{\textit{quelles sont les \textit{features} qui apportent plus d'information que la moyenne d'information contenue dans cette représentation vectorielle ?}}
		\end{leftBarAuthorOpinion}
	
	
	%%%
	%%% Subsection C.3.3: Activation des \textit{features} à l'aide de la \texttt{F-Measure}.
	%%%
	\subsection{Activation des \textit{features} à l'aide de la \texttt{F-Measure}}
	\label{annex:C.3.3-DESCRIPTION-IMPLEMENTATION-FEATURES-MAXIMIZATION-METRIC-ACTIVATION-FEATURES}
	
		% Introduction.
		L'objectif de cette dernière étape est de vérifier l'activation des composantes vectorielles pour chaque classe $c \in \mathcal{C}$.
		Pour cela, un seuil d'activation est défini pour chaque \textit{feature} $f \in \mathcal{F}$ grâce à la moyenne locale des scores de \textit{Features F-Measure}.
		\newline
		
		
		%%% FEATURES FMEASURE MARGINAL AVERAGE.
		
		% Description du calcul.
		La \textbf{\textit{Features Marginal Average}} d'une composante sélectionnée $f \in \mathcal{F}_{\texttt{SELECTED}}$, notée $\overline{\texttt{FM}[f]}$, est un score entre $0$ et $1$ qui est obtenu par la moyenne locale de la \texttt{Features F-Measure} pour la composante $f$ et pour toutes les classes.
		
		% Equation.
		\begin{equation}
			\label{equation:C.3.1-DESCRIPTION-IMPLEMENTATION-FEATURES-MAXIMIZATION-METRIC-FEATURES-MARGINAL-AVERAGE}
			\overline{\texttt{FM}[f]}~=~\frac{
				\sum\limits_{
					c' \in \mathcal{C}
				} \texttt{FM}[f][c']
			}{
				|\mathcal{F_{\texttt{SELECTED}}}|
			}
		\end{equation}
		
		% Interprétation.
		\begin{leftBarAuthorOpinion}
			Ce seuil répond à la question :
			\textguillemets{\textit{quelle est la moyenne d'information contenue par la feature $f$ dans cette représentation vectorielle ?}}
		\end{leftBarAuthorOpinion}
		
		%%% FEATURES ACTIVATION.
		
		% Description du calcul.
		L'\textbf{activation de \textit{features}} se fait en comparant les valeurs de \texttt{Features F-Measures} à cette moyenne locale : si une composante sélectionnée $f \in \mathcal{F}_{\texttt{SELECTED}}$ a un score \texttt{FM}[f][c] supérieur à la moyenne locale $\overline{\texttt{FM}[f]}$ pour une classe $c \in \mathcal{C}$, alors cette composante $f$ est activée pour cette classe $c$.
		
		% Equation.
		\begin{equation}
			\label{equation:C.3.1-DESCRIPTION-IMPLEMENTATION-FEATURES-MAXIMIZATION-METRIC-FEATURES-ACTIVATION}
			\begin{cases}
				f \texttt{active} c
					& si \texttt{FM}[f][c] \geq \overline{\texttt{FM}[f]} \\
				f \texttt{n'active pas} c
					& sinon
			\end{cases}
		\end{equation}
		
		% Interprétation.
		\begin{leftBarAuthorOpinion}
			Cette sélection répond à la question :
			\textguillemets{\textit{pour quelle(s) classe(s) la \textit{feature} $f$ est pertinente ?}}
		\end{leftBarAuthorOpinion}
	
	
	%%%
	%%% Subsection C.3.4: Application à l'analyse de la classification de textes.
	%%%
	\subsection{Application à l'analyse de la classification de textes}
	\label{annex:C.3.4-DESCRIPTION-IMPLEMENTATION-FEATURES-MAXIMIZATION-METRIC-APPLICATION-TEXTES}
	
		% Description.
		Plaçons nous dans le cas d'une classification de textes en intentions :
		\begin{itemize}
			% Data.
			\item $\mathcal{D}$ représente l'ensemble des textes de la base d'apprentissage ;
			% Features.
			\item $\mathcal{F}$ représente l'ensemble des composantes vectorielles permettant de décrire les textes ;
			% Classes.
			\item $\mathcal{C}$ représente l'ensemble des intentions possibles de la base d'apprentissage.
		\end{itemize}
		
		% Application au TF-IDF
		Si nous utilisons une représentation vectorielle basées sur une description statistique du vocabulaire (comme \texttt{Bag of Words} (\cite{harris:1954:distributional-structure}) ou \texttt{TF-IDF} (\cite{ramos:2003:using-tfidf-determine}), nous pouvons alors déduire quels sont les mots du vocabulaire qui décrivent le mieux chaque intention grâce à la \texttt{Maximisation des traits}.
		Il est possible de compléter l'analyse en considérant que :
		\begin{itemize}
			\item un pattern linguistique $f$ est caractéristique d'une classe $c$ s'il ne s'active que pour cette classe ;
			\item un pattern linguistique $f$ est ambigu s'il s'active pour plusieurs classes.
		\end{itemize}
		Nous utilisons cette technique dans la \textsc{Section~\ref{section:4.4.2-ETUDE-PERTINENCE-PATTERNS-LINGUISTIQUES}} pour déterminer la sémantique d'un \textit{cluster} sur la base de son vocabulaire caractéristique.


%%%%%--------------------------------------------------------------------
%%%%% Annexe D: ANNEXE TECHNIQUE
%%%%%--------------------------------------------------------------------
%\newpage
\Annex{Annexe technique}
\label{annex:D-ANNEXE-TECHNIQUE}

	% INTRODUCTION DE L'ANNEXE.
	\todo[inline]{à rédiger}

	% TABLE DES MATIÈRES DE L'ANNEXE.
	\minitoc

	%%%%%--------------------------------------------------------------------
	%%%%% Annexe D.1: \texttt{v-measure}: métrique d'évaluation de \textit{clustering}.
	%%%%%--------------------------------------------------------------------
	\section{\texttt{v-measure}: une métrique d'évaluation de \textit{clustering}}
	\label{annex:D.1-TECHNIQUE-VMEASURE}
	\todo[inline]{ANNEXE: v-measure}

	%%%%%--------------------------------------------------------------------
	%%%%% Annexe D.2: \texttt{FMC}: métrique de pertinence des patterns linguistiques.
	%%%%%--------------------------------------------------------------------
	\section{\texttt{FMC}: métrique de pertinence des patterns linguistiques}
	\label{annex:D.2-TECHNIQUE-FMC-CLUSTERING}
	\todo[inline]{ANNEXE: fmc}


%%%%%%%%%%%%%%%%%%%%%%%%%%%%%%%%%%%%%%%%%%%%%%%%%%%%%%%%%%%%%%%%%%%%%%%
%%%%% DIVERS
%%%%%%%%%%%%%%%%%%%%%%%%%%%%%%%%%%%%%%%%%%%%%%%%%%%%%%%%%%%%%%%%%%%%%%%

\PutLineInToc
\DontFrameChaptersInToc

%%%%%--------------------------------------------------------------------
%%%%% Bibliographie
%%%%%--------------------------------------------------------------------

%%% Affichage de la bibliographie:
\printbibliography

% Use "\nocite{*}" to add all .bib file.
% Use this command to see which references are note cited.

%%%%%--------------------------------------------------------------------
%%%%% Liste des TODOs
%%%%%--------------------------------------------------------------------

%%% Génère la liste des choses à faire:
%\listoftodos[Liste des TODOs]

%%%%%--------------------------------------------------------------------
%%%%% Liste des figures
%%%%%--------------------------------------------------------------------

%%% Génère la liste des figures:
\renewcommand{\listfigurename}{Liste des figures}
\listoffigures

%%%%%--------------------------------------------------------------------
%%%%% Liste des tableaux
%%%%%--------------------------------------------------------------------

%%% Génère la liste des tableaux:
\renewcommand{\listtablename}{Liste des tableaux}
\listoftables

%%%%%--------------------------------------------------------------------
%%%%% Liste des algorithmes
%%%%%--------------------------------------------------------------------

%%% Génère la liste des algorithmes:
\renewcommand{\listalgorithmcfname}{Liste des algorithmes}
\listofalgorithms

%%%%%--------------------------------------------------------------------
%%%%% Liste des codes
%%%%%--------------------------------------------------------------------

%%% Génère la liste des codes:
\renewcommand{\lstlistlistingname}{Liste de codes informatiques}
\lstlistoflistings

%%%%%--------------------------------------------------------------------
%%%%% Glossaire
%%%%%--------------------------------------------------------------------

%%% Affichage du glossaire:
%\printglossaries

%%%%%--------------------------------------------------------------------
%%%%% Index
%%%%%--------------------------------------------------------------------

%%% Affichage de l'index:
%\WriteThisInToc
%\PrintIndex

\end{document}