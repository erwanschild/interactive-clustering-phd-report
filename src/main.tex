% Exemple d'utilisation de la classe thesul
% ------------------------------------------
%
% (de manière generale, les commandes de thesul sont celles qui ne sont pas complètement en minuscules)
% Voir la documentation complète pour plus de détails.
%
% D. Roegel, 30 mars 2013
%
\documentclass[11pt]{thesul}
%----------------------------------------------------------------------
%                     Chargement de quelques packages
%----------------------------------------------------------------------

% Si l'on veut produire une version PDF avec des hyperliens:
%\usepackage[pageanchor=false]{tulhypref}

% Si on veut le style de bibliographie named:
%\usepackage{named}

% Pour les figures:
\usepackage{graphicx}

% Si on veut des mini-tables des matières (utiliser minitoc-hyper en conjonction avec tulhypref) :
\usepackage[french]{minitoc}


%-------------------------------------------------------------------
%                             Marges
%-------------------------------------------------------------------

% Pour positionner les vraies marges:
%\SetRealMargins{1mm}{1mm}


%-------------------------------------------------------------------
%                             En-têtes
%-------------------------------------------------------------------

% Les en-têtes: quelques exemples:
%\UppercaseHeadings 
%\UnderlineHeadings
%\newcommand\bfheadings[1]{{\bf #1}}
%\FormatHeadingsWith{\bfheadings}
%\FormatHeadingsWith{\uppercase}
%\FormatHeadingsWith{\underline}
\newcommand\upun[1]{\uppercase{\underline{\underline{#1}}}}
\FormatHeadingsWith\upun

\newcommand\itheadings[1]{\textit{#1}}
\FormatHeadingsWith{\itheadings}

% pour avoir un trait sous l'en-tete:
\setlength{\HeadRuleWidth}{0.4pt}

%-------------------------------------------------------------------
%                         Les références
%-------------------------------------------------------------------

\NoChapterNumberInRef
\NoChapterPrefix

%-------------------------------------------------------------------
%                           Brouillons
%-------------------------------------------------------------------

% Ceci ajoute une marque « brouillon » et la date:
%\ThesisDraft

%-------------------------------------------------------------------
%                   Pour collecter un glossaire et un index
%-------------------------------------------------------------------

\makeglossary
\makeindex

\begin{document}


\OddHead={{\leftmark\rightmark}{\hfil\slshape\rightmark}}
\EvenHead={{\leftmark}{{\slshape\leftmark}\hfil}}
\OddFoot={\hfil\thepage}
\EvenFoot={\thepage\hfil}
\pagestyle{ThesisHeadingsII}

%-------------------------------------------------------------------
%                          Encadrements
%-------------------------------------------------------------------

% encadre les chapitres dans la table des matières:
% (ces commandes doivent figurer apres \begin{document}

\FrameChaptersInToc
%\FramePartsInToc


%-------------------------------------------------------------------
%            Réinitialisation de la numérotation des chapitres
%-------------------------------------------------------------------

% Si la commande suivante est présente,
% elle doit figurer APRÈS \begin{document}
% et avant la première commande \part
\ResetChaptersAtParts

%-------------------------------------------------------------------
%               mini-tables des matières par chapitre
%-------------------------------------------------------------------

% préparer les mini-tables des matières par chapitre.
% (commande de minitoc.sty)
\dominitoc

%-------------------------------------------------------------------
%                         Page de titre:
%-------------------------------------------------------------------

\ThesisTitle{Comment j'ai r\'eussi \`a prouver mon existence de fou}
\ThesisDate{28 janvier 1986}
\ThesisAuthor{Auteur anonyme}

% Type de la these
\ThesisUL

% Jury:

% (ne pas mettre de \\ apres la dernière entree)

% Exemple de création d'une nouvelle catégorie dans le jury:

\NewJuryCategory{family}{\it Membre de la famille :}
{\it Membres de la famille :}

\family={Mon frère\\Ma sœur}

\def\blanc{\hspace*{1cm}}

\President    = {
Le président
}
\Rapporteurs  = {
Le rapporteur 1&de Paris\\
Le rapporteur 2\\
\blanc suite&taratata\\
Le rapporteur 3
}
\Examinateurs = {
L'examinateur 1&d'ici\\
L'examinateur 2
}

% Création de la page de titre:
\MakeThesisTitlePage

% on peut en faire plusieurs:
%\MakeThesisTitlePage

%-------------------------------------------------------------------


%-------------------------------------------------------------------
%                          remerciements
%-------------------------------------------------------------------

%\DontFrameThisInToc
\begin{ThesisAcknowledgments}
	Les remerciements.
\end{ThesisAcknowledgments}

%-------------------------------------------------------------------
%                            dédicace
%-------------------------------------------------------------------

\begin{ThesisDedication}
	Je dédie cette thèse\\
	à ma machine.\\
	Oui, à Pandore,\\
	qui fut la première de toutes.
\end{ThesisDedication}


%-------------------------------------------------------------------
%                  écriture de `Chapitre' et `Partie' 
%                      dans la table des matières
%-------------------------------------------------------------------

\WritePartLabelInToc
\WriteChapterLabelInToc

%-------------------------------------------------------------------
%                        table des matières
%-------------------------------------------------------------------

\tableofcontents

%-------------------------------------------------------------------
%              Exemple d'utilisation de \SpecialSection
%-------------------------------------------------------------------

\SpecialSection{Introduction générale}


%\FrameThisInToc
\DontNumberThisInToc
\part{Une partie}

% Pour ne pas avoir le mot « Chapitre » au début de chaque chapitre.
\NoChapterHead

\DontWriteThisInToc
\listoffigures

\WriteThisInToc
\FrameThisInToc
\NumberThisInToc
\part*{Introduction (La première partie)}

% La commande \mainmatter permet de passer
% à la numérotation arabe (ce que fait \pagenumbering{arabic}) 
% et de faire commencer la nouvelle page 1 sur une page impaire.
% On évitera donc d'utiliser directement \pagenumbering{arabic}.
\mainmatter

\NumberThisInToc
\chapter*{Introduction (Le premier chapitre)}
\section{Une première section}

Une « autre » page avec « plein » de texte « et » très varié.
Une autre page avec plein de texte très varié .
Une autre page avec plein de texte très varié.

\DontFrameThisInToc
\chapter*{Introduction (Le premier chapitre)}

Une autre page avec plein de texte très varié.
Une autre page avec plein de texte très varié.

\chapter{Encore un chapitre (test de \oe)}

Une autre page avec plein de texte très varié.
Une autre page avec plein de texte très varié.

\DontFrameThisInToc
\chapter{Et un chapitre (cha\^{\i}ne ou cha\^\i ne ?)}
\minitoc

Une autre page avec plein de texte très varié.
Une autre page avec plein de texte très varié.

\section{Ceci est une section}

Une autre page avec plein de texte très varié.
Une autre page avec plein de texte très varié.
Une autre page avec plein de texte très varié.
(cf. le label \S\ref{toto})
Une autre page avec plein de texte très varié.

\section{Une autre section}

Une autre page «avec» <<plein>> de texte très varié.
Une autre page avec plein de texte très varié.
Une autre page avec plein de texte très varié.

% On met une petite ligne de séparation dans la table des
% matières, ainsi qu'un saut de page.
%
\PutLineInToc
\PutNewPageInToc

%\DontFrameThisInToc
\WriteThisInToc
\chapter{Le second chapitre}
\minitoc

Une autre page avec plein de texte très varié.
Une autre page avec plein de texte très varié.

\section{Une section}

Un label.\label{toto}


Une autre page avec plein de texte très varié.
Une autre page avec plein de texte très varié.

\section{Encore une section}

Une autre page avec plein de texte très varié.
Une autre page avec plein de texte très varié.
Une autre page avec


\NumberThisInToc
\part{Une autre partie}

\FrameThisInToc
\chapter{Le troisième chapitre}
\minitoc
plein de texte très varié.
Une autre page avec plein de texte très varié.

\section{Une section du troisième chapitre}

Une autre page avec plein de texte très varié.
Une autre page avec plein de texte très varié.
Une autre page avec plein de texte très varié.

\PutLineInToc

\section{Une section}

Une autre page avec plein de texte très varié.

\section{Une section}

Une autre page\index{page} avec plein de texte très varié.
Une autre page avec plein de texte très varié.

\section{Une section}

Une autre page avec plein\index{plein} de texte très varié.
Une autre page avec plein de texte très varié.
Une autre page avec p
lein de texte très varié.

\section{Une section}

Une autre page avec plein de texte très varié.

\section{Une section}

Une autre page avec plein de texte très varié.

\section{Une section}

Une autre page avec plein de texte très varié.
Une autre page avec plein de texte très varié.

\chapter{Le quatri\`eme chapitre}

plein de texte très varié.
Une autre page avec plein de texte très varié.

\section{Une section}

Une autre page avec plein de texte très varié.
Une autre page avec plein de texte très varié.
Une autre page avec plein de texte très varié.

\section{Une section}

Une autre page avec plein de texte très varié.

% En cours de route, on peut changer le cadrage par défaut:
\DontFrameChaptersInToc

\Annexes

\Annex{Première annexe}

Plein de texte très varié.
Une autre page avec plein de texte très varié.

\section{Une section}

Une autre page avec plein de texte très varié.
Une autre page avec plein de texte très varié.
Une autre page avec plein de texte très varié.




\Glossary{Chat1}{animal}\Glossary{Chien1}{Autre animal}
\Glossary{Chat2}{animal}\Glossary{Chien2}{Autre animal}
\Glossary{Chat3}{animal}\Glossary{Chien3}{Autre animal}
\Glossary{Chat4}{animal}\Glossary{Chien4}{Autre animal}
\Glossary{Chat5}{animal}\Glossary{Chien5}{Autre animal}
\Glossary{Chat6}{animal}\Glossary{Chien6}{Autre animal}
\Glossary{Chat7}{animal}\Glossary{Chien7}{Autre animal}
\Glossary{Chat8}{animal}\Glossary{Chien8}{Autre animal}
\Glossary{Chat9}{animal}\Glossary{Chien9}{Autre animal}
\Glossary{Chat10}{animal}\Glossary{Chien10}{Autre animal}
\Glossary{Chat11}{animal}\Glossary{Chien11}{Autre animal}
\Glossary{Chat12}{animal}\Glossary{Chien12}{Autre animal}
\Glossary{Chat13}{animal}\Glossary{Chien13}{Autre animal}
\Glossary{Chat14}{animal}\Glossary{Chien14}{Autre animal}
\Glossary{Chat15}{animal}\Glossary{Chien15}{Autre animal}
\Glossary{Chat16}{animal}\Glossary{Chien16}{Autre animal}
\Glossary{Chat17}{animal}\Glossary{Chien17}{Autre animal}
\Glossary{Chat18}{animal}\Glossary{Chien18}{Autre animal}
\Glossary{Chat19}{animal}\Glossary{Chien19}{Autre animal}
\Glossary{Chat22}{animal}\Glossary{Chien22}{Autre animal}
\Glossary{Chat23}{animal}\Glossary{Chien23}{Autre animal}
\Glossary{Chat24}{animal}\Glossary{Chien24}{Autre animal}
\Glossary{Chat25}{animal}\Glossary{Chien25}{Autre animal}
\Glossary{Chat26}{animal}\Glossary{Chien26}{Autre animal}
\Glossary{Chat27}{animal}\Glossary{Chien27}{Autre animal}
\Glossary{Chat28}{animal}\Glossary{Chien28}{Autre animal}
\Glossary{Chat29}{animal}\Glossary{Chien29}{Autre animal}

%-------------------------------------------------------------------
%                         Le glossaire
%-------------------------------------------------------------------
\BeginGloWith{Voici un glossaire tout-à-fait fictif,
	introduit par un texte sur toute la largeur
	des deux colonnes.}
\twocolumn
\PrintGlossary

%-------------------------------------------------------------------
%              L'index (toujours sur deux colonnes)
%-------------------------------------------------------------------
\BeginIndWith{Voici un index}
\PrintIndex

\onecolumn

%-------------------------------------------------------------------
%                       La bibliographie
%-------------------------------------------------------------------

% La bibliographie (comme d'habitude)

%\nocite{*}
%\bibliographystyle{named}
%\bibliography{TC}

\begin{thebibliography}{Lam91a}

	\bibitem[CM88]{chandy88a}
	K.~M. Chandy and J.~Misra.
	\newblock \emph{Parallel program design: a foundation}.
	\newblock Addison-Wesley Publishing Company, 1988.

	\bibitem[Lam91a]{lamport91a}
	Leslie Lamport.
	\newblock {The Temporal Logic of Actions}.
	\newblock Technical Report~79, SRC, 1991.

\end{thebibliography}

%-------------------------------------------------------------------
%                          Les résumés
%-------------------------------------------------------------------
% (si le résumé apparaît sur une colonne étroite, avec la
% bibliographie à gauche, c'est sans doute parce que vous avez
% oublié de générer les fichiers d'index et de glossaire...)

\NumberAbstractPages
\begin{ThesisAbstract}
	\begin{FrenchAbstract}
		Le résumé.
		\KeyWords{chat, chien, puces.}
	\end{FrenchAbstract}
	\begin{EnglishAbstract}
		% résumé du DEC SRC Research Report #12
		% Date: June 23, 1986
		% 
		%     "Fractional Cascading."
		%     Bernard Chazelle and Leonidas J. Guibas.
		%     58 pages.
		In computational geometry many search problems and range queries
		can be solved by performing an iterative search for the same key
		in separate ordered lists.  In Part I of this report we show that,
		if these ordered lists can be put in a one-to-one correspondence
		with the nodes of a graph of degree  d  so that the iterative
		search always proceeds along edges of that graph, then we can
		do much better than the obvious sequence of binary searches. Without
		expanding the storage by more than a constant factor, we can build
		a data-structure, called a fractional cascading structure,
		in which all original searches after the first can be carried
		out at only  log d  extra cost per search.  Several results related
		to the dynamization of this structure are also presented. Part
		II gives  numerous applications of this technique to geometric
		problems.

		Examples include intersecting a polygonal path with
		a line, slanted range search, orthogonal range search, computing
		locus functions, and others. Some results on the optimality of
		fractional cascading, and certain extensions of the technique
		for retrieving additional information are also included.
		\KeyWords{cat, dog, flees.}
	\end{EnglishAbstract}
\end{ThesisAbstract}

\end{document}
