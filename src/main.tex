%%
%% * DOCUMENT:
%%   - title: `PhD report on Interactive Clustering`
%%   - autor: Erwan SCHILD
%%   - date: 02/09/2022
%% * LATEX TEMPLATE:
%%   - title: thesul v0.15
%%   - autor: Denis ROEGEL
%%   - date: 30 mars 2013, update 12 mars 2022
%%   - url: [thesul](https://members.loria.fr/DRoegel/TeX/TUL.html)
%%

%%%%%%%%%%%%%%%%%%%%%%%%%%%%%%%%%%%%%%%%%%%%%%%%%%%%%%%%%%%%%%%%%%%%%%%
%%%%% PARAMÈTRES DU DOCUMENT
%%%%%%%%%%%%%%%%%%%%%%%%%%%%%%%%%%%%%%%%%%%%%%%%%%%%%%%%%%%%%%%%%%%%%%%

%%% Pour la classe du document:
\documentclass[11pt]{thesul}

%%% Pour déclarer un brouillon:
% \ThesisDraft

%%% Pour les vraies marges:
% \SetRealMargins{1mm}{1mm}


%%%%%--------------------------------------------------------------------
%%%%% Chargement des packages
%%%%%--------------------------------------------------------------------

%%% Pour un PDF avec des hyperliens:
\usepackage[pageanchor=false]{tulhypref}

%%% Pour une bibliographie au styme 'named':
% \usepackage{named}

%%% Pour les tableaux:
% \usepackage{array}
% \usepackage{multirow}

%%% Pour les figures:
\usepackage{graphicx}

%%% Pour les algorithmes:
\usepackage{algorithm}

%%% Pour les mini-tables des matières:
\usepackage[french]{minitoc}

%%% Pour le glossaire:
\usepackage{glossaries}


%%%%%--------------------------------------------------------------------
%%%%% Paramètres des en-têtes
%%%%%--------------------------------------------------------------------

%%% Exemples d'en-têtes:
% \UppercaseHeadings 
% \UnderlineHeadings
% \newcommand\bfheadings[1]{{\bf #1}}
% \FormatHeadingsWith{\bfheadings}
% \FormatHeadingsWith{\uppercase}
% \FormatHeadingsWith{\underline}

\newcommand\upun[1]{\uppercase{\underline{\underline{#1}}}}
\FormatHeadingsWith\upun
\newcommand\itheadings[1]{\textit{#1}}
\FormatHeadingsWith{\itheadings}

% Pour un trait sous une en-tete:
\setlength{\HeadRuleWidth}{0.4pt}


%%%%%--------------------------------------------------------------------
%%%%% Paramètres des référence, du glossaire et de l'index
%%%%%--------------------------------------------------------------------

%%% Enlever les numéro et préfix des chapitres:
\NoChapterNumberInRef
\NoChapterPrefix

%%% Faire un glossaire:
\makeglossary

%%% Faire un index:
\makeindex


%%%%%%%%%%%%%%%%%%%%%%%%%%%%%%%%%%%%%%%%%%%%%%%%%%%%%%%%%%%%%%%%%%%%%%%
%%%%% DÉBUT DU DOCUMENT
%%%%%%%%%%%%%%%%%%%%%%%%%%%%%%%%%%%%%%%%%%%%%%%%%%%%%%%%%%%%%%%%%%%%%%%

\begin{document}

%%% Pour définir le style de pages:
\OddHead={{\leftmark\rightmark}{\hfil\slshape\rightmark}}
\EvenHead={{\leftmark}{{\slshape\leftmark}\hfil}}
\OddFoot={\hfil\thepage}
\EvenFoot={\thepage\hfil}
\pagestyle{ThesisHeadingsII}

%%%%%--------------------------------------------------------------------
%%%%% Paramétrage de la table des matières (encadrement, numérotation, ...)
%%%%%--------------------------------------------------------------------

%%% Ecrire "Partie" dans la table des matières:
% \WritePartLabelInToc
%%% Ecrire "Chapitre" dans la table des matières:
% \WriteChapterLabelInToc
%%% Ajouter ou non le prochain titre à la table des matières:
% \WriteThisInToc
% \DontWriteThisInToc

%%% Ajouter une ligne ou un saut de page dans la table des matières:
% \PutLineInToc
% \PutNewPageInToc

%%% Pour encadrer encadre les parties dans la table des matières:
%%% (nb: ces commandes doivent figurer apres \begin{document})
% \FramePartsInToc
\DontFramePartsInToc
%%% Pour encadrer encadre les chapitres dans la table des matières:
%%% (nb: ces commandes doivent figurer apres \begin{document})
\FrameChaptersInToc
% \DontFrameChaptersInToc
%%% Encadrer ou non le prochain titre:
% \FrameThisInToc
% \DontFrameThisInToc

%%% Pour réinitialiser la numérotation des chapitres à chaque partie:
%%% (nb: ces commandes doivent figurer apres \begin{document})
\ResetChaptersAtParts
%%% Numéroter ou non le prochain titre dans la table des matières:
% \NumberThisInToc
% \DontNumberThisInToc

%%%%%--------------------------------------------------------------------
%%%%% Mini-table des matières à chaque chapitre
%%%%%--------------------------------------------------------------------

%%% Préparer les mini-tables des matières par chapitre (commande de minitoc.sty):
\dominitoc

%%% Créer une mini table du chapitre à cet endroit:
% \minitoc

%%%%%--------------------------------------------------------------------
%%%%% Page de titre
%%%%%--------------------------------------------------------------------

%%% Titre:
\ThesisTitle{
    Faciliter la conception d'un assistant conversationnel avec le clustering interactif
}

%%% Date:
\ThesisDate{09 setpembre 2022}

%%% Auteur:
\ThesisAuthor{Erwan SCHILD}

%%% Type de these:
\ThesisUL

%%% Membres du jury:
\President={
M. Le président de la soutenance
}
\Rapporteurs ={
Le rapporteur 1\\
Le rapporteur 2
}
\Examinateurs={
L'examinateur 1\\
L'examinateur 2
}
\Encadrants={
Dr. Jean-Charles LAMIREL\\
Dr. Florian MICONI
}

%%% Création de la page de titre:
\MakeThesisTitlePage

%%%%%--------------------------------------------------------------------
%%%%% Abstract
%%%%%--------------------------------------------------------------------

%%% (si le résumé apparaît sur une colonne étroite, avec la bibliographie à gauche, c'est sans doute parce que vous avez oublié de générer les fichiers d'index et de glossaire...)

\NumberAbstractPages
\begin{ThesisAbstract}

    %%% Abstract en français
    \begin{FrenchAbstract}
        Le résumé.
        \KeyWords{chat, chien, puces.}
    \end{FrenchAbstract}

    %%% Abstract en anglais
    \begin{EnglishAbstract}
        In computational geometry many search problems and range queries can be solved by performing an iterative search for the same key in separate ordered lists.  In Part I of this report we show that, if these ordered lists can be put in a one-to-one correspondence with the nodes of a graph of degree  d  so that the iterative search always proceeds along edges of that graph, then we can do much better than the obvious sequence of binary searches. Without expanding the storage by more than a constant factor, we can build a data-structure, called a fractional cascading structure, in which all original searches after the first can be carried out at only  log d  extra cost per search.  Several results related to the dynamization of this structure are also presented. Part II gives  numerous applications of this technique to geometric problems.
        Examples include intersecting a polygonal path with a line, slanted range search, orthogonal range search, computing locus functions, and others. Some results on the optimality of fractional cascading, and certain extensions of the technique for retrieving additional information are also included.
        \KeyWords{cat, dog, flees.}
    \end{EnglishAbstract}
\end{ThesisAbstract}

%%%%%--------------------------------------------------------------------
%%%%% Page de remerciements
%%%%%--------------------------------------------------------------------

% \WriteThisInToc
\begin{ThesisAcknowledgments}
    Les remerciements.\\
    Ma femme.\\
    Mes amis.\\
    Mon chien.\\
\end{ThesisAcknowledgments}

%%%%%--------------------------------------------------------------------
%%%%% Page de dédicaces
%%%%%--------------------------------------------------------------------

% \WriteThisInToc
\begin{ThesisDedication}
    Je dédie cette thèse\\
    à ma machine.\\
    Oui, à Pandore,\\
    qui fut la première de toutes.
\end{ThesisDedication}

%%%%%--------------------------------------------------------------------
%%%%% Table des matières
%%%%%--------------------------------------------------------------------

%%% Génération de la table des matières:
\tableofcontents

%%% Génère la liste des figures:
\listoffigures

%%% Génère la liste des tableaux:
\listoftables

%%%%%%%%%%%%%%%%%%%%%%%%%%%%%%%%%%%%%%%%%%%%%%%%%%%%%%%%%%%%%%%%%%%%%%%
%%%%% CORPS DU DOCUMENT
%%%%%%%%%%%%%%%%%%%%%%%%%%%%%%%%%%%%%%%%%%%%%%%%%%%%%%%%%%%%%%%%%%%%%%%

%%% Commencer la numérotation arabe (cf. '\pagenumbering{arabic}') avec la page 1 sur une page impaire:
\mainmatter

%%%%%--------------------------------------------------------------------
%%%%% Introduction
%%%%%--------------------------------------------------------------------
\chapter*{Introduction}
    \label{chapter:INTRODUCTION}
	\todo[inline]{CHAPITRE À REFORMULER FAÇON SWALES}

    %%%%%--------------------------------------------------------------------
    %%%%% Section I.1:
    %%%%%--------------------------------------------------------------------
    \section{"\textit{Asset centrality}"}
	\todo[inline]{SECTION À RÉDIGER}

    \begin{todolist}
    
        \item Des enjeux ou problèmes actuels

            •	Accessibilité à l'information :
                o	Grosses bases documentaires, pas toujours ordonnées ;

            •	Relations client à distance
                o   Besoin d'un accessibilité h24 ;

        \item Utilisation de plus en plus fréquente des chatbots
        
            •   Description succinte ;

            •   Cas d'usage usuels ;

            •   Tous les canaux d'utilisation ;

            •   Avantages et Dérives potentiels de l'utilisation (emploi, biais, pertinence, ergonomie, ...) ;

        \item Révolution techniques fréquentes (règles, classification, modèles)

            •	Moteurs de règles :
                o	Basé sur la détecté de mots clés,
                o	(+) facile à mettre en œuvre,
                o	(-) peu robuste au langage naturel,
                o	Paramétrage des réponses ;

            •	Paramétrage intentions-entités :
                o	Classification d’intention et/ou détection d’entités,
                o	(+) plus robuste au langage naturel, facile à paramétrer, réponses contrôlées,
                o	(-) demande de l’entrainement, des données, …,
                o	Paramétrage des réponses ;

            •	Génération de réponse :
                o	Réseau de neurones avec attention,
                o	Transformers,
                o	(+) plus robuste,
                o	(-) plus complexe à mettre en œuvre, réponses non contrôlées,
                o	Réponses non paramétrées ;

            •	Approche hybride :
                o   Cumul des trois approches pour cumuler certains avantages suivant les besoins ;
	\end{todolist}

    %%%%%--------------------------------------------------------------------
    %%%%% Section I.2:
    %%%%%--------------------------------------------------------------------
	
	\section{"\textit{Estabilishing a Niche}"}
	\todo[inline]{SECTION À RÉDIGER}

    \begin{todolist}

        \item Cadre industriel

            •	Algorithme fixe

            •	Données spécifiques

        \item GAP: Besoins de données

            •	Collecte de données spécifiqus au domaine traité :
                o   extraction de base de données (solution simple),
                o   collecte manuelle (organisation complexe, biais de collecte),
                o   scraping (pas toujours fiable) ;

            •	Nombreux biais :
                o   Biais,,
                o   Réglementation,
                o   Compétences (NOTRE COEUR DU SUJET),
                o   ...
    \end{todolist}

    %%%%%--------------------------------------------------------------------
    %%%%% Section I.3:
    %%%%%--------------------------------------------------------------------
	
	\section{"\textit{Occupying the Niche}"}
	\todo[inline]{SECTION À RÉDIGER}

    \begin{todolist}
        \item Etude de l'organisation d'une entreprise pour concevoir ses jeux de données
        \item Etude de l'état de l'art pour concevoir des jeux de données
        \item proposition/contribution : une méthode adaptée pour un cadre industriel
    \end{todolist}

%%%%%--------------------------------------------------------------------
%%%%% Chapitre 1Table des matières
%%%%%--------------------------------------------------------------------
\chapter{État de l'art : concevons un jeu de données}
    \label{chapter:1-ETAT-DE-L-ART}

    Dans cette partie, nous allons faire un état des lieux des méthodes pour créer le premier jeu de données nécessaire à l'entrainement d'un assistant conversationnel.
    Cela comprend une description des acteurs du projet, un rappel de l'organisation usuelle en fonction de leur compétence, et une énumération des problèmes et solutions les plus communs.
	\todo[inline]{TRANSITIOn À COMPLÉTERR}
    \todo{Rappel des contraintes industrielles}

    \minitoc

    %%%%%--------------------------------------------------------------------
    %%%%% Section 1.1:
    %%%%%--------------------------------------------------------------------
    \section{Rappel sur le fonctionnement usuel d'un chatbot}
		\todo[inline]{SECTION À RÉDIGER}

        •	Description du cas d'un chatbot \index{chatbot} "classique" modélisé à base d'intention et d'entités
            o	 On se concentre sur ces implémentations car on peut y controller les réponses (image de marque en jeu)

        •	Classification \index{chatbot!classification} d'intention (règles, classification supervisée, ...)

        •	Extraction d'entités \index{chatbot!ner} (règles, ner, ...)

        •	Mapping des réponses sur la base du couple $(intention, enites)$

        •	\textbf{CITATION}

    %%%%%--------------------------------------------------------------------
    %%%%% Section 1.2:
    %%%%%--------------------------------------------------------------------
    \section{Les étapes usuelles de conception d'un chatbot}
		\todo[inline]{SECTION À RÉDIGER}

        Préambule : l'organisation peut bien entendu varier suivant les contextes, mais la description qui suit est représentative des organisation principales

        \subsection{Définition des acteurs}

            •	Data scientistes :
                o	Experts en IA
                o	Peu de connaissance métier, i.e. peu de regarde critique sur la pertinence des résultats (autre que statistique)
            
            •	Expert métier :
                o	Pas de connaissance en IA, i.e. nécessitent des formations
                o	Connaissance métier forte, i.e. peuvent décrire la pertinence d’un résultat
            
            •	Chef de projet
                o	Pas de connaissance en IA
                o	Pas de connaissance métier
                o	Connaissance du besoin (hypothèse non vérifiée car parfois ils ne savent pas ce qu’ils veulent dû à la méconnaissance des capacités de l’IA)

        \subsection{Cadrage du projet}

            •	Objectifs :
                o	Clarification du besoin,
                o	Définition du périmètre couvert (i.e. les fonctionnalités et réponses à proposer),

            •   Livrable : un cahier des charges

        \subsection{Collecte des données}

            •	Souvent pas de données à disposition :
                o	En R\&D, "80\%" sur la recherche d’algo sur des données publiques, d'où le besoin de datascientists,
                o	En entreprise, "80\%" sur la gestion des données privées/spécifiques sur des algo connus, d'où le besoin d'experts métiers ;
    
            •	Risque de biais dans les données :
                o	Biais d’échantillon : la collecte ne représente pas la réalité,
                o	Biais de sélection : le trie de la collecte ne représente plus la réalité,
                o	Biais de confirmation : on garde les données qui nous arrangent,
                o	Biais de valeur : les données ne sont pas éthiquement représentatives,
                o	Biais de contexte : les données d’un cas d’usage ne sont pas toujours réutilisables pour un autre cas d’usage (ex : différence entre les jargons des AV clients et celui des AV conseillers) ;
                o	\textbf{A COMPLETER}
            
            •   Livrable : une collecte de données brutes

        \subsection{Modélisation d’une structure et Labellisation des données}

            •	Le coeur "métier" de la création du projet ;

            •	Objectif : Définition d'une modélisation sur la base des besoins attendus restreints au périmètre à couvrir ;

            •	En théorie :
                o	Intention: verbe d'actions,
                o	Entités: informations complémentaires, personnes, date, lieux, montants, noms de produits, ... ;

            •   Complexité de la tâche :
                o	Intention abstraite : définition difficile voir subjective, ...
                o	Annotation difficile :  différence entre théorie et pratique, données ambigues, ...
                o	Plusieurs itérations car modélisation trop théorique / pas pratique
                o	Besoins de beaucoup de formation (pour donner la compétence aux experts) et d'atelier (pour se mettre d'accord)

            •   Livrable : un jeu de données annotées

        \subsection{Entrainement et tests}

            •	Le coeur "technique" de la création du projet ;

            •	Objectif : avoir un modèle qui soit adapté à son utilisation en production

            •	En théorie :
                o	Split en train et tests
                o	Entrainement et tests
                o	Association des réponses

            •   Complexité de la tâche :
                o	Modélisation précédente pas toujours adaptée : OK pour un métier, mais pas possible à entrainer à cause de déséquilibre, de manque de données, ...
                o	Algorithme fixe mais données variables : savoir quelle modélisation est la plus adaptée est compliqué à deviner
                o	Réponses pas toujours adaptées aux questions : décalage entre entrainement (modélisation théorique) et réponse (modélisation pratique)

        \subsection{Déploiement de la première version}

            •	RAS

            •	Parfois la modélisation est décalée par rapport à l'utilisation en production
                o	Comportement en moteur de recherche avec des questions courtes
                o	Vocabulaire non maitrisé par les utilisateurs
                o	problème d'ergo ou d'expérience utilisateur

        \subsection{Amélioration continue}

            •	Vérification du comportement ;

            •	Ajustement du modèle ;

            •	Déploiement des versions suivantes.

    %%%%%--------------------------------------------------------------------
    %%%%% Section 1.2:
    %%%%%--------------------------------------------------------------------
    \section{Zoom sur la partie Modélisation et Labellisation de la base d'apprentissage}
		\todo[inline]{SECTION À RÉDIGER}

        \subsection{Création « manuelle »}

            •	Enchainement de plusieurs ateliers/cycles :
                o	Définition d’une structure en atelier et Annotation des données
                o	Premier conflit : La structure est trop théorique
                o	Redéfinition et Ré-annotation
                o	Second conflit : Les structure ou les données ne sont pas adaptées
                o	Collecte complémentaire, Redéfinition et Ré-annotation

            •	Avantages :
                o	Transmission progressive du savoir aux datascientist
                o	Test des modélisations potentielles

            •	Inconvénients :
                o	Nombreux ateliers
                o	Nombreuses remises en questions / aller-retour de conception
                o	L'avis initiale sur le périmètre à couvrir est flou quand cela concerne une centaine de demandes clients
                o	Se base sur de la connaissance que les experts métiers n’ont pas
                o	Comment les aider dans ce problème d’organisation ?

        \subsection{Création assistée par des regroupements non-supervisés}

            •	Constat :
                o	Pour des jeux de données à taille humaine (moins de 20.000 données), le premier tri est parfois "optimisé" manuellement sur la base des patterns commun (ordonnancement alphabétique)

            •	Solution :
                o	Un clustering pourrait simplifier cette tâche !
                o   Rappel : grandes lignes du fonctionnement d'un algorithme de \gls{clustering} ?
                o	NB : une section ou une annexe détaillera les algorithmes de \index{clustering} les plus utilisés

            •	Avantages :
                o	Regroupement automatique
                o	Découverte de la structure

            •	Inconvénients :
                o	Les résultats sont souvent peu pertinents
                o	Similarité par entités, et pas par intentions
                o	Nuances métiers non comprises
                o	Plusieurs soucis si le jeu de données est déséquilibré ou spécifique
                o	Absence d’un modèle de langue spécifique au contexte...
                o   parfois besoin d'hyperparamètres complexes à déterminer

        \subsection{Conception assistée par des regroupements semi-supervisés}

            •	Solution :
                o	On peut envisager ainsi de corriger le clustering en y insérant des contraintes métiers\hspace{2em}
                \cite{lampert:2018}
                o	Méthodes semi-supervisée
                o	NB : une section ou une annexe détaillera les algorithmes de clustering sous contraintes

            •   Interactions possibles avec le clustering (sur la base de proposition de l'humain)
                o	Sur les données / sur le résultat : ajouts de contraintes sur les données, suppressions ou modifications manuelles de données, réorganisation manuelles des clusters, …
                o	Sur les paramètres : modifier les hyper-paramètres, modifier le nombre de clusters, modifier les embeddings, utiliser d’autres algorithmes, …
                o	Besoin de visualisation : vue des contraintes, de la représentation vectorielle, …

            •	Avantage :
                o	On a réglé les problèmes de pertinence en ajoutant des contraintes

            •	Inconvénients : 
                o	Choisir comment modéliser ces contraintes peut être complexe
                o	Surtout énorme en ajoutant des contraintes
                o	Choisir les contraintes pertinentes est une tâche difficile

        \subsection{Conception basée sur des méthodes d’apprentissage actif}

            •	Solution :
                o	On peut demander à la machine de définir les contraintes dont elle a besoin pour s’améliorer / confirmer son comportement
                o	On peut séparer et cibler les tâches pour que le clustering se nourrissent des commentaires de l’expert et que l’expert corrige ce qui semble utile au clustering
                o	Sous-entendu : Préférer la collaboration à la supériorité (que ce soit celle de la machine ou celle de l’expert)
                o	NB : une section ou une annexe détaillera les interactions possibles entre homme et machine

            •   Interactions possibles avec le clustering (sur la base de propositions de la machine)
                o	Sur les données / sur le résultat : proposition de suppression de données abhérrantes, proposition d'ajout de contraintes à des endroits stratégiques, …
                o	Sur les paramètres : réévaluation des paramètres, combiner plusieurs algorithmes et synthétiser le résultat, …

            •	Avantage :
                o	On a réglé les problèmes de pertinence et de coûts en ajoutant des contraintes

            •	Inconvénients / problème à résoudre : 
                o	Accepter de collaborer avec la machine (problème UX, ergo, accompagnement au changement)
                o	Il faut prouver cette méthode

%%%%%--------------------------------------------------------------------
%%%%% Chapitre 2
%%%%%--------------------------------------------------------------------
\chapter{Travaux}

    \label{CHAP_2_TRAVAUX}

    \minitoc

    \section{Dans la fiction}

        \subsection{Rêvons un peu...}
            Une « autre » page avec « plein » de texte « et » très varié.
            Une autre page avec plein de texte très varié .
            Une autre page avec plein de texte très varié.

        \subsection{Dans les films...}
            Une « autre » page avec « plein » de texte « et » très varié.
            Une autre page avec plein de texte très varié .
            Une autre page avec plein de texte très varié.

        \subsection{Dans les comics...}
            Une « autre » page avec « plein » de texte « et » très varié.
            Une autre page avec plein de texte très varié .
            Une autre page avec plein de texte très varié.

    \section{Dans la recherche}

        \subsection{Depuis 1980...}
            Une autre page avec plein de texte très varié.
            Une autre page avec plein de texte très varié.

        \subsection{Depuis 1990...}
            Une autre page avec plein de texte très varié.
            Une autre page avec plein de texte très varié.

        \subsection{Depuis 2000...}
            Une autre page avec plein de texte très varié.
            Une autre page avec plein de texte très varié.

            \begin{table}[htbp]
                \center\small
                \begin{tabular*}{0.8\linewidth}{cc}
                    \hline
                    \hline
                    INTENTION 
                        & EXEMPLE
                    \tabularnewline
                    \hline
                    \textit{carte avalée}
                        & Ma carte a été avalée par un distributeur !
                        \tabularnewline
                    \textit{commande de carte}
                        & Est-ce que je peux avoir une carte sans frais ?
                        \tabularnewline
                    \textit{consultation du solde}
                        & Comment savoir si je suis à découvert ?
                        \tabularnewline
                \end{tabular*}
                \caption{Exemple de questions pour chaque intention du jeu de données utilisé.}
                \label{TABLEAU_DATASET_TUTO_CARTE_BANCAIRE_V1}
            \end{table}

        \subsection{Depuis 2010...}
            Une autre page avec plein de texte très varié.
            Une autre page avec plein de texte très varié.

        \subsection{Depuis 2022...}
            Une autre page avec plein de texte très varié.
            Une autre page avec plein de texte très varié.

    \section{Dans l'industrie}
    \index{industrie}

        \subsection{Avec XXXXX...}
            \index{industrie!XXXXX}
            Une autre page avec plein de texte très varié.
            Une autre page avec plein de texte très varié.

        \subsection{Avec YYYY...}
        \index{industrie!YYYY}
            Une autre page avec plein de texte très varié.
            Une autre page avec plein de texte très varié.

        \subsection{Depuis ZZZZZ...}
        \index{industrie!ZZZZZ}
            Une autre page avec plein de texte très varié.
            Une autre page avec plein de texte très varié.

%%%%%--------------------------------------------------------------------
%%%%% Conclusions
%%%%%--------------------------------------------------------------------
\chapter*{Conclusion}
    \label{chapter:CONCLUSION}

    %%%%%--------------------------------------------------------------------
    %%%%% Section C.1:
    %%%%%--------------------------------------------------------------------
    \section{Rappel de la problématique ??}

    TODO

    %%%%%--------------------------------------------------------------------
    %%%%% Section C.2:
    %%%%%--------------------------------------------------------------------
    \section{Avantage et limites de la méthodes??}

    TODO

    %%%%%--------------------------------------------------------------------
    %%%%% Section C.3:
    %%%%%--------------------------------------------------------------------
    \section{Ouverture ??}

    TODO

%%%%%%%%%%%%%%%%%%%%%%%%%%%%%%%%%%%%%%%%%%%%%%%%%%%%%%%%%%%%%%%%%%%%%%%
%%%%% ANNEXES DU DOCUMENT
%%%%%%%%%%%%%%%%%%%%%%%%%%%%%%%%%%%%%%%%%%%%%%%%%%%%%%%%%%%%%%%%%%%%%%%

%%% Préparer les annexes et la bibliographie:
\PutLineInToc
% \PutNewPageInToc
\DontFrameChaptersInToc

%%%%%--------------------------------------------------------------------
%%%%% Annexes
%%%%%--------------------------------------------------------------------
\label{ANNEXES}
\Annexes

\Annex{Annexe théorique}
\label{annex:A-ANNEXE-THEORIQUE}

	\minitoc

    \section{Les algorithmes de clustering }

        \subsection{Kmeans}
        kmeans \index{clustering!kmeans}

        \subsection{Hierarchique}
        hierarchique \index{clustering!hierarchique}

        \subsection{Spectral}
        spectral \index{clustering!spectral}

        \subsection{DBScan}
        dbscan \index{clustering!dbscan}

        \subsection{Affinity Propagation}
        affinity propagation \index{clustering!affinity propagation}


    \section{Evaluation d’une clustering}

        \subsection{Homogénéité – Complétude – Vmeasure}
        la VMeasure \index{vmeasure} est la moyenne harmonique entre l'homogénéité et la complétude.
        
        \subsection{FMC}

\Annex{Annexe technique}
\label{annex:B-ANNEXE-TECHNIQUE}

	\minitoc

    \section{package pypi interactive-clustering}
    \section{package pypi interactive-clustering-gui}
    \section{package pypi features-maximization-metrics}
    \section{experimentations jupyter notebook}

\Annex{Annexe des jeux de données}
\label{annex:C-ANNEXE-DATASET}

	\minitoc

    \section{BANK CARDS: french bank cards}
	\label{annex:C.1-DATASET-BANK-CARDS}
	
		[FR] Jeu d'entraînement en français d'assistants conversationnels traitant des demandes courantes sur les cartes bancaires.
		
		Description : Cet ensemble de données représente des exemples de demandes usuelles des clients concernant la gestion des cartes bancaires. Il peut être utilisé comme jeu d'entraînement pour un assistant conversationnel destiné à traiter ces demandes courantes.
		
		Contenu : Les questions sont formulées en français. L'ensemble de données est divisé en 10 intentions de 50 questions chacune, pour un total de 500 questions.
		
		Périmètre des intentions : Les intentions sont construites de telle manière que toutes les questions issues d'une même intention ont la même réponse ou action. Le périmètre couvert concerne : la perte ou le vol de cartes ; la carte avalée ; la commande des cartes ; la consultation du solde bancaire ; l'assurance fournie par une carte ; le déverrouillage de la carte ; la gestion de cartes virtuelles ; la gestion du découvert bancaire ; la gestion des plafonds de paiement ; la gestion du mode sans contact.
		
		Origine : Le périmètre des intentions est inspiré par un chatbot actuellement en production, et la formulation des questions est inspirée de demandes courantes de clients.
	
    \section{MLSUM: press titles}
	\label{annex:C.2-DATASET-MLSUM-SUBSET-SCHILD}
	
		We present MLSUM, the first large-scale MultiLingual SUMmarization dataset.
		Obtained from online newspapers, it contains 1.5M+ article/summary pairs in five different languages -- namely, French, German, Spanish, Russian, Turkish. Together with English newspapers from the popular CNN/Daily mail dataset, the collected data form a large scale multilingual dataset which can enable new research directions for the text summarization community. We report cross-lingual comparative analyses based on state-of-the-art systems. These highlight existing biases which motivate the use of a multi-lingual dataset.
		
		For constraints annotation experiment based on data similarity, this dataset have been subsetted (randomly pick 75 articles in the following 14 most used topics: 'economie', 'politique', 'sport', 'planete' (renamed in 'ecologie'), 'sciences', 'police-justice', 'disparitions', 'emploi', 'sante', 'musiques', 'arts', 'educations', 'climat' (renamed in 'meteo'), 'immobilier') and filtered (keep articles that have an obvious topics regarding their titles, without their bodies). Two reviewers have working on this task in order to limit the subjectivity of the filtering. This subsetted dataset is used (1) to estimate needed time to annotate titles similarity with constraints (MUST-LINK, CANNOT-LINK) and (2) to test interactive clustering methodology (constraints annotation and constrained clustering).


%%%%%--------------------------------------------------------------------
%%%%% Bibliographie
%%%%%--------------------------------------------------------------------
%%% Définition de la bibliographie:
\bibliographystyle{apalike}
\bibliography{references/bibliography}

%%%%%--------------------------------------------------------------------
%%%%% glossaire
%%%%%--------------------------------------------------------------------
%%% Définition du glossaire:
\newglossaryentry{comics}{
    name=comics,
    description={une bande dessinée à parution régulière}}
\newglossaryentry{1980}{
    name={Années 1980},
    description={une decénie de musiques chouettes}}
\newglossaryentry{1990}{
    name={Années 1990},
    description={une decénie de musiques discutables}}

%%% Définition des accronymes:
\newacronym{svm4}{SVM}{Support Vector Machine}

\WriteThisInToc
\printglossaries

%%%%%--------------------------------------------------------------------
%%%%% Index
%%%%%--------------------------------------------------------------------
\WriteThisInToc
\printindex

\end{document}