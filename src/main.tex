%% * DOCUMENT:
%%   - title: `PhD report on Interactive Clustering`
%%   - autor: Erwan SCHILD
%%   - date: 02/09/2022
%%   - url: [erwanschild/interactive-clustering-phd-report](https://github.com/erwanschild/interactive-clustering-phd-report)
%% * LATEX TEMPLATE:
%%   - title: thesul v0.15
%%   - autor: Denis ROEGEL
%%   - date: 30 mars 2013, update 12 mars 2022
%%   - url: [thesul](https://members.loria.fr/DRoegel/TeX/TUL.html)

%%%%%%%%%%%%%%%%%%%%%%%%%%%%%%%%%%%%%%%%%%%%%%%%%%%%%%%%%%%%%%%%%%%%%%%
%%%%% PARAMÈTRES DU DOCUMENT
%%%%%%%%%%%%%%%%%%%%%%%%%%%%%%%%%%%%%%%%%%%%%%%%%%%%%%%%%%%%%%%%%%%%%%%

%%% Pour la classe du document:
\documentclass[11pt]{template/thesul}

%%% Pour déclarer un brouillon:
% \ThesisDraft

%%% Pour les vraies marges:
% \SetRealMargins{1mm}{1mm}

%%% Magie noire..
\usepackage{morewrites}  % for "! No room for a new \write."
%\morewritessetup{ allocate = 32 }

%%%%%--------------------------------------------------------------------
%%%%% Chargement des packages
%%%%%--------------------------------------------------------------------

%%% Pour la langue:
\usepackage[french]{babel}
\usepackage[T1]{fontenc}
\usepackage{lettrine}

%%% Pour un PDF avec des hyperliens:
\usepackage[pageanchor=true]{template/tulhypref}

%%% Pour une bibliographie au styme 'named':
% \usepackage{named}

%%% Pour les notes à faire:
\usepackage{todonotes}

%%% Pour les styles:
\usepackage{
	amssymb,
	pifont,
	xcolor,  % http://latexcolor.com/
}

\definecolor{colorDarkPastelRed}{HTML}{C23B21}  % Dark Pastel Red (#C23B21)
\definecolor{colorCarrotOrange}{HTML}{F39A27}  % Carrot Orange (#F39A27)
	%\definecolor{colorCadmiumOrange}{HTML}{ED872E}  % Cadmium Orange (#ED872E)
\definecolor{colorMinionYellow}{HTML}{EADA52}  % Minion Yellow (#EADA52)
\definecolor{colorDarkPastelGreen}{HTML}{03C03C}  % Dark Pastel Green (#03C03C)
	%\definecolor{colorDarkPastelGreen}{HTML}{03BF3D}  % Dark Pastel Green (#03BF3D)
\definecolor{colorSilverLakeBlue}{HTML}{579ABE}  % Silver Lake Blue (#579ABE)
	%\definecolor{colorDarkPastelBlue}{HTML}{789ECC}  % Dark Pastel Blue (#789ECC)
\definecolor{colorDarkPastelPurple}{HTML}{976ED7}  % Dark Pastel Purple (#976ED7)
\definecolor{colorDimGray}{HTML}{737373}  % Dim Gray (#737373)
\definecolor{colorBlack}{HTML}{000000}  % Black (#000000)

%%% Pour les symbols:
\usepackage{fontawesome5}  % https://www.ipgp.fr/~moguilny/LaTeX/fontawesome5Icons.pdf
	% Signet: \faBookmark \faBook
	% Warning: \faExclamation \faExclamationTriangle
	% Commentaire personnel: \faCommentDots
	% Idées: \faLightbulb
	% Check: \faCheckSquare
	% MUST-LINK: \faCheck \faCheckCircle \faEquals
	% CANNOT-LINK: \faTimes \faTimesCircle \faNotEqual
	% Information, Questions ou Plus d'info: \faInfoCircle \faQuestionCircle % \faPlusCircle
	% Eprouvette: \faVial
	% Github: \faGit \faGithub

%%% Pour les listes:
\usepackage{enumitem}
\setlist{topsep=6pt, itemsep=6pt}
\newlist{todolist}{itemize}{2}
\setlist[todolist]{label=$\square$}
\newcommand{\cmark}{\ding{51}}
\newcommand{\xmark}{\ding{55}}
\newcommand{\itemok}{\rlap{$\square$}{\raisebox{2pt}{\large\hspace{1pt}\cmark}}\hspace{-2.5pt}}
\newcommand{\itemko}{\rlap{$\square$}{\large\hspace{1pt}\xmark}}

%%% Pour les tableaux:
\usepackage{array}
\usepackage{arydshln}
\usepackage{multirow}
\definecolor{colorCaption}{HTML}{737373}  % Dim Gray (#737373)
\usepackage[
	labelsep=endash,
	font={color=colorCaption},
	labelfont={sc, bf},
	textfont={it},
]{caption}

%%% Pour les figures:
\usepackage{float}
\usepackage{graphicx}

%%% Pour les schémas:
\usepackage{tikz}
\usetikzlibrary{
	arrows,
	shapes,
	automata,
	petri,
	positioning,
	calc
}

%%% Pour les encadrés:
\usepackage{tcolorbox}
\definecolor{colorTcolorboxHypothesis}{HTML}{87A86B}  % Asparagus (#87A86B)

%%% Pour les paragraphes indentés:
\usepackage{framed}  % provide \leftbar

% Renommer la commande pour accepter la gestion des couleurs.
\renewenvironment{leftbar}[2][15]
{
    \def\FrameCommand
    {
        {\color{#2}\vrule width 3pt}
        \hspace{0pt}
        \fboxsep=\FrameSep\colorbox{#2!#1}
    }
    \MakeFramed{\hsize\hsize\advance\hsize-\width\FrameRestore}
}
{\endMakeFramed}

% Définition de l'encadré important.
\definecolor{colorLeftBarImportantGreen}{HTML}{03C03C}  % Dark Pastel Green (#03C03C)
\newenvironment{leftBarImportantGreen}{
	\begin{leftbar}{colorLeftBarImportantGreen}
	\noindent
}{
    \end{leftbar}
}  % \begin{leftBarImportantGreen} \lipsum[1] \end{leftBarImportantGreen}
\definecolor{colorLeftBarImportantRed}{HTML}{C23B21}  % Dark Pastel Red (#C23B21)
\newenvironment{leftBarImportantRed}{
	\begin{leftbar}{colorLeftBarImportantRed}
	\noindent
}{
    \end{leftbar}
}  % \begin{leftBarImportantRed} \lipsum[1] \end{leftBarImportantRed}

% Définition de l'encadré d'attention.
\definecolor{colorLeftBarWarning}{HTML}{F39A27}  % Carrot Orange (#F39A27)
	%\definecolor{colorLeftBarWarning}{HTML}{ED872E}  % Cadmium Orange (#ED872E)
\newenvironment{leftBarWarning}{
	\begin{leftbar}{colorLeftBarWarning}
	\noindent
	\textbf{\textcolor{colorLeftBarWarning}{\faExclamationTriangle}~Attention :}
}{
    \end{leftbar}
}  % \begin{leftBarWarning} \lipsum[1] \end{leftBarWarning}

% Définition de l'encadré des points à retenir.
\definecolor{colorLeftBarSummary}{HTML}{03C03C}  % Dark Pastel Green (#03C03C)
\newenvironment{leftBarSummary}{
	\begin{leftbar}{colorLeftBarSummary}
	\noindent
	\textbf{\textcolor{colorLeftBarSummary}{\faBookmark}~Points à retenir :}
}{
    \end{leftbar}
}  % \begin{leftBarSummary} \lipsum[1] \end{leftBarSummary}

% Définition de l'encadré d'idée.
\definecolor{colorLeftBarIdea}{HTML}{EADA52}  % Minion Yellow (#EADA52)
	%\definecolor{colorLeftBarIdea}{HTML}{FFE036}  % Banana Yellow (#FFE036)
\newenvironment{leftBarIdea}{
	\begin{leftbar}{colorLeftBarIdea}
	\noindent
	\textbf{\textcolor{colorLeftBarIdea}{\faLightbulb}~Idée :}
}{
    \end{leftbar}
}  % \begin{leftBarIdea} \lipsum[1] \end{leftBarIdea}

% Définition de l'encadré de notes de l'auteur.
\definecolor{colorLeftBarAuthorOpinion}{HTML}{579ABE}  % Silver Lake Blue (#579ABE)
	%\definecolor{colorLeftBarAuthorOpinion}{HTML}{21ABCC}  % Ball blue (#21ABCC)
\newenvironment{leftBarAuthorOpinion}{
	\begin{leftbar}{colorLeftBarAuthorOpinion}
	\noindent
	\textbf{\textcolor{colorLeftBarAuthorOpinion}{\faCommentDots}~Note de l'auteur :}
}{
    \end{leftbar}
}  % \begin{leftBarAuthorOpinion} \lipsum[1] \end{leftBarAuthorOpinion}

% Définition de l'encadré d'informations
\definecolor{colorLeftBarInformation}{HTML}{737373}  % Dim Gray (#737373)
	%\definecolor{colorLeftBarInformation}{HTML}{91A3B0}  % Cadet grey (#91A3B0)
\newenvironment{leftBarInformation}{
	\begin{leftbar}{colorLeftBarInformation}
	\noindent
	\textbf{\textcolor{colorLeftBarInformation}{\faInfoCircle}~Pour information :}
}{
    \end{leftbar}
}  % \begin{leftBarInformation} \lipsum[1] \end{leftBarInformation}

% Définition de l'encadré d'exemples
\definecolor{colorLeftBarExamples}{HTML}{737373}  % Dim Gray (#737373)
	%\definecolor{colorLeftBarExamples}{HTML}{91A3B0}  % Cadet grey (#91A3B0)
\newenvironment{leftBarExamples}{
	\begin{leftbar}{colorLeftBarExamples}
	\noindent
	\textbf{Exemples :}
}{
    \end{leftbar}
}  % \begin{leftBarExamples} \lipsum[1] \end{leftBarExamples}


%%% Pour les algorithmes:
\usepackage{algpseudocode}
\usepackage[
	vlined,
	boxed,
	algochapter,
	linesnumbered,
	french,
	onelanguage,
]{algorithm2e} % algorithm
\SetAlgoCaptionSeparator{--}
\SetAlCapFnt{ \bfseries\scshape }
\renewcommand\AlCapSty{ \color{colorCaption} }

%%% Pour le code Python:
\usepackage{listings}
\definecolor{colorCodeString}{HTML}{990000}  % Dark red (#990000)
\definecolor{colorCodeComment}{HTML}{008000}  % Dark Green (#008000)
\definecolor{colorCodeKeyword}{HTML}{0000B3}  % Medium Blue (#0000B3)
\definecolor{colorCodeEmphasize}{HTML}{F72673}  % Deep Pink (#F72673)
\definecolor{colorCodeBackground}{HTML}{91A3B0}  % Cadet grey (#91A3B0)
\renewcommand{\lstlistingname}{Code}
\lstset{
  inputencoding=utf8,
  breaklines=true,
  captionpos=b,
  escapeinside={\%*}{*)},
  frame=lines,
  numbers=left,
  showstringspaces=false,
  morekeywords=[1]{,as,assert,nonlocal,with,yield,self,True,False,None,}
  basicstyle=\tiny,%\ttfamily,
  backgroundcolor=\color{colorCodeBackground!15},
  commentstyle=\color{colorCodeComment},
  keywordstyle=\color{colorCodeKeyword}\bfseries,
  stringstyle=\color{colorCodeString},
  emphstyle=\color{colorCodeEmphasize}\underbar,
  tabsize=2,
  literate=
    {é}{{\'e}}{1}%
    {è}{{\`e}}{1}%
	{à}{{\`a}}{1}%
}

%%% Pour les mini-tables des matières:
\usepackage[french]{minitoc}

%%% Pour les références bibliographiques:
\usepackage[style=apa, natbib=true, backend=biber]{biblatex}
\addbibresource{references/bibliography.bib}
\usepackage{csquotes}

%%% Pour le glossaire:
\usepackage[automake]{glossaries}

%%% Pour le brouillon:
\usepackage{lipsum}

%%% Pour les partitions:
\usepackage{guitar}

%%%%%--------------------------------------------------------------------
%%%%% Paramètres des en-têtes
%%%%%--------------------------------------------------------------------

%%% Exemples d'en-têtes:
% \UppercaseHeadings 
% \UnderlineHeadings
% \newcommand\bfheadings[1]{{\bf #1}}
% \FormatHeadingsWith{\bfheadings}
% \FormatHeadingsWith{\uppercase}
% \FormatHeadingsWith{\underline}

\newcommand\upun[1]{\uppercase{\underline{\underline{#1}}}}
\FormatHeadingsWith\upun
\newcommand\itheadings[1]{\textit{#1}}
\FormatHeadingsWith{\itheadings}

% Pour un trait sous une en-tete:
\setlength{\HeadRuleWidth}{0.4pt}


%%%%%--------------------------------------------------------------------
%%%%% Paramètres des référence, du glossaire et de l'index
%%%%%--------------------------------------------------------------------

%%% Enlever les numéro et préfix des chapitres:
\NoChapterNumberInRef
\NoChapterPrefix

%%% Faire un glossaire:
\makeglossaries

%%% Définition du glossaire:
%%% Definition des termes du glossaire:
\newglossaryentry{clustering}{
    name={clustering},
    description={!!TODO!!}
}
\newglossaryentry{contrainte}{
    name={contrainte},
    description={!!TODO!!}
}
\newglossaryentry{mustlink}{
    name={MUST-LINK},
    description={!!TODO!!}
	parent=contrainte
}
\newglossaryentry{cannotlink}{
    name={CANNOT-LINK},
    description={!!TODO!!}
	parent=contrainte
}
\newglossaryentry{annotation}{
    name={annotation},
    description={!!TODO!!}
}
\newglossaryentry{vmeasure}{
    name={vmeasure},
    description={!!TODO!!}
}

%%% Définition des accronymes:
\newacronym{svm}{SVM}{Support Vector Machine}
\newacronym{fmc}{FMC}{Feature Maximizaton Contrast}

%%% Faire un index:
\makeindex


%%%%%%%%%%%%%%%%%%%%%%%%%%%%%%%%%%%%%%%%%%%%%%%%%%%%%%%%%%%%%%%%%%%%%%%
%%%%% DÉBUT DU DOCUMENT
%%%%%%%%%%%%%%%%%%%%%%%%%%%%%%%%%%%%%%%%%%%%%%%%%%%%%%%%%%%%%%%%%%%%%%%

\begin{document}

%%% Pour définir le style de pages:
\OddHead={{\leftmark\rightmark}{\hfil\slshape\rightmark}}
\EvenHead={{\leftmark}{{\slshape\leftmark}\hfil}}
\OddFoot={\hfil\thepage}
\EvenFoot={\thepage\hfil}
\pagestyle{ThesisHeadingsII}

%%%%%--------------------------------------------------------------------
%%%%% Paramétrage de la table des matières (encadrement, numérotation, ...)
%%%%%--------------------------------------------------------------------

%%% Ecrire "Partie" dans la table des matières:
% \WritePartLabelInToc
%%% Ecrire "Chapitre" dans la table des matières:
% \WriteChapterLabelInToc
%%% Ajouter ou non le prochain titre à la table des matières:
% \WriteThisInToc
% \DontWriteThisInToc

%%% Ajouter une ligne ou un saut de page dans la table des matières:
% \PutLineInToc
% \PutNewPageInToc

%%% Pour encadrer encadre les parties dans la table des matières:
%%% (nb: ces commandes doivent figurer apres \begin{document})
% \FramePartsInToc
\DontFramePartsInToc
%%% Pour encadrer encadre les chapitres dans la table des matières:
%%% (nb: ces commandes doivent figurer apres \begin{document})
\FrameChaptersInToc
% \DontFrameChaptersInToc
%%% Encadrer ou non le prochain titre:
% \FrameThisInToc
% \DontFrameThisInToc

%%% Pour réinitialiser la numérotation des chapitres à chaque partie:
%%% (nb: ces commandes doivent figurer apres \begin{document})
\ResetChaptersAtParts
%%% Numéroter ou non le prochain titre dans la table des matières:
% \NumberThisInToc
% \DontNumberThisInToc

%%%%%--------------------------------------------------------------------
%%%%% Mini-table des matières à chaque chapitre
%%%%%--------------------------------------------------------------------

%%% Préparer les mini-tables des matières par chapitre (commande de minitoc.sty):
\dominitoc

%%% Créer une mini table du chapitre à cet endroit:
% \minitoc

%%%%%--------------------------------------------------------------------
%%%%% Page de titre
%%%%%--------------------------------------------------------------------

%%% Titre:
\ThesisTitle{
    Faciliter la conception d'un assistant conversationnel avec le clustering interactif
}

%%% Date:
\ThesisDate{
    01 décembre 2023
}  % 8 semaine après la date du rendu

%%% Auteur:
\ThesisAuthor{
    Erwan SCHILD
}

%%% Type de these:
\ThesisUL

%%% Membres du jury:
\President={
    Dr. ??\\
}
\Rapporteurs ={
	% rapporteurs = membre du jury évaluant le manuscrit
    Dr. Pascale KUNTZ-COSPEREC\\
    Dr. Thomas LAMPERT
}
\Examinateurs={
	% examinateurs = membre du jury présents à la soutenance
    Dr. Adrien COULET (à demander)
}
\Encadrants={
    Dr. Jean-Charles LAMIREL\\
    Dr. Florian MICONI
}
\Invites={
    Dr. Gautier DURANTIN\\
    Dr. Mathieu POWALKA
}

%%% Création de la page de titre:
\MakeThesisTitlePage

%%%%%--------------------------------------------------------------------
%%%%% Abstract
%%%%%--------------------------------------------------------------------

%%% (si le résumé apparaît sur une colonne étroite, avec la bibliographie à gauche, c'est sans doute parce que vous avez oublié de générer les fichiers d'index et de glossaire...)

\NumberAbstractPages
\begin{ThesisAbstract}

    %%% Abstract en français
    \begin{FrenchAbstract}
        Le résumé à faire.
        \KeyWords{chat, chien, puces.}
    \end{FrenchAbstract}

    %%% Abstract en anglais
    \begin{EnglishAbstract}
        The abstract to do
        \KeyWords{cat, dog, flees.}
    \end{EnglishAbstract}
\end{ThesisAbstract}

%%%%%--------------------------------------------------------------------
%%%%% Page de remerciements
%%%%%--------------------------------------------------------------------

% \WriteThisInToc
\begin{ThesisAcknowledgments}

    Par la présente, je souhaites remercier:
    \begin{todolist}
        \item ma femme
			%
        \item ma tortue
			%
        \item ma famille
			% 
        \item mes amis
			%
        \item mes collègues
			% Gautier, Amélie, Mathieu
        \item mes encadrants
			% Adrien, Jean-Charles, Florian
        \item chatGPT qui m'a aidé à corriger ce template LaTeX
        \item ...
    \end{todolist}

\end{ThesisAcknowledgments}

%%%%%--------------------------------------------------------------------
%%%%% Page de dédicaces
%%%%%--------------------------------------------------------------------

% \WriteThisInToc
\begin{ThesisDedication}

    Je dédie cette thèse\\
    à quelqu'un de bien.

\end{ThesisDedication}

%%%%%--------------------------------------------------------------------
%%%%% Table des matières
%%%%%--------------------------------------------------------------------

%%% Génération de la table des matières:
\renewcommand{\contentsname}{Table des matières}
\tableofcontents


%%% TODO généraux.
\todo[inline]{TITRE A REVOIR: Faciliter => Accélérer ? Améliorer l'accessibilité ? Limiter les biais et erreurs ? ...}

%%%%%%%%%%%%%%%%%%%%%%%%%%%%%%%%%%%%%%%%%%%%%%%%%%%%%%%%%%%%%%%%%%%%%%%
%%%%% CORPS DU DOCUMENT
%%%%%%%%%%%%%%%%%%%%%%%%%%%%%%%%%%%%%%%%%%%%%%%%%%%%%%%%%%%%%%%%%%%%%%%

%%% Commencer la numérotation arabe (cf. '\pagenumbering{arabic}') avec la page 1 sur une page impaire:
\mainmatter

%%%%%--------------------------------------------------------------------
%%%%% Introduction
%%%%%--------------------------------------------------------------------

\part{Introduction (La première partie)}
\label{PARTIE_1_INTRODUCTION}

\minitoc

    \chapter{Dans la fiction}

        \section{Rêvons un peu...}
            Une « autre » page avec « plein » de texte « et » très varié.
            Une autre page avec plein de texte très varié .
            Une autre page avec plein de texte très varié.

        \section{Dans les films...}
            Une « autre » page avec « plein » de texte « et » très varié.
            Une autre page avec plein de texte très varié .
            Une autre page avec plein de texte très varié.

        \section{Dans les comics...}
            Une « autre » page avec « plein » de texte « et » très varié.
            Une autre page avec plein de texte très varié .
            Une autre page avec plein de texte très varié.

    \chapter{Dans la recherche}

        \section{Depuis 1980...}
            Une autre page avec plein de texte très varié.
            Une autre page avec plein de texte très varié.

        \section{Depuis 1990...}
            Une autre page avec plein de texte très varié.
            Une autre page avec plein de texte très varié.

        \section{Depuis 2000...}
            Une autre page avec plein de texte très varié.
            Une autre page avec plein de texte très varié.

        \section{Depuis 2010...}
            Une autre page avec plein de texte très varié.
            Une autre page avec plein de texte très varié.

        \section{Depuis 2022...}
            Une autre page avec plein de texte très varié.
            Une autre page avec plein de texte très varié.

    \chapter{Dans l'industrie}

        \section{Avec XXXXX...}
            Une autre page avec plein de texte très varié.
            Une autre page avec plein de texte très varié.

        \section{Avec YYYY...}
            Une autre page avec plein de texte très varié.
            Une autre page avec plein de texte très varié.

        \section{Depuis ZZZZZ...}
            Une autre page avec plein de texte très varié.
            Une autre page avec plein de texte très varié.

%%%%%--------------------------------------------------------------------
%%%%% Chapitre: Etat de l'art
%%%%%--------------------------------------------------------------------
\chapter{État de l'art : concevons un jeu de données}
    \label{chapter:2-ETAT-DE-L-ART}

    Dans cette partie, nous allons faire un état des lieux des méthodes pour créer le premier jeu de données nécessaire à l’entraînement d'un assistant conversationnel.
    Cela comprend une description des acteurs du projet, un rappel de l'organisation usuelle en fonction de leur compétence, et une énumération des problèmes et solutions les plus communs.
	\todo[inline]{TRANSITION À COMPLÉTER}
    \todo{Rappel des contraintes industrielles}

    \minitoc

    %%%%%--------------------------------------------------------------------
    %%%%% Section 2.1:
    %%%%%--------------------------------------------------------------------
    \section{Rappel sur le fonctionnement d'un chatbot}
    \todo{titre: Approche statistique vs symbolique}
		\todo[inline]{SECTION À RÉDIGER}
		\todo[inline]{
			Remarque Gautier 20/02/2023:
			Le "usuel" est clairement à discuter ici.
			Il y a deux approches à la connaissance, qui sont ici à discuter, je pense :
			- une approche statistique, qui cherche DIRECTEMENT à générer la connaissance à partir de la masse de données ingérée (on y retrouve les approches génératives, par exemple)
			- une approche symbolique, dans laquelle on décide de passer par des représentations symboliques intermédiaires (les intentions et entités) comme médiateur de la réponse qu'on apporte au client
			Il n'y a pas d'approche qui soit "usuelle", à mon sens, mais uniquement deux approches de la connaissance différentes, chacune à ses avantages, et en l'occurrence on peut apprécier le pragmatisme de l'approche symbolique, puisque ça a un côté très efficace et ça permet de garder le contrôle sur le vocabulaire (les symboles) qu'on souhaite couvrir.
			Quelle que soit ta position sur le sujet, je ne pense pas que tu puisses directement parler de fonctionnement usuel sans passer d'abord en revue les différentes approches qu'on peut choisir pour concevoir un chatbot 
		}

        •	Description du cas d'un chatbot \index{chatbot} supervisé / à base d'intention et d'entités
            o	 On se concentre sur ces implémentations car on peut y contrôler les réponses (image de marque en jeu)
            o	 Classification \index{chatbot!classification} d'intention (règles, classification supervisée, ...)
            o	 Extraction d'entités \index{chatbot!ner} (règles, ner, ...)
            o	 Mapping des réponses sur la base du couple $(intention, enites)$
            o	 \todo{citation}
            
        •	Description du cas d'un chatbot \index{chatbot} non-supervisé / à base d'un modèle de langage
            o	 \todo{citation}

    %%%%%--------------------------------------------------------------------
    %%%%% Section 2.2:
    %%%%%--------------------------------------------------------------------
    \section{Les étapes usuelles de conception d'un chatbot}
		\todo[inline]{SECTION À RÉDIGER}

        Préambule : l'organisation peut bien entendu varier suivant les contextes, mais la description qui suit est représentative des organisation principales
        
        \todo[inline]{
        	a distinguer suivant l'approche statistiques et l'approche symbolique
        }

        \subsection{Définition des acteurs}
		\todo[inline]{
			Remarque Gautier 20/02/2023:
        	Vu le chaos du monde du travail concernant la définition du data scientist, et en quoi il est différent d'un data engineer, analyst, etc..., ce sera important que tu livres ta définition et ton point de vue sur ce qu'est un DS.
        	En fait on pourrait imaginer trouver des experts métiers et des chefs de projets qui connaissent l'IA. On peut même les y former (c'est une des approches qu'on suit souvent). Mais c'est juste pas pratique à faire.
			Je me demande, à la lecture de cette section, si le problème n'est pas plutôt un problème de division des compétences ici, plutôt que de acteurs. On divise les compétences (connaissance des algorithmes, des données, du métier, de l'organisation d'un projet), et c'est de cette division que naissent les différents acteurs d'un projet. Ca serait intéressant de trouver un exemple d'un chatbot conçu par une seule personne qui prend en charge touts les aspects.
		}
		\todo[inline]{
			reformuler cette section par "compétences nécessaires" et montrer qu'elles sont en générales réparties entre plusieurs acteurs
		}

            •	Data scientistes :
                o	Experts en IA
                o	Peu de connaissance métier, i.e. peu de regarde critique sur la pertinence des résultats (autre que statistique)
            
            •	Expert métier :
                o	Peu de connaissance en IA, i.e. nécessitent des formations
                o	Connaissance métier forte, i.e. peuvent décrire la pertinence d’un résultat
            
            •	Chef de projet
                o	Peu de connaissance en IA
                o	Peu de connaissance métier
                o	Connaissance du besoin (hypothèse non vérifiée car parfois ils ne savent pas ce qu’ils veulent dû à la méconnaissance des capacités de l’IA)

        \subsection{Cadrage du projet}

            •	Objectifs :
                o	Clarification du besoin,
                o	Définition du périmètre couvert (i.e. les fonctionnalités et réponses à proposer),

            •   Livrable : un cahier des charges
            
            \todo[inline]{
				Remarque Gautier 20/02/2023:
				La aussi, ça mérite presque une digression (et ton point de vue perso) sur les méthodes de travail et l'agilité en particulier. Le cahier des charges et la spécification ont l'avantage de contractualiser le travail à faire, et lorque le travail est très divisé c'est important. Mais dans la pratique, aujourd'hui tout le monde dit qu'il est Agile. hors, dans l'agilité, on n'est pas sensé avoir de contractualisation. Pourquoi en faire une ici ?
            }

        \subsection{Collecte des données}

            •	Souvent pas de données à disposition :
                o	En R\&D, "80\%" sur la recherche d’algo sur des données publiques, d'où le besoin de datascientists,
                o	En entreprise, "80\%" sur la gestion des données privées/spécifiques sur des algo connus, d'où le besoin d'experts métiers ;
    
            •	Risque de biais dans les données :
                o	Biais d’échantillon : la collecte ne représente pas la réalité,
                o	Biais de sélection : le trie de la collecte ne représente plus la réalité,
                o	Biais de confirmation : on garde les données qui nous arrangent,
                o	Biais de valeur : les données ne sont pas éthiquement représentatives,
                o	Biais de contexte : les données d’un cas d’usage ne sont pas toujours réutilisables pour un autre cas d’usage (ex : différence entre les jargons des AV clients et celui des AV conseillers) ;
                o	\textbf{A COMPLETER}
            
            •   Livrable : une collecte de données brutes

        \subsection{Modélisation d’une structure et Labellisation des données}
            
            \todo[inline]{
				Remarque Gautier 20/02/2023:
				Au dela de ce que tu écris (avec lequel je suis d'accord), on a aussi un problème plus large. En choisissant une approche symbolique (cf mon commentaire plus haut), ça implique que la création et l'utilisation des chatbots fait se rencontrer deux mondes symboliques : 
				- le monde symbolique des experts travaillant dans le métier (i.e. les banquiers)
				- le monde symbolique des utilisateurs (i.e les clients)
				Il serait intéressant de discuter les raisons pour lesquels ces mondes symboliques peuvent converger (objectifs identiques et partagés, caractère humain...) et diverger (compétences et connaissances très inégales). Ca permet d'avoir un regard critique sur l'organisation du travail, et justement de prôner l'idée que l'on doit retirer le plus possible les facteurs de divergence durant la symbolisation de la connaissance.
			}

            •	Le coeur "métier" de la création du projet ;

            •	Objectif : Définition d'une modélisation sur la base des besoins attendus restreints au périmètre à couvrir ;

            •	En théorie :
                o	Intention: verbe d'actions,
                o	Entités: informations complémentaires, personnes, date, lieux, montants, noms de produits, ... ;

            •   Complexité de la tâche :
                o	Intention abstraite : définition difficile voir subjective, ...
                o	Annotation difficile :  différence entre théorie et pratique, données ambiguës, ...
                o	Plusieurs itérations car modélisation trop théorique / pas pratique
                o	Besoins de beaucoup de formation (pour donner la compétence aux experts) et d'atelier (pour se mettre d'accord)

            •   Livrable : un jeu de données annotées

        \subsection{Entraînement et tests}

            •	Le coeur "technique" de la création du projet ;

            •	Objectif : avoir un modèle qui soit adapté à son utilisation en production

            •	En théorie :
                o	Split en train et tests
                o	Entraînement et tests
                o	Association des réponses

            •   Complexité de la tâche :
                o	Modélisation précédente pas toujours adaptée : OK pour un métier, mais pas possible à entraîner à cause de déséquilibre, de manque de données, ...
                o	Algorithme fixe mais données variables : savoir quelle modélisation est la plus adaptée est compliqué à deviner
                o	Réponses pas toujours adaptées aux questions : décalage entre entraînement (modélisation théorique) et réponse (modélisation pratique)

        \subsection{Déploiement de la première version}

            •	RAS

            •	Parfois la modélisation est décalée par rapport à l'utilisation en production
                o	Comportement en moteur de recherche avec des questions courtes
                o	Vocabulaire non maitrisé par les utilisateurs
                o	problème d'ergo ou d'expérience utilisateur
            
            \todo[inline]{
				Remarque Gautier 20/02/2023:
				oui,cf mon commentaire plus haut sur la rencontre des mondes symboliques. C'est pour moi un désavantage de cette approche, et ça explique peut etre en partie le succès des approches non supervisées style ChatGPT
			}

        \subsection{Amélioration continue}

            •	Vérification du comportement ;

            •	Ajustement du modèle ;

            •	Déploiement des versions suivantes.
            
            \todo[inline]{
				Remarque Gautier 20/02/2023:
				Quels sont les objectifs de l'AC ? C'est seulement d'améliorer le tux de bonnes réponse ? Ou c'est plu large que ça ? (corriger les erreurs d'interprétation, faire converger les conceptions symboliques, éduquer les équipes, etc...)
			}

    %%%%%--------------------------------------------------------------------
    %%%%% Section 2.2:
    %%%%%--------------------------------------------------------------------
    \section{Zoom sur la partie Modélisation et Labellisation de la base d'apprentissage}
		\todo[inline]{SECTION À RÉDIGER}

        \subsection{Création « manuelle »}

            •	Enchaînement de plusieurs ateliers/cycles :
                o	Définition d’une structure en atelier et Annotation des données
                o	Premier conflit : La structure est trop théorique
                o	Redéfinition et Ré-annotation
                o	Second conflit : Les structure ou les données ne sont pas adaptées
                o	Collecte complémentaire, Redéfinition et Ré-annotation

            •	Avantages :
                o	Transmission progressive du savoir aux data scientists
                o	Test des modélisations potentielles

            •	Inconvénients :
                o	Nombreux ateliers
                o	Nombreuses remises en questions / aller-retour de conception
                o	L'avis initiale sur le périmètre à couvrir est flou quand cela concerne une centaine de demandes clients
                o	Se base sur de la connaissance que les experts métiers n’ont pas
                o	Comment les aider dans ce problème d’organisation ?

        \subsection{Création assistée par des regroupements non-supervisés}

            •	Constat :
                o	Pour des jeux de données à taille humaine (moins de 20.000 données), le premier tri est parfois "optimisé" manuellement sur la base des patterns commun (ordonnancement alphabétique)

            •	Solution :
                o	Un clustering pourrait simplifier cette tâche !
                \todo{clustering, topic modeling, ...}
                o   Rappel : grandes lignes du fonctionnement d'un algorithme de \gls{clustering} ?
                o	NB : une section ou une annexe détaillera les algorithmes de \index{clustering} les plus utilisés					o	KMeans : Classique, Incontournable, Rapide, Efficace
					o	Hiérarchique : Lent mais facile à implémenter
					o	Spectral : Permet des topologies complexes
					o	DBScan : Classique, Incontournable, Rapide, Efficace, Peu d'hyperparamètre
					o	Affinity propagation : 
					o	Metric learning : Lent mais plus adapté au corpus
					o	…

            •	Avantages :
                o	Regroupement automatique
                o	Découverte de la structure

            •	Inconvénients :
                o	Les résultats sont souvent peu pertinents
                o	Similarité par entités, et pas par intentions
                o	Nuances métiers non comprises
                o	Plusieurs soucis si le jeu de données est déséquilibré ou spécifique
                o	Absence d’un modèle de langue spécifique au contexte...
                o   parfois besoin d'hyperparamètres complexes à déterminer

        \subsection{Conception assistée par des regroupements semi-supervisés}

            •	Solution :
                o	On peut envisager ainsi de corriger le clustering en y insérant des contraintes métiers\hspace{2em}
                \cite{lampert:2018}
                o	Méthodes semi-supervisée
                o	NB : une section ou une annexe détaillera les algorithmes de clustering sous contraintes
					o	KMeans : Classique, Incontournable, Rapide, Efficace
					o	Hiérarchique : Lent mais facile à implémenter
					o	Spectral : Permet des topologies complexes
					o	DBScan : Classique, Incontournable, Rapide, Efficace, Peu d'hyperparamètre
					o	Affinity propagation : 
					o	Metric learning : Lent mais plus adapté au corpus
					o	…

            •   Interactions possibles avec le clustering (sur la base de proposition de l'humain)
                o	Sur les données / sur le résultat : ajouts de contraintes sur les données, suppressions ou modifications manuelles de données, réorganisation manuelles des clusters, …
                o	Sur les paramètres : modifier les hyper-paramètres, modifier le nombre de clusters, modifier les embeddings, utiliser d’autres algorithmes, …
                o	Besoin de visualisation : vue des contraintes, de la représentation vectorielle, …

            •	Avantage :
                o	On a réglé les problèmes de pertinence en ajoutant des contraintes

            •	Inconvénients : 
                o	Choisir comment modéliser ces contraintes peut être complexe
                o	Surtout énorme en ajoutant des contraintes
                o	Choisir les contraintes pertinentes est une tâche difficile

        \subsection{Conception basée sur des méthodes d’apprentissage actif}

            •	Solution :
                o	On peut demander à la machine de définir les contraintes dont elle a besoin pour s’améliorer / confirmer son comportement
                o	On peut séparer et cibler les tâches pour que le clustering se nourrissent des commentaires de l’expert et que l’expert corrige ce qui semble utile au clustering
                o	Sous-entendu : Préférer la collaboration à la supériorité (que ce soit celle de la machine ou celle de l’expert)
                o	NB : une section ou une annexe détaillera les interactions possibles entre homme et machine

            •   Interactions possibles avec le clustering (sur la base de propositions de la machine)
                o	Sur les données / sur le résultat : proposition de suppression de données aberrantes, proposition d'ajout de contraintes à des endroits stratégiques, …
                o	Sur les paramètres : réévaluation des paramètres, combiner plusieurs algorithmes et synthétiser le résultat, …

            •	Avantage :
                o	On a réglé les problèmes de pertinence et de coûts en ajoutant des contraintes

            •	Inconvénients / problème à résoudre : 
                o	Accepter de collaborer avec la machine (problème UX, ergo, accompagnement au changement)
                o	Il faut prouver cette méthode

%%%%%--------------------------------------------------------------------
%%%%% Chapitre: Clustering Interactif
%%%%%--------------------------------------------------------------------
\chapter{Proposition d'un Clustering Interactif}
    \label{chapter:3-CLUSTERING-INTERACTIF}
    
    % RÉSUMÉ DES ÉPISODES PRÉCÉDENTES: L'ANNOTATION C'EST COMPLIQUÉ !
    Dans le chapitre précédent, nous avons vu les points essentiels suivants :
	\begin{todolist}
	    % 1. Importance du jeu de données.
		\item[\itemok] Dans un cadre industriel, le choix de l'algorithme utilisé pour l'entrainement d'un modèle est déterminé à l'avance, donc la qualité de l'assistant repose principalement sur la fiabilité et la pertinence de son jeu de données ;
		% 2. Experts métiers avec connaissance métiers, pas de connaissance en datascience.
		\item[\itemok] Pour concevoir ce jeu de données, il est nécessaire de faire appel à des experts maîtrisant le domaine à couvrir par l'assistant car les données sont en général spécifiques ou privées ;
		% 3. Experts métiers ayant peu de donnaissance en data science.
		\item[\itemok] L'intervention de ces experts métiers au sein du projet est en général laborieuse :
		d'une part à cause de leur manque de connaissances en data science (ce n'est pas leur domaine d'expertise),
		d'autre part à cause de la complexité inhérente des tâches de modélisation et d'annotation des données.
		% 4. Tache manuelle avec peu d'assistance.
		\item[\itemok] Par manque de compétences, de connaissances ou d'ergonomie, la tâche de conception d'un jeu de données reste manuelle et est encore mal assistée par ordinateur.\todo{à reformuler plus tard.}
	\end{todolist}
	
	% ANNONCE DU BUT DU CHAPITRE: MA CONTRIBUTION !
	Dans cette partie, nous proposons une alternative à l'organisation manuelle destinée à la conception d'un jeu de données. Notre proposition vise à remplir un double objectif :
	\begin{todolist}
		% 1. Efficacité de création du trainset.
		\item Proposer une méthode permettant d'assister la modélisation et l'annotation des données pour créer plus efficacement une base d'apprentissage pour la classification d'intention d'un assistant conversationnel ;
		% 2. Efficacité d'intervention d'un expert.
		\item Redéfinir les tâches et les objectifs des différents acteurs afin de rester au plus proche de leurs compétences réelles, particulièrement en ce qui concerne les experts métiers intervenants dans le projet.
	\end{todolist}

	% TABLE DES MATIÈRES DU CHAPITRE
    \minitoc


    %%%%%--------------------------------------------------------------------
    %%%%% Section 3.1: Intuitions à l'origine de notre méthode.
    %%%%%--------------------------------------------------------------------
    \section{Intuitions à l'origine de notre méthode}
    \label{section:3.1-DESCRIPTION-INTUITIONS}

		% 1. L'annotation de contraintes est plus intuitif pour un expert métier.
		La pierre angulaire de notre méthode repose sur le fait qu'il est difficile pour un expert métier de classer une question suivant une modélisation abstraite prédéfinie :
		cela l'éloigne de ses compétences initiales, nécessite en contre-partie de nombreuses formations, et introduit de nombreuses erreurs d'annotations.
		\todo{citation}
        \todo[inline]{
			Remarque Gautier 20/02/2023:
			erreur de routine, erreur par manque de connaissance, ...
			Il faudra discuter les causes de ces erreurs
		}
		De fait, il semble plus adéquat de demander à l'expert métier de discriminer deux questions sur la base de leurs réponses :
		une telle approche demande une charge de travail plus faible et est plus intuitive car elle est plus proche des compétences réelles de l'annotateur.
		\todo{citation}
		Ainsi, nous basons notre méthode sur l'annotation de contraintes sur les données.
		
		% 2. Un expert métier seul ne peut trouver une modélisation adéquate, il faut se reposer sur l'interaction homme-machine.
		Toutefois, l'annotation de contraintes semble elle aussi fastidieuse.
		En effet, pour faire émerger une base d'apprentissage, il faut annoter un grand nombre de contraintes et être attentifs aux éventuelles incohérences pour ne pas introduire de contraintes contradictoires.
		\todo{citation}
		Pour assister l'expert dans cette tâche, nous avons donc décidé de l'intégrer dans une stratégie d'apprentissage actif en essayant de tirer parti des interactions possibles avec la machine. Ce choix est motivé entre autre par l'intuition qu'il est possible de coopérer avec la machine pour obtenir plus efficacement un résultat pertinent.\todo{à reformuler plus tard.}

		% TR:
		C'est sur la combinaison de ces deux éléments que repose notre méthode d'annotation pour concevoir le jeu d’entraînement de notre assistant conversationnel.


    %%%%%--------------------------------------------------------------------
    %%%%% Section 3.2: Description théorique de notre clustering interactif.
    %%%%%--------------------------------------------------------------------
    \section{Description théorique de notre clustering interactif}
	\label{section:3.2-DESCRIPTION-THEORIQUE}
    
	    	% Rappel de l'objectif.
		Nous proposons la méthode suivante pour transformer une collecte de données brut en une base d'apprentissage nécessaire à l’entraînement d'un assistant conversationnel. Cette méthode, que nous appelons "\textit{clustering interactif}", est décrite formellement à l'aide du pseudo-code figurant dans Alg.~\ref{algorithm:3.2-CLUSTERING-INTERACTIF}.

		\begin{algorithm}
		    \begin{algorithmic}[1]
		        \Require données non segmentées ; budget à disposition
				\State \textbf{initialisation}: créer une liste vide de contraintes
		        \State \textit{optionnel}: évaluer les hyper-paramètres de la segmentation automatique
		        \State \textbf{segmentation initial}: regrouper les données par similarité
		        \Repeat
		            \State \textit{optionnel}: évaluer les hyper-paramètres de l'échantillonnage
		            \State \textbf{échantillonnage}: sélectionner une partie de la segmentation à corriger
		            \State \textbf{annotation}: corriger la segmentation en ajoutant des contraintes sur l'échantillon
		            \State \textit{optionnel}: ré-évaluer les hyper-paramètres de la segmentation automatique
			        \State \textbf{segmentation}: regrouper les données par similarité avec les contraintes
			        \State \textbf{évaluation (1)}: estimer la pertinence et la stabilité de la segmentation
			        \State \textbf{évaluation (2)}: estimer le budget restant et les coûts restant à investir
		        \Until{segmentation satisfaisante OU budget épuisé}
		        \Ensure données segmentées (i.e. base d'apprentissage)
		    \end{algorithmic}
		    \caption{Description en pseudo-code de la méthode d'annotation proposée employant le clustering interactif}
		    \label{algorithm:3.2-CLUSTERING-INTERACTIF}
		\end{algorithm}
		
		% Présentation succinte.
		La méthode repose principalement sur l'alternance successive entre deux phases clefs :
		\begin{itemize}
			\item[\(\bullet\)] une phase d'\textbf{annotation de contraintes}
			par un expert sur la base des connaissances qu'il détient ;
			\item[\(\bullet\)] une phase de \textbf{segmentation automatique} des données
			\todo{utiliser l'appellation clustering ou segmentation ?}
			par une machine sur la base de la proximité sémantique des données et des contraintes précédemment annotées.
		\end{itemize}
		
		% Objectif recherché
		L'objectif recherché en associant ces deux phases est la création d'un cercle vertueux pour améliorer itérativement la qualité de la base d'apprentissage en cours de construction.
		En effet, à chaque itération, l'expert métier obtiendra une proposition de segmentation des données qu'il pourra raffiner pour corriger le fonctionnement de la machine et ainsi obtenir une segmentation plus pertinente à l'itération suivante.
		
		% Description de l'initialisation.
		Pour l'\textbf{initialisation} de la méthode (cf. Alg.~\ref{algorithm:3.2-CLUSTERING-INTERACTIF}, \textit{lignes 1 à 3}), nous définissons une liste vide de contraintes : tout au long du processus, cette liste contiendra l'ensemble de la connaissance que l'expert transmettra au système sous la forme de contraintes simple sur les données (nous entrerons en détails en décrivant la phase d'annotation).
		De plus, il faut une première segmentation des données par la machine : celle-ci se réalise par l'exécution d'un algorithme de clustering. Nous estimons qu'il n'est pas du ressort de l'expert métier de choisir de l'algorithme de clustering et ses hyper-paramètres. Ces derniers pourront être déterminés par un data scientist en fonction du problème à traiter ou laissés par défaut\todo{cf. partie étude}.
		Il est à noter que cette segmentation des données est réalisée sans bénéficier de la connaissance de l'expert, il est donc peu probable que le résultat soit pertinent à ce stade.
		
		% Description de l'échantillonnage.
		Nous entrons dans le coeur de la boucle itérative par la phase d'\textbf{échantillonnage} (cf. Alg.~\ref{algorithm:3.2-CLUSTERING-INTERACTIF}, \textit{lignes 5 et 6}).
		Comme mentionné au préalable, savoir quelles contraintes ajouter pour corriger efficacement le clustering est un problème NP-difficile (le nombre de possibilité croît proportionnellement au carré du nombre de données). De plus, l'intervention d'expert est chiffrée et représente en général la majeure partie des coûts à investir dans un projet\todo{citation}.
		Il est donc inconcevable de laisser un expert métier annoter des contraintes "seul" et "au hasard".
		Ainsi, pour optimiser ses interventions, il convient de déterminer là où l'expert aura le plus d'impact lors de sa transmission de connaissance. C'est pourquoi la phase d'échantillonnage est primordiale dans la méthode proposée : Nous proposons d'y sélectionner des couples de données sur la base de leur similarité, de leur segmentation ou encore de leur relations avec d'autres données déjà liées par d'autres contraintes.
		\todo{description technique plus tard ? ref subsection:3.3.4}
		
		% Description de l'annotation.
		Sur la base de cet échantillon, l'expert peut entamer son étape d'\textbf{annotation de contraintes} (cf. Alg.~\ref{algorithm:3.2-CLUSTERING-INTERACTIF}, \textit{ligne 7}).
		Pour alléger la charge d'annotation, nous avons décidé de discriminer les données de l'échantillon par des contraintes binaires simples : \texttt{MUST-LINK} et \texttt{CANNOT-LINK}. Ces contraintes représentent respectivement la similitude ou la différence entre deux données, et seront utilisées pour regrouper ou séparer certaines données dans la prochain segmentation.
		En fonction de l'orientation du projet et afin de rester au plus proche des compétences réelles de l'expert, la formulation de l'énoncer d'annotation doit être judicieusement définie : par exemple, les contraintes peuvent représenter une similitude
		sur la thématique concernée\footnote{thématique : \textit{crédit} vs. \textit{assurance} ; \textit{sport} vs. \textit{culture}, ...},
		sur l'action désirée\footnote{action : \textit{souscrire} vs. \textit{résilier} ; \textit{activer} vs. \textit{désactiver} ; \textit{s'informer} vs. \textit{réaliser}, ...},
		ou encore sur le besoin de l'utilisateur\footnote{besoin : \textit{souscrire un crédit} vs. \textit{souscrire une assurance} ; \textit{s'informer en sport} vs. \textit{s'informer en culture}, ...}.
		On notera que des incohérences peuvent s'introduire, ayant pour conclusions de devoir à la fois considérer comme similaire et différentes deux données\todo{figure, ref subsection:3.3.2.}.
		
		% Description du clustering.
		Pour finir, la dernière phase de cette boucle est composée d'une nouvelle \textbf{segmentation} des données (cf. Alg.~\ref{algorithm:3.2-CLUSTERING-INTERACTIF}, \textit{lignes 8 et 9}). Cette devra respecter les contraintes préalablement définies par l'expert, nous nous tournons donc vers l'utilisation d'un clustering sous contraintes.
		Au fur et à mesure des itérations, de plus en plus de contraintes seront ajoutées pour corriger le clustering. ainsi, au bout d'un certain nombre d'itérations, la segmentation des données reflétera la vision que l'expert aura voulu transmettre.
		Comme précédemment, nous estimons qu'il n'est pas du ressort de l'expert métier de choisir de l'algorithme de clustering et ses hyper-paramètres. Ces derniers pourront être déterminés par un data scientist en fonction du problème à traiter, estimés en fonction de l'itération et des contraintes disponibles, ou laissés par défaut.\todo{cf. partie étude}
		\todo{description technique plus tard ? ref subsection:3.3.3}
		
		% Description de l'évaluation.
		Comme la méthode est itérative, il faut pouvoir estimer des \textbf{cas d'arrêt} (cf. Alg.~\ref{algorithm:3.2-CLUSTERING-INTERACTIF}, \textit{lignes 10 à 12}).
		Le cas d'arrêt le plus évident n'est pas technique mais relatif aux coûts investis dans l'opération : si le projet n'a plus de budget dédié à l'annotation, il faudra créer la base d'apprentissage avec le résultat à disposition, quel que soit la pertinence de la segmentation obtenue sur les données. Ce cas d'arrêt par défaut peut malheureusement être synonyme d'échec pour le projet si les résultats sont inexploitables.
		D'autres cas d'arrêts plus techniques peuvent être envisagés en fonction de la qualité de la segmentation.
		D'une part, nous pouvons comparer l'évolution de la segmentation des données : si les segmentations sont similaires sur plusieurs itérations, il est possible que la modélisation atteint un optimum local ou un palier de performance. D'autre part, nous pouvons aussi comparer l'évolution de l'accord entre la segmentation obtenue et l'annotation de l'expert : en effet, si l'expert ne contredit plus la répartition proposée des données, il est probable que sa vision et la vision de la machine aient convergé.
		Dans les deux cas, l'analyse de l'expert métier reste nécessaire pour valider si la modélisation des données est pertinente ou si elle comporte encore des incohérences à corriger.
		\todo{description technique plus tard ? ref subsection:3.3.X}

		
    %%%%%--------------------------------------------------------------------
    %%%%% Section 3.3: Description technique et implémentation.
    %%%%%--------------------------------------------------------------------
    \section{Description technique et implémentation}
	\label{section:3.3-DESCRIPTION-IMPLEMENTATION}
		
		% Généralités.
		Nous avons réalisé une implémentation en Python de notre \textit{clustering interactif}. Celle-ci est répartie en trois librairies :
		\begin{enumerate}
			\item \texttt{cognitivefactory-interactive-clustering}, regroupant les gestions de données, de contraintes et les algorithmes de \textit{Machine Learning} qui ont été implémentés ;
			\item \texttt{cognitivefactory-features-maximization-metrics}, disposant d'une méthode de comparaison entre deux modélisations thématiques d'un même jeu de données ;
			\item \texttt{cognitivefactory-interactive-clustering-gui}, encapsulant les algorithmes précédents et intégrant la logique de la méthode dans une interface graphique.
		\end{enumerate}
		
		% Exemple.
		Pour les sections suivantes, nous suivrons l'exemple suivant (cf. Code~\ref{code:3.3-IMPLEMENTATION-DATA}) pour présenter nos implémentations.
		
		\begin{lstlisting}[
			language=Python,
			caption={Jeu exemple pour présenter notre implémentation du clustering interactif.},
			label={code:3.3-IMPLEMENTATION-DATA},
		]
# Définir les données.
dict_of_texts = {
    "0": "Comment signaler un vol de carte bancaire ?",
    "1": "J'ai égaré ma carte bancaire, que faire ?",
    "2": "J'ai perdu ma carte de paiement",
    "3": "Le distributeur a avalé ma carte !",
    "4": "En retirant de l'argent, le GAB a gardé ma carte...",
    "5": "Le distributeur ne m'a pas rendu ma carte bleue.",
    # ...
    "N": "Pourquoi le sans contact ne fonctionne pas ?",
}
		\end{lstlisting}
		
		
		%%% Subsection 3.3.1: Gestion des données.
		\subsection{Gestion des données}
				
		% cognitivefactory.interactive-clustering.utils
		Tout d'abord, en ce qui concerne la \textbf{manipulation de données}, nous utilisons le module \texttt{utils} de la librairie \texttt{cognitivefactory-interactive-clustering}. Les données sont stockées dans un dictionnaire Python afin de tracer les manipulations à l'aide d'une clé servant d'identifiant de la donnée.
		
		% cognitivefactory.interactive-clustering.utils.preprocessing : Implémentation.
		Nous avons d'une part la partie \texttt{utils.preprocessing}\footnote{\url{https://cognitivefactory.github.io/interactive-clustering/reference/cognitivefactory/interactive_clustering/utils/preprocessing/}} qui permet de normaliser les données.
		Par défaut :

		\begin{itemize}
			\item[\(\bullet\)] le texte est passé en \textit{minuscule} (de "\texttt{Bonjour}" à "\texttt{bonjour}"),
			\item[\(\bullet\)] la \textit{ponctuation} est supprimée, %(\texttt{.}, \texttt{,}, \texttt{;}, \texttt{:}, \texttt{!}, \texttt{¡}, \texttt{?}, \texttt{¿}, \texttt{…}, \texttt{•}, \texttt{(}, \texttt{)}, \texttt{\{}, \texttt{\}}, \texttt{[}, \texttt{]}, \texttt{«}, \texttt{»}, \texttt{^}, \texttt{\`}, \texttt{'}, \texttt{"}, \texttt{\\}, \texttt{/}, \texttt{|}, \texttt{-}, \texttt{\_}, \texttt{#}, \texttt{\&}, \texttt{\~}, \texttt{\@}),
			\item[\(\bullet\)] les \textit{accents} sont enlevés (de "\texttt{crédit}" à "\texttt{credit}"),
			\item[\(\bullet\)] et les multiples \textit{espaces blancs} sont convertis en un unique espace simple (de "\texttt{au~~~~revoir}" à "\texttt{au revoir}").
		\end{itemize}
		
		Si besoin, trois options "avancées" sont disponibles pour réaliser un prétraitement plus destructif :

		\begin{itemize}
			\item[\(\bullet\)] la suppression des mots vides (\textit{stopwords}\todo{citation}),
			\item[\(\bullet\)] la conversion des mots vers leur forme racine (\textit{lemmatisation}\todo{citation})
			\item[\(\bullet\)] et la suppression des mots en fonction de leur profondeur dans l'arbre de dépendance syntaxique(\textit{dependency parsing})\todo{citation}.
		\end{itemize}
		
		% cognitivefactory.interactive-clustering.utils.preprocessing : Dépendances.
		Ces traitements sont réalisés en bénéficiant des fonctionnalités mises à disposition d'un modèle de langue de type SpaCy\todo{citation}, avec par défaut l'utilisation du modèle \texttt{fr-core-news-md}\todo{citation}.
		
		% cognitivefactory.interactive-clustering.utils.preprocessing : Par défaut.
		Pour nos études, nous définissons quatre niveaux de prétraitements facilement identifiables :
		\begin{enumerate}
			\item L'\textbf{absence de prétraitement}, soit la conservation de la donnée brute, noté \texttt{prep.no} ;
			\item Le \textbf{prétraitement simple}, correspondant au traitement traitement de base (minuscules, ponctuations, accents, espaces blancs), noté \texttt{prep.simple} ; 
			\item Le \textbf{prétraitement lemmatisé}, correspondant au traitement de base auquel s'ajoute la lemmatisation des mots, noté \texttt{prep.lemma} ;
			\item le \textbf{prétraitement avec filtre}, correspondant au traitement de base avec l'élagage de l'arbre de dépendance syntaxique de la phrase, noté \texttt{prep.filter}.
		\end{enumerate}
		
		
		% cognitivefactory.interactive-clustering.utils.vectorization
		D'autre part, la partie \texttt{utils.vectorization}\footnote{\url{https://cognitivefactory.github.io/interactive-clustering/reference/cognitivefactory/interactive_clustering/utils/vectorization/}} permet de transformer les données en une représentation exploitable pour la machine.
		Deux modes de vectorisation sont mis à disposition :
		
		\begin{enumerate}
			\item \textbf{TF-IDF}\todo{citation}, utilisant la fréquence d'occurrence des mots pour représenter une phrase, et noté \texttt{vect.tfidf} pour nos études ;
			\item \textbf{SpaCy}\todo{citation}, utilisant le modèle de langue \texttt{fr-core-news-md}, et noté \texttt{vect.frcorenewsmd}.
		\end{enumerate}
		
		% cognitivefactory.interactive-clustering.utils : Exemple.
		Vous avez un exemple d'utilisation des modules de prétraitements et de vectorisation dans Code~\ref{code:3.3-IMPLEMENTATION-GESTION-DONNEES}.
		
		\begin{lstlisting}[
			language=Python,
			caption={Démonstration de notre implémentation du prétraitement et de la vectorisation sur le jeu d'exemple.},
			label={code:3.3-IMPLEMENTATION-GESTION-DONNEES},
		]
# Import des dépendances.
from cognitivefactory.interactive_clustering.utils.preprocessing import preprocess
from cognitivefactory.interactive_clustering.utils.vectorization import vectorize

# Prétraitement des données.
dict_of_preprocess_texts = preprocess(
    dict_of_texts=dict_of_texts,
    apply_stopwords_deletion=False,
    apply_parsing_filter=False,
    apply_lemmatization=False,
    spacy_language_model="fr_core_news_md",
)
"""
    {"0": "comment signaler un vol de carte bancaire",
     "1": "j ai egare ma carte bancaire, que faire",
     "2": "j ai perdu ma carte de paiement",
     "3": "le distributeur a avale ma carte",
     "4": "en retirant de l argent le gab a garde ma carte",
     "5": "le distributeur ne m a pas rendu ma carte bleue",
     # ...
     "N": "pourquoi le sans contact ne fonctionne pas"}
"""

# Vectorisation des données.
dict_of_vectors = vectorize(
    dict_of_texts=dict_of_preprocess_texts,
    vectorizer_type="tfidf",
)
		\end{lstlisting}
		
		%%%	Subsection 3.3.2: Gestion des contraintes
		\subsection{Gestion des contraintes}
		
		% cognitivefactory.interactive-clustering.constraints
		En ce qui concerne la \textbf{manipulation de contraintes}, nous utilisons le module \texttt{contraints}\footnote{\url{https://cognitivefactory.github.io/interactive-clustering/reference/cognitivefactory/interactive_clustering/constraints/}} de la librairie \texttt{cognitivefactory-interactive-clustering}.
		
		% cognitivefactory.interactive-clustering.constraints: Types
		Deux types de contraintes sont prises en charge :
		
		\begin{itemize}
			\item[\(\bullet\)] les contraintes \texttt{MUST-LINK} permettant de réunir deux données,
			\item[\(\bullet\)] et les contraintes \texttt{CANNOT-LINK} permettant à l'inverse de les séparer.
		\end{itemize}

		% cognitivefactory.interactive-clustering.constraints: Transitivité.
		Ces types de contraintes respectent les propriétés de transitivités suivantes (cf. Fig.~\ref{figure:3.3-CONTRAINTES-TRANSITIVITE}):
		
		\begin{equation}
			(\forall D_1,D_2,D_3) \\
			\texttt{MUSTLINK}(D_1,D_2) \wedge \texttt{MUSTLINK}(D_2,D_3)
			\Rightarrow \texttt{MUSTLINK}(D_1,D_3)
		\end{equation}
		
		\begin{equation}
			(\forall D_1,D_2,D_3) \\
			\texttt{MUSTLINK}(D_1,D_2) \wedge \texttt{CANNOTLINK}(D_2,D_3)
			\Rightarrow \texttt{CANNOTLINK}(D_1,D_3)
		\end{equation}
		
		Pour respecter ces propriétés, le gestionnaire de contraintes doit calculer les transitivités à chaque ajout ou suppression de contraintes. On distinguera donc une contrainte ajoutée (\texttt{added}) d'une contrainte déduite par transitivité (\texttt{inferred}).
		
		% cognitivefactory.interactive-clustering.constraints: Conflits.
		Il se peut que la contrainte en cours d'ajout contredise les contraintes déduites : nous parlons alors d'incohérence ou de conflit (cf. Fig.~\ref{figure:3.3-CONTRAINTES-TRANSITIVITE}). Dans ce cas, l'ajout de la dernière contrainte n'est pas prise en compte et le gestionnaire renvoie une erreur permettant d'identifier ce conflit.
		Ce conflit peut venir simplement venir d'une erreur d’inattention, mais peut aussi venir d'une déduction basée sur des ajouts antérieurs erronés .
		
		\begin{equation}
			\begin{cases}
				& \textcolor{red}{\texttt{CANNOTLINK}(D_0,D_n)} \\
				(\exists \{D_i | i\in[0,n]\})
				& \bigwedge_{i\in[0,n-1]} \texttt{MUSTLINK}(D_i,D_{i+1})
			\end{cases}
		\end{equation}
		\todo{figure}
		
		\begin{equation}
			\begin{cases}
				& \textcolor{red}{\texttt{MUSTLINK}(D_0,D'_0)} \\
				(\exists \{D_i | i\in[0,n]\})
				& \bigwedge_{i\in[0,n-1]} \texttt{MUSTLINK}(D_i,D_{i+1})\\
				(\exists \{D'_j | j\in[0,m]\})
				& \bigwedge_{j\in[0,m-1]} \texttt{MUSTLINK}(D'_j,D'_{j+1})\\
				& \texttt{CANNOTLINK}(D_n,D'_m)
			\end{cases}
		\end{equation}
		\todo{figure}
		
		% cognitivefactory.interactive-clustering.constraints: Composants connexe.
		A partir d'une donnée \(D\), et par application de la propriété de transitivité des \texttt{MUST-LINK}, nous appelons \textbf{composant connexe} de \(D\) l'ensemble des données \(D_i\) liées par une succession de contraintes \texttt{MUST-LINK} à \(D\) (cf. Fig.~\ref{figure:3.3-CONTRAINTES-TRANSITIVITE}).
		Ce composant peut être vu comme un noyau de \textit{cluster}. Il pourra être associé à d'autres noyau par similarité pour former un plus \textit{cluster} plus conséquent, ou être distingué d'autres noyaux pour former plusieurs \textit{clusters}.
		\todo{figure}

		\begin{figure}[H]
			\centering
			\includegraphics[width=0.7\textwidth]{figures/example-constraints-transitivity}
			\caption{Exemples des propriétés de transitivité des contraintes \texttt{MUST-LINK} (flèches vertes) et \texttt{CANNOT-LINK} (flèches rouges). \textbf{(1)} et \textbf{(2)} représente les possibilités de déduction d'une contrainte (\texttt{(c)}) en fonction des deux autres (\texttt{(a)} et \texttt{(b)}). \textbf{(3)} représente deux composants connexes définis par la transitivité des contraintes \texttt{MUST-LINK}. Enfin, \textbf{(4)} représente un cas de conflit où une contrainte (\texttt{(c)}) ne correspond pas à sa déduction faite à partir des autres contraintes (\texttt{(a)} et \texttt{(b)}).}
			\label{figure:3.3-CONTRAINTES-TRANSITIVITE}
		\end{figure}
		
		% cognitivefactory.interactive-clustering.constraints : Exemple.
		Un exemple d'utilisation du module de gestion de contraintes est consultable dans Code~\ref{code:3.3-IMPLEMENTATION-GESTION-CONTRAINTES}.
		
		\begin{lstlisting}[
			language=Python,
			caption={Démonstration de notre implémentation de gestion des contraintes sur le jeu d'exemple.},
			label={code:3.3-IMPLEMENTATION-GESTION-CONTRAINTES},
		]
# Import des dépendances.
from cognitivefactory.interactive_clustering.constraints.factory import managing_factory

# Création du gestionnaire de contraintes.
constraints_manager = managing_factory(
    manager="binary",
    list_of_data_IDs = list(dict_of_texts.keys()),  # ["0", "1", "2", "3", "4", "5", ..., "N"]
)

# Ajout de contraintes.
constraints_manager.add_constraint(
    data_ID1="0",  # "Comment signaler un vol de carte bancaire ?"
    data_ID2="1",  # "J'ai égaré ma carte bancaire, que faire ?"
    constraint_type="MUST_LINK",
)
constraints_manager.add_constraint(
    data_ID1="3",  # "Le distributeur a avalé ma carte !"
    data_ID2="4",  # "En retirant de l'argent, le GAB a gardé ma carte..."
    constraint_type="MUST_LINK",
)
constraints_manager.add_constraint(
    data_ID1="0",  # "Comment signaler un vol de carte bancaire ?"
    data_ID2="N",  # "Pourquoi le sans contact ne fonctionne pas ?"
    constraint_type="CANNOT_LINK",
)
	# NB: ajouter une contrainte "MUST_LINK" entre "1" et "N" lèverait une erreur.

# Récupération des composants connexes.
# Récupérartion des composants connexes.
connected_components = constraints_manager.get_connected_components()
"""
	[['0', '1'],
	 ['2'],
	 ['3', '4'],
	 ['5'],
	 ['N']]
"""
		\end{lstlisting}
		
		
		%%%	Subsection 3.3.3: Algorithme de clustering sous contraintes.
		\subsection{Algorithme de clustering sous contraintes}
		
		% cognitivefactory.interactive-clustering.clustering
		En ce qui concerne le \textbf{regroupement automatique} des données par similarité, nous utilisons le module \texttt{clustering}\footnote{\url{https://cognitivefactory.github.io/interactive-clustering/reference/cognitivefactory/interactive_clustering/clustering/}} de la librairie \texttt{cognitivefactory-interactive-clustering}.
		
		% cognitivefactory.interactive-clustering.utils.clustering : Implémentation.
		Ce module met à disposition six algorithmes de \textit{clustering} sous contraintes (référez-vous à la section \todo{ref:section2:clustering} pour les détails de fonctionnement des algorithmes non contraints) :
		
		\begin{enumerate}
			\item \textbf{KMeans}, dans sa version \textit{COP}\todo{citation} et sa version \textit{MPC}\todo{citation}, notés \texttt{clust.kmeans.cop} et \texttt{clust.kmeans.mpc} pour nos études ;
			\item \textbf{DBscan}, dans sa version \textit{C-DBScan}\todo{citation}, noté \texttt{clust.cdbscan} ;
			\item \textbf{Hiérarchique}, dans sa version xxx\todo{citation}, avec quatre métriques de distances : \textit{single} (noté \texttt{clust.hier.sing}), \textit{complete} (noté \texttt{clust.hier.comp}), \textit{average} (noté \texttt{clust.hier.avg}) et \textit{ward} (noté \texttt{clust.hier.ward}) ;
			\item \textbf{Spectral}, dans sa version \textit{SPEC}\todo{citation}, noté \texttt{clust.spec} ;
			\item \textbf{Propagation par affinité}, dans sa version xxx\todo{citation}, noté \texttt{clust.affprop}.
		\end{enumerate}
		
		Une classe abstraite définit les prérequis des algorithmes implémentés (avoir une méthode \texttt{cluster}) et une \textit{factory} est disponible pour instancier rapidement un objet de \textit{clustering}.
		Pour plus de détails, les descriptions en pseudo-code de ces algorithmes sont disponibles en annexe\todo{ref}.
		% cognitivefactory.interactive-clustering.clustering : Exemple.
		Enfin, un exemple d'utilisation ce module est consultable dans Code~\ref{code:3.3-IMPLEMENTATION-CLUSTERING}.
		
		
		\begin{lstlisting}[
			language=Python,
			caption={Démonstration de notre implémentation du clustering sous contraintes sur le jeu d'exemple.},
			label={code:3.3-IMPLEMENTATION-CLUSTERING},
		]
# Import des dépendances.
from cognitivefactory.interactive_clustering.clustering.factory import clustering_factory

# Initialiser un objet de clustering.
clustering_model = clustering_factory(
    algorithm="kmeans",
	model="COP",
    random_seed=42,
)

# Lancer le clustering.
clustering_result = clustering_model.cluster(
    constraints_manager=constraints_manager,  # contient les contraintes
    nb_clusters=2,
    vectors=dict_of_vectors,
)
"""
    {"0": 0,  # "Comment signaler un vol de carte bancaire ?"
     "1": 0,  # "J'ai égaré ma carte bancaire, que faire ?"
     "2": 0,  # "J'ai perdu ma carte de paiement"
     "3": 1,  # "Le distributeur a avalé ma carte !"
     "4": 1,  # "En retirant de l'argent, le GAB a gardé ma carte..."
     "5": 1,  # "Le distributeur ne m'a pas rendu ma carte bleue."
     # ...
     "N": 1}  # "Pourquoi le sans contact ne fonctionne pas ?"
"""
		\end{lstlisting}
		
		% cognitivefactory.interactive-clustering.utils.clustering : Historique.
		Les algorithmes KMeans (COP), hiérarchique et spectral (SPEC) ont été implémentés au début de ce doctorat.
		Dans le cadre d'un projet industriel au sein de l'école Télécom Physique Strasbourg, les implémentations des algorithmes KMeans (MPC), C-DBScan et propagation par affinité ont été ajoutées\todo{citation + footnote}. Les élèves ont conclu ce projet d'extension en suggérant de se concentrer sur l'étude du C-DBScan car les deux autres algorithmes étaient soit trop instables, soit trop gourmand en temps de calcul\todo{travaux TPS en annexe ?}.
		
		
		%%%	Subsection 3.3.4: Algorithme d'échantillonnage de contraintes.
		\subsection{Algorithme d'échantillonnage de contraintes}
		
		% cognitivefactory.interactive-clustering.sampling
		En ce qui concerne l'\textbf{échantillonnage} de contraintes à annoter, nous utilisons le module \texttt{sampling}\footnote{\url{https://cognitivefactory.github.io/interactive-clustering/reference/cognitivefactory/interactive_clustering/sampling/}} de la librairie \texttt{cognitivefactory-interactive-clustering}.
		
		% cognitivefactory.interactive-clustering.utils.sampling : Implémentation.
		Cet échantillonnage correspond à la sélection de couple de données.
		Par défaut, l'échantillonnage est purement aléatoire.
		Cependant, plusieurs options sont disponibles :
		
		\begin{itemize}
			\item[\(\bullet\)] une restriction sur la \textit{distance} pouvant imposer aux données d'être les plus proches ou les plus éloignées du corpus ;
			\item[\(\bullet\)] une restriction sur le \textit{résultat du clustering} pouvant imposer aux données d'être issues d'un même cluster ou de clusters différents,
			\item[\(\bullet\)] une restriction pour exclure les contraintes \textit{déjà annotées},
			\item[\(\bullet\)] et enfin une restriction pour exclure les contraintes \textit{déjà déduites} par transitivité.
		\end{itemize}
		
		% cognitivefactory.interactive-clustering.utils.sampling : Par défaut.
		Sur cette base, nous définissons quatre niveaux d'échantillonnage facilement identifiables pour nos études :
		
		\begin{enumerate}
			\item Un échantillonnage \textbf{purement aléatoire} en excluant toutes les contraintes déjà annotées ou déduites, noté \texttt{samp.random.full} ;
			\item Un échantillonnage \textbf{pseudo-aléatoire} de données issues d'un \textbf{même cluster}, en excluant toutes les contraintes déjà annotées ou déduites, noté \texttt{samp.random.same} ;
			\item Un échantillonnage des données issues d'un \textbf{même cluster} et étant \textbf{les plus éloignées} les unes des autres, noté \texttt{samp.farhtest.same} ;
			\item Un échantillonnage des données issues de \textbf{clusters différents} et étant \textbf{les plus proches} les unes des autres, noté \texttt{samp.closest.diff} ;
		\end{enumerate}
		
		\todo[inline]{figure}

		Une classe abstraite définit les prérequis des algorithmes implémentés (avoir une méthode \texttt{sample}) et une \textit{factory} est disponible pour instancier rapidement un objet d'échantillonnage.
		% cognitivefactory.interactive-clustering.sampling : Exemple.
		Un exemple d'utilisation ce module est consultable dans Code~\ref{code:3.3-IMPLEMENTATION-SAMPLING}.
		
		\begin{lstlisting}[
			language=Python,
			caption={Démonstration de notre implémentation de l'échantillonnage sur le jeu d'exemple.},
			label={code:3.3-IMPLEMENTATION-SAMPLING},
		]
# Import des dépendances.
from cognitivefactory.interactive_clustering.sampling.factory import sampling_factory

# Initialiser un objet d'échantillonnage.
sampler = sampling_factory(
	algorithm="random",
    random_seed=42,
)

# Run sampling.
selection = sampler.sample(
	constraints_manager=constraints_manager,
	nb_to_select=2,
	clustering_result=clustering_result,  # optionnel pour "random"
	vectors=dict_of_vectors,  # optionnel pour "random"
)
"""
	[("0", '5"),  # "Comment signaler un vol de carte bancaire ?" vs "Le distributeur ne m'a pas rendu ma carte bleue."
	 ("0", '2"),  # "Comment signaler un vol de carte bancaire ?" vs "J'ai perdu ma carte de paiement"
	 ("2", 'N")]  # "J'ai perdu ma carte de paiement" vs "Pourquoi le sans contact ne fonctionne pas ?"
"""
		\end{lstlisting}
		

		%%%	Subsection 3.3.X:
		\subsection{todo}
		\todo[inline]{SECTION À RÉDIGER: FMC}
		\todo[inline]{SECTION À RÉDIGER: IC-GUI page d'annotation}
		\todo[inline]{SECTION À RÉDIGER: IC-GUI gestion d'état de l'application}
		\todo[inline]{SECTION À RÉDIGER: IC-GUI page d'analyse (en cours)}

    
    %%%%%--------------------------------------------------------------------
    %%%%% Section 3.4:
    %%%%%--------------------------------------------------------------------
    \section{Espoirs de la méthode proposée}
		\todo[inline]{SECTION À RÉDIGER}

        •	Moins de formations, d’ateliers, …
        •	Se concentrer sur son domaine de compétence (i.e. pas de datascience pour les experts métiers)
        •	Permettre de trouver la base d’apprentissage
        •	Méthode réaliste / pas trop coûteuse
        •	…
 
    
    %%%%%--------------------------------------------------------------------
    %%%%% Section 3.5:
    %%%%%--------------------------------------------------------------------
    \section{Protocole d'utilisation : Mode d'emploi associé (??CONCLUSION??)}
		\todo[inline]{SECTION À RÉDIGER}

        •	Collecte des données
        •	Itération de clustering > échantillonnage > annotation
        •	A chaque conflit : correction nécessaire
        •	A la fin d’un clustering : caractériser la pertinence métier avec FMC
        •	A chaque itération : voir l’évolution par rapport à la précédente
        NB : la démonstration de cette proposition de protocole sera démontrée dans la partie 3.

%%%%%--------------------------------------------------------------------
%%%%% Chapitre: Etude de la méthode
%%%%%--------------------------------------------------------------------
\chapter{Étude de la méthode}
    \label{chapter:4-ETUDES}
	
	% RÉSUMÉ DES ÉPISODES PRÉCÉDENTES:
	Dans le chapitre précédent, nous avons présenté une méthode de création d'un jeu de données d'entrainement pour un assistant conversationnel, que nous appelons "\textit{clustering interactif}" :
	\begin{todolist}
	    % 1. Structure de la méthode.
		\item[\itemok] La méthode proposée repose sur la combinaison entre un regroupement automatique des données par la machine et l'annotation de contraintes binaires par un expert métier pour corriger le regroupement proposé ;
		% 2. Enjeu 1 de la méthode : moins de technique.
		\item[\itemok] Une telle approche devrait limiter les pré-requis techniques actuellement exigés à un expert métier en les déléguant à la machine.
		% 2. Enjeu 2 de la méthode : plus de connaissance métier.
		\item[\itemok] En échange, l'expert se concentre d'avantage sur la transmission de ses connaissances avec une annotation caractérisant la similitude métier entre deux données.
		% 3. Divers.
		\item[\itemok] ...\todo{divers à compléter (technique ? méthode ? ...).}
	\end{todolist}
	
	% ANNONCE DU BUT DU CHAPITRE: TEST DE LA MÉTHODE.
	Comme nous l'avons détaillé dans le chapitre~\ref{chapter:2-ETAT-DE-L-ART}, des procédés d'annotation similaires existent pour des données facilement visualisables, comme dans le cadre du traitement d'images. Cependant, l'application d'une telle approche dans le cadre de la classification de données textuelles est peu détaillée dans la littérature. Ainsi, dans cette partie, nous étudierons la faisabilité d'un \textit{clustering interactif} pour des données textuelles en explorant les questions suivantes :
	\begin{todolist}
		% 1. Efficacité.
		\item Peut-on obtenir une base d'apprentissage à l'aide de notre proposition d'implémentation de la méthodologie d'\textit{clustering} interactif ? (cf. hypothèse d'\textbf{efficacité} en section~\ref{section:4.1-HYPOTHESE-EFFICACITE})
		% 2. Efficience.
		\item Peut-on déterminer un paramétrage optimal de cette implémentation pour obtenir plus rapidement une base d'apprentissage ? (cf. hypothèse d'\textbf{efficience} en section~\ref{section:4.2-HYPOTHESE-EFFICIENCE})
		% 3. Pertinence.
		\item A un instant donné, peut-on estimer la pertinence métier d'une base d'apprentissage en cours de construction ? (cf. hypothèse de \textbf{pertinence} en section~\ref{section:4.3-HYPOTHESE-PERTINENCE})
		% 4. Coûts.
		\item D'après les données initiales, peut-on approximer l'investissement nécessaire pour obtenir une base d'apprentissage exploitable ? (cf. hypothèse sur les \textbf{coûts} en section~\ref{section:4.4-HYPOTHESE-COUTS})
		% 5. Impact.
		\item A un instant donné, peut-on estimer les gains potentiels d'une nouvelle étape de raffinage de la base d'apprentissage en cours de construction ? (cf. hypothèse d'\textbf{impact} en section~\ref{section:4.5-HYPOTHESE-IMPACT})
		% 6. Robustesse.
		\item Peut-on estimer l'influence d'une erreur ou d'une différence d'annotation dans la construction de la base d'apprentissage ? (cf. hypothèse de \textbf{robustesse} en section~\ref{section:4.6-HYPOTHESE-ROBUSTESSE})
	\end{todolist}
	
	% ILLUSTRATION: SCHEMA DES HYPOTHESES
	Afin d'illustrer ces interrogations, nous vous proposons de considérer de la figure~\ref{figure:HYPOTHESE-00-DEFAULT}. Dans les sections suivantes, cette figure évoluera pour résumer les études réalisées.
	
	\begin{figure}[H]
		\centering
		\includegraphics[width=0.8\textwidth]{figures/hypotheses-00-default}
		\caption{Illustration des études réalisées sur le \textit{clustering} interactif (\textit{étape 0/6}) en schématisant l'évolution de la performance (\textit{v-measure par rapport à une vérité terrain}) d'une base d'apprentissage en cours de construction en fonction du nombre d'itérations de la méthode (\textit{nombre d'annotations}).}
		\label{figure:HYPOTHESE-00-DEFAULT}
	\end{figure}
	
	% PRÉAMBULE TECHNIQUE : CPU + scrips + datasets.
	Pour ces études, l'exécution des différentes expériences a été réalisée sur des CPU \textit{Intel(R) Xeon(R) CPU E5-2660 v4 \@ 2.00GHz} et parallélisé avec la librairie Python \textit{multiprocessing} (un worker par CPU).
	Les scripts d'exécution et d'analyse de ces expériences, rédigés au sein de notebooks Python et/ou R, sont disponibles dans~\cite{schild:cognitivefactory-interactive-clustering-comparative-study:2021}.
	Enfin, les jeux de données utilisés pour ces études sont détaillés en Annexe~\ref{annex:C-ANNEXE-DATASET}.
	
	
	% TABLE DES MATIÈRES DU CHAPITRE
    \minitoc

    %%%%%--------------------------------------------------------------------
    %%%%% Section 4.1: Hypothèse d'efficacité.
    %%%%%--------------------------------------------------------------------
    \section{Hypothèse d'efficacité : « \textit{est-ce que la méthode fonctionne ?} »}
	\label{section:4.1-HYPOTHESE-EFFICACITE}
	
		%%% Formulation des hypothèses:
		Nous aimerions vérifier l'hypothèse d'efficacité suivante :
		\todo{à compléter}

		\begin{tcolorbox}[
			title=\textbf{Hypothèse d'efficacité},
			colback=gray!20,
			colframe=gray!50!black!75,
			width=\linewidth
		]
			« Une méthodologie d'annotation basée sur le \textit{clustering} interactif \textbf{peut converger} vers une vérité terrain préalablement établie (cf. figure~\ref{figure:HYPOTHESE-EFFICACITE}. »
			
			\begin{figure}[H]
				\centering
				\includegraphics[width=0.8\textwidth]{figures/hypotheses-01-efficacite}
				\caption{Illustration des études réalisées sur le \textit{clustering} interactif (\textit{étape 1/6}) en schématisant l'évolution de la performance (\textit{v-measure par rapport à une vérité terrain}) d'une base d'apprentissage en cours de construction en fonction du nombre d'itérations de la méthode (\textit{nombre d'annotations}).}
				\label{figure:HYPOTHESE-EFFICACITE}
			\end{figure}

		\end{tcolorbox}
		
		%%%
		%%% Subsection 4.1.1: Étude de convergence
		%%%
		\subsection{Étude de convergence}
	
			%%% Protocole expérimental.
			\subsubsection{Protocole expérimental}
				\todo[inline]{Description succincte du protocole expérimental dans l'encadré d'hypothèse ?}
			
				% Description rapide du protocole.
				Pour vérifier l'hypothèse d'efficacité, nous proposons une simulation de création d'un jeu d’entraînement d'un assistant conversationnel en employant notre méthodologie d'annotation basée sur le \textit{clustering} interactif.
				Pour cela, nous ré-annotons une version non labellisée d'une vérité terrain à disposition, en commençant sans aucune contrainte et en terminant lorsque toutes les contraintes possibles entre les données soient annotées.
				
				% Description de la simulation de l'annotation.
				Pour simuler l'annotation de l'expert métier entre deux données, nous utilisons la comparaison des labels de la vérité terrain : ainsi, deux données auront une contrainte \textit{MUST-LINK} si elles ont le même label, et une contrainte \textit{CANNOT-LINK} sinon. Cela traduit le prérequis d'avoir un annotateur qui soit capable de caractériser la similitude entre deux données.
				
				% Description implémentation de l'interactive clustering.
				\todo{Description implémentation de l'interactive clustering (au préalable ?)}
				\todo{Description des paramètres testés}
				\todo{Description de l'évaluation}
				
				% Pseudo-code.
				La protocole expérimental est aussi décrit à l'aide du pseudo-code figurant dans Alg.~\ref{algorithm:4.1-PROTOCOLE-EXPERIMENTAL}.

				\begin{algorithm}
					\begin{algorithmic}[1]
						\Require données non segmentées, algorithmes et paramètres à tester
						\State \textbf{prétraitement}: suppression du bruits dans les données
						\State \textbf{vectorisation}: transformation des données en vecteurs
						\State \textbf{initialisation}: créer une liste vide de contraintes
						\State \textbf{clustering initial}: regrouper les données par similarité
						\State \textbf{évaluation}: comparer le clustering à la vérité terrain
						\Repeat
							\State \textbf{échantillonnage}: sélection de nouvelles contraintes à annoter
							\State \textbf{annotation}: simulation d'annotation en comparant les labels de la vérité terrain
							\State \textbf{clustering}: regrouper les données par similarité et avec les contraintes
							\State \textbf{évaluation}: comparer le clustering à la vérité terrain
						\Until{annotation de toutes les contraintes possibles}
						\Ensure données segmentées, toutes les contraints possibles sont annotées
					\end{algorithmic}
					\caption{Description en pseudo-code du protocole expérimental visant à estimer la faisabilité technique du clustering interactif.}
					\label{algorithm:4.1-PROTOCOLE-EXPERIMENTAL}
				\end{algorithm}

			%%% Résultats
			\subsubsection{Résultats obtenus}
				\todo[inline]{SECTION À RÉDIGER}
				
				\todo{max de clustering à itération 0}
				\todo{convergence}
				\todo{graphe évolution moyenne}

			%%% Discussion
			\subsubsection{Discussion}
				\todo[inline]{SECTION À RÉDIGER}
				
				\todo{convergence lente, mais convergence !}
				\todo{pas besoin d'une structure établie, mais de savoir distinguer}
				\todo{on peut optimiser (moyenne + écart-type itération convergence)}
	

    %%%%%--------------------------------------------------------------------
    %%%%% Section 4.2: Hypothèse d'efficience.
    %%%%%--------------------------------------------------------------------
    \section{Hypothèse d'efficience : « \textit{est-ce que l'implémentation est optimale ?} »}
	\label{section:4.2-HYPOTHESE-EFFICIENCE}
	
		%%% Formulation des hypothèses:
		Nous aimerions vérifier l'hypothèse d'efficience suivante :
		\todo{à compléter}

		\begin{tcolorbox}[
			title=\textbf{Hypothèse d'efficience},
			colback=gray!20,
			colframe=gray!50!black!75,
			width=\linewidth
		]
			« La vitesse de convergence du \textit{clustering} interactif \textbf{peut être optimisée} en réglant différents paramètres. Nous étudierons l'influence du prétraitement des données, de la vectorisation des données, de l'échantillonnage des contraintes à annoter et du \textit{clustering} sous contraintes (cf. figure~\ref{figure:HYPOTHESE-EFFICIENCE}. »
			
			
			\begin{figure}[H]
				\centering
				\includegraphics[width=0.8\textwidth]{figures/hypotheses-02-efficience}
				\caption{Illustration des études réalisées sur le \textit{clustering} interactif (\textit{étape 2/6}) en schématisant l'évolution de la performance (\textit{v-measure par rapport à une vérité terrain}) d'une base d'apprentissage en cours de construction en fonction du nombre d'itérations de la méthode (\textit{nombre d'annotations}).}
				\label{figure:HYPOTHESE-EFFICIENCE}
			\end{figure}

		\end{tcolorbox}
		
		%%%
		%%% Subsection 4.2.1: Étude des paramètres de convergence optimaux
		%%%
		\subsection{Étude des paramètres de convergence optimaux}
	
			%%% Protocole expérimental.
			\subsubsection{Protocole expérimental}
				\todo[inline]{Description succincte du protocole expérimental dans l'encadré d'hypothèse ?}
				\todo{Reprendre 4.1.1, mais pour estimer les meilleurs paramètres}
				\todo{seuils étudiés}
				\todo{ANOVA répété}

			%%% Résultats
			\subsubsection{Résultats obtenus}
				\todo{3 tableaux de paramètres}

			%%% Discussion
			\subsubsection{Discussion}
				\todo{meilleur paramétrage}
	

    %%%%%--------------------------------------------------------------------
    %%%%% Section 4.3: Hypothèse de pertinence.
    %%%%%--------------------------------------------------------------------
    \section{Hypothèse de pertinence : « \textit{est-ce le résultat est exploitable ?} »}
	\label{section:4.3-HYPOTHESE-PERTINENCE}
	
		%%% Formulation des hypothèses:
		Nous aimerions vérifier l'hypothèse de pertinence suivante :
		\todo{à compléter}

		\begin{tcolorbox}[
			title=\textbf{Hypothèse de pertinence},
			colback=gray!20,
			colframe=gray!50!black!75,
			width=\linewidth
		]
			« La vitesse de convergence du \textit{clustering} interactif \textbf{peut être optimisée} en réglant différents paramètres. Nous étudierons l'influence du prétraitement des données, de la vectorisation des données, de l'échantillonnage des contraintes à annoter et du \textit{clustering} sous contraintes (cf. figure~\ref{figure:HYPOTHESE-PERTINENCE}. »
			
			
			\begin{figure}[H]
				\centering
				\includegraphics[width=0.8\textwidth]{figures/hypotheses-03-pertinence}
				\caption{Illustration des études réalisées sur le \textit{clustering} interactif (\textit{étape 3/6}) en schématisant l'évolution de la performance (\textit{v-measure par rapport à une vérité terrain}) d'une base d'apprentissage en cours de construction en fonction du nombre d'itérations de la méthode (\textit{nombre d'annotations}).}
				\label{figure:HYPOTHESE-PERTINENCE}
			\end{figure}

		\end{tcolorbox}
		
		%%%
		%%% Subsection 4.3.1: Étude de la cohérence statistique de la base d'apprentissage en cours de construction
		%%%
		\subsection{Étude de la cohérence statistique de la base d'apprentissage en cours de construction}
		
			%%% Protocole expérimental.
			\subsubsection{Protocole expérimental}
				\todo[inline]{Description succincte du protocole expérimental dans l'encadré d'hypothèse ?}

			%%% Résultats
			\subsubsection{Résultats obtenus}

			%%% Discussion
			\subsubsection{Discussion}
		
		%%%
		%%% Subsection 4.3.2: Étude de la pertinence sémentique de la base d'apprentissage en cours de construction
		%%%
		\subsection{Étude de la pertinence sémentique de la base d'apprentissage en cours de construction}
		
			%%% Protocole expérimental.
			\subsubsection{Protocole expérimental}
				\todo[inline]{Description succincte du protocole expérimental dans l'encadré d'hypothèse ?}

			%%% Résultats
			\subsubsection{Résultats obtenus}

			%%% Discussion
			\subsubsection{Discussion}
	

    %%%%%--------------------------------------------------------------------
    %%%%% Section 4.4: Hypothèse sur les coûts.
    %%%%%--------------------------------------------------------------------
    \section{Hypothèse sur les coûts : « \textit{combien dois-je investir ?} »}
	\label{section:4.4-HYPOTHESE-COUTS}
	
		%%% Formulation des hypothèses:
		Nous aimerions vérifier l'hypothèse sur les coûts suivante :
		\todo{à compléter}

		\begin{tcolorbox}[
			title=\textbf{Hypothèse sur les coûts},
			colback=gray!20,
			colframe=gray!50!black!75,
			width=\linewidth
		]
			« Il est possible de \textbf{mesurer le temps nécessaire} à une méthodologie d'annotation basée sur le \textit{clustering} interactif pour obtenir un résultat exploitable (cf. figure~\ref{figure:HYPOTHESE-COUTS}. »
			
			
			\begin{figure}[H]
				\centering
				\includegraphics[width=0.8\textwidth]{figures/hypotheses-04-couts}
				\caption{Illustration des études réalisées sur le \textit{clustering} interactif (\textit{étape 4/6}) en schématisant l'évolution de la performance (\textit{v-measure par rapport à une vérité terrain}) d'une base d'apprentissage en cours de construction en fonction du nombre d'itérations de la méthode (\textit{nombre d'annotations}).}
				\label{figure:HYPOTHESE-COUTS}
			\end{figure}

		\end{tcolorbox}
		
		%%%
		%%% Subsection 4.4.1: Étude d'estimation du temps d'annotation par un expert métier
		%%%
		\subsection{Étude d'estimation du temps d'annotation par un expert métier}
		
			%%% Protocole expérimental.
			\subsubsection{Protocole expérimental}
				\todo[inline]{Description succincte du protocole expérimental dans l'encadré d'hypothèse ?}

			%%% Résultats
			\subsubsection{Résultats obtenus}

			%%% Discussion
			\subsubsection{Discussion}
		
		%%%
		%%% Subsection 4.4.2: Étude d'estimation du temps de calcul des algorithmes
		%%%
		\subsection{Étude d'estimation du temps de calcul des algorithmes}
		
			%%% Protocole expérimental.
			\subsubsection{Protocole expérimental}
				\todo[inline]{Description succincte du protocole expérimental dans l'encadré d'hypothèse ?}

			%%% Résultats
			\subsubsection{Résultats obtenus}

			%%% Discussion
			\subsubsection{Discussion}
		
		%%%
		%%% Subsection 4.4.3: Étude d'estimation du temps total d'un projet d'annotation
		%%%
		\subsection{Étude d'estimation du temps total d'un projet d'annotation}
		
			%%% Protocole expérimental.
			\subsubsection{Protocole expérimental}
				\todo[inline]{Description succincte du protocole expérimental dans l'encadré d'hypothèse ?}

			%%% Résultats
			\subsubsection{Résultats obtenus}

			%%% Discussion
			\subsubsection{Discussion}
	

    %%%%%--------------------------------------------------------------------
    %%%%% Section 4.5: Hypothèse d'impact.
    %%%%%--------------------------------------------------------------------
    \section{Hypothèse d'impact : « \textit{quel gain à chaque itération ?} »}
	\label{section:4.5-HYPOTHESE-IMPACT}
	
		%%% Formulation des hypothèses:
		Nous aimerions vérifier l'hypothèse d'impact :
		\todo{à reformuler}

		\begin{tcolorbox}[
			title=\textbf{Hypothèse d'impact},
			colback=gray!20,
			colframe=gray!50!black!75,
			width=\linewidth
		]
			« Il est possible d'estimer quand méthodologie d'annotation basée sur le \textit{clustering} interactif \textbf{a converger} vers un résultat satisfaisant (cf. figure~\ref{figure:HYPOTHESE-IMPACT}. »
			
			
			\begin{figure}[H]
				\centering
				\includegraphics[width=0.8\textwidth]{figures/hypotheses-05-impact}
				\caption{Illustration des études réalisées sur le \textit{clustering} interactif (\textit{étape 5/6}) en schématisant l'évolution de la performance (\textit{v-measure par rapport à une vérité terrain}) d'une base d'apprentissage en cours de construction en fonction du nombre d'itérations de la méthode (\textit{nombre d'annotations}).}
				\label{figure:HYPOTHESE-IMPACT}
			\end{figure}

		\end{tcolorbox}
		
		%%%
		%%% Subsection 4.5.1: Étude d'estimation des cas d'arrêts de la méthode
		%%%
		\subsection{Étude d'estimation des cas d'arrêts de la méthode}
		
			%%% Protocole expérimental.
			\subsubsection{Protocole expérimental}
				\todo[inline]{Description succincte du protocole expérimental dans l'encadré d'hypothèse ?}

			%%% Résultats
			\subsubsection{Résultats obtenus}

			%%% Discussion
			\subsubsection{Discussion}
	

    %%%%%--------------------------------------------------------------------
    %%%%% Section 4.6: Hypothèse de robustesse.
    %%%%%--------------------------------------------------------------------
    \section{Hypothèse de robustesse : « \textit{quelle influence d'une erreur ?} »}
	\label{section:4.6-HYPOTHESE-ROBUSTESSE}
	
		%%% Formulation des hypothèses:
		Nous aimerions vérifier l'hypothèse de robustesse :
		\todo{à reformuler}

		\begin{tcolorbox}[
			title=\textbf{Hypothèse de robustesse},
			colback=gray!20,
			colframe=gray!50!black!75,
			width=\linewidth
		]
			« Il est possible d'\textbf{estimer l'influence d'une différence d'annotation} lors d'une méthodologie d'annotation basée sur le \textit{clustering} interactif (cf. figure~\ref{figure:HYPOTHESE-ROBUSTESSE}. »
			
			
			\begin{figure}[H]
				\centering
				\includegraphics[width=0.8\textwidth]{figures/hypotheses-06-robustesse}
				\caption{Illustration des études réalisées sur le \textit{clustering} interactif (\textit{étape 6/6}) en schématisant l'évolution de la performance (\textit{v-measure par rapport à une vérité terrain}) d'une base d'apprentissage en cours de construction en fonction du nombre d'itérations de la méthode (\textit{nombre d'annotations}).}
				\label{figure:HYPOTHESE-ROBUSTESSE}
			\end{figure}

		\end{tcolorbox}
		
		%%%
		%%% Subsection 4.6.1: Étude de simulation d'erreurs d'annotations
		%%%
		\subsection{Étude de simulation d'erreurs d'annotations}
		
			%%% Protocole expérimental.
			\subsubsection{Protocole expérimental}
				\todo[inline]{Description succincte du protocole expérimental dans l'encadré d'hypothèse ?}

			%%% Résultats
			\subsubsection{Résultats obtenus}

			%%% Discussion
			\subsubsection{Discussion}
		
		%%%
		%%% Subsection 4.6.2: Étude d'annotation avec des paradigmes différents
		%%%
		\subsection{Étude d'annotation avec des paradigmes différents}
		
			%%% Protocole expérimental.
			\subsubsection{Protocole expérimental}
				\todo[inline]{Description succincte du protocole expérimental dans l'encadré d'hypothèse ?}

			%%% Résultats
			\subsubsection{Résultats obtenus}

			%%% Discussion
			\subsubsection{Discussion}
			
	
    %%%%%--------------------------------------------------------------------
    %%%%% Section 4.7:
    %%%%%--------------------------------------------------------------------
    \section{Autres études à réaliser}
	\label{section:4.7-ETUDES-DIVERSES}
	\todo[inline]{SECTION À RÉDIGER}

        \subsection{Choix du nombre de clusters ==> problème de recherche complexe}
            o	Piste de résolution : plusieurs clusterings + vote collaboratif ? algorithmes sans le nombre de clusters en hyper-paramètres

        \subsection{Impact d'un modèle de langage ==> nécessite de nombreuses données spécifiques au domaine}
            o	Piste de résolution : script d'étude comparative déjà prêt, mais il manque les données opensources… 

        \subsection{Paradigme d’annotation (intention vs dialogue) ==> problème d'UX + objectif métier}
            o	Etude Ergo, sort de mon domaine d'expertise

        \subsection{(et plein d'autres que j'ajouterai au fur et à mesure de ma rédaction)}
            o	


%%%%%--------------------------------------------------------------------
%%%%% Chapitre: Guide d'utilisation
%%%%%--------------------------------------------------------------------
\chapter{Guide d'utilisation de la méthode}
\label{chapter:5-GUIDE}

    %%%%%--------------------------------------------------------------------
    %%%%% Section 5.1:
    %%%%%--------------------------------------------------------------------
    \section{Organisation \texttt{ITTER}}
		\label{section:5.1-GUIDE-ITTER}
		\todo[inline]{SECTION À RÉDIGER}
	
	
    %%%%%--------------------------------------------------------------------
    %%%%% Section 5.2:
    %%%%%--------------------------------------------------------------------
    \section{Conseils pratiques}
		\label{section:5.2-GUIDE-CONSEILS}
		\todo[inline]{SECTION À RÉDIGER}

%%%%%--------------------------------------------------------------------
%%%%% Conclusion
%%%%%--------------------------------------------------------------------
\input{chapters/6_conclusion.tex}


%%%%%%%%%%%%%%%%%%%%%%%%%%%%%%%%%%%%%%%%%%%%%%%%%%%%%%%%%%%%%%%%%%%%%%%
%%%%% ANNEXES
%%%%%%%%%%%%%%%%%%%%%%%%%%%%%%%%%%%%%%%%%%%%%%%%%%%%%%%%%%%%%%%%%%%%%%%

\PutLineInToc
\Annexes

\Annex{Annexe théorique}

    \section{Les algorithmes de clustering }

        \subsection{Kmeans}
        \subsection{Hierarchique}
        \subsection{Spectral}
        \subsection{DBScan}
        \subsection{Affinity Propagation}

    \section{Evaluation d’une clustering (??ANNEXE??)}

        \subsection{Homogénéité – Complétude – Vmeasure}
        \subsection{FMC}

\Annex{Annexe technique}

    \section{package pypi interactive-clustering}
    \section{package pypi interactive-clustering-gui}
    \section{package pypi features-maximization-metrics}
    \section{experimentations jupyter notebook}

\Annex{Datasets}

    \section{french bank cards}
    \section{DNA press title}


%%%%%%%%%%%%%%%%%%%%%%%%%%%%%%%%%%%%%%%%%%%%%%%%%%%%%%%%%%%%%%%%%%%%%%%
%%%%% DIVERS
%%%%%%%%%%%%%%%%%%%%%%%%%%%%%%%%%%%%%%%%%%%%%%%%%%%%%%%%%%%%%%%%%%%%%%%

\PutLineInToc
\DontFrameChaptersInToc

%%%%%--------------------------------------------------------------------
%%%%% Bibliographie
%%%%%--------------------------------------------------------------------

%%% Affichage de la bibliographie:
\printbibliography

% Use "\nocite{*}" to add all .bib file.
% Use this command to see which references are note cited.

%%%%%--------------------------------------------------------------------
%%%%% Liste des TODOs
%%%%%--------------------------------------------------------------------

%%% Génère la liste des choses à faire:
\listoftodos[Liste des TODOs]

%%% Certaines choses à faire:
\todo[inline]{\textbf{Style d'écriture}: "je" ou "nous" ou "on" ?}

%%%%%--------------------------------------------------------------------
%%%%% Liste des figures
%%%%%--------------------------------------------------------------------

%%% Génère la liste des figures:
\renewcommand{\listfigurename}{Liste des figures}
\listoffigures

%%%%%--------------------------------------------------------------------
%%%%% Liste des tableaux
%%%%%--------------------------------------------------------------------

%%% Génère la liste des tableaux:
\renewcommand{\listtablename}{Liste des tableaux}
\listoftables

%%%%%--------------------------------------------------------------------
%%%%% Liste des algorithmes
%%%%%--------------------------------------------------------------------

%%% Génère la liste des algorithmes:
%\renewcommand{\listalgorithmname}{Liste des algorithmes}
\renewcommand{\listalgorithmcfname}{Liste des algorithmes}
\listofalgorithms  % ERROR TABLE MATIERES

%%%%%--------------------------------------------------------------------
%%%%% Liste des codes
%%%%%--------------------------------------------------------------------

%%% Génère la liste des codes:
\renewcommand{\lstlistlistingname}{Liste de codes informatiques}
\lstlistoflistings

%%%%%--------------------------------------------------------------------
%%%%% Glossaire
%%%%%--------------------------------------------------------------------

%%% Affichage du glossaire:
\printglossaries

%%%%%--------------------------------------------------------------------
%%%%% Index
%%%%%--------------------------------------------------------------------

%%% Affichage de l'index:
\WriteThisInToc
\PrintIndex

\end{document}